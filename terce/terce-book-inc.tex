% !TEX TS-program = lualatex
% !TEX encoding = UTF-8

% This is a simple template for a LuaLaTeX document using gregorio scores.

% easter can be from march 22 to april 25

\usepackage{../definepsalms}
\usepackage{titlesec}
\usepackage{titletoc}
\usepackage{titleps}
\usepackage{letltxmacro}
\usepackage{changepage} % gives us \ifoddpage use [strict]
\usepackage[super]{nth}
\usepackage[savepos]{zref}
\usepackage{xparse}
\usepackage{setspace}
\usepackage{amsfonts}
\usepackage{refcount}
\usepackage{metalogo}
%\nofiles
%\includeonly{inc-grassi}
\LetLtxMacro{\oldnth}{\nth}
\renewcommand{\nth}[1]{{\addfontfeature{Numbers=Lining}\oldnth{#1}}}
\LetLtxMacro{\oldneedspace}{\needspace}
\renewcommand{\needspace}[1]{
	\checkoddpage\ifoddpage\oldneedspace{#1}\else\fi
}
\let\gredagger=\dag
\newcommand*\cleartoleftpage{%
  \clearpage
  \ifodd\value{page}\hbox{}\newpage\fi
}
\hyphenation{GregoBase}

%\usepackage{hyperref}
\newcommand{\phantomsection}{}

\setcounter{secnumdepth}{-1}

\ifthenelse{\boolean{lettersize}}{
	\def\mywidth{8.5in}
	\def\myheight{11in}
}{
	\def\mywidth{6in}
	\def\myheight{9in}
}

% !TEX TS-program = lualatex
% !TEX encoding = UTF-8
% usual packages loading:
%\usepackage{luatextra}
%\usepackage{graphicx} % support the \includegraphics command and options
\usepackage{geometry} % See geometry.pdf to learn the layout options. There are lots.
\ifx\undefined\mywidth
    \geometry{letterpaper} % or letterpaper (US) or a5paper or....
\else
    \geometry{papersize={\mywidth,\myheight}}
\fi
\usepackage{expl3}
\let\luatexlocalrightbox\localrightbox
\let\luatexlocalleftbox\localleftbox
\usepackage{gregoriotex} % for gregorio score inclusion
\usepackage{import}

% If you use usual TeX fonts, here is a starting point:
%\usepackage{palatino}
%\input{glyphtounicode} \pdfglyphtounicode{f_f}{FB00} \pdfglyphtounicode{f_f_i}{FB03} \pdfglyphtounicode{f_f_l}{FB04}
%\pdfglyphtounicode{Q_u}{E048} \pdfglyphtounicode{O_e}{0152} \pdfglyphtounicode{o_e}{0153}
%\pdfgentounicode=1
% to change the font to something better, you can install the kpfonts package (if not already installed). To do so
% go open the "TeX Live Manager" in the Menu Start->All Programs->TeX Live 2010
% the additional width of the additional lines (compared to the width of the glyph they're associated with)
\grechangedim{additionallineswidth}{0.14584 cm}{scalable}%
% width of the additional lines, used only for the custos (maybe should depend on the width of the custos...)
% the width is the one for the custos at end of lines, the line for custos in the middle of a score is the same
% multiplied by 2.
\grechangedim{additionalcustoslineswidth}{0.09114 cm}{scalable}%
% null space
\grechangedim{zerowidthspace}{0 cm}{scalable}%
% space between glyphs in the same element
\grechangedim{interglyphspace}{0.06927 cm plus 0.00363 cm minus 0.00363 cm}{scalable}%
% space between an alteration (flat or natural) and the next glyph
\grechangedim{alterationspace}{0.07747 cm plus 0.01276 cm minus 0.00455 cm}{scalable}%
% space between a clef and a flat (for clefs with flat)
\grechangedim{clefflatspace}{0.05469 cm plus 0.00638 cm minus 0.00638 cm}{scalable}%
% space before a choral sign
\grechangedim{beforelowchoralsignspace}{0.04556 cm plus 0.00638 cm minus 0.00638 cm}{scalable}%
% when bolshifts are enabled, minimal space between a clef at the beginning of the line and a leading alteration glyph (should be larger than clefflatspace so that a flatted clef can be distinguished from a flat which is part of the first glyph on a line, but also smaller than spaceafterlineclef, the distance from the clef to the first notes)
\grechangedim{beforealterationspace}{0.1 cm}{scalable}%
% space between elements
\grechangedim{interelementspace}{0.06927 cm plus 0.00182 cm minus 0.00363 cm}{scalable}%
% larger space between elements
\grechangedim{largerspace}{0.10938 cm plus 0.01822 cm minus 0.00911 cm}{scalable}%
% space between elements in ancient notation
\grechangedim{nabcinterelementspace}{0.06927 cm plus 0.00182 cm minus 0.00363 cm}{scalable}%
% larger space between elements in ancient notation
\grechangedim{nabclargerspace}{0.10938 cm plus 0.01822 cm minus 0.00911 cm}{scalable}%
% space between elements which has the size of a note
\grechangedim{glyphspace}{0.21877 cm plus 0.01822 cm minus 0.01822 cm}{scalable}%
% space before custos
\grechangedim{spacebeforecustos}{0.1823 cm plus 0.31903 cm minus 0.0638 cm}{scalable}%
% space before punctum mora and augmentum duplex
\grechangedim{spacebeforesigns}{0.05469 cm plus 0.00455 cm minus 0.00455 cm}{scalable}%
% space after punctum mora and augmentum duplex
\grechangedim{spaceaftersigns}{0.08203 cm plus 0.0082 cm minus 0.0082 cm}{scalable}%
% space after a clef at the beginning of a line
\grechangedim{spaceafterlineclef}{0.27345 cm plus 0.14584 cm minus 0.01367 cm}{scalable}%
% minimal space between notes of different words
%\grechangedim{interwordspacenotes}{0.27 cm plus 0.15 cm minus 0.05 cm}{scalable}%
\grechangedim{interwordspacenotes}{0.27 cm plus 0.08 cm minus 0.05 cm}{scalable}%
% minimal space between notes of the same syllable.
% Warning: always keep minus to 0; also keep plus very low, or some words won't be hyphenated
%\grechangedim{intersyllablespacenotes}{0.24 cm plus 0.04cm minus 0cm}{scalable}%
\grechangedim{intersyllablespacenotes}{0.24 cm plus 0.04cm minus 0cm}{scalable}%
% minimal space between letters of different words. Makes sense to have
% the same plus and minus as interwordspacenotes.
%\grechangedim{interwordspacetext}{0.38 cm plus 0.15 cm minus 0.05 cm}{scalable}%
\grechangedim{interwordspacetext}{0.18 cm plus 0.08 cm minus 0.05 cm}{scalable}%
% Versions of interword spaces for euouae blocks
%\grechangedim{interwordspacenotes@euouae}{0.19 cm plus 0.1 cm minus 0.05 cm}{scalable}%
\grechangedim{interwordspacenotes@euouae}{0.13 cm plus 0.1 cm minus 0.05 cm}{1}%
%\grechangedim{interwordspacetext@euouae}{0.27 cm plus 0.1 cm minus 0.05 cm}{scalable}%
\grechangedim{interwordspacetext@euouae}{0.13 cm plus 0.1 cm minus 0.05 cm}{1}%
% space between notes of a bivirga or trivirga
\grechangedim{bitrivirspace}{0.06927 cm plus 0.00182 cm minus 0.00546 cm}{scalable}%
% space between notes of a bistropha or tristrophae
\grechangedim{bitristrospace}{0.06927 cm plus 0.00182 cm minus 0.00546 cm}{scalable}%
% space between two punctum inclinatum
\grechangedim{punctuminclinatumshift}{-0.03918 cm plus 0.0009 cm minus 0.0009 cm}{scalable}%
% space before puncta inclinata
\grechangedim{beforepunctainclinatashift}{0.05286 cm plus 0.00728 cm minus 0.00455 cm}{scalable}%
% space between a punctum inclinatum and a punctum inclinatum deminutus
\grechangedim{punctuminclinatumanddebilisshift}{-0.02278 cm plus 0.0009 cm minus 0.0009 cm}{scalable}%
% space between two punctum inclinatum deminutus
\grechangedim{punctuminclinatumdebilisshift}{-0.00728 cm plus 0.0009 cm minus 0.0009 cm}{scalable}%
% space between puncta inclinata, larger ambitus (range=3rd)
\grechangedim{punctuminclinatumbigshift}{0.07565 cm plus 0.0009 cm minus 0.0009 cm}{scalable}%
% space between puncta inclinata, larger ambitus (range=4th -or more?-)
\grechangedim{punctuminclinatummaxshift}{0.17865 cm plus 0.0009 cm minus 0.0009 cm}{scalable}%
% space for the bars (inside syllables)
%first for virgula and divisio minima
\grechangedim{spacearoundsmallbar}{0.1823 cm plus 0.22787 cm minus 0.00469 cm}{scalable}%
%then divisio minor
\grechangedim{spacearoundminor}{0.1823 cm plus 0.22787 cm minus 0.00469 cm}{scalable}%
%divisio major
\grechangedim{spacearoundmaior}{0.1823 cm plus 0.22787 cm minus 0.00469 cm}{scalable}%
%divisio finalis
\grechangedim{spacearoundfinalis}{0.1823 cm plus 0.22787 cm minus 0.00469 cm}{scalable}%
%a special space for finalis, for when it is the last glyph
\grechangedim{spacebeforefinalfinalis}{0.29169 cm plus 0.07292 cm minus 0.27345 cm}{scalable}%
% additional space that will appear around bars that are preceded by a custos and followed by a key.
\grechangedim{spacearoundclefbars}{0.03645 cm plus 0.00455 cm minus 0.0009 cm}{scalable}%
% space between the text and the text of the bar
\grechangedim{textbartextspace}{0.24611 cm plus 0.13672 cm minus 0.04921 cm}{scalable}%
% minimal space between a note and a bar
\grechangedim{notebarspace}{0.31903 cm plus 0.27345 cm minus 0.02824 cm}{scalable}%
% maximal space between two syllables for which we consider a dash is not needed
\grechangedim{maximumspacewithoutdash}{0.00 cm}{scalable}%
% an extensible space for the beginning of lines
\grechangedim{afterclefnospace}{0 cm plus 0.27345 cm minus 0 cm}{scalable}%
% space between the initial and the beginning of the score
\grechangedim{afterinitialshift}{0.2457 cm}{scalable}%
% space before the initial
\grechangedim{beforeinitialshift}{0.2457 cm}{scalable}%
% when bolshifts are enabled, minimum space between beginning of line and first syllable text
\grechangedim{minimalspaceatlinebeginning}{0.05 cm}{scalable}%
% space to force the initial width to.  Ignored when 0.
\grechangedim{manualinitialwidth}{0 cm}{scalable}%
% distance to move the initial up by
\grechangedim{initialraise}{0 cm}{scalable}%
% Space between lines in the annotation
\grechangedim{annotationseparation}{0.05cm}{scalable}%
% Amount to raise (positive) or lower (negative) the annotations from the default position (base line of top annotation aligned with top line of staff)
\grechangedim{annotationraise}{0cm}{scalable}%
% space at the beginning of the lines if there is no clef
\grechangedim{noclefspace}{0.1 cm}{scalable}%
% space around a clef change
\grechangedim{clefchangespace}{0.01768 cm plus 0.00175 cm minus 0.01768 cm}{scalable}%
%When \gre@clivisalignment is 2, this distance is the maximum length of the consonants after vowels for which the clivis will be aligned on its center.
\grechangedim{clivisalignmentmin}{0.3 cm}{scalable}%



%%%%%%%%%%%%%%%%%%
% vertical spaces
%%%%%%%%%%%%%%%%%%

% first, we have two spaces for the chironomic signs
\grechangedim{abovesignsspace}{0.8 cm}{scalable}%
\grechangedim{belowsignsspace}{0 cm}{scalable}%
% the amount to shift down:
% (a) low choral signs that are not lower than the note, regardless of whether
%     it's on a line or in a space
% (b) high choral signs and low choral signs that are lower than the note which
%     are in a space
\grechangedim{choralsigndownshift}{0.00911 cm}{scalable}%
% the amount to shift up:
% (a) high choral signs and low choral signs that are lower than the note which
%     are on a line
\grechangedim{choralsignupshift}{0.04556 cm}{scalable}%
% the space for the translation
\grechangedim{translationheight}{0.5 cm}{scalable}%
%the space above the lines
\grechangedim{spaceabovelines}{0.45576 cm plus 0.36461 cm minus 0.09114 cm}{scalable}%
%the space between the lines and the bottom of the text
\grechangedim{spacelinestext}{0.60617 cm}{scalable}%
%the space beneath the text
\grechangedim{spacebeneathtext}{0 cm}{scalable}%
% height of the text above the note line
\grechangedim{abovelinestextraise}{-0.1 cm}{scalable}%
% height that is added at the top of the lines if there is text above the lines (it must be bigger than the text for it to be taken into consideration)
\grechangedim{abovelinestextheight}{0.3 cm}{scalable}%
% an additional shift you can give to the brace above the bars if you don't like it
\grechangedim{braceshift}{0 cm}{scalable}%
% a shift you can give to the accentus above the curly brace
\grechangedim{curlybraceaccentusshift}{-0.05 cm}{scalable}%


%\def\greinitialformat#1{{\fontsize{37}{37}\selectfont #1}}
%small > footnotesize > scriptsize > tiny


% my stuff
\usepackage[garamond]{../mypackage}
% end my stuff

\setgrefactor{17}

%\marginsize{25pt}{25pt}{25pt}{30pt}
\usepackage{calc}
%\setlength\headsep{20pt}
%\setlength\footskip{15pt}
\setlength\headheight{15pt}
\setlength\headsep{22pt}
\ifx\undefined\tenebrae
    \geometry{outer=25pt,inner=25pt,top=22pt+\headsep+\headheight,bottom=25pt+\footskip}
\else
    \ifx\undefined\mywidth
        %had been .3 outer, .4 inner
        %let's try .75 for inner and .5 for outer
        %now let's go to .35 outer, .9 inner
        %this time let's try .4 and .85
        \ifbook{\geometry{outer=0.4in,inner=0.85in,top=25pt+\headsep+\headheight,bottom=25pt+\footskip,twoside=true}}
        \ifnotbook{\geometry{outer=0.625in,inner=0.625in,top=25pt+\headsep+\headheight,bottom=25pt+\footskip,twoside=true}}
    \else
        \setlength\headsep{0.25in}
        \setlength\footskip{0.3in}
        \geometry{outer=0.5in,inner=0.5in,top=0.25in+\headsep+\headheight,bottom=0.25in+\footskip,twoside=true}
    \fi
\fi

\pagestyle{fancy} % no header or footers
\let\oldheadrulewidth\headrulewidth
\renewcommand\headrulewidth{\ifnum\thepage=1
0pt
\else
\oldheadrulewidth
\fi}

\ifx\undefined\ifbook
    \newcommand{\ifbook}[1]{}
    \newcommand{\ifnotbook}[1]{#1}
\fi
\ifx\undefined\ifsmallbook
    \newcommand{\ifsmallbook}[1]{}
    \newcommand{\ifnotsmallbook}[1]{#1}
\fi
%\cfoot{\thepage}

\gresetbarspacing{new}
%\gresetlastline{justified}
\setlength\headheight{0.25in+15pt}
\setlength\headsep{1pc}
\setlength\topskip{0pc}
\setlength\footskip{1pc}
\geometry{outer=0.4in,inner=0.85in,top=0pc+\headheight+\headsep,bottom=0.4in,twoside=true}
\newpagestyle{main}{
%\setheadrule{0pt}
\sethead[\garamond{\thepage}][\garamond{\chaptertitle}][] % even
{}{\garamond{\sectiontitle}}{\garamond{\thepage}} % odd
\setfoot[][][] % even
{}{}{} % odd
}
\pagestyle{main}

%\grechangeglyph{Porrectus*}{*}{.alt}
%\grechangeglyph{TorculusResupinus*}{*}{.alt}

\titleformat
{\section} % command
[block] % shape
{\phantomsection\large\addfontfeature{Numbers=Lining}} % format
{} % label
{} % sep
{
    % \rule{\textwidth}{1pt}
    % \vspace{1ex}
    \centering
} % before-code
%[
% \vspace{-0.5ex}%
% \rule{\textwidth}{0.3pt}
%] % after-code
 
 
\titleformat{\chapter}[block]
{\thispagestyle{empty}\phantomsection\Large\scshape\addfontfeature{Numbers=Lining}}
{}{0.5em}{\centering}
 
\titlespacing{\chapter}{0pt}{6pt-\headheight}{1pc}
\titlespacing{\section}{0pt}{*2.5}{*1}
\titleclass{\chapter}{top}
\newcommand{\chapterbreak}{\clearpage}
%\titleclass{\section}{top}

\contentsmargin{2pc}
%\dottedcontents{chapter}[2.3em]{}{2.3em}{1pc}
\titlecontents{chapter}[2.3em]{}{\contentslabel{2.3em}}{\hspace*{-2.3em}}{}
\dottedcontents{section}[5.5em]{}{3.2em}{1pc}

\newcommand{\printnote}[1]{
	{\normalsize \emph{#1}}
}
\newcommand{\subtitle}[1]{
{
	\centering
	{\addfontfeature{Numbers=Lining} \normalsize \emph{#1}}\par
}
}

\newcommand{\deusinadjutorium}{\noindent\printnote{\Vbar~\emph{Deus in adjutórium}, p.~\pageref{deusinadjutorium}}}
%\newcommand{\deusinadjutoriumsolemn}{\noindent\printnote{\Vbar~\emph{Deus in adjutórium}, p.~\pageref{deusinadjutoriumsolemn}.}}
\newcommand{\printcollect}[2]{
	\ifx\undefined\begincollectcols\def\begincollectcols{\begin{parcolumns}[rulebetween]{2}}\fi
	\ifx\printcollectheading\undefined\def\printcollectheading{T}\fi
	\if\printcollectheading T
	\oldneedspace{3\baselineskip}
	\medskip
	{\centering\large Collect.\par}
	\smallskip
	\fi
	\begincollectcols
	\sloppy
	\prayer{#1}{#2}
	\end{parcolumns}
	\let\begincollectcols=\undefined
}
\newcommand{\benedicamusdominoreference}[1]{%
	\ifthenelse{\boolean{includebenedicamusdominoreferences}}{
		\Vbar~\emph{Benedicámus Dómino \IfInteger{#1}{#1}{\csname benedicamusdominoname#1\endcsname}}, page \pageref{benedicamusdomino-#1}.
	}{}
}
\newcommand{\benedicamusdominoreferencelentoreaster}{%
	\ifthenelse{\boolean{includebenedicamusdominoreferences}}{
		In Lent, \Vbar~\emph{Benedicámus Dómino \benedicamusdominonamelent}, page \pageref{benedicamusdomino-lent}, or in Easter, \emph{\benedicamusdominonameeaster}, page \pageref{benedicamusdomino-easter}.
	}{}
}
\newcommand{\benedicamusdominomaster}[1]{
	\ifthenelse{\boolean{includebenedicamusdominoreferences}}{
		\noindent\printnote{\benedicamusdominoreference{#1}}
		\ifx\postbenedicamusdomino\undefined\else\postbenedicamusdomino\fi
	}{}
	\bigskip
	\hrule
}
\newcommand{\benedicamusdominolentoreaster}{
	\ifthenelse{\boolean{includebenedicamusdominoreferences}}{
		\noindent\printnote{\benedicamusdominoreferencelentoreaster}
		\ifx\postbenedicamusdomino\undefined\else\postbenedicamusdomino\fi
	}{}
	\bigskip
	\hrule
}

\ifthenelse{\boolean{lettersize}}{
	\newcommand{\psalmcolsoverride}[1][0]{
	}	
}{
	\newcommand{\psalmcolsoverride}[1][0]{
		\def\beginpsalmcols{\begin{parcolumns}[rulebetween,colwidths={1=0.45\linewidth}]{2}}
		\ifnum#1=110
		\def\beginpsalmcols{\begin{parcolumns}[rulebetween,colwidths={1=0.465\linewidth}]{2}}
		\fi
		\ifnum#1=111
		\def\beginpsalmcols{\begin{parcolumns}[rulebetween,colwidths={1=0.475\linewidth}]{2}}
		\fi
		%\ifnum#1=112
		%\def\beginpsalmcols{\begin{parcolumns}[rulebetween,colwidths={1=0.445\linewidth}]{2}}
		%\fi
		\ifnum#1=116
		\def\beginpsalmcols{\begin{parcolumns}[rulebetween,colwidths={1=0.5655\linewidth}]{2}}
		\fi
		\ifnum#1=121
		\def\beginpsalmcols{\begin{parcolumns}[rulebetween,colwidths={1=0.47\linewidth}]{2}}
		\fi
		\ifnum#1=125
		\def\beginpsalmcols{\begin{parcolumns}[rulebetween,colwidths={1=0.49\linewidth}]{2}}
		\fi
		\ifnum#1=129
		\def\beginpsalmcols{\begin{parcolumns}[rulebetween,colwidths={1=0.475\linewidth}]{2}}
		\fi
		\ifnum#1=131
		\def\beginpsalmcols{\begin{parcolumns}[rulebetween,colwidths={1=0.4875\linewidth},distance=1em]{2}}
		\fi
		\ifnum#1=137
		\def\beginpsalmcols{\begin{parcolumns}[rulebetween]{2}}
		\fi
		\ifnum#1=147
		%\def\beginpsalmcols{\begin{parcolumns}[rulebetween,colwidths={1=0.465\linewidth}]{2}}
		\def\beginpsalmcols{\begin{parcolumns}[rulebetween,colwidths={1=0.475\linewidth}]{2}}
		\fi
	}
}
\newcommand{\printvrcommem}{
		{\normalsize
		\ifx\beginvrcols\undefined\def\beginvrcols{\begin{parcolumns}[rulebetween]{2}}\fi
		\beginvrcols
		\colchunk{%
			\selectlanguage{latin}%
	    \Vbar{}~\commvlatin{}%
    }
		\colchunk{%
			\selectlanguage{american}%
	    \Vbar{}~\commvtranslation{}%
		}%
		\colplacechunks%
    \colchunk{%
			\selectlanguage{latin}%
	    \Rbar{}~\commrlatin{}%
		}%
		\colchunk{%
			\selectlanguage{american}%
	    \Rbar{}~\commrtranslation{}%
		}
		\end{parcolumns}
		}
}
\newcommand{\printvrdirigatur}{
	\smallskip{}
	\oldneedspace{3\baselineskip}
	\noindent\printnote{If the Sunday is being commemorated:\\}\vspace{-0.5\baselineskip}
	{
		\ifthenelse{\boolean{birmingham}}{
			\def\beginvrcols{\begin{parcolumns}[rulebetween,colwidths={1=0.53\linewidth}]{2}}
		}{
			\def\beginvrcols{\begin{parcolumns}[rulebetween,colwidths={1=0.51\linewidth}]{2}}
		}
		\def\commvlatin{Dirigátur Dómine orátio \textbf{mé}a.}
		\def\commrlatin{Sicut incénsum in conspéctu \textbf{tú}o.}
		\def\commvtranslation{\noindent{}Let my prayer be directed, O Lord.}
		\def\commrtranslation{\noindent{}As incense in Thy sight.}
		\printvrcommem{}
	}
}
\newcommand{\printvrmanenobiscum}{
	\smallskip{}
	\oldneedspace{3\baselineskip}
	\noindent\printnote{If the Sunday is being commemorated:\\}\vspace{-0.5\baselineskip}
	{
		\def\commvlatin{Mane nobíscum Dómine, alle\textbf{lú}ia.}
		\def\commrlatin{Quóniam advesperáscit, alle\textbf{lú}ia.}
		\def\commvtranslation{Stay with us O Lord, alleluia.}
		\def\commrtranslation{Because it is towards evening, alleluia.}
		\printvrcommem{}
	}
}
\newcommand{\printvrdominusincaelo}{
	\smallskip{}
	\oldneedspace{3\baselineskip}
	\noindent\printnote{If the Sunday is being commemorated:\\}\vspace{-0.5\baselineskip}
	{
		\def\commvlatin{Dóminus in cælo, alle\textbf{lú}ia.}
		\def\commrlatin{Parávit sedem suam, alle\textbf{lú}ia.}
		\def\commvtranslation{The Lord in heaven, alleluia.}
		\def\commrtranslation{Has prepared His throne, alleluia.}
		\printvrcommem{}
	}
}
\newcommand{\printcommemoration}[2][.]{
	\def\gabcfolder{#1}
	\input{\gabcfolder/#2}
	\subtitle{\comheadingtext}
	\sectionmark{\comheadingtext}
	{
	\def\noeuouae{F}
	%\printgabc{At Magn.}{\oldstylenums{\commagantlinetwo}}{\commagantinitial}{\commaganttex}
		{%
			\gresetinitiallines{0}
			\ifx\gabcfolder\undefined
		        \gregorioscore{\commaganttex}
		    \else
		        \gregorioscore{\gabcfolder/\commaganttex}
		    \fi

			\greseteolcustos{auto}%
		}
	}
	\translation[]{\englishcommagantiphon}

	\smallskip
	\ifx\commvlatin\undefined
		\printvr[\greseteolcustos{manual}]{\commvrtex}{\commvtranslation}{\commrtranslation}
			\greseteolcustos{auto}
	\else
	{
		\oldneedspace{3\baselineskip}
		\printvrcommem{}
	}
	\fi

	\printcollect{\latincomcollect}{\englishcomcollect}
}
\DeclareDocumentCommand{\printbothversions}{ O{\undefined} O{\undefined} m m }{
% #1 & #2 are the label
% #3 is what to check for
% #4 is the body of what to print
	\ifx#3\undefined
		\ifx\definevesperspropersalt\undefined\else
	  {
			\ifx\vesperspropersaltnote\undefined\else
		    \oldneedspace{3\baselineskip}
				\printnote{\vesperspropersaltnote}
			\fi
			\definevesperspropersalt
			\ifx#2\undefined\else\oldneedspace{5\baselineskip}\label{#2}\fi
	  	#4
		}
		\medskip
		\fi
		\ifx\definevesperspropers\undefined\else
	  {
			\ifx\vesperspropersnote\undefined\else
	    	\oldneedspace{3\baselineskip}
				\printnote{\vesperspropersnote}
			\fi
			\definevesperspropers
			\ifx#1\undefined\else\oldneedspace{5\baselineskip}\label{#1}\fi
	  	#4
		}
		\fi
	\else
  {
		\ifx#1\undefined\else\label{#1}\fi
  	#4
	}
	\fi
}
\newcommand{\printpsalms}{
	\printpsalm{1}{\psalmonenum}{\psalmonetone\psalmoneend}{\antonetex}{\antoneinitial}
	\medskip
	\needspace{3\baselineskip}
	\printpsalm{2}{\psalmtwonum}{\psalmtwotone\psalmtwoend}{\anttwotex}{\anttwoinitial}
	\medskip
	\needspace{3\baselineskip}
	\printpsalm{3}{\psalmthreenum}{\psalmthreetone\psalmthreeend}{\antthreetex}{\antthreeinitial}
	\medskip
	\needspace{3\baselineskip}
	\printpsalm{4}{\psalmfournum}{\psalmfourtone\psalmfourend}{\antfourtex}{\antfourinitial}
	\medskip
	\ifx\psalmfivenum\undefined{
		\ifx\antfivetex\undefined{%then it is a different antiphon for each
			\ifx\definevesperspropersalt\undefined\else{
				\definevesperspropersalt
				\ifx\vesperspropersaltnote\undefined\else
					\oldneedspace{3\baselineskip}
					\noindent\printnote{\vesperspropersaltnote}
				\fi
				\printpsalm{5}{\psalmfivenum}{\psalmfivetone\psalmfiveend}{\antfivetex}{\antfiveinitial}
			}\fi
			\ifx\definevesperspropers\undefined\else{
				\definevesperspropers
				\ifx\vesperspropersnote\undefined\else
					\oldneedspace{3\baselineskip}
					\noindent\printnote{\vesperspropersnote}
				\fi
				\printpsalm{5}{\psalmfivenum}{\psalmfivetone\psalmfiveend}{\antfivetex}{\antfiveinitial}
			}\fi
		}\else{%they share the same antiphon
			\ifx\definevesperspropersalt\undefined\else{
				\definevesperspropersalt
				\ifx\vesperspropersaltnote\undefined\else
					\let\oldprepsalmtitle=\prepsalmtitlefive
					\def\prepsalmtitlefive{
						\smallskip
						\oldneedspace{3\baselineskip}
						\noindent\printnote{\vesperspropersaltnote}

						\ifx\oldprepsalmtitle\undefined\else\oldprepsalmtitle\fi
					}
				\fi
				\printpsalm{5}{\psalmfivenum}{\psalmfivetone\psalmfiveend}{\antfivetex}{\antfiveinitial}
			}\fi
			\def\onlyoneant{T}
			\ifx\definevesperspropers\undefined\else{
				\definevesperspropers
				\ifx\vesperspropersnote\undefined\else
					\oldneedspace{3\baselineskip}
					\noindent\printnote{\vesperspropersnote}
				\fi
				\let\oldprepsalmtitle=\prepsalmtitlefive
				\def\prepsalmtitlefive{\def\onlyoneant{F}\ifx\oldprepsalmtitle\undefined\else\oldprepsalmtitle\fi}
				\def\printrepeatantiphon{T}
				\printpsalm{5}{\psalmfivenum}{\psalmfivetone\psalmfiveend}{\antfivetex}{\antfiveinitial}
			}\fi
		}\fi
	}\else{
		\printpsalm{5}{\psalmfivenum}{\psalmfivetone\psalmfiveend}{\antfivetex}{\antfiveinitial}
	}\fi
}
\newcommand{\printterce}[2][.]{
	{
		\grechangestaffsize{15}
		\def\gabcfolder{#1}
		\input{\gabcfolder/#2}

		\deusinadjutorium{}

		\ifx\printhymn\undefined\else\printhymn\fi

		\printgabc{Ant.}{\anttwolinetwo}{\anttwoinitial}{\anttwotex .noEuouae}

		\translation[]{\anttwotranslation}

		% psalms

		\printchapter{\chaptertext}{\chaptertranslation}

		\medskip

		\ifx\printshortresp\undefined\else\printshortresp\fi

		% vr

		\printcollect{\collect}{\collecttranslation}

		% benedicamus domino
	}
}

\sloppy
%\nofiles
\begin{document}
\normalsize
\grechangestaffsize{15}
{
	\thispagestyle{empty}
	% title page
	% general
	\vspace*{5\baselineskip}

	{\centering

	{\Huge
	Terce with Gregorian Chant

	}

	{\Large\medskip
	\emph{for}

	\medskip}

	{\Huge
	Sundays \& Holy Days

	}
	\vfill}
	\pagebreak
	% copyright page
	\thispagestyle{empty}
	\noindent{}Terce with Gregorian Chant for Sundays \& Holy Days: \emph{newly typeset, based on \emph{The Liber Usualis}, edited by the Benedictines of Solesmes (Desclee Company, 1961).}

	\bigskip{}\noindent{}Many thanks to Elie Roux for creating Gregorio, to all the people answering questions on the Gregorio forums, to Olivier Berton for creating GregoBase (http://gregobase.selapa.net) which makes it much easier to find all the chant that Andrew Hinkley and others have contributed to the public domain.  Also many thanks to my brother Benjamin, whose website (http://bbloomf.github.io/jgabc) makes it incredibly easy to typeset the psalms and substantially easier than otherwise to typeset Gregorian chant with Gregorio.

	\begin{flushright}
	\emph{Albert Bloomfield}\\
	Cincinnati, Ohio
	\end{flushright}

	\bigskip{}\noindent{}http://asbloomf.github.io/gabc-chants

	\vfill
%editions
	\bigskip{}\noindent{}%
	First edition, 9 December 2020.

	\hangindent=1em % indent all subsequent lines
    \bigskip

\makeatletter
	\noindent{}Typeset using \LuaLaTeX{} and Gregorio version \gre@gregoriotexversion{}
\makeatother

	\bigskip{}\noindent{}This work is free of known copyright restrictions.

	\bigskip{}\noindent{}%CreateSpace
	ISBN: \isbn{}
	%
  %\noindent{}Lulu ISBN: 978-1-329-59990-1
}
%\frontmatter
%\pagenumbering{roman}

\tableofcontents%

%\pagenumbering{arabic}
%\mainmatter

%\chapter{Common of Festal Vespers}

\sectionmark{Deus in adjutórium}
\addcontentsline{toc}{section}{Deus in adjutórium}

%\subtitle{Festal Tone}
\label{deusinadjutorium}
\printnote{The Festal Tone may be sung on any Sunday or Feast.}
\printnote{From Septuagesima to Wednesday in Holy Week, the \emph{Laus tibi} is said instead of \emph{Allelúia}.}

\bigskip
\def\deusinadjutoriumsolemn{F}
\printdeusinadjutorium{}
\vfil

%\subtitle{Solemn Tone}
\printnote{The Solemn Tone may only be sung on Feasts of the First or Second Class.}
\printnote{From Septuagesima to Wednesday in Holy Week, the \emph{Laus tibi} is said instead of \emph{Allelúia}.}

\bigskip
{
  \def\deusinadjutoriumsolemn{T}
  \ifthenelse{\boolean{birmingham}}{
    \def\includetranslation{F}
  }{}
  \label{deusinadjutoriumsolemn}
  \printdeusinadjutorium{}
  \let\deusinadjutoriumsolemn=\undefined
}

\printnote{Vespers then proceed with the Proper Antiphons, Psalms, Chapter, Hymn, Versicle,
Magnificat, and Collect given for the respective Sunday or Feast.

For Sundays throughout the year, \emph{p.~\pageref{sundayvespers}}.

For Sundays in Paschaltide, \emph{p.~\pageref{sundayvespers-easter}}.
}

\bigskip

\newcommand{\printvrwithtranslation}{
    {\normalsize
    \ifx\beginvrcols\undefined\def\beginvrcols{\begin{parcolumns}[rulebetween]{2}}\fi
    \beginvrcols
    \colchunk{
      \par\vspace{-\baselineskip}\noindent\selectlanguage{latin}%
      \Vbar{}~\vlatin{}
    }
    \colchunk{%
      \par\vspace{-\baselineskip}\noindent\selectlanguage{american}%
      \Vbar{}~\vtranslation{}
    }%
    \colplacechunks%
    \ifx\rlatin\undefined\else
    \colchunk{
      \par\vspace{-\baselineskip}\noindent\selectlanguage{latin}%
      \Rbar{}~\rlatin{}%
    }
    \colchunk{%
      \par\vspace{-\baselineskip}\noindent\selectlanguage{american}%
      \Rbar{}~\rtranslation{}%
    }%
    \colplacechunks%
    \fi
    \end{parcolumns}
    }
}

{
  \printnote{All Vespers conclude with the following:}\par
  \def\beginvrcols{\begin{parcolumns}[rulebetween,colwidths={1=0.48\linewidth}]{2}}
  \def\vlatin{Dóminus vobíscum.}
  \def\rlatin{Et cum spíritu tuo.}
  \def\vtranslation{The Lord be with you.}
  \def\rtranslation{And with thy spirit.}
  \printvrwithtranslation{}

  %\Vbar{}~Dóminus vobíscum.\\%\hspace{5em}
  %\Rbar{}~Et cum spíritu tuo.

  \ifthenelse{\boolean{birmingham}}{% TRUE is empty
  }{% False
    \printnote{Or, in the absence of a priest or deacon:}\par
    \def\vlatin{Dómine, exáudi oratiónem meam.}
    \def\rlatin{Et clamor meus ad te véniat.}
    \def\vtranslation{Lord, hear my prayer.}
    \def\rtranslation{And let my cry come unto Thee.}
    \printvrwithtranslation{}
  }

  %\Vbar{}~Dómine, exáudi oratiónem meam.\\
  %\Rbar{}~Et clamor meus ad te véniat.

  \bigskip

  \noindent\printnote{The ``Benedicámus Dómino'' is then sung\ifthenelse{\boolean{birmingham}}{}{ in one of the tones on the next pages}.}

  \bigskip

  \let\rlatin\undefined
  \let\rtranslation\undefined
  \def\vlatin{Fidélium ánimæ per misericórdiam Dei requiéscant in pace.\\\Rbar~Amen.}
  %\def\rlatin{Amen.}
  \def\vtranslation{May the souls of the faithful departed through the mercy of God rest in peace.\hfill\Rbar~Amen.}
  %\def\rtranslation{Amen.}
  \printvrwithtranslation{}
}
%\noindent\Vbar{}~Fidélium ánimæ per misericóridam Dei requiéscant in pace.\\
%\Rbar{}~Amen.

\def\noeuouae{T}
\def\dontrepeatantiphon{T}

\ifthenelse{\boolean{testrun}}{

}{

	%{
\newcommand{\printbenedicamusdomino}[2]{
	\greseteolcustos{manual}
	\def\annot{\small{#1}}
	\alsetinitialspacing{B}
	\gregorioscore{#2}
  \greseteolcustos{auto}
}

\pagebreak
\phantomsection\printnote{The ``Benedicámus Dómino'' is then sung in one of the different tones as follows:}
\bigskip
\label{benedicamusdomino}
\sectionmark{Benedicámus Dómino}
\addcontentsline{toc}{section}{Benedicámus Dómino}

%\vfil
\grechangestaffsize{15}
\label{benedicamusdomino-1}
{{\centering \bfseries 1.~On feasts of the I class.\\}
\smallskip
{\centering \normalsize At \nth{1} Vespers.\\}
\smallskip
\def\breakbeforeresp{T}
\printbenedicamusdomino{2.}{../BenedicamusDomino/BenedicamusDomino_1class1stvespers}
\bigskip\needspace{8\baselineskip}
%\vfil
{\centering \normalsize At \nth{2} Vespers.\\}
\smallskip
{\def\breakbeforeresp{T}
\printbenedicamusdomino{6.}{../BenedicamusDomino/BenedicamusDomino_1class2ndvespers}
}
\medskip
\emph{\normalsize or more commonly:}

\smallskip
\def\breakbeforeresp{T}
\printbenedicamusdomino{5.}{../BenedicamusDomino/BenedicamusDomino_1class2ndVespersAlt}}
%\vfil
{\bigskip
\def\dotting{\leaders\hbox to 1em{\hfil.\hfil}\hfill}
\emph{Marian anthems can be found on the following pages:}\\
Alma Redemptoris Mater \emph{(Advent through February 2)},\dotting p.~\pageref{almaredemptorismater}\\
Ave Regina Cælorum \emph{(February 2 through Lent)},\dotting p.~\pageref{avereginacaelorum}\\
Regina Cæli \emph{(Easter through the Friday after Pentecost)},\dotting p.~\pageref{reginacaeli}\\
Salve Regina \emph{(1st Vespers of Trinity Sunday until just before Advent)},\dotting p.~\pageref{salveregina}

}
\pagebreak

\label{benedicamusdomino-2}
{{\centering \bfseries 2.~On feasts of the II class.\\}
\smallskip
{\centering \normalsize At \nth{1} Vespers.\\}
{\def\breakbeforeresp{T}
\printbenedicamusdomino{2.}{../BenedicamusDomino/BenedicamusDomino_2class1stvespers}
}
\bigskip
{\centering \normalsize At \nth{2} Vespers.\\}
{%\def\breakbeforeresp{T}
\printbenedicamusdomino{8.}{../BenedicamusDomino/BenedicamusDomino_2class2ndvespers}}
}
\bigskip

%{{\centering \bfseries On feasts of the III class.\\}
%\printbenedicamusdomino{2.}{../BenedicamusDomino/BenedicamusDomino_3class}}
%\bigskip\vfill

\needspace{3\baselineskip}
\label{benedicamusdomino-mary}
\gdef\benedicamusdominonamemary{3}
{{\centering \bfseries 3.~On feasts of the Blessed Virgin.\\}
\smallskip
%\def\breakbeforeresp{T}
\printbenedicamusdomino{1.}{../BenedicamusDomino/BenedicamusDomino_blessedVirgin}}
\bigskip\bigskip

\needspace{6\baselineskip}
\label{benedicamusdomino-sunday}
\gdef\benedicamusdominonamesunday{4}
{{\centering \bfseries 4.~On Sundays during the Year\\and Septuagesima, Sexagesima, and Quinquagesima.\\}
\smallskip
%\def\breakbeforeresp{T}
\printbenedicamusdomino{1.}{../BenedicamusDomino/BenedicamusDomino_Sundays}}
\bigskip\bigskip

\label{benedicamusdomino-lent}\label{benedicamusdomino-advent}
\gdef\benedicamusdominonamelent{5}
\gdef\benedicamusdominonameadvent{\benedicamusdominonamelent}
{{\centering \bfseries 5.~On Sundays of Advent and Lent.\\}
\smallskip
%\def\breakbeforeresp{T}
\printbenedicamusdomino{6.}{../BenedicamusDomino/BenedicamusDomino_SundaysOfAdventAndLent}}
\bigskip\bigskip

\label{benedicamusdomino-easter}
\gdef\benedicamusdominonameeaster{6}
{{\centering \bfseries 6.~On Sundays of Paschal Time.\\}
\smallskip
%\def\breakbeforeresp{T}
\printbenedicamusdomino{7.}{../BenedicamusDomino/BenedicamusDomino_SundaysOfPaschalTime}}

}

	\clearpage

	\makeatletter  
\newcounter{score}
\newcounter{tabstop}[score]
\newcommand{\grealign}{%
  \@bsphack%
  \ifgre@boxing\else%
    \kern\gre@dimen@begindifference%
    \stepcounter{tabstop}%
    \expandafter\zsavepos{stop-\thescore-\thetabstop}%
    \kern-\gre@dimen@begindifference%
  \fi%
  \@esphack%
}

\newcommand{\setstops}{%
  \gdef\nstabbing@stops{%
    \checkoddpage
    \hspace*{-\ifoddpage\oddsidemargin\else\evensidemargin\fi}\hspace{-1in}%
    \hspace*{\zposx{stop-\thescore-1} sp}\=%
  }%
  \count@=\@ne
  \loop\ifnum\count@<\value{tabstop}%
    \begingroup\edef\x{\endgroup
      \noexpand\g@addto@macro\noexpand\nstabbing@stops{%
        \noexpand\hspace{-\noexpand\zposx{stop-\thescore-\the\count@} sp}%
        \noexpand\hspace{\noexpand\zposx{stop-\thescore-\the\numexpr\count@+1} sp}\noexpand\=%
      }%
    }\x
    \advance\count@\@ne
  \repeat
  \nstabbing@stops\kill
}
\makeatother

\newenvironment{nstabbing}
  {\setlength{\topsep}{0pt}%
   \setlength{\partopsep}{0pt}%
   \tabbing%
   \setstops}
  {\endtabbing\stepcounter{score}}

\def\beginpsalmcols{\begin{parcolumns}[rulebetween]{2}}
\def\endpsalmcols{\end{parcolumns}}
\newcommand{\printpsalmtitle}[1]{%
    {\addfontfeature{Numbers=Lining}\centering Psalm #1\par}
  }

\newcommand{\printpsalmwithtranslation}[3]{
	% #1 psalm number
	% #2 ending
	% #3 0 to start on 1st verse, 1 for second
	\setlength{\columnsep}{18pt}
	\setlength{\columnseprule}{.4pt}
	\beginpsalmcols
	\colchunk{\vspace{-12pt}%
	\begin{psalmverses}[#3]
	\vspace{-\baselineskip}%
	\input{psalms/Psalm#1-#2-verses}
	\end{psalmverses}
	}

	\selectlanguage{american}
	\colchunk{\vspace{-12pt}%
	\sloppy
	\begin{psalmverses}
	\vspace{-\baselineskip}%
	\input{psalms/Psalm#1-verses-en}
	\end{psalmverses}
	}
	\selectlanguage{latin}
	\endpsalmcols
}

\renewcommand{\printpsalm}[3]{
  % #1 psalm number
  % #2 ending
  % #3 0 to start on 1st verse, 1 for second
  \setlength{\columnsep}{18pt}
  \setlength{\columnseprule}{.4pt}
  \begin{multicols}{2}
  \begin{psalmverses}[#3]
  \input{psalms/Psalm#1-#2-verses}
  \end{psalmverses}
  \end{multicols}
}



\newcommand{\printpsalmtranslation}[3]{
  % #1 psalm number
  % #2 ending
  % #3 0 to start on 1st verse, 1 for second
  \setlength{\columnsep}{18pt}
  \setlength{\columnseprule}{.4pt}
  \begin{multicols}{2}
  \begin{psalmverses}[#3]
  \input{psalms/Psalm#1-verses-en}
  \end{psalmverses}
  \end{multicols}
}

\def\beginchaptercols{\begin{parcolumns}[rulebetween]{2}}
\newcommand{\printchapternew}[4]{
	% #1 ref
	% #2 latin first words
	% #3 latin remaining
	% #4 english text
	{\noindent\hspace{3em}Chapter.\hfill\emph{#1}\hspace{3em}}
    \beginchaptercols{}
    \colchunk{\sloppy\dropcap{latin}{#2} #3\hfill{}\Rbar{}~Deo~grátias.}
    \colchunk{\sloppy \dropcap{american}{#4}}
    \end{parcolumns}
}
\newcommand{\printvrwithtranslation}{
    {\normalsize
    \ifx\beginvrcols\undefined\def\beginvrcols{\begin{parcolumns}[rulebetween]{2}}\fi
    \beginvrcols
    \colchunk{
      \par\vspace{-\baselineskip}\noindent\selectlanguage{latin}%
      \Vbar{}~\vlatin{}
    }
    \colchunk{%
      \par\vspace{-\baselineskip}\noindent\selectlanguage{american}%
      \Vbar{}~\vtranslation{}
    }%
    \colplacechunks%
    \ifx\rlatin\undefined\else
    \colchunk{
      \par\vspace{-\baselineskip}\noindent\selectlanguage{latin}%
      \Rbar{}~\rlatin{}%
    }
    \colchunk{%
      \par\vspace{-\baselineskip}\noindent\selectlanguage{american}%
      \Rbar{}~\rtranslation{}%
    }%
    \colplacechunks%
    \fi
    \end{parcolumns}
    }
}
\newcommand{\printtercepsalms}[1]{
  % #1 is psalm tone like 2D
  \printpsalmtitle{118. III.}

  %print first verse in chant
  {
    \gresetinitiallines{0}
    \grechangedim{interwordspacetext}{0.15 cm plus 0.15 cm minus 0.10 cm}{scalable}%
    \gregorioscore{psalms/Psalm118.3-#1}
    % the additional width of the additional lines (compared to the width of the glyph they're associated with)
\grechangedim{additionallineswidth}{0.14584 cm}{scalable}%
% width of the additional lines, used only for the custos (maybe should depend on the width of the custos...)
% the width is the one for the custos at end of lines, the line for custos in the middle of a score is the same
% multiplied by 2.
\grechangedim{additionalcustoslineswidth}{0.09114 cm}{scalable}%
% null space
\grechangedim{zerowidthspace}{0 cm}{scalable}%
% space between glyphs in the same element
\grechangedim{interglyphspace}{0.06927 cm plus 0.00363 cm minus 0.00363 cm}{scalable}%
% space between an alteration (flat or natural) and the next glyph
\grechangedim{alterationspace}{0.07747 cm plus 0.01276 cm minus 0.00455 cm}{scalable}%
% space between a clef and a flat (for clefs with flat)
\grechangedim{clefflatspace}{0.05469 cm plus 0.00638 cm minus 0.00638 cm}{scalable}%
% space before a choral sign
\grechangedim{beforelowchoralsignspace}{0.04556 cm plus 0.00638 cm minus 0.00638 cm}{scalable}%
% when bolshifts are enabled, minimal space between a clef at the beginning of the line and a leading alteration glyph (should be larger than clefflatspace so that a flatted clef can be distinguished from a flat which is part of the first glyph on a line, but also smaller than spaceafterlineclef, the distance from the clef to the first notes)
\grechangedim{beforealterationspace}{0.1 cm}{scalable}%
% space between elements
\grechangedim{interelementspace}{0.06927 cm plus 0.00182 cm minus 0.00363 cm}{scalable}%
% larger space between elements
\grechangedim{largerspace}{0.10938 cm plus 0.01822 cm minus 0.00911 cm}{scalable}%
% space between elements in ancient notation
\grechangedim{nabcinterelementspace}{0.06927 cm plus 0.00182 cm minus 0.00363 cm}{scalable}%
% larger space between elements in ancient notation
\grechangedim{nabclargerspace}{0.10938 cm plus 0.01822 cm minus 0.00911 cm}{scalable}%
% space between elements which has the size of a note
\grechangedim{glyphspace}{0.21877 cm plus 0.01822 cm minus 0.01822 cm}{scalable}%
% space before custos
\grechangedim{spacebeforecustos}{0.1823 cm plus 0.31903 cm minus 0.0638 cm}{scalable}%
% space before punctum mora and augmentum duplex
\grechangedim{spacebeforesigns}{0.05469 cm plus 0.00455 cm minus 0.00455 cm}{scalable}%
% space after punctum mora and augmentum duplex
\grechangedim{spaceaftersigns}{0.08203 cm plus 0.0082 cm minus 0.0082 cm}{scalable}%
% space after a clef at the beginning of a line
\grechangedim{spaceafterlineclef}{0.27345 cm plus 0.14584 cm minus 0.01367 cm}{scalable}%
% minimal space between notes of different words
%\grechangedim{interwordspacenotes}{0.27 cm plus 0.15 cm minus 0.05 cm}{scalable}%
\grechangedim{interwordspacenotes}{0.27 cm plus 0.08 cm minus 0.05 cm}{scalable}%
% minimal space between notes of the same syllable.
% Warning: always keep minus to 0; also keep plus very low, or some words won't be hyphenated
%\grechangedim{intersyllablespacenotes}{0.24 cm plus 0.04cm minus 0cm}{scalable}%
\grechangedim{intersyllablespacenotes}{0.24 cm plus 0.04cm minus 0cm}{scalable}%
% minimal space between letters of different words. Makes sense to have
% the same plus and minus as interwordspacenotes.
%\grechangedim{interwordspacetext}{0.38 cm plus 0.15 cm minus 0.05 cm}{scalable}%
\grechangedim{interwordspacetext}{0.18 cm plus 0.08 cm minus 0.05 cm}{scalable}%
% Versions of interword spaces for euouae blocks
%\grechangedim{interwordspacenotes@euouae}{0.19 cm plus 0.1 cm minus 0.05 cm}{scalable}%
\grechangedim{interwordspacenotes@euouae}{0.13 cm plus 0.1 cm minus 0.05 cm}{1}%
%\grechangedim{interwordspacetext@euouae}{0.27 cm plus 0.1 cm minus 0.05 cm}{scalable}%
\grechangedim{interwordspacetext@euouae}{0.13 cm plus 0.1 cm minus 0.05 cm}{1}%
% space between notes of a bivirga or trivirga
\grechangedim{bitrivirspace}{0.06927 cm plus 0.00182 cm minus 0.00546 cm}{scalable}%
% space between notes of a bistropha or tristrophae
\grechangedim{bitristrospace}{0.06927 cm plus 0.00182 cm minus 0.00546 cm}{scalable}%
% space between two punctum inclinatum
\grechangedim{punctuminclinatumshift}{-0.03918 cm plus 0.0009 cm minus 0.0009 cm}{scalable}%
% space before puncta inclinata
\grechangedim{beforepunctainclinatashift}{0.05286 cm plus 0.00728 cm minus 0.00455 cm}{scalable}%
% space between a punctum inclinatum and a punctum inclinatum deminutus
\grechangedim{punctuminclinatumanddebilisshift}{-0.02278 cm plus 0.0009 cm minus 0.0009 cm}{scalable}%
% space between two punctum inclinatum deminutus
\grechangedim{punctuminclinatumdebilisshift}{-0.00728 cm plus 0.0009 cm minus 0.0009 cm}{scalable}%
% space between puncta inclinata, larger ambitus (range=3rd)
\grechangedim{punctuminclinatumbigshift}{0.07565 cm plus 0.0009 cm minus 0.0009 cm}{scalable}%
% space between puncta inclinata, larger ambitus (range=4th -or more?-)
\grechangedim{punctuminclinatummaxshift}{0.17865 cm plus 0.0009 cm minus 0.0009 cm}{scalable}%
% space for the bars (inside syllables)
%first for virgula and divisio minima
\grechangedim{spacearoundsmallbar}{0.1823 cm plus 0.22787 cm minus 0.00469 cm}{scalable}%
%then divisio minor
\grechangedim{spacearoundminor}{0.1823 cm plus 0.22787 cm minus 0.00469 cm}{scalable}%
%divisio major
\grechangedim{spacearoundmaior}{0.1823 cm plus 0.22787 cm minus 0.00469 cm}{scalable}%
%divisio finalis
\grechangedim{spacearoundfinalis}{0.1823 cm plus 0.22787 cm minus 0.00469 cm}{scalable}%
%a special space for finalis, for when it is the last glyph
\grechangedim{spacebeforefinalfinalis}{0.29169 cm plus 0.07292 cm minus 0.27345 cm}{scalable}%
% additional space that will appear around bars that are preceded by a custos and followed by a key.
\grechangedim{spacearoundclefbars}{0.03645 cm plus 0.00455 cm minus 0.0009 cm}{scalable}%
% space between the text and the text of the bar
\grechangedim{textbartextspace}{0.24611 cm plus 0.13672 cm minus 0.04921 cm}{scalable}%
% minimal space between a note and a bar
\grechangedim{notebarspace}{0.31903 cm plus 0.27345 cm minus 0.02824 cm}{scalable}%
% maximal space between two syllables for which we consider a dash is not needed
\grechangedim{maximumspacewithoutdash}{0.00 cm}{scalable}%
% an extensible space for the beginning of lines
\grechangedim{afterclefnospace}{0 cm plus 0.27345 cm minus 0 cm}{scalable}%
% space between the initial and the beginning of the score
\grechangedim{afterinitialshift}{0.2457 cm}{scalable}%
% space before the initial
\grechangedim{beforeinitialshift}{0.2457 cm}{scalable}%
% when bolshifts are enabled, minimum space between beginning of line and first syllable text
\grechangedim{minimalspaceatlinebeginning}{0.05 cm}{scalable}%
% space to force the initial width to.  Ignored when 0.
\grechangedim{manualinitialwidth}{0 cm}{scalable}%
% distance to move the initial up by
\grechangedim{initialraise}{0 cm}{scalable}%
% Space between lines in the annotation
\grechangedim{annotationseparation}{0.05cm}{scalable}%
% Amount to raise (positive) or lower (negative) the annotations from the default position (base line of top annotation aligned with top line of staff)
\grechangedim{annotationraise}{0cm}{scalable}%
% space at the beginning of the lines if there is no clef
\grechangedim{noclefspace}{0.1 cm}{scalable}%
% space around a clef change
\grechangedim{clefchangespace}{0.01768 cm plus 0.00175 cm minus 0.01768 cm}{scalable}%
%When \gre@clivisalignment is 2, this distance is the maximum length of the consonants after vowels for which the clivis will be aligned on its center.
\grechangedim{clivisalignmentmin}{0.3 cm}{scalable}%



%%%%%%%%%%%%%%%%%%
% vertical spaces
%%%%%%%%%%%%%%%%%%

% first, we have two spaces for the chironomic signs
\grechangedim{abovesignsspace}{0.8 cm}{scalable}%
\grechangedim{belowsignsspace}{0 cm}{scalable}%
% the amount to shift down:
% (a) low choral signs that are not lower than the note, regardless of whether
%     it's on a line or in a space
% (b) high choral signs and low choral signs that are lower than the note which
%     are in a space
\grechangedim{choralsigndownshift}{0.00911 cm}{scalable}%
% the amount to shift up:
% (a) high choral signs and low choral signs that are lower than the note which
%     are on a line
\grechangedim{choralsignupshift}{0.04556 cm}{scalable}%
% the space for the translation
\grechangedim{translationheight}{0.5 cm}{scalable}%
%the space above the lines
\grechangedim{spaceabovelines}{0.45576 cm plus 0.36461 cm minus 0.09114 cm}{scalable}%
%the space between the lines and the bottom of the text
\grechangedim{spacelinestext}{0.60617 cm}{scalable}%
%the space beneath the text
\grechangedim{spacebeneathtext}{0 cm}{scalable}%
% height of the text above the note line
\grechangedim{abovelinestextraise}{-0.1 cm}{scalable}%
% height that is added at the top of the lines if there is text above the lines (it must be bigger than the text for it to be taken into consideration)
\grechangedim{abovelinestextheight}{0.3 cm}{scalable}%
% an additional shift you can give to the brace above the bars if you don't like it
\grechangedim{braceshift}{0 cm}{scalable}%
% a shift you can give to the accentus above the curly brace
\grechangedim{curlybraceaccentusshift}{-0.05 cm}{scalable}%


    \greseteolcustos{auto}
  }

  \setlength{\columnsep}{18pt}
  \setlength{\columnseprule}{.4pt}
  \selectlanguage{latin}
  \begin{multicols}{2}
  \begin{psalmverses}[1]
  \input{psalms/Psalm118.3-#1-verses}
  \end{psalmverses}
  \printpsalmtitle{118. IV.}
  \vspace{-0.8\baselineskip}
  \begin{psalmverses}
  \input{psalms/Psalm118.4-#1-verses}
  \end{psalmverses}
  \printpsalmtitle{118. V.}
  \vspace{-0.8\baselineskip}
  \begin{psalmverses}
  \input{psalms/Psalm118.5-#1-verses}
  \end{psalmverses}
  \end{multicols}
}
\newcommand{\printtercepsalmstranslation}{
  \setlength{\columnsep}{18pt}
  \setlength{\columnseprule}{.4pt}
  \selectlanguage{american}
  \begin{multicols}{2}
  \printpsalmtitle{118. III.}
  \vspace{-0.5\baselineskip}
  \begin{psalmverses}
  \item Set before me for a law the way of thy justifications, O Lord: * and I will always seek after it.
\item Give me understanding, and I will search thy law; * and I will keep it with my whole heart.
\item Lead me into the path of thy commandments; * for this same I have desired.
\item Incline my heart into thy testimonies * and not to covetousness.
\item Turn away my eyes that they may not behold vanity: * quicken me in thy way.
\item Establish thy word to thy servant, * in thy fear.
\item Turn away my reproach, which I have apprehended: * for thy judgments are delightful.
\item Behold I have longed after thy precepts: * quicken me in thy justice.
\item Let thy mercy also come upon me, O Lord: * thy salvation according to thy word.
\item So shall I answer them that reproach me in any thing; * that I have trusted in thy words.
\item And take not thou the word of truth utterly out of my mouth: * for in thy words, I have hoped exceedingly.
\item So shall I always keep thy law, * for ever and ever.
\item And I walked at large: * because I have sought after thy commandments.
\item And I spoke of thy testimonies before kings: * and I was not ashamed.
\item I meditated also on thy commandments, * which I loved.
\item And I lifted up my hands to thy commandments, which I loved: * and I was exercised in thy justifications.
  \end{psalmverses}
  \printpsalmtitle{118. IV.}
  \vspace{-0.5\baselineskip}
  \begin{psalmverses}
  \item Be thou mindful of thy word to thy servant, * in which thou hast given me hope.
\item This hath comforted me in my humiliation: * because thy word hath enlivened me.
\item The proud did iniquitously altogether: * but I declined not from thy law.
\item I remembered, O Lord, thy judgments of old: * and I was comforted.
\item A fainting hath taken hold of me, * because of the wicked that forsake thy law.
\item Thy justifications were the subject of my song, * in the place of my pilgrimage.
\item In the night I have remembered thy name, O Lord: * and have kept thy law.
\item This happened to me: * because I sought after thy justifications.
\item O Lord, my portion, * I have said, I would keep thy law.
\item I entreated thy face with all my heart: * have mercy on me according to thy word.
\item I have thought on my ways: * and turned my feet unto thy testimonies.
\item I am ready, and am not troubled: * that I may keep thy commandments.
\item The cords of the wicked have encompassed me: * but I have not forgotten thy law.
\item I rose at midnight to give praise to thee; * for the judgments of thy justification.
\item I am a partaker with all them that fear thee, * and that keep thy commandments.
\item The earth, O Lord, is full of thy mercy: * teach me thy justifications.
  \end{psalmverses}
  \printpsalmtitle{118. V.}
  \vspace{-0.5\baselineskip}
  \begin{psalmverses}
  \item Thou hast done well with thy servant, O Lord, * according to thy word.
\item Teach me goodness and discipline and knowledge; * for I have believed thy commandments.
\item Before I was humbled I offended; * therefore have I kept thy word.
\item Thou art good; * and in thy goodness teach me thy justifications.
\item The iniquity of the proud hath been multiplied over me: * but I will seek thy commandments with my whole heart.
\item Their heart is curdled like milk: * but I have meditated on thy law.
\item It is good for me that thou hast humbled me, * that I may learn thy justifications.
\item The law of thy mouth is good to me, * above thousands of gold and silver.
\item Thy hands have made me and formed me: * give me understanding, and I will learn thy commandments.
\item They that fear thee shall see me, and shall be glad: * because I have greatly hoped in thy words.
\item I know, O Lord, that thy judgments are equity: * and in thy truth thou hast humbled me.
\item O! let thy mercy be for my comfort, * according to thy word unto thy servant.
\item Let thy tender mercies come unto me, and I shall live: * for thy law is my meditation.
\item Let the proud be ashamed, because they have done unjustly towards me: * but I will be employed in thy commandments.
\item Let them that fear thee turn to me: * and they that know thy testimonies.
\item Let my heart be undefiled in thy justifications, * that I may not be confounded.
  \end{psalmverses}
  \end{multicols}
}

{
  \label{advent}
  \chapter{Sunday at Terce}
  \section{Sunday at Terce throughout the year}

  \printgabc{\Vbar}{}{D}{deus-in-adjutorium}

  \bigskip

  {\centering Hymn.\par}

  {\centering 1.~On Ordinary Sundays.\par}

  \printgabc{2.}{}{N}{hy--nunc_sancte_nobis_(in_dominicis_per_annum)--solesmes_1961}

  \oldneedspace{10\baselineskip}
  {\centering 2.~On Solemn Feasts.\par}

  \printgabc{8.}{}{N}{hy--nunc_sancte_nobis_(in_festis)--solesmes_1961}

  \begin{multicols}{2}
  \begin{psalmverses}
\item Come Holy Ghost, who ever One
Art with the Father and the Son,
It is the hour, our souls possess
With thy full flood of holiness.

\item Let flesh and heart and lips and mind
Sound forth our witness to mankind;
And love light up our mortal frame,
Till others catch the living flame.

\item Almighty Father, hear our cry,
Through Jesus Christ, our Lord most High,
Who, with the Holy Ghost and thee,
Doth live and reign eternally.
Amen.
  \end{psalmverses}
  \end{multicols}

\bigskip



% print antiphon
%    \printgabc[\preant]{\antlineone}{\antlinetwo}{\antinitial}{\anttex}
  \bigskip
  \printgabc{Ant.}{2. D}{A}{an--alleluia._(sund._at_terce)--solesmes}
% print translation
  \translation[]{Alleluia, Lead me into the path of thy commandments, alleluia, alleluia.}

\bigskip
  \printtercepsalms{2D}

  \bigskip
  \printtercepsalmstranslation{}

% \printpsalm{118.3}{2D}{1}
% \printpsalmtitle{118. IV.}
% \printpsalm{118.4}{2D}{0}
% \printpsalmtitle{118. V.}
% \printpsalm{118.5}{2D}{0}

% print antiphon
%    \printgabc[\preant]{\antlineone}{\antlinetwo}{\antinitial}{\anttex}
  \printgabc{Ant.}{2. D}{A}{an--alleluia._(sund._at_terce)--solesmes}
% print translation
  \translation[]{Alleluia, Lead me into the path of thy commandments, alleluia, alleluia.}

\bigskip
\printchapternew{1. John 4.}{Deus cáritas}{est~:~\dag{} et qui manet in caritáte, in Deo manet,~* et Deus in eo.}{God is charity: and he that abideth in charity, abideth in God, and God in him.}

\bigskip
\printgabc{Short}{Resp.}{I}{re--inclina_cor_meum--solesmes}

\translation[]{\Vbar{}~Incline my heart into thy testimonies.
\Rbar{}~Incline\dots{}
\Vbar{}~Turn away my eyes that they may not behold vanity: quicken me in thy way.
\Rbar{}~Into thy testimonies.
\Vbar{}~Glory be to the Father, and to the Son, and to the Holy Ghost.
\Vbar{}~Incline\dots{}}

\bigskip
\gresetinitiallines{0}
\gregorioscore{vr-ego-dixi}
\newlength{\myhwidth}
\settowidth{\myhwidth}{tib}
\begin{nstabbing}
\>\Rbar{}~Sana ánimam meam, quia peccávi \>\hspace{-\myhwidth}tibi.\\
\end{nstabbing}

\translation[]{\Vbar{}~I said: O Lord, be thou merciful to me.\\
\Rbar{}~Heal my soul, for I have sinned against thee.}

\bigskip
\def\beginvrcols{\begin{parcolumns}[rulebetween,colwidths={1=0.48\linewidth}]{2}}
\def\vlatin{Dóminus vobíscum.}
\def\rlatin{Et cum spíritu tuo.}
\def\vtranslation{The Lord be with you.}
\def\rtranslation{And with thy spirit.}
{\centering Collect.\par}
\printvrwithtranslation{}

\bigskip
\emph{See proper of time.}

\bigskip
\printvrwithtranslation{}
\bigskip

\gresetinitiallines{0}
\gregorioscore{vr-benedicamus-domino}
\bigskip

\def\beginvrcols{\begin{parcolumns}[rulebetween,colwidths={1=0.44\linewidth}]{2}}
\def\vlatin{Fidélium ánimæ per misericórdiam Dei requiéscant in pace.\hfill{}Amen.}
\let\rlatin=\undefined
\def\vtranslation{May the souls of the faithful departed through the mercy of God rest in peace.\hfill{}Amen.}
\let\rtranslation=\undefined
\printvrwithtranslation{} 
}

	\chapter{Proper of the Time -- Advent Season}
{
\newcommand{\benedicamusdomino}[1][advent]{
	\benedicamusdominomaster{#1}
}
\newcommand{\printhymnnote}{
	\noindent\printnote{Hymn.~\emph{Creátor alme síderum}, page \pageref{hymn-creatoralmesiderum}.
	\Vbar~\emph{Roráte}, page \pageref{vr-rorate}.}
}
\newcommand{\printoant}[2]{
%	#1 number
%	#2 translation
{
\oldneedspace{5\baselineskip}
\subtitle{December #1.}
\printgabc{\small At Magn.}{\small \oldstylenums{Ant.~2.~D}}{O}{December#1-MagAntiphon}
\translation[]{#2}
\medskip
\emph{\emph{Magnificat}.~page \pageref{oantiphon-magnificat}.}
}
}

\normalsize
%\def\nogloriapatri{T}
%\def\breakbeforeEuouae{T}
%1st sunday of advent
{
\section{First Sunday of Advent}
\subtitle{\nth{1} Class, Violet}

\def\premag{\def\noeuouae{T}}
\def\premagverses{\greseteolcustos{manual}}
\def\printfullhymn{
	\label{hymn-creatoralmesiderum}
	{
	\def\gabcfolder{../Advent}
	\def\prehymntranslation{\vspace{-\baselineskip}}
	\printhymn{\oldstylenums{4.}}{C}{hymn-CreatorAlmeSiderum}
	{\item Creator of the stars of night
	Thy people's everlasting light,
	Jesu, Redeemer, save us all,
	And hear Thy servants when they call.

	\item Thou, lest the demon's ancient curse
	Should doom to death a universe,
	In love wast made, Thyself alone,
	The means to save a world undone.

	\item Towards the Cross Thou wentest forth,
	That Thou might'st heal the crimes of earth;
	Proceeding from a virgin shrine,
	The spotless Victim all divine.

	\item At whose dread Name, majestic now,
	All knees must bend, all hearts must bow;
	And things celestial Thee shall own,
	And things terrestrial, Lord alone.

	\item O Thou, whose coming is with dread,
	To judge and doom the quick and dead.
	Thy heavenly grace on us bestow,
	To shield us from our ghostly foe.

	\item To God the Father, God the Son,
	And God the Spirit, Three in One,
	Laud, honour, might, and glory be
	From age to age eternally.
	Amen.}

	{
		\def\vrlinebreak{T}
		\label{vr-rorate}
		\printvr[\greseteolcustos{manual}]{vr}
		{Ye heavens, drop down dew from above, and let the clouds rain down the Just One.}
		{Let the earth open and bud forth the Saviour.}
	}
	}
}
\def\prepsalmfive{\greseteolcustos{manual}}
\def\prevespers{%
	\let\oldthing\antfourtranslation
	\def\antfourtranslation{\oldthing\vspace{-\baselineskip}}%
	\let\oldthingb\antfivetranslation
	\def\antfivetranslation{\oldthingb\vspace{-\baselineskip}}%
}
\printvespers[../Advent1]{inc-Advent1}
\bigskip
\benedicamusdomino{}
}



% Advent 2
{
\section{Second Sunday of Advent}
\subtitle{\nth{1} Class, Violet}
\printnote{If today is December 8, the Second Vespers of the Immaculate Conception is sung with a Commemoration of this Sunday, page \pageref{immaculateconception}.}
\medskip
\def\preanttwo{\needspace{5\baselineskip}}
\def\prepsalmthree{\needspace{5\baselineskip}}
\def\prepsalmfive{\greseteolcustos{manual}}
\def\premag{\def\noeuouae{T}}
\def\premagverses{\greseteolcustos{manual}}
%\def\prechapter{\vspace{-\baselineskip}}
\printvespers[../Advent2]{inc-Advent2-Vespers2}

\noindent
\printnote{If today is December 7, the First Vespers of the Immaculate Conception is commemorated as follows.  Otherwise \benedicamusdominoreference{advent}}
\bigskip
\hrule
}



% commemoration of First Vespers of Immaculate Conception
{
\def\beginvrcols{\begin{parcolumns}[rulebetween,colwidths={1=0.45\linewidth}]{2}}
\printcommemoration[../December8-ImmaculateConception]{commemorationImmaculateConception-Vespers1}

\bigskip
\benedicamusdomino{}
}


% 3rd Sunday of Advent
{
\section{Third Sunday of Advent}
\subtitle{\nth{1} Class, Violet or Rose}

\def\prepsalmfive{\greseteolcustos{manual}}
\def\preantfour{\needspace{15\baselineskip}}
\def\premag{\def\noeuouae{T}
\printnote{When the third Sunday of Advent falls on December 17, the Antiphon \emph{O Sapiéntia} on page \pageref{osapientia} is sung in place of \emph{Beáta es}.}
\medskip

\label{magant-beataes}
}
\def\premagverses{\greseteolcustos{manual}}
\def\postmag{
	\def\nomagtitle{T}
	\def\magsolemn{T}
	%\definemag{2}{D}
	\def\gabcfolder{../Advent}
	\def\noeuouae{T}
		%print o sapientia
	%\printoant{17}{O Wisdom, which camest out of the mouth of the Most High, reaching from end to end and ordering all things mightily and sweetly: come and teach us the way of prudence.}
	\renewcommand{\anttranslation}{O Wisdom, which camest out of the mouth of the Most High, reaching from end to end and ordering all things mightily and sweetly: come and teach us the way of prudence.}
	\oldneedspace{5\baselineskip}
	\subtitle{December 17.}
	\label{osapientia}
	\def\preverses{\greseteolcustos{manual}}
	\printmag{2D}{O}{December17-MagAntiphon}
}
\def\prechapter{\vspace{-\baselineskip}}
\def\beginchaptercols{\begin{parcolumns}[rulebetween,colwidths={1=0.48\linewidth}]{2}}
\printvespers[../Advent3]{inc-Advent3}
\bigskip
\benedicamusdomino{}
}



%Advent 4
{
\section{Fourth Sunday of Advent}
\subtitle{\nth{1} Class, Violet}
\printnote{If today is December 24, First Vespers of Christmas is used on page \pageref{christmas}.}
\medskip

\def\prepsalmtwoverses{\oldneedspace{2\baselineskip}}
\def\prepsalmfive{\greseteolcustos{manual}}
\def\magreplacement{
	%\pagebreak
	\bigskip
	{
	\def\magsolemn{T}
	\definemag{2}{D}
	\def\gabcfolder{../Advent}
	\def\noeuouae{T}
	\printoant{18}{O Adonai, and Leader of the house of Israël, who didst appear to Moses in the flame of the burning bush and didst give unto him the law on Sinai: come and with an outstretched arm redeem us.}
	\printoant{19}{O Root of Jesse, which standest for an ensign of the people, before whom kings shall keep silence, whom the Gentiles shall beseech: come and deliver us, and tarry not.}
	\printoant{20}{O Key of David, and Sceptre of the house of Israël, that openest and no man shutteth, and shuttest and no man openeth: come and bring the prisoner forth from the prison-house, and him that sitteth in darkness and in the shadow of death.}
	\printoant{21}{O Day-spring, Brightness of light eternal, and Sun of Justice, come and enlighten them that sit in darkness and in the shadow of death.}
	\printoant{22}{O King of the Gentiles and the desire thereof, Thou cornerstone that makest both one, come and deliver mankind, whom Thou didst form out of clay.}
	\printoant{23}{O Emmanuel, our King and Lawgiver, the desire of the nations and the Saviour thereof, come to save us, O Lord our God.}

	\bigskip
	\pagebreak
	\vfil
	{\greseteolcustos{manual}
	\label{oantiphon-magnificat}
	\begin{magnificat}{\magtex}
	\magverses
	\end{magnificat}
	\noindent\emph{Repeat antiphon.}}
	\vfil
	}
}
\printvespers[../Advent4]{inc-Advent4}
\bigskip
\benedicamusdomino{}
}
}

}
\end{document}

