% !TEX TS-program = lualatex
% !TEX encoding = UTF-8

% This is a simple template for a LuaLaTeX document using gregorio scores.

% easter can be from march 22 to april 25

\usepackage{../definepsalms}
\usepackage{titlesec}
\usepackage{titletoc}
\usepackage{titleps}
\usepackage{letltxmacro}
\usepackage{changepage} % gives us \ifoddpage use [strict]
\usepackage[super]{nth}
\usepackage[savepos]{zref}
\usepackage{xparse}
\usepackage{setspace}
\usepackage{amsfonts}
\usepackage{refcount}
\usepackage{metalogo}
\usepackage{enumitem}
\usepackage{zref-titleref}
\makeatletter
\newcommand*{\currentname}{\zref@getcurrent{title}}
% or \newcommand*{\currentname}{\zref@titleref@current}
\makeatother
%\nofiles
%\includeonly{inc-grassi}
\LetLtxMacro{\oldnth}{\nth}
\renewcommand{\nth}[1]{{\addfontfeature{Numbers=Lining}\oldnth{#1}}}
\LetLtxMacro{\oldneedspace}{\needspace}
\renewcommand{\needspace}[1]{
	\checkoddpage\ifoddpage\oldneedspace{#1}\else\fi
}
\let\gredagger=\dag
\newcommand*\cleartoleftpage{%
  \clearpage
  \ifodd\value{page}\hbox{}\newpage\fi
}
\hyphenation{GregoBase}

%\usepackage{hyperref}
\ifx\phantomsection\undefined%
	\newcommand{\phantomsection}{}
\fi

\setcounter{secnumdepth}{-1}

\ifthenelse{\boolean{lettersize}}{
	\def\mywidth{8.5in}
	\def\myheight{11in}
}{
	\def\mywidth{6in}
	\def\myheight{9in}
}

% !TEX TS-program = lualatex
% !TEX encoding = UTF-8
% usual packages loading:
%\usepackage{luatextra}
%\usepackage{graphicx} % support the \includegraphics command and options
\usepackage{geometry} % See geometry.pdf to learn the layout options. There are lots.
\ifx\undefined\mywidth
    \geometry{letterpaper} % or letterpaper (US) or a5paper or....
\else
    \geometry{papersize={\mywidth,\myheight}}
\fi
\usepackage{expl3}
\let\luatexlocalrightbox\localrightbox
\let\luatexlocalleftbox\localleftbox
\usepackage{gregoriotex} % for gregorio score inclusion
\usepackage{import}

% If you use usual TeX fonts, here is a starting point:
%\usepackage{palatino}
%\input{glyphtounicode} \pdfglyphtounicode{f_f}{FB00} \pdfglyphtounicode{f_f_i}{FB03} \pdfglyphtounicode{f_f_l}{FB04}
%\pdfglyphtounicode{Q_u}{E048} \pdfglyphtounicode{O_e}{0152} \pdfglyphtounicode{o_e}{0153}
%\pdfgentounicode=1
% to change the font to something better, you can install the kpfonts package (if not already installed). To do so
% go open the "TeX Live Manager" in the Menu Start->All Programs->TeX Live 2010
% the additional width of the additional lines (compared to the width of the glyph they're associated with)
\grechangedim{additionallineswidth}{0.14584 cm}{scalable}%
% width of the additional lines, used only for the custos (maybe should depend on the width of the custos...)
% the width is the one for the custos at end of lines, the line for custos in the middle of a score is the same
% multiplied by 2.
\grechangedim{additionalcustoslineswidth}{0.09114 cm}{scalable}%
% null space
\grechangedim{zerowidthspace}{0 cm}{scalable}%
% space between glyphs in the same element
\grechangedim{interglyphspace}{0.06927 cm plus 0.00363 cm minus 0.00363 cm}{scalable}%
% space between an alteration (flat or natural) and the next glyph
\grechangedim{alterationspace}{0.07747 cm plus 0.01276 cm minus 0.00455 cm}{scalable}%
% space between a clef and a flat (for clefs with flat)
\grechangedim{clefflatspace}{0.05469 cm plus 0.00638 cm minus 0.00638 cm}{scalable}%
% space before a choral sign
\grechangedim{beforelowchoralsignspace}{0.04556 cm plus 0.00638 cm minus 0.00638 cm}{scalable}%
% when bolshifts are enabled, minimal space between a clef at the beginning of the line and a leading alteration glyph (should be larger than clefflatspace so that a flatted clef can be distinguished from a flat which is part of the first glyph on a line, but also smaller than spaceafterlineclef, the distance from the clef to the first notes)
\grechangedim{beforealterationspace}{0.1 cm}{scalable}%
% space between elements
\grechangedim{interelementspace}{0.06927 cm plus 0.00182 cm minus 0.00363 cm}{scalable}%
% larger space between elements
\grechangedim{largerspace}{0.10938 cm plus 0.01822 cm minus 0.00911 cm}{scalable}%
% space between elements in ancient notation
\grechangedim{nabcinterelementspace}{0.06927 cm plus 0.00182 cm minus 0.00363 cm}{scalable}%
% larger space between elements in ancient notation
\grechangedim{nabclargerspace}{0.10938 cm plus 0.01822 cm minus 0.00911 cm}{scalable}%
% space between elements which has the size of a note
\grechangedim{glyphspace}{0.21877 cm plus 0.01822 cm minus 0.01822 cm}{scalable}%
% space before custos
\grechangedim{spacebeforecustos}{0.1823 cm plus 0.31903 cm minus 0.0638 cm}{scalable}%
% space before punctum mora and augmentum duplex
\grechangedim{spacebeforesigns}{0.05469 cm plus 0.00455 cm minus 0.00455 cm}{scalable}%
% space after punctum mora and augmentum duplex
\grechangedim{spaceaftersigns}{0.08203 cm plus 0.0082 cm minus 0.0082 cm}{scalable}%
% space after a clef at the beginning of a line
\grechangedim{spaceafterlineclef}{0.27345 cm plus 0.14584 cm minus 0.01367 cm}{scalable}%
% minimal space between notes of different words
%\grechangedim{interwordspacenotes}{0.27 cm plus 0.15 cm minus 0.05 cm}{scalable}%
\grechangedim{interwordspacenotes}{0.27 cm plus 0.08 cm minus 0.05 cm}{scalable}%
% minimal space between notes of the same syllable.
% Warning: always keep minus to 0; also keep plus very low, or some words won't be hyphenated
%\grechangedim{intersyllablespacenotes}{0.24 cm plus 0.04cm minus 0cm}{scalable}%
\grechangedim{intersyllablespacenotes}{0.24 cm plus 0.04cm minus 0cm}{scalable}%
% minimal space between letters of different words. Makes sense to have
% the same plus and minus as interwordspacenotes.
%\grechangedim{interwordspacetext}{0.38 cm plus 0.15 cm minus 0.05 cm}{scalable}%
\grechangedim{interwordspacetext}{0.18 cm plus 0.08 cm minus 0.05 cm}{scalable}%
% Versions of interword spaces for euouae blocks
%\grechangedim{interwordspacenotes@euouae}{0.19 cm plus 0.1 cm minus 0.05 cm}{scalable}%
\grechangedim{interwordspacenotes@euouae}{0.13 cm plus 0.1 cm minus 0.05 cm}{1}%
%\grechangedim{interwordspacetext@euouae}{0.27 cm plus 0.1 cm minus 0.05 cm}{scalable}%
\grechangedim{interwordspacetext@euouae}{0.13 cm plus 0.1 cm minus 0.05 cm}{1}%
% space between notes of a bivirga or trivirga
\grechangedim{bitrivirspace}{0.06927 cm plus 0.00182 cm minus 0.00546 cm}{scalable}%
% space between notes of a bistropha or tristrophae
\grechangedim{bitristrospace}{0.06927 cm plus 0.00182 cm minus 0.00546 cm}{scalable}%
% space between two punctum inclinatum
\grechangedim{punctuminclinatumshift}{-0.03918 cm plus 0.0009 cm minus 0.0009 cm}{scalable}%
% space before puncta inclinata
\grechangedim{beforepunctainclinatashift}{0.05286 cm plus 0.00728 cm minus 0.00455 cm}{scalable}%
% space between a punctum inclinatum and a punctum inclinatum deminutus
\grechangedim{punctuminclinatumanddebilisshift}{-0.02278 cm plus 0.0009 cm minus 0.0009 cm}{scalable}%
% space between two punctum inclinatum deminutus
\grechangedim{punctuminclinatumdebilisshift}{-0.00728 cm plus 0.0009 cm minus 0.0009 cm}{scalable}%
% space between puncta inclinata, larger ambitus (range=3rd)
\grechangedim{punctuminclinatumbigshift}{0.07565 cm plus 0.0009 cm minus 0.0009 cm}{scalable}%
% space between puncta inclinata, larger ambitus (range=4th -or more?-)
\grechangedim{punctuminclinatummaxshift}{0.17865 cm plus 0.0009 cm minus 0.0009 cm}{scalable}%
% space for the bars (inside syllables)
%first for virgula and divisio minima
\grechangedim{spacearoundsmallbar}{0.1823 cm plus 0.22787 cm minus 0.00469 cm}{scalable}%
%then divisio minor
\grechangedim{spacearoundminor}{0.1823 cm plus 0.22787 cm minus 0.00469 cm}{scalable}%
%divisio major
\grechangedim{spacearoundmaior}{0.1823 cm plus 0.22787 cm minus 0.00469 cm}{scalable}%
%divisio finalis
\grechangedim{spacearoundfinalis}{0.1823 cm plus 0.22787 cm minus 0.00469 cm}{scalable}%
%a special space for finalis, for when it is the last glyph
\grechangedim{spacebeforefinalfinalis}{0.29169 cm plus 0.07292 cm minus 0.27345 cm}{scalable}%
% additional space that will appear around bars that are preceded by a custos and followed by a key.
\grechangedim{spacearoundclefbars}{0.03645 cm plus 0.00455 cm minus 0.0009 cm}{scalable}%
% space between the text and the text of the bar
\grechangedim{textbartextspace}{0.24611 cm plus 0.13672 cm minus 0.04921 cm}{scalable}%
% minimal space between a note and a bar
\grechangedim{notebarspace}{0.31903 cm plus 0.27345 cm minus 0.02824 cm}{scalable}%
% maximal space between two syllables for which we consider a dash is not needed
\grechangedim{maximumspacewithoutdash}{0.00 cm}{scalable}%
% an extensible space for the beginning of lines
\grechangedim{afterclefnospace}{0 cm plus 0.27345 cm minus 0 cm}{scalable}%
% space between the initial and the beginning of the score
\grechangedim{afterinitialshift}{0.2457 cm}{scalable}%
% space before the initial
\grechangedim{beforeinitialshift}{0.2457 cm}{scalable}%
% when bolshifts are enabled, minimum space between beginning of line and first syllable text
\grechangedim{minimalspaceatlinebeginning}{0.05 cm}{scalable}%
% space to force the initial width to.  Ignored when 0.
\grechangedim{manualinitialwidth}{0 cm}{scalable}%
% distance to move the initial up by
\grechangedim{initialraise}{0 cm}{scalable}%
% Space between lines in the annotation
\grechangedim{annotationseparation}{0.05cm}{scalable}%
% Amount to raise (positive) or lower (negative) the annotations from the default position (base line of top annotation aligned with top line of staff)
\grechangedim{annotationraise}{0cm}{scalable}%
% space at the beginning of the lines if there is no clef
\grechangedim{noclefspace}{0.1 cm}{scalable}%
% space around a clef change
\grechangedim{clefchangespace}{0.01768 cm plus 0.00175 cm minus 0.01768 cm}{scalable}%
%When \gre@clivisalignment is 2, this distance is the maximum length of the consonants after vowels for which the clivis will be aligned on its center.
\grechangedim{clivisalignmentmin}{0.3 cm}{scalable}%



%%%%%%%%%%%%%%%%%%
% vertical spaces
%%%%%%%%%%%%%%%%%%

% first, we have two spaces for the chironomic signs
\grechangedim{abovesignsspace}{0.8 cm}{scalable}%
\grechangedim{belowsignsspace}{0 cm}{scalable}%
% the amount to shift down:
% (a) low choral signs that are not lower than the note, regardless of whether
%     it's on a line or in a space
% (b) high choral signs and low choral signs that are lower than the note which
%     are in a space
\grechangedim{choralsigndownshift}{0.00911 cm}{scalable}%
% the amount to shift up:
% (a) high choral signs and low choral signs that are lower than the note which
%     are on a line
\grechangedim{choralsignupshift}{0.04556 cm}{scalable}%
% the space for the translation
\grechangedim{translationheight}{0.5 cm}{scalable}%
%the space above the lines
\grechangedim{spaceabovelines}{0.45576 cm plus 0.36461 cm minus 0.09114 cm}{scalable}%
%the space between the lines and the bottom of the text
\grechangedim{spacelinestext}{0.60617 cm}{scalable}%
%the space beneath the text
\grechangedim{spacebeneathtext}{0 cm}{scalable}%
% height of the text above the note line
\grechangedim{abovelinestextraise}{-0.1 cm}{scalable}%
% height that is added at the top of the lines if there is text above the lines (it must be bigger than the text for it to be taken into consideration)
\grechangedim{abovelinestextheight}{0.3 cm}{scalable}%
% an additional shift you can give to the brace above the bars if you don't like it
\grechangedim{braceshift}{0 cm}{scalable}%
% a shift you can give to the accentus above the curly brace
\grechangedim{curlybraceaccentusshift}{-0.05 cm}{scalable}%


%\def\greinitialformat#1{{\fontsize{37}{37}\selectfont #1}}
%small > footnotesize > scriptsize > tiny


% my stuff
\usepackage[garamond]{../mypackage}
% end my stuff

\setgrefactor{17}

%\marginsize{25pt}{25pt}{25pt}{30pt}
\usepackage{calc}
%\setlength\headsep{20pt}
%\setlength\footskip{15pt}
\setlength\headheight{15pt}
\setlength\headsep{22pt}
\ifx\undefined\tenebrae
    \geometry{outer=25pt,inner=25pt,top=22pt+\headsep+\headheight,bottom=25pt+\footskip}
\else
    \ifx\undefined\mywidth
        %had been .3 outer, .4 inner
        %let's try .75 for inner and .5 for outer
        %now let's go to .35 outer, .9 inner
        %this time let's try .4 and .85
        \ifbook{\geometry{outer=0.4in,inner=0.85in,top=25pt+\headsep+\headheight,bottom=25pt+\footskip,twoside=true}}
        \ifnotbook{\geometry{outer=0.625in,inner=0.625in,top=25pt+\headsep+\headheight,bottom=25pt+\footskip,twoside=true}}
    \else
        \setlength\headsep{0.25in}
        \setlength\footskip{0.3in}
        \geometry{outer=0.5in,inner=0.5in,top=0.25in+\headsep+\headheight,bottom=0.25in+\footskip,twoside=true}
    \fi
\fi

\pagestyle{fancy} % no header or footers
\let\oldheadrulewidth\headrulewidth
\renewcommand\headrulewidth{\ifnum\thepage=1
0pt
\else
\oldheadrulewidth
\fi}

\ifx\undefined\ifbook
    \newcommand{\ifbook}[1]{}
    \newcommand{\ifnotbook}[1]{#1}
\fi
\ifx\undefined\ifsmallbook
    \newcommand{\ifsmallbook}[1]{}
    \newcommand{\ifnotsmallbook}[1]{#1}
\fi
%\cfoot{\thepage}

\gresetbarspacing{new}
%\gresetlastline{justified}
\setlength\headheight{0.25in+15pt}
\setlength\headsep{1pc}
\setlength\topskip{0pc}
\setlength\footskip{1pc}
\geometry{outer=0.4in,inner=0.85in,top=0pc+\headheight+\headsep,bottom=0.4in,twoside=true}
\newpagestyle{main}{
%\setheadrule{0pt}
\sethead[\garamond{\thepage}][\garamond{\chaptertitle}][] % even
{}{\garamond{\sectiontitle}}{\garamond{\thepage}} % odd
\setfoot[][][] % even
{}{}{} % odd
}
\pagestyle{main}

%\grechangeglyph{Porrectus*}{*}{.alt}
%\grechangeglyph{TorculusResupinus*}{*}{.alt}

\titleformat
{\section} % command
[block] % shape
{\phantomsection\large\addfontfeature{Numbers=Lining}} % format
{} % label
{} % sep
{
    % \rule{\textwidth}{1pt}
    % \vspace{1ex}
    \centering
} % before-code
%[
% \vspace{-0.5ex}%
% \rule{\textwidth}{0.3pt}
%] % after-code
 
 
\titleformat{\chapter}[block]
{\thispagestyle{empty}\phantomsection\Large\scshape\addfontfeature{Numbers=Lining}}
{}{0.5em}{\centering}
 
\titlespacing{\chapter}{0pt}{6pt-\headheight}{1pc}
\titlespacing{\section}{0pt}{*2.5}{*1}
\titleclass{\chapter}{top}
\newcommand{\chapterbreak}{\clearpage}
%\titleclass{\section}{top}

\contentsmargin{2pc}
%\dottedcontents{chapter}[2.3em]{}{2.3em}{1pc}
\titlecontents{chapter}[2.3em]{}{\contentslabel{2.3em}}{\hspace*{-2.3em}}{}
\dottedcontents{section}[5.5em]{}{3.2em}{1pc}

\newcommand{\printnote}[1]{
	{\normalsize \emph{#1}}%
}
\newcommand{\subtitle}[1]{
{
	\centering
	{\addfontfeature{Numbers=Lining} \normalsize \emph{#1}}\par
}
}
\makeatletter  
\newcounter{score}
\newcounter{tabstop}[score]
\newcommand{\grealign}{%
  \@bsphack%
  \ifgre@boxing\else%
    \kern\gre@dimen@begindifference%
    \stepcounter{tabstop}%
    \expandafter\zsavepos{stop-\thescore-\thetabstop}%
    \kern-\gre@dimen@begindifference%
  \fi%
  \@esphack%
}
\newcommand{\setstops}{%
  \gdef\nstabbing@stops{%
    \checkoddpage
    \hspace*{-\ifoddpage\oddsidemargin\else\evensidemargin\fi}\hspace{-1in}%
    \hspace*{\zposx{stop-\thescore-1} sp}\=%
  }%
  \count@=\@ne
  \loop\ifnum\count@<\value{tabstop}%
    \begingroup\edef\x{\endgroup
      \noexpand\g@addto@macro\noexpand\nstabbing@stops{%
        \noexpand\hspace{-\noexpand\zposx{stop-\thescore-\the\count@} sp}%
        \noexpand\hspace{\noexpand\zposx{stop-\thescore-\the\numexpr\count@+1} sp}\noexpand\=%
      }%
    }\x
    \advance\count@\@ne
  \repeat
  \nstabbing@stops\kill
}
\makeatother

\newenvironment{nstabbing}
  {\setlength{\topsep}{0pt}%
   \setlength{\partopsep}{0pt}%
   \tabbing%
   \setstops}
  {\endtabbing\stepcounter{score}}

\def\beginpsalmcols{\begin{parcolumns}[rulebetween]{2}}
\def\endpsalmcols{\end{parcolumns}}

\newcommand{\deusinadjutorium}{\noindent{\Vbar~Deus in adjutórium, p.~\pageref{deusinadjutorium}}}
%\newcommand{\deusinadjutoriumsolemn}{\noindent\printnote{\Vbar~\emph{Deus in adjutórium}, p.~\pageref{deusinadjutoriumsolemn}.}}
\newcommand{\printcollect}[2]{
	\ifx\undefined\begincollectcols\def\begincollectcols{\begin{parcolumns}[rulebetween]{2}}\fi
	\ifx\printcollectheading\undefined\def\printcollectheading{T}\fi
	\if\printcollectheading T
		\oldneedspace{36pt}
		%\penalty -100% this is to try to replace the oldneedspace
		\medskip
		{\centering\large Collect.\par}
		%\smallskip
	\fi
	\ifx\precollect\undefined\else\precollect\fi
	\begincollectcols
	\sloppy
	\prayer{#1}{#2}
	\end{parcolumns}
	\let\begincollectcols=\undefined
}
\newcommand{\benedicamusdominoreference}[1]{%
	\ifthenelse{\boolean{includebenedicamusdominoreferences}}{
		\Vbar~\emph{Benedicámus Dómino \IfInteger{#1}{#1}{\csname benedicamusdominoname#1\endcsname}}, p.~\pageref{benedicamusdomino-#1}.
	}{}
}
\newcommand{\benedicamusdominoreferencelentoreaster}{%
	\ifthenelse{\boolean{includebenedicamusdominoreferences}}{
		In Lent, \Vbar~\emph{Benedicámus Dómino \benedicamusdominonamelent}, p.~\pageref{benedicamusdomino-lent}, or in Easter, \emph{\benedicamusdominonameeaster}, page \pageref{benedicamusdomino-easter}.
	}{}
}
\newcommand{\benedicamusdominomaster}[1]{
	\ifthenelse{\boolean{includebenedicamusdominoreferences}}{
		\noindent\printnote{\benedicamusdominoreference{#1}}
		\ifx\postbenedicamusdomino\undefined\else\postbenedicamusdomino\fi
	}{}
	\bigskip
	\hrule
}
\newcommand{\benedicamusdominolentoreaster}{
	\ifthenelse{\boolean{includebenedicamusdominoreferences}}{
		\noindent\printnote{\benedicamusdominoreferencelentoreaster}
		\ifx\postbenedicamusdomino\undefined\else\postbenedicamusdomino\fi
	}{}
	\bigskip
	\hrule
}

\ifthenelse{\boolean{lettersize}}{
	\newcommand{\psalmcolsoverride}[1][0]{
	}	
}{
	\newcommand{\psalmcolsoverride}[1][0]{
		\def\beginpsalmcols{\begin{parcolumns}[rulebetween,colwidths={1=0.45\linewidth}]{2}}
		\ifnum#1=110
		\def\beginpsalmcols{\begin{parcolumns}[rulebetween,colwidths={1=0.465\linewidth}]{2}}
		\fi
		\ifnum#1=111
		\def\beginpsalmcols{\begin{parcolumns}[rulebetween,colwidths={1=0.475\linewidth}]{2}}
		\fi
		%\ifnum#1=112
		%\def\beginpsalmcols{\begin{parcolumns}[rulebetween,colwidths={1=0.445\linewidth}]{2}}
		%\fi
		\ifnum#1=116
		\def\beginpsalmcols{\begin{parcolumns}[rulebetween,colwidths={1=0.5655\linewidth}]{2}}
		\fi
		\ifnum#1=121
		\def\beginpsalmcols{\begin{parcolumns}[rulebetween,colwidths={1=0.47\linewidth}]{2}}
		\fi
		\ifnum#1=125
		\def\beginpsalmcols{\begin{parcolumns}[rulebetween,colwidths={1=0.49\linewidth}]{2}}
		\fi
		\ifnum#1=129
		\def\beginpsalmcols{\begin{parcolumns}[rulebetween,colwidths={1=0.475\linewidth}]{2}}
		\fi
		\ifnum#1=131
		\def\beginpsalmcols{\begin{parcolumns}[rulebetween,colwidths={1=0.4875\linewidth},distance=1em]{2}}
		\fi
		\ifnum#1=137
		\def\beginpsalmcols{\begin{parcolumns}[rulebetween]{2}}
		\fi
		\ifnum#1=147
		%\def\beginpsalmcols{\begin{parcolumns}[rulebetween,colwidths={1=0.465\linewidth}]{2}}
		\def\beginpsalmcols{\begin{parcolumns}[rulebetween,colwidths={1=0.475\linewidth}]{2}}
		\fi
	}
}
\DeclareDocumentCommand{\printbothversions}{ O{\undefined} O{\undefined} m m }{
% #1 & #2 are the label
% #3 is what to check for
% #4 is the body of what to print
	\ifx#3\undefined
		\ifx\definevesperspropersalt\undefined\else
	  {
			\ifx\vesperspropersaltnote\undefined\else
		    \oldneedspace{3\baselineskip}
				\printnote{\vesperspropersaltnote}
			\fi
			\definevesperspropersalt
			\ifx#2\undefined\else\oldneedspace{5\baselineskip}\label{#2}\fi
	  	#4
		}
		\medskip
		\fi
		\ifx\definevesperspropers\undefined\else
	  {
			\ifx\vesperspropersnote\undefined\else
	    	\oldneedspace{3\baselineskip}
				\printnote{\vesperspropersnote}
			\fi
			\definevesperspropers
			\ifx#1\undefined\else\oldneedspace{5\baselineskip}\label{#1}\fi
	  	#4
		}
		\fi
	\else
  {
		\ifx#1\undefined\else\label{#1}\fi
  	#4
	}
	\fi
}
\let\printhymn=\undefined
\newcommand{\printpsalmtitle}[1]{%
  {\addfontfeature{Numbers=Lining}\centering Psalm #1\\}%
}
\newcommand{\printtercepsalms}[1]{
  % #1 is psalm tone like 2D
  \printpsalmtitle{118. III.}

  %print first verse in chant
  {
    \gresetinitiallines{0}
    \grechangedim{interwordspacetext}{0.15 cm plus 0.15 cm minus 0.10 cm}{scalable}%
    \gregorioscore{psalms/Psalm118.3-#1}
    % the additional width of the additional lines (compared to the width of the glyph they're associated with)
\grechangedim{additionallineswidth}{0.14584 cm}{scalable}%
% width of the additional lines, used only for the custos (maybe should depend on the width of the custos...)
% the width is the one for the custos at end of lines, the line for custos in the middle of a score is the same
% multiplied by 2.
\grechangedim{additionalcustoslineswidth}{0.09114 cm}{scalable}%
% null space
\grechangedim{zerowidthspace}{0 cm}{scalable}%
% space between glyphs in the same element
\grechangedim{interglyphspace}{0.06927 cm plus 0.00363 cm minus 0.00363 cm}{scalable}%
% space between an alteration (flat or natural) and the next glyph
\grechangedim{alterationspace}{0.07747 cm plus 0.01276 cm minus 0.00455 cm}{scalable}%
% space between a clef and a flat (for clefs with flat)
\grechangedim{clefflatspace}{0.05469 cm plus 0.00638 cm minus 0.00638 cm}{scalable}%
% space before a choral sign
\grechangedim{beforelowchoralsignspace}{0.04556 cm plus 0.00638 cm minus 0.00638 cm}{scalable}%
% when bolshifts are enabled, minimal space between a clef at the beginning of the line and a leading alteration glyph (should be larger than clefflatspace so that a flatted clef can be distinguished from a flat which is part of the first glyph on a line, but also smaller than spaceafterlineclef, the distance from the clef to the first notes)
\grechangedim{beforealterationspace}{0.1 cm}{scalable}%
% space between elements
\grechangedim{interelementspace}{0.06927 cm plus 0.00182 cm minus 0.00363 cm}{scalable}%
% larger space between elements
\grechangedim{largerspace}{0.10938 cm plus 0.01822 cm minus 0.00911 cm}{scalable}%
% space between elements in ancient notation
\grechangedim{nabcinterelementspace}{0.06927 cm plus 0.00182 cm minus 0.00363 cm}{scalable}%
% larger space between elements in ancient notation
\grechangedim{nabclargerspace}{0.10938 cm plus 0.01822 cm minus 0.00911 cm}{scalable}%
% space between elements which has the size of a note
\grechangedim{glyphspace}{0.21877 cm plus 0.01822 cm minus 0.01822 cm}{scalable}%
% space before custos
\grechangedim{spacebeforecustos}{0.1823 cm plus 0.31903 cm minus 0.0638 cm}{scalable}%
% space before punctum mora and augmentum duplex
\grechangedim{spacebeforesigns}{0.05469 cm plus 0.00455 cm minus 0.00455 cm}{scalable}%
% space after punctum mora and augmentum duplex
\grechangedim{spaceaftersigns}{0.08203 cm plus 0.0082 cm minus 0.0082 cm}{scalable}%
% space after a clef at the beginning of a line
\grechangedim{spaceafterlineclef}{0.27345 cm plus 0.14584 cm minus 0.01367 cm}{scalable}%
% minimal space between notes of different words
%\grechangedim{interwordspacenotes}{0.27 cm plus 0.15 cm minus 0.05 cm}{scalable}%
\grechangedim{interwordspacenotes}{0.27 cm plus 0.08 cm minus 0.05 cm}{scalable}%
% minimal space between notes of the same syllable.
% Warning: always keep minus to 0; also keep plus very low, or some words won't be hyphenated
%\grechangedim{intersyllablespacenotes}{0.24 cm plus 0.04cm minus 0cm}{scalable}%
\grechangedim{intersyllablespacenotes}{0.24 cm plus 0.04cm minus 0cm}{scalable}%
% minimal space between letters of different words. Makes sense to have
% the same plus and minus as interwordspacenotes.
%\grechangedim{interwordspacetext}{0.38 cm plus 0.15 cm minus 0.05 cm}{scalable}%
\grechangedim{interwordspacetext}{0.18 cm plus 0.08 cm minus 0.05 cm}{scalable}%
% Versions of interword spaces for euouae blocks
%\grechangedim{interwordspacenotes@euouae}{0.19 cm plus 0.1 cm minus 0.05 cm}{scalable}%
\grechangedim{interwordspacenotes@euouae}{0.13 cm plus 0.1 cm minus 0.05 cm}{1}%
%\grechangedim{interwordspacetext@euouae}{0.27 cm plus 0.1 cm minus 0.05 cm}{scalable}%
\grechangedim{interwordspacetext@euouae}{0.13 cm plus 0.1 cm minus 0.05 cm}{1}%
% space between notes of a bivirga or trivirga
\grechangedim{bitrivirspace}{0.06927 cm plus 0.00182 cm minus 0.00546 cm}{scalable}%
% space between notes of a bistropha or tristrophae
\grechangedim{bitristrospace}{0.06927 cm plus 0.00182 cm minus 0.00546 cm}{scalable}%
% space between two punctum inclinatum
\grechangedim{punctuminclinatumshift}{-0.03918 cm plus 0.0009 cm minus 0.0009 cm}{scalable}%
% space before puncta inclinata
\grechangedim{beforepunctainclinatashift}{0.05286 cm plus 0.00728 cm minus 0.00455 cm}{scalable}%
% space between a punctum inclinatum and a punctum inclinatum deminutus
\grechangedim{punctuminclinatumanddebilisshift}{-0.02278 cm plus 0.0009 cm minus 0.0009 cm}{scalable}%
% space between two punctum inclinatum deminutus
\grechangedim{punctuminclinatumdebilisshift}{-0.00728 cm plus 0.0009 cm minus 0.0009 cm}{scalable}%
% space between puncta inclinata, larger ambitus (range=3rd)
\grechangedim{punctuminclinatumbigshift}{0.07565 cm plus 0.0009 cm minus 0.0009 cm}{scalable}%
% space between puncta inclinata, larger ambitus (range=4th -or more?-)
\grechangedim{punctuminclinatummaxshift}{0.17865 cm plus 0.0009 cm minus 0.0009 cm}{scalable}%
% space for the bars (inside syllables)
%first for virgula and divisio minima
\grechangedim{spacearoundsmallbar}{0.1823 cm plus 0.22787 cm minus 0.00469 cm}{scalable}%
%then divisio minor
\grechangedim{spacearoundminor}{0.1823 cm plus 0.22787 cm minus 0.00469 cm}{scalable}%
%divisio major
\grechangedim{spacearoundmaior}{0.1823 cm plus 0.22787 cm minus 0.00469 cm}{scalable}%
%divisio finalis
\grechangedim{spacearoundfinalis}{0.1823 cm plus 0.22787 cm minus 0.00469 cm}{scalable}%
%a special space for finalis, for when it is the last glyph
\grechangedim{spacebeforefinalfinalis}{0.29169 cm plus 0.07292 cm minus 0.27345 cm}{scalable}%
% additional space that will appear around bars that are preceded by a custos and followed by a key.
\grechangedim{spacearoundclefbars}{0.03645 cm plus 0.00455 cm minus 0.0009 cm}{scalable}%
% space between the text and the text of the bar
\grechangedim{textbartextspace}{0.24611 cm plus 0.13672 cm minus 0.04921 cm}{scalable}%
% minimal space between a note and a bar
\grechangedim{notebarspace}{0.31903 cm plus 0.27345 cm minus 0.02824 cm}{scalable}%
% maximal space between two syllables for which we consider a dash is not needed
\grechangedim{maximumspacewithoutdash}{0.00 cm}{scalable}%
% an extensible space for the beginning of lines
\grechangedim{afterclefnospace}{0 cm plus 0.27345 cm minus 0 cm}{scalable}%
% space between the initial and the beginning of the score
\grechangedim{afterinitialshift}{0.2457 cm}{scalable}%
% space before the initial
\grechangedim{beforeinitialshift}{0.2457 cm}{scalable}%
% when bolshifts are enabled, minimum space between beginning of line and first syllable text
\grechangedim{minimalspaceatlinebeginning}{0.05 cm}{scalable}%
% space to force the initial width to.  Ignored when 0.
\grechangedim{manualinitialwidth}{0 cm}{scalable}%
% distance to move the initial up by
\grechangedim{initialraise}{0 cm}{scalable}%
% Space between lines in the annotation
\grechangedim{annotationseparation}{0.05cm}{scalable}%
% Amount to raise (positive) or lower (negative) the annotations from the default position (base line of top annotation aligned with top line of staff)
\grechangedim{annotationraise}{0cm}{scalable}%
% space at the beginning of the lines if there is no clef
\grechangedim{noclefspace}{0.1 cm}{scalable}%
% space around a clef change
\grechangedim{clefchangespace}{0.01768 cm plus 0.00175 cm minus 0.01768 cm}{scalable}%
%When \gre@clivisalignment is 2, this distance is the maximum length of the consonants after vowels for which the clivis will be aligned on its center.
\grechangedim{clivisalignmentmin}{0.3 cm}{scalable}%



%%%%%%%%%%%%%%%%%%
% vertical spaces
%%%%%%%%%%%%%%%%%%

% first, we have two spaces for the chironomic signs
\grechangedim{abovesignsspace}{0.8 cm}{scalable}%
\grechangedim{belowsignsspace}{0 cm}{scalable}%
% the amount to shift down:
% (a) low choral signs that are not lower than the note, regardless of whether
%     it's on a line or in a space
% (b) high choral signs and low choral signs that are lower than the note which
%     are in a space
\grechangedim{choralsigndownshift}{0.00911 cm}{scalable}%
% the amount to shift up:
% (a) high choral signs and low choral signs that are lower than the note which
%     are on a line
\grechangedim{choralsignupshift}{0.04556 cm}{scalable}%
% the space for the translation
\grechangedim{translationheight}{0.5 cm}{scalable}%
%the space above the lines
\grechangedim{spaceabovelines}{0.45576 cm plus 0.36461 cm minus 0.09114 cm}{scalable}%
%the space between the lines and the bottom of the text
\grechangedim{spacelinestext}{0.60617 cm}{scalable}%
%the space beneath the text
\grechangedim{spacebeneathtext}{0 cm}{scalable}%
% height of the text above the note line
\grechangedim{abovelinestextraise}{-0.1 cm}{scalable}%
% height that is added at the top of the lines if there is text above the lines (it must be bigger than the text for it to be taken into consideration)
\grechangedim{abovelinestextheight}{0.3 cm}{scalable}%
% an additional shift you can give to the brace above the bars if you don't like it
\grechangedim{braceshift}{0 cm}{scalable}%
% a shift you can give to the accentus above the curly brace
\grechangedim{curlybraceaccentusshift}{-0.05 cm}{scalable}%


    \greseteolcustos{auto}
  }
  \vspace{0pt plus 4pt minus 8pt}
  \setlength\parsep{0pt}%
  \setlength\topsep{0pt}%
  \setlength\partopsep{0pt}%
  \setlength\multicolsep{0pt}%
  \setlength{\columnsep}{18pt}
  \setlength{\columnseprule}{.4pt}
  \selectlanguage{latin}
  \begin{multicols}{2}
  \begin{psalmverses}[1]
  \input{psalms/Psalm118.3-#1-verses}
  \end{psalmverses}
  %\medskip
  \printpsalmtitle{118. IV.}
  \begin{psalmverses}
  \input{psalms/Psalm118.4-#1-verses}
  \end{psalmverses}
  %\medskip
  \printpsalmtitle{118. V.}
  \begin{psalmverses}
  \input{psalms/Psalm118.5-#1-verses}
  \end{psalmverses}
  \setlength{\parindent}{6ex}
  \ifx\printrepeatantiphon\undefined
	  \IfStrEq{#1}{no-tone}{}{%if not no-tone

	    \emph{Repeat antiphon.}%
	  }
	\else

		\printrepeatantiphon
	\fi
  \end{multicols}
}
\newcommand{\printtercepsalmstranslation}{
  \setlength\parsep{0pt}%
  \setlength\topsep{0pt}%
  \setlength\partopsep{0pt}%
  \setlength\multicolsep{0pt}%
  \setlength{\columnsep}{18pt}
  \setlength{\columnseprule}{.4pt}
  \selectlanguage{american}
  \begin{multicols}{2}
  \printpsalmtitle{118. III.}
  \begin{psalmverses}
  \emph{\item Set before me for a law the way of thy justifications, O Lord: * and I will always seek after it.
\item Give me understanding, and I will search thy law; * and I will keep it with my whole heart.
\item Lead me into the path of thy commandments; * for this same I have desired.
\item Incline my heart into thy testimonies * and not to covetousness.
\item Turn away my eyes that they may not behold vanity: * quicken me in thy way.
\item Establish thy word to thy servant, * in thy fear.
\item Turn away my reproach, which I have apprehended: * for thy judgments are delightful.
\item Behold I have longed after thy precepts: * quicken me in thy justice.
\item Let thy mercy also come upon me, O Lord: * thy salvation according to thy word.
\item So shall I answer them that reproach me in any thing; * that I have trusted in thy words.
\item And take not thou the word of truth utterly out of my mouth: * for in thy words, I have hoped exceedingly.
\item So shall I always keep thy law, * for ever and ever.
\item And I walked at large: * because I have sought after thy commandments.
\item And I spoke of thy testimonies before kings: * and I was not ashamed.
\item I meditated also on thy commandments, * which I loved.
\item And I lifted up my hands to thy commandments, which I loved: * and I was exercised in thy justifications.}
  \end{psalmverses}
  \medskip
  \printpsalmtitle{118. IV.}
  \begin{psalmverses}
  \emph{\item Be thou mindful of thy word to thy servant, * in which thou hast given me hope.
\item This hath comforted me in my humiliation: * because thy word hath enlivened me.
\item The proud did iniquitously altogether: * but I declined not from thy law.
\item I remembered, O Lord, thy judgments of old: * and I was comforted.
\item A fainting hath taken hold of me, * because of the wicked that forsake thy law.
\item Thy justifications were the subject of my song, * in the place of my pilgrimage.
\item In the night I have remembered thy name, O Lord: * and have kept thy law.
\item This happened to me: * because I sought after thy justifications.
\item O Lord, my portion, * I have said, I would keep thy law.
\item I entreated thy face with all my heart: * have mercy on me according to thy word.
\item I have thought on my ways: * and turned my feet unto thy testimonies.
\item I am ready, and am not troubled: * that I may keep thy commandments.
\item The cords of the wicked have encompassed me: * but I have not forgotten thy law.
\item I rose at midnight to give praise to thee; * for the judgments of thy justification.
\item I am a partaker with all them that fear thee, * and that keep thy commandments.
\item The earth, O Lord, is full of thy mercy: * teach me thy justifications.}
  \end{psalmverses}
  \medskip
  \printpsalmtitle{118. V.}
  \begin{psalmverses}
  \emph{\item Thou hast done well with thy servant, O Lord, * according to thy word.
\item Teach me goodness and discipline and knowledge; * for I have believed thy commandments.
\item Before I was humbled I offended; * therefore have I kept thy word.
\item Thou art good; * and in thy goodness teach me thy justifications.
\item The iniquity of the proud hath been multiplied over me: * but I will seek thy commandments with my whole heart.
\item Their heart is curdled like milk: * but I have meditated on thy law.
\item It is good for me that thou hast humbled me, * that I may learn thy justifications.
\item The law of thy mouth is good to me, * above thousands of gold and silver.
\item Thy hands have made me and formed me: * give me understanding, and I will learn thy commandments.
\item They that fear thee shall see me, and shall be glad: * because I have greatly hoped in thy words.
\item I know, O Lord, that thy judgments are equity: * and in thy truth thou hast humbled me.
\item O! let thy mercy be for my comfort, * according to thy word unto thy servant.
\item Let thy tender mercies come unto me, and I shall live: * for thy law is my meditation.
\item Let the proud be ashamed, because they have done unjustly towards me: * but I will be employed in thy commandments.
\item Let them that fear thee turn to me: * and they that know thy testimonies.
\item Let my heart be undefiled in thy justifications, * that I may not be confounded.}
  \end{psalmverses}
  \end{multicols}
}
\def\printbenedicamusdominoref{%
  \noindent{\Vbar{}~Benedicámus Dómino, p.~\pageref{benedicamus-domino-sunday}.}
}
\def\printhymnsundayrefnospace{%
  \emph{Hymn 1}, p.~\pageref{hymn-sunday}%
}
\def\printhymnsundayref{%
  , \printhymnsundayrefnospace{}.

  \medskip
}
\def\printhymnfeastref{%
  , \emph{Hymn 2}, p.~\pageref{hymn-feast}.

  \medskip
}
\def\printhymnbvmref{%
  , \emph{Hymn 3}, p.~\pageref{hymn-bvm}.

  \medskip
}
\let\printhymn=\printhymnsundayref
\newcommand{\printterce}[3][.]{
	{
		\grechangestaffsize{15}
		\ifx\gabcfolder\undefined
			\def\gabcfolder{#1}
			\input{\gabcfolder/#2}
		\fi

		\deusinadjutorium{}%
		\ifx\printhymn\undefined%
		\else%
		  \printhymn
		\fi

		\ifx\anttwotex\undefined\else%
			\setlabel{#3}{ant}
			\printgabc{Ant.}{\anttwolinetwo}{\anttwoinitial}{\anttwotex .noEuouae}

			\ifx\anttwotranslation\undefined\else%
				\translation[]{\anttwotranslation}
			\fi

			\printtercepsalms{\anttwotoneendexpanded}
		\fi
		\ifx\chaptertext\undefined\else
			\ifx\prechapter\undefined\else%
				\prechapter
			\fi
			\setlabel{#3}{chapter}
			\printchapter{\chaptertext}{\chaptertranslation}

			\medskip
		\fi

		\ifx\printshortresp\undefined%
			\noindent\emph{Short Resp. \emph{Inclina cor meum}}, p. \pageref{shortresp-sunday}, \Vbar{}~Ego dixi, p. \pageref{vr-sunday}.
		\else
			\printshortresp
		\fi

		\ifx\collect\undefined%
			\let\collect=\latincollect
			\let\collecttranslation=\englishcollect
		\fi
		\setlabel{#3}{collect}
		\printcollect{\collect}{\collecttranslation}

		\medskip

		\ifx\printbenedicamusdominoref\undefined\else\printbenedicamusdominoref\fi
	}
}

\sloppy
%\nofiles
\begin{document}
\normalsize
\grechangestaffsize{15}
{
	\thispagestyle{empty}
	% title page
	% general
	\vspace*{5\baselineskip}

	{\centering

	{\Huge
	Terce with Gregorian Chant

	}

	{\Large\medskip
	\emph{for}

	\medskip}

	{\Huge
	Sundays \& Holy Days

	}
	\vfill}
	\pagebreak
	% copyright page
	\thispagestyle{empty}
	\noindent{}Terce with Gregorian Chant for Sundays \& Holy Days: \emph{newly typeset, based on \emph{The Liber Usualis}, edited by the Benedictines of Solesmes (Desclee Company, 1961).}

	\bigskip{}
	\noindent{}%
	Elie Roux's Gregorio (https://gregorio-project.github.io), Olivier Berton's GregoBase (http://gregobase.selapa.net), and my brother Benjamin's chant tools (http://bbloomf.github.io/jgabc) were indispensible in the creation of this book.

	\begin{flushright}
	\emph{Albert Bloomfield}\\
	Cincinnati, Ohio
	\end{flushright}

	\bigskip{}\noindent{}http://asbloomf.github.io/gabc-chants

	\vfill
%editions
	\bigskip{}\noindent{}%
	First edition, 13 July 2021.

	\hangindent=1em % indent all subsequent lines
    \bigskip

\makeatletter
	\noindent{}Typeset using \LuaLaTeX{} and Gregorio version \gre@gregoriotexversion{}
\makeatother

	\bigskip{}\noindent{}This work is free of known copyright restrictions.

	\bigskip{}\noindent{}%CreateSpace
	ISBN: \isbn{}
	%
  %\noindent{}Lulu ISBN: 978-1-329-59990-1
}
%\frontmatter
%\pagenumbering{roman}

\tableofcontents%

%\pagenumbering{arabic}
%\mainmatter

% redefine the  label command
\newcommand{\setlabel}[2]{
% #1 label part 1 (like easter2)
% #2 label part 2 (any of these 3: ant/chapter/collect)
	\label{#1-#2}% now execute the original label command
}

%\chapter{Common of Festal Vespers}

\sectionmark{Deus in adjutórium}
\addcontentsline{toc}{section}{Deus in adjutórium}

%\subtitle{Festal Tone}
\label{deusinadjutorium}
\printnote{The Festal Tone may be sung on any Sunday or Feast.}
\printnote{From Septuagesima to Wednesday in Holy Week, the \emph{Laus tibi} is said instead of \emph{Allelúia}.}

\bigskip
\def\deusinadjutoriumsolemn{F}
\printdeusinadjutorium{}
\vfil

%\subtitle{Solemn Tone}
\printnote{The Solemn Tone may only be sung on Feasts of the First or Second Class.}
\printnote{From Septuagesima to Wednesday in Holy Week, the \emph{Laus tibi} is said instead of \emph{Allelúia}.}

\bigskip
{
  \def\deusinadjutoriumsolemn{T}
  \ifthenelse{\boolean{birmingham}}{
    \def\includetranslation{F}
  }{}
  \label{deusinadjutoriumsolemn}
  \printdeusinadjutorium{}
  \let\deusinadjutoriumsolemn=\undefined
}

\printnote{Vespers then proceed with the Proper Antiphons, Psalms, Chapter, Hymn, Versicle,
Magnificat, and Collect given for the respective Sunday or Feast.

For Sundays throughout the year, \emph{p.~\pageref{sundayvespers}}.

For Sundays in Paschaltide, \emph{p.~\pageref{sundayvespers-easter}}.
}

\bigskip

\newcommand{\printvrwithtranslation}{
    {\normalsize
    \ifx\beginvrcols\undefined\def\beginvrcols{\begin{parcolumns}[rulebetween]{2}}\fi
    \beginvrcols
    \colchunk{
      \par\vspace{-\baselineskip}\noindent\selectlanguage{latin}%
      \Vbar{}~\vlatin{}
    }
    \colchunk{%
      \par\vspace{-\baselineskip}\noindent\selectlanguage{american}%
      \Vbar{}~\vtranslation{}
    }%
    \colplacechunks%
    \ifx\rlatin\undefined\else
    \colchunk{
      \par\vspace{-\baselineskip}\noindent\selectlanguage{latin}%
      \Rbar{}~\rlatin{}%
    }
    \colchunk{%
      \par\vspace{-\baselineskip}\noindent\selectlanguage{american}%
      \Rbar{}~\rtranslation{}%
    }%
    \colplacechunks%
    \fi
    \end{parcolumns}
    }
}

{
  \printnote{All Vespers conclude with the following:}\par
  \def\beginvrcols{\begin{parcolumns}[rulebetween,colwidths={1=0.48\linewidth}]{2}}
  \def\vlatin{Dóminus vobíscum.}
  \def\rlatin{Et cum spíritu tuo.}
  \def\vtranslation{The Lord be with you.}
  \def\rtranslation{And with thy spirit.}
  \printvrwithtranslation{}

  %\Vbar{}~Dóminus vobíscum.\\%\hspace{5em}
  %\Rbar{}~Et cum spíritu tuo.

  \ifthenelse{\boolean{birmingham}}{% TRUE is empty
  }{% False
    \printnote{Or, in the absence of a priest or deacon:}\par
    \def\vlatin{Dómine, exáudi oratiónem meam.}
    \def\rlatin{Et clamor meus ad te véniat.}
    \def\vtranslation{Lord, hear my prayer.}
    \def\rtranslation{And let my cry come unto Thee.}
    \printvrwithtranslation{}
  }

  %\Vbar{}~Dómine, exáudi oratiónem meam.\\
  %\Rbar{}~Et clamor meus ad te véniat.

  \bigskip

  \noindent\printnote{The ``Benedicámus Dómino'' is then sung\ifthenelse{\boolean{birmingham}}{}{ in one of the tones on the next pages}.}

  \bigskip

  \let\rlatin\undefined
  \let\rtranslation\undefined
  \def\vlatin{Fidélium ánimæ per misericórdiam Dei requiéscant in pace.\\\Rbar~Amen.}
  %\def\rlatin{Amen.}
  \def\vtranslation{May the souls of the faithful departed through the mercy of God rest in peace.\hfill\Rbar~Amen.}
  %\def\rtranslation{Amen.}
  \printvrwithtranslation{}
}
%\noindent\Vbar{}~Fidélium ánimæ per misericóridam Dei requiéscant in pace.\\
%\Rbar{}~Amen.

\def\noeuouae{T}
\def\dontrepeatantiphon{T}

\ifthenelse{\boolean{testrun}}{
	\makeatletter  
\newcounter{score}
\newcounter{tabstop}[score]
\newcommand{\grealign}{%
  \@bsphack%
  \ifgre@boxing\else%
    \kern\gre@dimen@begindifference%
    \stepcounter{tabstop}%
    \expandafter\zsavepos{stop-\thescore-\thetabstop}%
    \kern-\gre@dimen@begindifference%
  \fi%
  \@esphack%
}

\newcommand{\setstops}{%
  \gdef\nstabbing@stops{%
    \checkoddpage
    \hspace*{-\ifoddpage\oddsidemargin\else\evensidemargin\fi}\hspace{-1in}%
    \hspace*{\zposx{stop-\thescore-1} sp}\=%
  }%
  \count@=\@ne
  \loop\ifnum\count@<\value{tabstop}%
    \begingroup\edef\x{\endgroup
      \noexpand\g@addto@macro\noexpand\nstabbing@stops{%
        \noexpand\hspace{-\noexpand\zposx{stop-\thescore-\the\count@} sp}%
        \noexpand\hspace{\noexpand\zposx{stop-\thescore-\the\numexpr\count@+1} sp}\noexpand\=%
      }%
    }\x
    \advance\count@\@ne
  \repeat
  \nstabbing@stops\kill
}
\makeatother

\newenvironment{nstabbing}
  {\setlength{\topsep}{0pt}%
   \setlength{\partopsep}{0pt}%
   \tabbing%
   \setstops}
  {\endtabbing\stepcounter{score}}

\def\beginpsalmcols{\begin{parcolumns}[rulebetween]{2}}
\def\endpsalmcols{\end{parcolumns}}
\newcommand{\printpsalmtitle}[1]{%
    {\addfontfeature{Numbers=Lining}\centering Psalm #1\par}
  }

\newcommand{\printpsalmwithtranslation}[3]{
	% #1 psalm number
	% #2 ending
	% #3 0 to start on 1st verse, 1 for second
	\setlength{\columnsep}{18pt}
	\setlength{\columnseprule}{.4pt}
	\beginpsalmcols
	\colchunk{\vspace{-12pt}%
	\begin{psalmverses}[#3]
	\vspace{-\baselineskip}%
	\input{psalms/Psalm#1-#2-verses}
	\end{psalmverses}
	}

	\selectlanguage{american}
	\colchunk{\vspace{-12pt}%
	\sloppy
	\begin{psalmverses}
	\vspace{-\baselineskip}%
	\input{psalms/Psalm#1-verses-en}
	\end{psalmverses}
	}
	\selectlanguage{latin}
	\endpsalmcols
}

\renewcommand{\printpsalm}[3]{
  % #1 psalm number
  % #2 ending
  % #3 0 to start on 1st verse, 1 for second
  \setlength{\columnsep}{18pt}
  \setlength{\columnseprule}{.4pt}
  \begin{multicols}{2}
  \begin{psalmverses}[#3]
  \input{psalms/Psalm#1-#2-verses}
  \end{psalmverses}
  \end{multicols}
}



\newcommand{\printpsalmtranslation}[3]{
  % #1 psalm number
  % #2 ending
  % #3 0 to start on 1st verse, 1 for second
  \setlength{\columnsep}{18pt}
  \setlength{\columnseprule}{.4pt}
  \begin{multicols}{2}
  \begin{psalmverses}[#3]
  \input{psalms/Psalm#1-verses-en}
  \end{psalmverses}
  \end{multicols}
}

\def\beginchaptercols{\begin{parcolumns}[rulebetween]{2}}
\newcommand{\printchapternew}[4]{
	% #1 ref
	% #2 latin first words
	% #3 latin remaining
	% #4 english text
	{\noindent\hspace{3em}Chapter.\hfill\emph{#1}\hspace{3em}}
    \beginchaptercols{}
    \colchunk{\sloppy\dropcap{latin}{#2} #3\hfill{}\Rbar{}~Deo~grátias.}
    \colchunk{\sloppy \dropcap{american}{#4}}
    \end{parcolumns}
}
\newcommand{\printvrwithtranslation}{
    {\normalsize
    \ifx\beginvrcols\undefined\def\beginvrcols{\begin{parcolumns}[rulebetween]{2}}\fi
    \beginvrcols
    \colchunk{
      \par\vspace{-\baselineskip}\noindent\selectlanguage{latin}%
      \Vbar{}~\vlatin{}
    }
    \colchunk{%
      \par\vspace{-\baselineskip}\noindent\selectlanguage{american}%
      \Vbar{}~\vtranslation{}
    }%
    \colplacechunks%
    \ifx\rlatin\undefined\else
    \colchunk{
      \par\vspace{-\baselineskip}\noindent\selectlanguage{latin}%
      \Rbar{}~\rlatin{}%
    }
    \colchunk{%
      \par\vspace{-\baselineskip}\noindent\selectlanguage{american}%
      \Rbar{}~\rtranslation{}%
    }%
    \colplacechunks%
    \fi
    \end{parcolumns}
    }
}
\newcommand{\printtercepsalms}[1]{
  % #1 is psalm tone like 2D
  \printpsalmtitle{118. III.}

  %print first verse in chant
  {
    \gresetinitiallines{0}
    \grechangedim{interwordspacetext}{0.15 cm plus 0.15 cm minus 0.10 cm}{scalable}%
    \gregorioscore{psalms/Psalm118.3-#1}
    % the additional width of the additional lines (compared to the width of the glyph they're associated with)
\grechangedim{additionallineswidth}{0.14584 cm}{scalable}%
% width of the additional lines, used only for the custos (maybe should depend on the width of the custos...)
% the width is the one for the custos at end of lines, the line for custos in the middle of a score is the same
% multiplied by 2.
\grechangedim{additionalcustoslineswidth}{0.09114 cm}{scalable}%
% null space
\grechangedim{zerowidthspace}{0 cm}{scalable}%
% space between glyphs in the same element
\grechangedim{interglyphspace}{0.06927 cm plus 0.00363 cm minus 0.00363 cm}{scalable}%
% space between an alteration (flat or natural) and the next glyph
\grechangedim{alterationspace}{0.07747 cm plus 0.01276 cm minus 0.00455 cm}{scalable}%
% space between a clef and a flat (for clefs with flat)
\grechangedim{clefflatspace}{0.05469 cm plus 0.00638 cm minus 0.00638 cm}{scalable}%
% space before a choral sign
\grechangedim{beforelowchoralsignspace}{0.04556 cm plus 0.00638 cm minus 0.00638 cm}{scalable}%
% when bolshifts are enabled, minimal space between a clef at the beginning of the line and a leading alteration glyph (should be larger than clefflatspace so that a flatted clef can be distinguished from a flat which is part of the first glyph on a line, but also smaller than spaceafterlineclef, the distance from the clef to the first notes)
\grechangedim{beforealterationspace}{0.1 cm}{scalable}%
% space between elements
\grechangedim{interelementspace}{0.06927 cm plus 0.00182 cm minus 0.00363 cm}{scalable}%
% larger space between elements
\grechangedim{largerspace}{0.10938 cm plus 0.01822 cm minus 0.00911 cm}{scalable}%
% space between elements in ancient notation
\grechangedim{nabcinterelementspace}{0.06927 cm plus 0.00182 cm minus 0.00363 cm}{scalable}%
% larger space between elements in ancient notation
\grechangedim{nabclargerspace}{0.10938 cm plus 0.01822 cm minus 0.00911 cm}{scalable}%
% space between elements which has the size of a note
\grechangedim{glyphspace}{0.21877 cm plus 0.01822 cm minus 0.01822 cm}{scalable}%
% space before custos
\grechangedim{spacebeforecustos}{0.1823 cm plus 0.31903 cm minus 0.0638 cm}{scalable}%
% space before punctum mora and augmentum duplex
\grechangedim{spacebeforesigns}{0.05469 cm plus 0.00455 cm minus 0.00455 cm}{scalable}%
% space after punctum mora and augmentum duplex
\grechangedim{spaceaftersigns}{0.08203 cm plus 0.0082 cm minus 0.0082 cm}{scalable}%
% space after a clef at the beginning of a line
\grechangedim{spaceafterlineclef}{0.27345 cm plus 0.14584 cm minus 0.01367 cm}{scalable}%
% minimal space between notes of different words
%\grechangedim{interwordspacenotes}{0.27 cm plus 0.15 cm minus 0.05 cm}{scalable}%
\grechangedim{interwordspacenotes}{0.27 cm plus 0.08 cm minus 0.05 cm}{scalable}%
% minimal space between notes of the same syllable.
% Warning: always keep minus to 0; also keep plus very low, or some words won't be hyphenated
%\grechangedim{intersyllablespacenotes}{0.24 cm plus 0.04cm minus 0cm}{scalable}%
\grechangedim{intersyllablespacenotes}{0.24 cm plus 0.04cm minus 0cm}{scalable}%
% minimal space between letters of different words. Makes sense to have
% the same plus and minus as interwordspacenotes.
%\grechangedim{interwordspacetext}{0.38 cm plus 0.15 cm minus 0.05 cm}{scalable}%
\grechangedim{interwordspacetext}{0.18 cm plus 0.08 cm minus 0.05 cm}{scalable}%
% Versions of interword spaces for euouae blocks
%\grechangedim{interwordspacenotes@euouae}{0.19 cm plus 0.1 cm minus 0.05 cm}{scalable}%
\grechangedim{interwordspacenotes@euouae}{0.13 cm plus 0.1 cm minus 0.05 cm}{1}%
%\grechangedim{interwordspacetext@euouae}{0.27 cm plus 0.1 cm minus 0.05 cm}{scalable}%
\grechangedim{interwordspacetext@euouae}{0.13 cm plus 0.1 cm minus 0.05 cm}{1}%
% space between notes of a bivirga or trivirga
\grechangedim{bitrivirspace}{0.06927 cm plus 0.00182 cm minus 0.00546 cm}{scalable}%
% space between notes of a bistropha or tristrophae
\grechangedim{bitristrospace}{0.06927 cm plus 0.00182 cm minus 0.00546 cm}{scalable}%
% space between two punctum inclinatum
\grechangedim{punctuminclinatumshift}{-0.03918 cm plus 0.0009 cm minus 0.0009 cm}{scalable}%
% space before puncta inclinata
\grechangedim{beforepunctainclinatashift}{0.05286 cm plus 0.00728 cm minus 0.00455 cm}{scalable}%
% space between a punctum inclinatum and a punctum inclinatum deminutus
\grechangedim{punctuminclinatumanddebilisshift}{-0.02278 cm plus 0.0009 cm minus 0.0009 cm}{scalable}%
% space between two punctum inclinatum deminutus
\grechangedim{punctuminclinatumdebilisshift}{-0.00728 cm plus 0.0009 cm minus 0.0009 cm}{scalable}%
% space between puncta inclinata, larger ambitus (range=3rd)
\grechangedim{punctuminclinatumbigshift}{0.07565 cm plus 0.0009 cm minus 0.0009 cm}{scalable}%
% space between puncta inclinata, larger ambitus (range=4th -or more?-)
\grechangedim{punctuminclinatummaxshift}{0.17865 cm plus 0.0009 cm minus 0.0009 cm}{scalable}%
% space for the bars (inside syllables)
%first for virgula and divisio minima
\grechangedim{spacearoundsmallbar}{0.1823 cm plus 0.22787 cm minus 0.00469 cm}{scalable}%
%then divisio minor
\grechangedim{spacearoundminor}{0.1823 cm plus 0.22787 cm minus 0.00469 cm}{scalable}%
%divisio major
\grechangedim{spacearoundmaior}{0.1823 cm plus 0.22787 cm minus 0.00469 cm}{scalable}%
%divisio finalis
\grechangedim{spacearoundfinalis}{0.1823 cm plus 0.22787 cm minus 0.00469 cm}{scalable}%
%a special space for finalis, for when it is the last glyph
\grechangedim{spacebeforefinalfinalis}{0.29169 cm plus 0.07292 cm minus 0.27345 cm}{scalable}%
% additional space that will appear around bars that are preceded by a custos and followed by a key.
\grechangedim{spacearoundclefbars}{0.03645 cm plus 0.00455 cm minus 0.0009 cm}{scalable}%
% space between the text and the text of the bar
\grechangedim{textbartextspace}{0.24611 cm plus 0.13672 cm minus 0.04921 cm}{scalable}%
% minimal space between a note and a bar
\grechangedim{notebarspace}{0.31903 cm plus 0.27345 cm minus 0.02824 cm}{scalable}%
% maximal space between two syllables for which we consider a dash is not needed
\grechangedim{maximumspacewithoutdash}{0.00 cm}{scalable}%
% an extensible space for the beginning of lines
\grechangedim{afterclefnospace}{0 cm plus 0.27345 cm minus 0 cm}{scalable}%
% space between the initial and the beginning of the score
\grechangedim{afterinitialshift}{0.2457 cm}{scalable}%
% space before the initial
\grechangedim{beforeinitialshift}{0.2457 cm}{scalable}%
% when bolshifts are enabled, minimum space between beginning of line and first syllable text
\grechangedim{minimalspaceatlinebeginning}{0.05 cm}{scalable}%
% space to force the initial width to.  Ignored when 0.
\grechangedim{manualinitialwidth}{0 cm}{scalable}%
% distance to move the initial up by
\grechangedim{initialraise}{0 cm}{scalable}%
% Space between lines in the annotation
\grechangedim{annotationseparation}{0.05cm}{scalable}%
% Amount to raise (positive) or lower (negative) the annotations from the default position (base line of top annotation aligned with top line of staff)
\grechangedim{annotationraise}{0cm}{scalable}%
% space at the beginning of the lines if there is no clef
\grechangedim{noclefspace}{0.1 cm}{scalable}%
% space around a clef change
\grechangedim{clefchangespace}{0.01768 cm plus 0.00175 cm minus 0.01768 cm}{scalable}%
%When \gre@clivisalignment is 2, this distance is the maximum length of the consonants after vowels for which the clivis will be aligned on its center.
\grechangedim{clivisalignmentmin}{0.3 cm}{scalable}%



%%%%%%%%%%%%%%%%%%
% vertical spaces
%%%%%%%%%%%%%%%%%%

% first, we have two spaces for the chironomic signs
\grechangedim{abovesignsspace}{0.8 cm}{scalable}%
\grechangedim{belowsignsspace}{0 cm}{scalable}%
% the amount to shift down:
% (a) low choral signs that are not lower than the note, regardless of whether
%     it's on a line or in a space
% (b) high choral signs and low choral signs that are lower than the note which
%     are in a space
\grechangedim{choralsigndownshift}{0.00911 cm}{scalable}%
% the amount to shift up:
% (a) high choral signs and low choral signs that are lower than the note which
%     are on a line
\grechangedim{choralsignupshift}{0.04556 cm}{scalable}%
% the space for the translation
\grechangedim{translationheight}{0.5 cm}{scalable}%
%the space above the lines
\grechangedim{spaceabovelines}{0.45576 cm plus 0.36461 cm minus 0.09114 cm}{scalable}%
%the space between the lines and the bottom of the text
\grechangedim{spacelinestext}{0.60617 cm}{scalable}%
%the space beneath the text
\grechangedim{spacebeneathtext}{0 cm}{scalable}%
% height of the text above the note line
\grechangedim{abovelinestextraise}{-0.1 cm}{scalable}%
% height that is added at the top of the lines if there is text above the lines (it must be bigger than the text for it to be taken into consideration)
\grechangedim{abovelinestextheight}{0.3 cm}{scalable}%
% an additional shift you can give to the brace above the bars if you don't like it
\grechangedim{braceshift}{0 cm}{scalable}%
% a shift you can give to the accentus above the curly brace
\grechangedim{curlybraceaccentusshift}{-0.05 cm}{scalable}%


    \greseteolcustos{auto}
  }

  \setlength{\columnsep}{18pt}
  \setlength{\columnseprule}{.4pt}
  \selectlanguage{latin}
  \begin{multicols}{2}
  \begin{psalmverses}[1]
  \input{psalms/Psalm118.3-#1-verses}
  \end{psalmverses}
  \printpsalmtitle{118. IV.}
  \vspace{-0.8\baselineskip}
  \begin{psalmverses}
  \input{psalms/Psalm118.4-#1-verses}
  \end{psalmverses}
  \printpsalmtitle{118. V.}
  \vspace{-0.8\baselineskip}
  \begin{psalmverses}
  \input{psalms/Psalm118.5-#1-verses}
  \end{psalmverses}
  \end{multicols}
}
\newcommand{\printtercepsalmstranslation}{
  \setlength{\columnsep}{18pt}
  \setlength{\columnseprule}{.4pt}
  \selectlanguage{american}
  \begin{multicols}{2}
  \printpsalmtitle{118. III.}
  \vspace{-0.5\baselineskip}
  \begin{psalmverses}
  \item Set before me for a law the way of thy justifications, O Lord: * and I will always seek after it.
\item Give me understanding, and I will search thy law; * and I will keep it with my whole heart.
\item Lead me into the path of thy commandments; * for this same I have desired.
\item Incline my heart into thy testimonies * and not to covetousness.
\item Turn away my eyes that they may not behold vanity: * quicken me in thy way.
\item Establish thy word to thy servant, * in thy fear.
\item Turn away my reproach, which I have apprehended: * for thy judgments are delightful.
\item Behold I have longed after thy precepts: * quicken me in thy justice.
\item Let thy mercy also come upon me, O Lord: * thy salvation according to thy word.
\item So shall I answer them that reproach me in any thing; * that I have trusted in thy words.
\item And take not thou the word of truth utterly out of my mouth: * for in thy words, I have hoped exceedingly.
\item So shall I always keep thy law, * for ever and ever.
\item And I walked at large: * because I have sought after thy commandments.
\item And I spoke of thy testimonies before kings: * and I was not ashamed.
\item I meditated also on thy commandments, * which I loved.
\item And I lifted up my hands to thy commandments, which I loved: * and I was exercised in thy justifications.
  \end{psalmverses}
  \printpsalmtitle{118. IV.}
  \vspace{-0.5\baselineskip}
  \begin{psalmverses}
  \item Be thou mindful of thy word to thy servant, * in which thou hast given me hope.
\item This hath comforted me in my humiliation: * because thy word hath enlivened me.
\item The proud did iniquitously altogether: * but I declined not from thy law.
\item I remembered, O Lord, thy judgments of old: * and I was comforted.
\item A fainting hath taken hold of me, * because of the wicked that forsake thy law.
\item Thy justifications were the subject of my song, * in the place of my pilgrimage.
\item In the night I have remembered thy name, O Lord: * and have kept thy law.
\item This happened to me: * because I sought after thy justifications.
\item O Lord, my portion, * I have said, I would keep thy law.
\item I entreated thy face with all my heart: * have mercy on me according to thy word.
\item I have thought on my ways: * and turned my feet unto thy testimonies.
\item I am ready, and am not troubled: * that I may keep thy commandments.
\item The cords of the wicked have encompassed me: * but I have not forgotten thy law.
\item I rose at midnight to give praise to thee; * for the judgments of thy justification.
\item I am a partaker with all them that fear thee, * and that keep thy commandments.
\item The earth, O Lord, is full of thy mercy: * teach me thy justifications.
  \end{psalmverses}
  \printpsalmtitle{118. V.}
  \vspace{-0.5\baselineskip}
  \begin{psalmverses}
  \item Thou hast done well with thy servant, O Lord, * according to thy word.
\item Teach me goodness and discipline and knowledge; * for I have believed thy commandments.
\item Before I was humbled I offended; * therefore have I kept thy word.
\item Thou art good; * and in thy goodness teach me thy justifications.
\item The iniquity of the proud hath been multiplied over me: * but I will seek thy commandments with my whole heart.
\item Their heart is curdled like milk: * but I have meditated on thy law.
\item It is good for me that thou hast humbled me, * that I may learn thy justifications.
\item The law of thy mouth is good to me, * above thousands of gold and silver.
\item Thy hands have made me and formed me: * give me understanding, and I will learn thy commandments.
\item They that fear thee shall see me, and shall be glad: * because I have greatly hoped in thy words.
\item I know, O Lord, that thy judgments are equity: * and in thy truth thou hast humbled me.
\item O! let thy mercy be for my comfort, * according to thy word unto thy servant.
\item Let thy tender mercies come unto me, and I shall live: * for thy law is my meditation.
\item Let the proud be ashamed, because they have done unjustly towards me: * but I will be employed in thy commandments.
\item Let them that fear thee turn to me: * and they that know thy testimonies.
\item Let my heart be undefiled in thy justifications, * that I may not be confounded.
  \end{psalmverses}
  \end{multicols}
}

{
  \label{advent}
  \chapter{Sunday at Terce}
  \section{Sunday at Terce throughout the year}

  \printgabc{\Vbar}{}{D}{deus-in-adjutorium}

  \bigskip

  {\centering Hymn.\par}

  {\centering 1.~On Ordinary Sundays.\par}

  \printgabc{2.}{}{N}{hy--nunc_sancte_nobis_(in_dominicis_per_annum)--solesmes_1961}

  \oldneedspace{10\baselineskip}
  {\centering 2.~On Solemn Feasts.\par}

  \printgabc{8.}{}{N}{hy--nunc_sancte_nobis_(in_festis)--solesmes_1961}

  \begin{multicols}{2}
  \begin{psalmverses}
\item Come Holy Ghost, who ever One
Art with the Father and the Son,
It is the hour, our souls possess
With thy full flood of holiness.

\item Let flesh and heart and lips and mind
Sound forth our witness to mankind;
And love light up our mortal frame,
Till others catch the living flame.

\item Almighty Father, hear our cry,
Through Jesus Christ, our Lord most High,
Who, with the Holy Ghost and thee,
Doth live and reign eternally.
Amen.
  \end{psalmverses}
  \end{multicols}

\bigskip



% print antiphon
%    \printgabc[\preant]{\antlineone}{\antlinetwo}{\antinitial}{\anttex}
  \bigskip
  \printgabc{Ant.}{2. D}{A}{an--alleluia._(sund._at_terce)--solesmes}
% print translation
  \translation[]{Alleluia, Lead me into the path of thy commandments, alleluia, alleluia.}

\bigskip
  \printtercepsalms{2D}

  \bigskip
  \printtercepsalmstranslation{}

% \printpsalm{118.3}{2D}{1}
% \printpsalmtitle{118. IV.}
% \printpsalm{118.4}{2D}{0}
% \printpsalmtitle{118. V.}
% \printpsalm{118.5}{2D}{0}

% print antiphon
%    \printgabc[\preant]{\antlineone}{\antlinetwo}{\antinitial}{\anttex}
  \printgabc{Ant.}{2. D}{A}{an--alleluia._(sund._at_terce)--solesmes}
% print translation
  \translation[]{Alleluia, Lead me into the path of thy commandments, alleluia, alleluia.}

\bigskip
\printchapternew{1. John 4.}{Deus cáritas}{est~:~\dag{} et qui manet in caritáte, in Deo manet,~* et Deus in eo.}{God is charity: and he that abideth in charity, abideth in God, and God in him.}

\bigskip
\printgabc{Short}{Resp.}{I}{re--inclina_cor_meum--solesmes}

\translation[]{\Vbar{}~Incline my heart into thy testimonies.
\Rbar{}~Incline\dots{}
\Vbar{}~Turn away my eyes that they may not behold vanity: quicken me in thy way.
\Rbar{}~Into thy testimonies.
\Vbar{}~Glory be to the Father, and to the Son, and to the Holy Ghost.
\Vbar{}~Incline\dots{}}

\bigskip
\gresetinitiallines{0}
\gregorioscore{vr-ego-dixi}
\newlength{\myhwidth}
\settowidth{\myhwidth}{tib}
\begin{nstabbing}
\>\Rbar{}~Sana ánimam meam, quia peccávi \>\hspace{-\myhwidth}tibi.\\
\end{nstabbing}

\translation[]{\Vbar{}~I said: O Lord, be thou merciful to me.\\
\Rbar{}~Heal my soul, for I have sinned against thee.}

\bigskip
\def\beginvrcols{\begin{parcolumns}[rulebetween,colwidths={1=0.48\linewidth}]{2}}
\def\vlatin{Dóminus vobíscum.}
\def\rlatin{Et cum spíritu tuo.}
\def\vtranslation{The Lord be with you.}
\def\rtranslation{And with thy spirit.}
{\centering Collect.\par}
\printvrwithtranslation{}

\bigskip
\emph{See proper of time.}

\bigskip
\printvrwithtranslation{}
\bigskip

\gresetinitiallines{0}
\gregorioscore{vr-benedicamus-domino}
\bigskip

\def\beginvrcols{\begin{parcolumns}[rulebetween,colwidths={1=0.44\linewidth}]{2}}
\def\vlatin{Fidélium ánimæ per misericórdiam Dei requiéscant in pace.\hfill{}Amen.}
\let\rlatin=\undefined
\def\vtranslation{May the souls of the faithful departed through the mercy of God rest in peace.\hfill{}Amen.}
\let\rtranslation=\undefined
\printvrwithtranslation{} 
}

	{
	\chapter{Proper of the Time -- Time After Pentecost}
  \def\deusinadjutorium{%
    \noindent\printnote{All from \emph{Sunday at Terce throughout the year}, p.~\pageref{deusinadjutorium}}
  }
  \def\printshortresp{}
  \def\printhymn{\medskip}
  \newcommand{\printterceafterpentecost}[1]{
    \section{\nth{#1} \ifnum#1=24{or Last }\fi Sunday after Pentecost}
    \printterce[../TimeAfterPentecost]{inc-VespersMagnificatPentecost#1}{pentecost#1}
  }
%\tracingmacros=1
  \printterceafterpentecost{2}
  \printterceafterpentecost{3}
  \printterceafterpentecost{4}
  \printterceafterpentecost{5}
  \printterceafterpentecost{6}
  \printterceafterpentecost{7}
  \printterceafterpentecost{8}
  \printterceafterpentecost{9}
  \printterceafterpentecost{10}
  \printterceafterpentecost{11}
  \printterceafterpentecost{12}
  \printterceafterpentecost{13}
  \printterceafterpentecost{14}
  \printterceafterpentecost{15}
  \printterceafterpentecost{16}
  \printterceafterpentecost{17}
  \printterceafterpentecost{18}
  \printterceafterpentecost{19}
  \printterceafterpentecost{20}
  \printterceafterpentecost{21}
  \printterceafterpentecost{22}
  \printterceafterpentecost{23}
  \printterceafterpentecost{24}
}
	%\clearpage
\chapter{Proper of the Saints}
\vspace{-0.75\baselineskip}
{
\let\oldsection=\section
\renewcommand{\section}[1]{
  \oldsection{#1}
  \vspace{-0.35\baselineskip}
}
\newcommand{\benedicamusdomino}[1][1]{
  \benedicamusdominomaster{#1}
}
\newcommand{\sundaycommemnote}{%
  If today is Sunday, \emph{Vespers of the Sunday} is commemorated with \emph{Magnificat antiphon}, \emph{\Vbar{}~Dirigátur}, and \emph{Collect}.
}
\newcommand{\sundaycommemnoteeaster}{%
  If today is Sunday, \emph{Vespers of the Sunday} is commemorated with \emph{Magnificat antiphon}, \emph{\Vbar{}~Mane nobíscum}, and \emph{Collect}.
}
\newcommand{\sundaycommemnoteeasterpentecost}{%
  If today is Sunday, \emph{Vespers of the Sunday} is commemorated with \emph{Magnificat antiphon}, \emph{\Vbar{}}, and \emph{Collect}.
}

\ifthenelse{\boolean{testrun}}{}{
%December 8: Immaculate Conception
%Second Vespers of Immaculate Conception
{
\label{immaculateconception}
\section{December 8: Immaculate Conception}
\subtitle{\nth{1} Class, White or Gold}
\subtitle{I \& II Vespers}
\smallskip

\def\preanttranslationone{\vspace{-0.35\baselineskip}}
\def\prepsalmtitleone{\vspace{-0.15\baselineskip}}
\def\prepsalmone{\vspace{-0.1\baselineskip}}
\def\prepsalmoneverses{\vspace{-0.05\baselineskip}}
\def\prerepeatantiphonone{}

\def\prepsalmtitlethree{\vspace{-0.75\baselineskip}}

\def\preanttranslationfive{\vspace{-0.35\baselineskip}}
\def\prepsalmtitlefive{\vspace{-\baselineskip}}
\def\prepsalmfive{\vspace{-0.2\baselineskip}}
\def\prepsalmfiveverses{\vspace{-0.025\baselineskip}}
\def\prerepeatantiphonfive{}
\def\premagverses{\greseteolcustos{manual}}
\def\definevesperspropersalt{\def\noeuouae{T}\newcommand{\maganttex}{MagnificatAntiphon1}
\newcommand{\magantinitial}{B}
\newcommand{\maganttranslation}{All generations shall call me blessed, because he that is mighty, hath done great things for me, alleluia.}
\def\magpsalmclef{c3}
\definemag{8}{G}
}
\def\definevesperspropers{\def\noeuouae{T}\newcommand{\maganttex}{MagnificatAntiphon2}
\newcommand{\magantinitial}{H}
\newcommand{\maganttranslation}{This day a rod came forth from the root of Jesse: this day Mary was conceived without any stain of sin: this day the head of the old serpent was crushed by her.  Alleluia.}
\definemag{1}{f}
}
\def\vesperspropersaltnote{At I Vespers:}
\def\vesperspropersnote{At II Vespers:}
\def\prehymn{\printnote{All kneel for the first stanza of the following hymn.}}
\def\hymnlabel{hymn-avemarisstella}
\def\vrlinebreak{F}
%\def\prechapter{\vspace{-\baselineskip}}
\printvespers[../December8-ImmaculateConception]{inc-ImmaculateConceptionVespers}
}

{
\bigskip

\bigskip
\noindent
\printnote{Then follows a Commemoration of the Advent Sunday or Feria according to the day of the week.  The antiphon is followed by \emph{\Vbar{} Roráte cæli}, page \pageref{vr-rorate-dec8}, and then the appropriate Collect, page \pageref{collect-dec8}.}
\bigskip
}

{
  \oldneedspace{5\baselineskip}
  \subtitle{First Week of Advent.}
  \subtitle{\small{Thursday.}}
  {
  \def\noeuouae{T}
  \printgabc{At Magn.}{\oldstylenums{Ant.~4.}}{E}{../Advent1/MagAntThursday-Exspectabo}
  }
  \translation[]{I will look for the Lord my Saviour, and await Him, while He is near, alleluia.}
  \medskip

  \oldneedspace{5\baselineskip}
  \subtitle{\small{Friday.}}
  {
  \def\noeuouae{T}
  \printgabc{At Magn.}{\oldstylenums{Ant.~4.}}{E}{../Advent1/MagAntFriday-ExAegypto}
  }
  \translation[]{Out of Egypt I have called my Son; he shall come to save His people.}
  \medskip

  \oldneedspace{5\baselineskip}
  \subtitle{Second Week of Advent.}
  \subtitle{\small{Saturday.}}
  {
  \def\noeuouae{T}
  \printgabc{At Magn.}{\oldstylenums{Ant.~7.}}{V}{../Advent1/MagAntSaturday-VeniDomine}
  }
  \translation[]{Come, Lord, to visit us in peace, that we may rejoice before Thee with a perfect heart.}
  \medskip

  \oldneedspace{5\baselineskip}
  \subtitle{\small{Sunday.}}
  {
  \def\noeuouae{T}
  \printgabc{At Magn.}{\oldstylenums{Ant.~8.~G}}{T}{../Advent2/MagnificatAntiphon-noEuouae}
  }
  \translation[]{Art Thou He that art to come, or look we for another? Relate to John what you have seen: The blind recover their sight, the dead rise again, the poor have the Gospel preached to them, alleluia.}
  \medskip

  \oldneedspace{5\baselineskip}
  \subtitle{\small{Monday.}}
  {
  \def\noeuouae{T}
  \printgabc{At Magn.}{\oldstylenums{Ant.~4.}}{E}{../Advent2/MagAntMonday-EcceRexVeniet}
  }
  \translation[]{Behold, the King shall come, the Lord of the land; and He shall take away the yoke of our captivity.}
  \medskip

  \oldneedspace{5\baselineskip}
  \subtitle{\small{Tuesday.}}
  {
  \def\noeuouae{T}
  \printgabc{At Magn.}{\oldstylenums{Ant.~5.}}{V}{../Advent2/MagAntTuesday-VoxClamantis}
  }
  \translation[]{A voice of one crying in the desert, Prepare ye the way of the Lord, make straight His paths}
  \medskip

  \oldneedspace{5\baselineskip}
  \subtitle{\small{Wednesday.}}
  {
  \def\noeuouae{T}
  \printgabc{At Magn.}{\oldstylenums{Ant.~4.}}{S}{../Advent2/MagAntWednesday-Sion}
  }
  \translation[]{Sion, thou shalt be restored, and shalt see the Just One who shall appear in thee.}
  \medskip

  \oldneedspace{5\baselineskip}
  \subtitle{\small{Thursday.}}
  {
  \def\noeuouae{T}
  \printgabc{At Magn.}{\oldstylenums{Ant.~4.}}{Q}{../Advent2/MagAntThursday-QuiPostMeVenit}
  }
  \translation[]{He that shall come after me is preferred before me; whose shoes I am not worthy to loose.}
  \medskip
  \hrule
  \medskip
  {
      \label{vr-rorate-dec8}
      \newcommand{\commvlatin}{Roráte cæli désuper, et nubes pluant \textbf{ju}stum.}
      \newcommand{\commrlatin}{Aperiátur terra, et gérminet Salva\textbf{tó}rem.}
      \newcommand{\commvtranslation}{Ye heavens, drop down dew from above, and let the clouds rain down the Just One.}
      \newcommand{\commrtranslation}{Let the earth open and bud forth the Saviour.}
  \printvrcommem{}
  }

  \oldneedspace{3\baselineskip}
  \label{collect-dec8}
  \begin{center}{\large Collect.}\end{center}
  \vspace{-1.5\baselineskip}
  \def\printcollectheading{F}
  {
  \begin{center}{First Week of Advent.}\end{center}
  \def\gabcfolder{../Advent1}
  \newcommand{\antonetex}{Ant1-InIllaDie}
\newcommand{\antoneinitial}{I}
\newcommand{\antonetranslation}{In that day the mountains shall drop down sweetness, and the hills shall flow with milk and honey, alleluia.}
\definepsalm{1}{109}{8}{G}

\newcommand{\anttwotex}{Ant2-Jucundare}
\newcommand{\anttwoinitial}{J}
\newcommand{\anttwotranslation}{Shout for joy, O daughter of Sion, rejoice greatly, O daughter of Jerusalem, alleluia.}
\definepsalm{2}{110}{8}{G*}

\newcommand{\antthreetex}{Ant3-EcceDominusVeniet}
\newcommand{\antthreeinitial}{E}
\newcommand{\antthreetranslation}{Behold, the Lord shall come, and all His Saints with Him: and there shall be in that day a great light, alleluia.}
\definepsalm{3}{111}{5}{a}

\newcommand{\antfourtex}{Ant4-Omnes}
\newcommand{\antfourinitial}{O}
\newcommand{\antfourtranslation}{All ye that thirst come to the waters: seek the Lord while He can be found, alleluia.}
\definepsalm{4}{112}{7}{c}

\newcommand{\antfivetex}{Ant5-EcceVeniet}
\newcommand{\antfiveinitial}{E}
\newcommand{\antfivetranslation}{Behold there shall come a great Prophet, and He shall renew Jerusalem, alleluia.\vspace{1ex}}
\definepsalm{5}{113}{4}{A*}

\newcommand{\chaptertext}{\dropcap{latin}{Fratres~: Hora est jam nos de somno} \textbf{súr}\-ge\-re~:~\gredagger{} nunc enim própior est \emph{no\-stra} \textbf{sá}\-lus,~* quam cum cre\-\textbf{dí}\-dimus.}
\newcommand{\chaptertranslation}{Brethren: it is now the hour for us to rise from sleep.  For now our salvation is nearer than when we believed.}

\newcommand{\magantinitial}{N}
\newcommand{\maganttex}{MagnificatAntiphon}
\newcommand{\maganttranslation}{Fear not, Mary, for thou hast found grace with the Lord: behold thou shalt conceive and bring forth a son, alleluia.}
\def\magsolemn{F}
\definemag{8}{G}

\newcommand{\collect}{Excita, quaésumus Dómine, poténtiam tuam, et veni~:~† ut ab imminéntibus peccatórum nostrórum perículis, te mereámur protegénte éripi,~* te liberánte salvári.  Qui vivis et regnas cum Deo Patre in unitáte Spíritus Sancti Deus~:~* per ómnia saécula sæculórum.}
\newcommand{\collecttranslation}{Stir up Thy power, we beseech Thee, O Lord, and come: that from the threatening dangers of our sins we may deserve to be rescued by Thy protection, and to be saved by Thy deliverance: Who livest and reignest with God the Father in the unity of the Holy Ghost, world without end.}

  \printcollect{\collect}{\collecttranslation}
  }
  {
  \begin{center}{Second Week of Advent.}\end{center}
  \def\gabcfolder{../Advent2}
  % !TEX TS-program = lualatex
% !TEX encoding = UTF-8

% This is a simple template for a LuaLaTeX document using gregorio scores.

\newcommand{\comheadingtext}{Commemoration of 2nd Sunday of Advent}

\newcommand{\latincomcollect}{Excita Dómine corda nostra ad præparándas Unigéniti tui vias~:~† ut per ejus advéntum,~* purificátis tibi méntibus servíre mereámur. Qui tecum.}
\newcommand{\englishcomcollect}{Stir up our hearts, O Lord, to prepare the ways of Thine only-begotten Son: that through His coming we may deserve to serve Thee with purified minds: Who with Thee.}

\newcommand{\englishcommagantiphon}{Art Thou He that art to come, or look we for another? Relate to John what you have seen: The blind recover their sight, the dead rise again, the poor have the Gospel preached to them, alleluia.}

\newcommand{\commagantlinetwo}{Ant. 8. G}
\newcommand{\commaganttex}{MagnificatAntiphon-noEuouae}
\newcommand{\commagantinitial}{T}
\newcommand{\commagantinitialsize}{35}

\newcommand{\commvrtex}{../Advent/vr-commemoration}
\newcommand{\commvtranslation}{Ye heavens, drop down dew from above, and let the clouds rain down the Just One.}
\newcommand{\commrtranslation}{Let the earth open and bud forth the Saviour.}

\grechangestaffsize{17}
  \printcollect{\latincomcollect}{\englishcomcollect}
  }

  \bigskip
  \benedicamusdomino{}
}


\ifthenelse{\boolean{includejan29}}{
	%January 29, St Francis de Sales
	{
\section{January 29: St Francis de Sales}
\subtitle{\nth{1} Class (patron of the Archdiocese of Cincinnati), White or Gold}
\subtitle{From Common of Confessor Bishop \& Common of Doctors}
\subtitle{I \& II Vespers}
\medskip

\def\deusinadjutoriumsolemn{T}
\def\definevesperspropers{\definepsalm{5}{131}{3}{g}

\newcommand{\vrtex}{vrJustumDeduxit}
\newcommand{\vtranslation}{The Lord led the just by right ways.}
\newcommand{\rtranslation}{And showed him the kingdom of God.}

\def\prepsalmtitlefive{\vspace{-\baselineskip}}
\def\prerepeatantiphonfive{}
}
\def\definevesperspropersalt{\definepsalm{5}{116}{3}{g}

\newcommand{\vrtex}{vrAmavitEum}
\newcommand{\vtranslation}{The Lord lloved him and adorned him.}
\newcommand{\rtranslation}{He clothed him with a robe of glory.}

\def\prepsalmtitlefive{\vspace{-\baselineskip}}
}
\def\prepsalmtitleone{\vspace{-0.5\baselineskip}}
\def\prepsalmtitletwo{\vspace{-0.5\baselineskip}}
\def\vesperspropersnote{At II Vespers:}
\def\vesperspropersaltnote{At I Vespers:}
%\def\premagverses{\vspace{-0.1\baselineskip}}
\def\prevespers{%
  %\let\oldthing=\maganttranslation
  %\def\maganttranslation{\vspace{-0.25\baselineskip}\oldthing\vspace{-0.25\baselineskip}}
  %\def\premagtexverses{\smallskip}
}
\def\premagtitle{\oldneedspace{8\baselineskip}}
%\def\premaganttranslation{\vspace{-2ex}}
%\def\precollect{\pagebreak}%\vspace{-0.5\baselineskip}}
\printvespers[../CommonOfConfessorBishop]{inc-StFrancisDeSales}
\medskip
\printnote{\sundaycommemnote{}

\medskip{}

\emph{\nth{4} Sunday after Epiphany}, p.~\pageref{epiphany4}.

\emph{Septuagesima}, p.~\pageref{septuagesima}.

\emph{Sexagesima}, p.~\pageref{sexagesima}.
}

\medskip{}
\benedicamusdomino{}
}
}{}

%Purification & Presentation (2nd class no commem of Sunday because it is a feast of the Lord)
{
\section{February 2: Purification \& Presentation}
\subtitle{\nth{2} Class, White}
\subtitle{I \& II Vespers}
\medskip
\printnote{At I Vespers, Psalms and Antiphons of the Circumcision, p.~\pageref{circumcision}, continuing with the chapter on p.~\pageref{purification-chapter}.}

\def\definevesperspropers{\newcommand{\maganttex}{an--hodie_beata_virgo--solesmes}
\newcommand{\magantinitial}{H}
\newcommand{\maganttranslation}{Today did the Blessed Virgin Mary present the Child Jesus in the temple; and Simeon, filled with the Holy Ghost, took Him up in his arms, and blessed God for ever and ever.\vspace{-2ex}}
\definemag{8}{G*}
}
\def\definevesperspropersalt{\newcommand{\maganttex}{an--senex_puerum--solesmes}
\newcommand{\magantinitial}{S}
\newcommand{\maganttranslation}{This day did the Blessed Virgin Mary present the Child Jesus in the temple; and Simeon, filled with the Holy Ghost, took Him up in his arms, and blessed God for ever and ever.}
\definemag{1}{D}
}
%\def\prepsalmtitletwo{\needspace{12\baselineskip}}
\def\vesperspropersnote{At II Vespers:}
\def\vesperspropersaltnote{At I Vespers:}
\def\prepsalmtitlefour{\vspace{-0.5\baselineskip}}
\def\prerepeatantiphonfour{}
\def\prepsalmfourverses{}
\def\preantfive{\vspace{-0.5\baselineskip}}
\def\prechapter{\vspace{-\baselineskip}\label{purification-chapter}}
\def\vrlabel{vr-feb2}
\def\printhymnnote{
  {
    \oldneedspace{3\baselineskip}
    \printnote{Hymn.~\emph{Ave Maris Stella}, p.~\pageref{hymn-avemarisstella}.\\}
    %
    % \def\vrlinebreak{T}
    % \oldneedspace{3\baselineskip}
    % \printvr[\greseteolcustos{manual}]{\vrtex}{\vtranslation}{\rtranslation}
  }
}
\printvespers[../February2-PurificationOfBlessedVirginMary]{inc-purification}
\bigskip{}
\benedicamusdomino[2]{}
}

%March 25: Annunciation 
{
\def\vrlinebreak{F}
\section{March 25: Annunciation of the Blessed Virgin Mary}
\subtitle{\nth{1} Class}
\subtitle{I \& II Vespers}
\medskip

\def\deusinadjutoriumsolemn{T}
\def\definevesperspropers{\newcommand{\maganttex}{an--gabriel_angelus_ave--solesmes}
\newcommand{\magantinitial}{G}
\newcommand{\maganttranslation}{The Angel Gabriel spoke to Mary, saying: Hail, full of grace, the Lord is with thee; blessed art thou amongst women.}
\def\magsolemn{T}
\def\psalmclef{c3}
\definemag{7}{d}

  \def\prepsalmfive{\greseteolcustos{manual}}
}
\def\definevesperspropersalt{\newcommand{\maganttex}{an--spiritus_sanctus--solesmes}
\newcommand{\magantinitial}{S}
\newcommand{\maganttranslation}{The Holy Ghost shall come down upon thee, Mary, and the power of the Most High shall overshadow thee.}
\def\magsolemn{T}
\definemag{8}{G}
}
\def\vesperspropersnote{At II Vespers:}
\def\vesperspropersaltnote{At I Vespers:}
%\def\premag{\def\noeuouae{T}}
\ifthenelse{\boolean{birmingham}}{
  \def\preantthree{\vspace{-0.75\baselineskip}}
  \def\prepsalmtitlethree{\vspace{-0.25\baselineskip}}
  \def\preantfive{\vspace{-0.75\baselineskip}}
  \def\prepsalmtitlefive{\vspace{-0.75\baselineskip}}
  \def\premagtitle{\oldneedspace{2\baselineskip}}
  \def\precollect{\vspace{-0.5\baselineskip}}
}{
  \def\prepsalmtitlefive{\vspace{-0.5\baselineskip}}
}
\def\prepsalmtitletwo{\vspace{-0.5\baselineskip}}
\def\premagverses{\greseteolcustos{manual}}
\def\printhymnnote{
  {
    \oldneedspace{3\baselineskip}
    \printnote{Hymn.~\emph{Ave Maris Stella}, p.~\pageref{hymn-avemarisstella}.\\}

    \printnote{At II Vespers in Paschal time, add Alleluia to the versicle and response.\\}
    %
    % \def\vrlinebreak{T}
    % \oldneedspace{3\baselineskip}
    % \printvr[\greseteolcustos{manual}]{\vrtex}{\vtranslation}{\rtranslation}
  }
}
\def\beginchaptercols{\begin{parcolumns}[rulebetween,colwidths={1=0.44\linewidth}]{2}}
\def\begincollectcols{\begin{parcolumns}[rulebetween,colwidths={1=0.43\linewidth}]{2}}
%\def\prepsalmthreeverses{\pagebreak}
\def\prevespers{
  %\let\oldthing=\anttwotranslation  
  %\def\anttwotranslation{\oldthing\vspace{-\baselineskip}}
  %\let\oldthingb=\antfourtranslation  
  %\def\antfourtranslation{\oldthing\pagebreak}
}
\def\vrlabel{vr-march25}

\printvespers[../March25-Annunciation]{inc-Annunciation}
}

{
\bigskip
\noindent
\printnote{Then follows a Commemoration of the Lenten Feria.  Finally, \benedicamusdominoreference{1}}
\bigskip
}


% Easter can be as early as March 22 or as late as April 25, so we basically need to include all the lenten ferias
% I think if it falls on or after Palm Sunday, it is translated to Monday after Low Sunday
% so I should only need to go from Wednesday of the second week unless I want to add St Joseph
{
\newcommand{\printcommem}[8][../Lent]{
% #1 Folder (../Lent)
% #2 name of the day of the week (Thursday)
% #3 Tone (2)
% #4 First letter of antiphon (Q)
% #5 gabc (an--qui_me_sanum_fecit--solesmes)
% #6 ant translation (He who healed me commanded me: Take up thy mat and walk in peace.)
% #7 collect
% #8 collect translation
  \oldneedspace{4\baselineskip}
  \sectionmark{#2 in \beforecommemweek{}\commemweek{}.}
  \subtitle{\small{#2.}}
  \smallskip
  {
  \def\noeuouae{T}
  \printgabc{At Magn.}{\oldstylenums{Ant.~#3.}}{#4}{#1/#5}
  }
  \translation[]{#6}
  \smallskip
  \ifx\vrtitle\undefined%
  {
    \label{\vrlabel}
    \printvrcommem{}
  }\else%
  \printnote{\emph{\Vbar{}~\vrtitle},
  \ifnum\getpagerefnumber{\vrlabel}=\thepage
  below%
  \else
  page \pageref{\vrlabel}%
  \fi.}
  \fi
  \ifx\precollect\undefined\else\precollect\fi
  \printcollect{#7}{#8}
  \bigskip
  \hrule
  \bigskip
}
\newcommand{\printweektitle}[2][the ]{
  \edef\commemweek{#2}%
  \edef\beforecommemweek{#1}%
  \subtitle{#2.}%
}
{
  \newcommand{\vrtitle}{Roráte cæli}
  \newcommand{\vrlabel}{commvrrorate}
  \newcommand{\commvlatin}{Roráte cæli désuper, et nubes pluant \textbf{ju}stum.}
  \newcommand{\commrlatin}{Aperiátur terra, et gérminet Salva\textbf{tó}rem.}
  \newcommand{\commvtranslation}{Ye heavens, drop down dew from above, and let the clouds rain down the Just One.}
  \newcommand{\commrtranslation}{Let the earth open and bud forth the Saviour.}
  % \oldneedspace{5\baselineskip}
  % \subtitle{First Week of Lent.}\vspace{-0.5\baselineskip}
  % \printcommem{Thursday}{4}{O}{an--o_mulier--solesmes}{O woman, great is thy faith: let it be unto thee as thou hast asked.}
  % {Da, quǽsumus Dómine, pópulis christiánis, et quæ profiténtur agnóscere~:~* et cæléste munus dilígere, quod frequéntant. Per Dóminum.}{Grant, Lord, we beseech Thee, unto all Christian people, that what they now believe they may one day know and see in love unchecked, that heavenly gift whereof now they are the worshippers and the partakers.}

  % \printcommem{Friday}{1}{Q}{an--qui_me_sanum_fecit--solesmes}{He who healed me commanded me: Take up thy bed and walk in peace.}
  % {Exáudi nos miséricors Deus~:~* et méntibus nostris grátiæ tuæ lumen osténde. Per Dóminum nostrum.}
  % {Hear us, O merciful God, and cause the bright beams of thy grace to shine upon our souls.}
  
  \oldneedspace{5\baselineskip}
  \printweektitle{Second Week of Lent}
  % \printcommem{Monday}{1}{Q}{an--qui_me_misit--solesmes}
  % {He who sent Me is with Me and hath not left Me alone: for I always do those things that please Him.}
  % {Adésto supplicatiónibus nostris omnípotesn Deus~:~* et quibus fidúciam sperándæ pietátis indúlges, consuétæ misericórdiæ tríbue benígnus efféctum. Per Dóminum.}
  % {Graciously hear our prayers, O Almighty God, and as Thou dost give us to look with confidence for Thy favour for which we hope, so grant us, in Thy goodness, the manifestation of Thine accustomed mercy.}

  % \printcommem{Tuesday}{4}{O}{an--omnes_autem_vos--solesmes}
  % {And all ye are brethren, and call no man your father upon earth for One is your Father, Which is in heaven neither be ye called masters, for One is your Master, even Christ.}
  % {Propitiáre Dómine supplicatiónibus nostris, et animárum nostrárum medére languóribus~:~* ut remissióne percépta, in tua semper benedictióne lætémur. Per Dóminum.}
  % {Tend thy merciful ears, O Lord, we beseech thee, unto our supplications, and heal the sickness of our souls, that we, receiving thy pardon, may rejoice forever in thy blessing.}
  
  \printcommem{Wednesday}{1}{T}{an--tradetur_enim_gentibus--solesmes}
  {He will be handed over to the gentiles to be mocked and scourged and crucified.}
  {Deus innocéntiæ restitútor et amátor, dírige ad te tuórum corda servórum~:~* ut spíritus tui fervóre concépto, et in fide inveniántur stábiles, et in ópere efficáces. Per Dóminum.}
  {God, the Renewer and Lover of innocence, turn the hearts of Thy servants to Thyself, so that they, enkindled with the fire of Thy Spirit, may be found ever rooted in faith, and fruitful in works.}

  %\newcommand{\vrtitle}{Roráte cæli}
  \printcommem{Thursday}{7}{D}{an--dives_ille_guttam--solesmes}
  {That rich man who had refused Lazarus bread-crumbs, cried for a drop of water.}
  {Adésto Dómine fámulis tuis, et perpétuam benignitátem largíre poscéntibus~:~* ut iis, qui te auctóre et gubernatóre gloriántur, et congregáta restáures, et restauráta consérves. Per Dóminum.}
  {Lord, be present to Thy servants, and grant unto those asking an abiding mercy; that as they boast in Thee, their creator and governer, so Thou wilt renew in them the gifts bestowed, and preserve what Thou hast renewed.}

  \printcommem{Friday}{3}{Q}{an--quaerentes_eum_tenere--solesmes}
  {Seeking to lay hands on Him, they feared the multitude, because they took Him for a Prophet.}
  {Da, quǽsumus Dómine, pópulo tuo salútem mentis et córporis~:~* ut bonis opéribus inhæréndo, tuæ semper virtútis mereátur protectióne deféndi. Per Dóminum.}
  {Grant unto Thy people, O Lord, we beseech Thee, soundness of mind and body, that they, cleaving unto good works, may evermore worthily be defended by the protection of Thy might.}

  \oldneedspace{5\baselineskip}
  \printweektitle{Third Week of Lent}
  \printcommem{Monday}{1}{J}{an--jesus_autem_transiens--solesmes}
  {But Jesus passing through their midst, went His way.}
  {Subvéniat nobis Dómine misericórdia tua~:~* ut ab imminéntibus peccatórum nostrórum perículis te mereámur protegénte éripi, te liberánte salvári. Per Dóminum.}
  {Let our help, O Lord, be in Thy mercy, that we over whom Thy wrath doth most justly hang because of our sins, may in all dangers worthily be shielded by Thy protection and delivered by Thy salvation.}

  %\let\vrtitle=\undefined
  \printcommem{Tuesday}{4}{U}{an--ubi_duo_vel_tres--solesmes}
  {Where two or three are gathered in my name, I am in their midst, saith the Lord.}
  {Tua nos Dómine protectióne defénde~:~* et ab omni semper iniquitáte costódi. Per Dóminum.}
  {O Lord, shield us by Thy protection, and keep us ever from all iniquity.}

  %\newcommand{\vrtitle}{Roráte cæli}
  \printcommem{Wednesday}{7}{N}{an--non_lotis_manibus--solesmes}
  {To eat with unwashen hands, defileth not a man.}
  {Concéde, quǽsumus omnípotens Deus~: ut qui protectiónis tuæ grátiam quǽrimus,~* liberáti a malis ómnibus, secúra tibi mente serviámus. Per Dóminum.}
  {Grant, we beseech Thee, Almighty God, that we who seek the proctection of Thy grace, freed from all evils, may serve Thee in peace and quietness of spirit.}

  %\let\vrtitle=\undefined
  \printcommem{Thursday}{1}{O}{an--omnes_qui_habebant--solesmes}
  {All who had any sick brought them to Jesus, and they were healed.}
  {Subjéctum tibi pópulum, quǽsumus Dómine, propitiátio cæléstis amplíficet~:~* et tuis semper fáciat servíre mandátis. Per Dóminum.}
  {Lord, we beseech Thee that Thine heavenly Peace-Offering may so effectually work for all Thy people, bowing down before Thee, that they may ever continue to keep Thy commandments.}

  %\newcommand{\vrtitle}{Roráte cæli}
  \printcommem{Friday}{3}{D}{an--domine_ut_video--solesmes}
  {Lord, I see that Thou art a prophet: our fathers worshipped on this mountain.}
  {Præsta, quǽsumus omnípotens Deus~:~\dag{} ut qui in tua protectióne confídimus,~* cuncta nobis adversántia te adjuvánte vincámus. Per Dóminum.}
  {Grant, we beseech Thee, Almighty God, that we who trust is Thy protection, may, with Thy help, overcome all evils that rise up against us.}

  \oldneedspace{5\baselineskip}
  \printweektitle{Fourth Week of Lent}
  \printcommem{Monday}{5}{S}{an--solvite_templum_hoc--solesmes}
  {Thus saith the Lord: Destroy this temple, and in three days I will rebuild it. But He was speaking of the temple of His Body.}
  {Deprecatiónem nostram, quǽ\-su\-mus Dómine, benígnus exáudi~:~* et quibus supplicándi præstas afféctum, tríbue defensiónis auxílium. Per Dóminum.}
  {Lord, we beseech Thee, graciously hear our supplication, and evermore help and defend all those to whom Thou hast given the mind to pray.}

  \printcommem{Tuesday}{1}{N}{an--nemo_in_eum_misit--solesmes}
  {No man laid hands on Him; because His hour was not yet come.}
  {Miserére Dómine pópulo tuo~:~* et contínuis tribulatiónibus laborántem, propítius respiráre concéde. Per Dóminum.}
  {Lord, have mercy upon Thy people, and be graciously pleased to grant relief unto the same, who are ever toiling amid the storms of diverse tribulations.}

  %\newcommand{\vrtitle}{Roráte cæli}
  \printcommem{Wednesday}{1}{I}{an--ille_homo--solesmes}
  {The man that is called Jesus, made clay of spittle, and anointed mine eyes, and now I see.}
  {Páteant aures misericórdiæ tuæ Dómine précibus supplicántium~:~* et ut peténtibus desideráta concédas, fac eos quæ tibi sunt plácita postuláre. Per Dóminum.}
  {Let Thy merciful ears, Lord, be open unto the prayers of those entreating Thee, and that Thou mayest grant what we ask, teach us ever to ask what is pleasing to Thee.}

  \printcommem{Thursday}{4}{P}{an--propheta_magnus--solesmes}
  {A great prophet is risen up among us, and God hath visited His people.}
  {Pópuli tui Deus institútor et rector, peccáta quibus impugnátur, expélle~:~* ut semper tibi plácitus, et tuo munímine sit secúrus. Per Dóminum.}
  {O God, Teacher and Shepherd of Thy people, free the same from all sins assailing them, that they may be ever pleasing in Thy sight and safe under Thy shelter.}

  \let\vrtitle=\undefined
  \printcommem{Friday}{1}{D}{an--domine_si_hic_fuisses--solesmes}
  {Lord, if Thou hadst been here, Lazarus had not died; behold already he stinketh, for he hath lain in the grave four days.}
  {Da nobis, quǽsumus omnípotens Deus~:~* ut qui infirmitátis nostræ cónscii, de tua virtúte confídimus, sub tua semper pietáte gaudeámus. Per Dóminum.}
  {Grant, we beseech Thee, Almighty God, unto us who know our weakness and who trust in Thy strength, to ever rejoice in Thy loving kindness.}
}
  
{
  \newcommand{\commvlatin}{Eripe me, Dómine, ab hómine \textbf{ma}lo.}
  \newcommand{\commrlatin}{A viro iníquo éri\textbf{pe} me.}
  \newcommand{\commvtranslation}{Deliver me, O Lord, from the wicked man.}
  \newcommand{\commrtranslation}{And save me from the evil-doer.}
  \newcommand{\vrtitle}{Eripe me}
  \newcommand{\vrlabel}{commvreripeme}
  \oldneedspace{7\baselineskip}
  \printweektitle[]{Passiontide}
  {
  \ifthenelse{\boolean{birmingham}}{
    \def\precollect{\vspace{-0.75\baselineskip}}
  }{
    \def\precollect{\vspace{-0.5\baselineskip}}
  }
  \printcommem{Monday}{4}{S}{an--si_quis_sitit--solesmes}
    {If anyone thirst, let him come and drink; and from his belly will flow living water.}
    {Da, quǽsumus Dómine, pópulo tuo salútem mentis et córporis~:~* ut bonis opéribus inhæréndo, tua semper mereátur protectióne deféndi. Per Dóminum.}
    {Grant unto Thy people, O Lord, we beseech Thee, soundness of mind and body, that they, cleaving unto good works, may evermore worthily be defended by Thy protection.}
  }
  {%\newcommand{\vrtitle}{Eripe me}
  \ifthenelse{\boolean{birmingham}}{
    \def\precollect{\vspace{-0.75\baselineskip}}
  }{}
  \printcommem{Tuesday}{1}{V}{an--vos_ascendite--solesmes}
  {Go ye up unto this Feast; I go not, for My time is not yet full come.}
  {Da nobis, quǽsumus Dómine, perseverántem in tua voluntáte famulátum~:~* ut in diébus nostris, et mérito et número pópulus tibi sérviens augeátur. Per Dóminum.}
  {Grant, we beseech Thee, Lord, to give us grace to endure to the end in doing Thy will, that in our days Thy people which serve Thee may have increase, both in merit and number.}
  }

  {
  \def\precollect{\vspace{-0.5\baselineskip}}
  \printcommem{Wednesday}{4}{M}{an--multa_bona_opera--solesmes}
    {I have wrought many good works among you: on account of which work do you want to kill me?}
    {Adésto supplicatiónibus nostris omnípotens Deus~:~* et quibus fidúciam sperándæ pietátis indúlges, consuétæ misericórdiæ tríbue benígnus efféctum. Per Dóminum.}
    {Gratiously hear our prayers, Almighty God, and as Thou dost give us to look with confidence for Thy favour for which we hope, so grant us, in Thy goodness, Thine accustomed mercy.}
  }
  %\newcommand{\vrtitle}{Eripe me}
  \printcommem{Thursday}{4}{D}{an--desiderio_desideravi--solesmes}
  {With desire I have desired to eat this Pasch with you before I suffer.}
  {Esto, quǽsumus Dómine, propítius plebi tuæ~:~* ut quæ tibi non placent respuéntes, tuórum pótius repleántur delectatiónibus mandatórum. Per Dóminum.}
  {Lord, we beseech Thee, deal mercifully with Thy people, and fill plentifully with the rich things of Thy commandments those who shun that which displeaseth Thee.}

  \let\vrtitle=\undefined
  \printcommem{Friday}{1}{P}{an--principes_consilium--solesmes}
  {The chief Priests consulted that they might kill Jesus, but they said: Not on the Feast-day, lest there be an uproar among the people.}
  {Concéde, quǽsumus omnípotens Deus~:~* ut qui protectiónis tuæ grátiam quǽrimus, liberáti a malis ómnibus, secúra tibi mente serviámus. Per Dóminum.}
  {Grant, we beseech Thee, Almighty God, that we who seek the proctection of Thy grace, freed from all evils, may serve Thee in peace and quietness of spirit.}
}
\vspace{-\baselineskip}
  % \bigskip
  % \benedicamusdomino{}
}


\ifthenelse{\boolean{includeapril23}}{
	%April 23: St George
	{
\section{April 23: St George}
\subtitle{? Class}
\subtitle{II Vespers}
\medskip

%\def\deusinadjutoriumsolemn{T}
%\def\prepsalmtitleone{\vspace{-0.75\baselineskip}}
%\def\prepsalmthreeverses{\vspace{-0.1\baselineskip}}
%\def\prerepeatontiphonthree{}
%\def\prepsalmtitlefour{\needspace{8\baselineskip}}
%\def\premagtitle{\bigskip}

%\printvespers[../April23-StGeorge]{inc-StGeorge-2Vespers}
\printvespers[../CommonOfApostles]{inc-StGeorge-2Vespers}
\medskip
\printnote{\sundaycommemnoteeaster{}

\begin{multicols}{2}
\noindent\emph{\nth{2} Sunday after Easter}, p.~\pageref{easter2-mag}.\\
\emph{\nth{3} Sunday after Easter}, p.~\pageref{easter3-mag}.\\
\emph{\nth{4} Sunday after Easter}, p.~\pageref{easter4-mag}.
\end{multicols}
}
\bigskip
\benedicamusdomino{}
}
}{}

\ifthenelse{\boolean{includemay1}}{
	%May 1: St Joseph the Worker (1st class)
	{
\section{1 May: St Joseph the Worker}
\subtitle{\nth{1} Class}
\subtitle{I \& II Vespers}
\medskip

\def\deusinadjutoriumsolemn{T}
\def\definevesperspropers{\newcommand{\vrtex}{vrOraProNobis}
\newcommand{\vtranslation}{Pray for us, St Joseph, alleluia.}
\newcommand{\rtranslation}{Faithful protector of our labors, alleluia.}

\newcommand{\maganttex}{an--et_ipse_jesus--solesmes}
\newcommand{\magantinitial}{E}
\newcommand{\maganttranslation}{}
\definemag{7}{d}

  \let\oldthing=\maganttranslation
  \def\maganttranslation{\oldthing\needspace{10\baselineskip}}
}
\def\definevesperspropersalt{\newcommand{\vrtex}{vrSolemnitasEstHodie}
\newcommand{\vtranslation}{Today is the solemnity of St Joseph, alleluia.}
\newcommand{\rtranslation}{Who ministered with his hands to the Son of God, alleluia.}

\newcommand{\maganttex}{an--christus_dominus--solesmes}
\newcommand{\magantinitial}{C}
\newcommand{\maganttranslation}{Christ the Lord deigned to be thought the son of a carpenter, alleluia.}
\definemag{7}{c2}

  \def\vrlinebreak{T}
  \let\oldthing=\maganttranslation
  \def\maganttranslation{\oldthing\needspace{10\baselineskip}}
}
\def\vesperspropersnote{At II Vespers:}
\def\vesperspropersaltnote{At I Vespers:}
\ifthenelse{\boolean{birmingham}}{
	%\def\postpsalmtitletwo{\needspace{12\baselineskip}}
	\def\preanttwo{\vspace{-0.3\baselineskip}}
	\def\preantfive{\vspace{-0.4\baselineskip}}
	\def\postpsalmtitlethree{\needspace{15\baselineskip}}
	\def\prepsalmtitlefive{\vspace{-0.1\baselineskip}}
}{
	\def\premagtitle{\bigskip}
}
\def\prepsalmtitleone{\vspace{-0.75\baselineskip}}
%\def\prepsalmthreeverses{\vspace{-0.1\baselineskip}}
%\def\prerepeatontiphonthree{}
\def\prepsalmtitlefour{\needspace{8\baselineskip}}

\printvespers[../May1-StJosephWorker]{inc-StJosephWorker}
%if feast of St Joseph the worker falls from 2nd through 5th Sunday after Easter, it outranks the Sunday and the Sunday is commemorated
\medskip
\printnote{\sundaycommemnoteeaster{}

\begin{multicols}{2}
\noindent\emph{\nth{2} Sunday after Easter}, p.~\pageref{easter2-mag}.\\
\emph{\nth{3} Sunday after Easter}, p.~\pageref{easter3-mag}.\\
\emph{\nth{4} Sunday after Easter}, p.~\pageref{easter4-mag}.\\
\emph{\nth{5} Sunday after Easter}, p.~\pageref{easter5-mag}.
\end{multicols}
}
\ifthenelse{\boolean{birmingham}}{
	%
}{
	\bigskip
}
\benedicamusdomino{}
}
}{}

%May 26: St Philip Neri (1)
{
\section{May 26: St Philip Neri}
\subtitle{\nth{1} Class (proper to the Oratory), White or Gold}
\subtitle{I \& II Vespers}
\medskip

\def\prepsalmtitleone{\needspace{8\baselineskip}}
\def\prepsalmtwoverses{}
%\def\prerepeatantiphonfour{}
%\def\preantfive{\vspace{-0.2\baselineskip}}
%\def\prepsalmtitlefive{\vspace{-0.3\baselineskip}}
%\def\prerepeatantiphonfive{}
\def\definevesperspropers{\newcommand{\maganttex}{MagnificatAntiphon2}
\newcommand{\magantinitial}{H}
\newcommand{\maganttranslation}{Come, children, hearken to me: I will teach you the fear of the Lord.}
\definemag{4}{A}

}
\def\definevesperspropersalt{\newcommand{\maganttex}{MagnificatAntiphon1}
\newcommand{\magantinitial}{D}
\newcommand{\maganttranslation}{My house shall be called a house of prayer, says the Lord.}

}
\def\vesperspropersnote{
\oldneedspace{8\baselineskip}
At II Vespers:}
\def\vesperspropersaltnote{At I Vespers:}
\def\prehymn{\vfill}

\printvespers[../StPhilipNeri]{inc-StPhilipNeri}
% feast can fall from 5th sunday after easter to 2nd sunday after pentecost
% pentecost and trinity would outrank it
\medskip
\printnote{\sundaycommemnoteeasterpentecost{}

\bigskip{}
\emph{\nth{5} Sunday after Easter}, p.~\pageref{easter5}.

\emph{Sunday after the Ascension}, p.~\pageref{easter6}.

% pentecost and trinity sunday would outrank this feast
\emph{\nth{2} Sunday after Pentecost}, p.~\pageref{pentecost2}.
}

\bigskip{}
\benedicamusdomino{}
%\vfil
%\pagebreak
}

\ifthenelse{\boolean{birmingham}}{
	%June 23: Dedication of Oratory Church
	{
\section{23 June: Dedication of Birmingham Oratory Church}
\subtitle{\nth{1} Class}
\medskip
\printnote{If today is Sunday, 22 June then \emph{First Vespers of the Dedication}, on page \pageref{commondedicationofchurch}, is sung without a commemoration of the Sunday.

If today is Sunday, 23 June then \emph{Dedication Vespers} will not be sung but \emph{First Vespers of the Nativity of St John the Baptist}, which follows, takes precedence with a commemoration of \emph{Second Vespers of the Dedication}.
}

\bigskip
\hrule
\bigskip
}
}{}

%June 24: Nativity of St John the Baptist (1st class)
{
\section{June 24: Nativity of St John the Baptist}
\subtitle{\nth{1} Class}
\subtitle{I \& II Vespers}
\medskip

\def\deusinadjutoriumsolemn{T}
\def\definevesperspropers{\newcommand{\antonetex}{Ant1-ElisabethZachariae}
\newcommand{\antoneinitial}{E}
\newcommand{\antonetranslation}{Elizabeth, the wife of Zacharias, gave birth to a great man, John the Baptist, the forerunner of the Lord.}
\definepsalm{1}{109}{3}{a}

\newcommand{\anttwotex}{Ant2-Innuebant}
\newcommand{\anttwoinitial}{I}
\newcommand{\anttwotranslation}{They made signs unto his father, by what name he should be called: and he wrote, saying: His name is John.}
\definepsalm{2}{110}{4}{E}

\newcommand{\antthreetex}{Ant3-JoannesVocabitur}
\newcommand{\antthreeinitial}{J}
\newcommand{\antthreetranslation}{His name shall be called John, and many shall rejoice in his birth.}
\definepsalm{3}{111}{1}{f}

\newcommand{\antfourtex}{Ant4-InterNatos}
\newcommand{\antfourinitial}{I}
\newcommand{\antfourtranslation}{Among those born of women, there hath not risen a greater than John the Baptist.}
\definepsalm{4}{112}{3}{b}

\newcommand{\antfivetex}{Ant5-TuPuer}
\newcommand{\antfiveinitial}{T}
\newcommand{\antfivetranslation}{Thou, child, shalt be called the Prophet of the Most High: thou shalt go before the Lord to prepare His ways.}
\definepsalm{5}{116}{3}{b}

\newcommand{\vrtex}{vrIstePuerMagnus}
\newcommand{\vtranslation}{This child is great before the Lord.}
\newcommand{\rtranslation}{For in truth His hand is with him.}

\newcommand{\maganttex}{MagnificatAntiphon2}
\newcommand{\magantinitial}{P}
\newcommand{\maganttranslation}{The child that is born to us is more than a prophet; for this is he of whom the Saviour said: Among those born of women there hath not risen a greater than John the Baptist.}
\def\magsolemn{T}
\definemag{7}{d}

  %  \def\preantone{\bigskip}
  %  \def\prepsalmtitleone{\bigskip}
  \def\preantfour{\bigskip}
  \def\prepsalmtitlefour{\needspace{14\baselineskip}}
  \ifthenelse{\boolean{birmingham}}{
    \def\prepsalmtitleone{\bigskip}
    \def\postpsalmtitlethree{\needspace{10\baselineskip}}
    \def\prepsalmthreeverses{\vspace{-0.1\baselineskip}}
    \def\prerepeatantiphonthree{}
  }{}
}
\def\definevesperspropersalt{\newcommand{\antonetex}{Ant1-IpsePraeibit}
\newcommand{\antoneinitial}{I}
\newcommand{\antonetranslation}{He shall go before Him in the spirit and power of Elias, to prepare unto the Lord a perfect people.\vspace{0ex plus 0ex minus 3ex}}
\definepsalm{1}{109}{7}{a}

\newcommand{\anttwotex}{Ant2-Joannes}
\newcommand{\anttwoinitial}{J}
\newcommand{\anttwotranslation}{John is his name.  Wine and strong drink he shall not drink, and many shall rejoice in his birth.}
\definepsalm{2}{110}{8}{G}

\newcommand{\antthreetex}{Ant3-ExUteroSenectutis}
\newcommand{\antthreeinitial}{E}
\newcommand{\antthreetranslation}{From the barren womb of age was born John, the forerunner of the Lord.}
\definepsalm{3}{111}{1}{f}

\newcommand{\antfourtex}{Ant4-IstePuer}
\newcommand{\antfourinitial}{I}
\newcommand{\antfourtranslation}{This child is great before the Lord, for the hand of God is with him.}
\definepsalm{4}{112}{4}{A*}

\newcommand{\antfivetex}{Ant5-Nazaraeus}
\newcommand{\antfiveinitial}{N}
\newcommand{\antfivetranslation}{This child shall be called a Nazarite; wine and strong drink he shall not drink, and from his mother's womb he shall eat nothing unclean.}
\definepsalm{5}{116}{5}{a}

\newcommand{\vrtex}{vrFuitHomo}
\newcommand{\vtranslation}{There was a man sent from God.}
\newcommand{\rtranslation}{Whose name was John.}

\newcommand{\maganttex}{MagnificatAntiphon1}
\newcommand{\magantinitial}{I}
\newcommand{\maganttranslation}{When Zacharias had entered the temple of the Lord, there appeared to him the Angel Gabriel, standing at the right hand of the altar of incense.\vspace{-4pt plus 4pt}}
\def\magpsalmclef{c3}
\def\magsolemn{T}
\definemag{8}{G}

  \def\preantthree{\needspace{8\baselineskip}}
  \ifthenelse{\boolean{birmingham}}
  {}
  {
    \def\prepsalmtitletwo{\vspace{-0.5\baselineskip}}
    \def\preantfour{\bigskip}
  }
}
\def\vesperspropersnote{At II Vespers:}
\def\vesperspropersaltnote{At I Vespers:}
\def\prevesperspsalms{\noindent\printnote{Chapter and following, page \pageref{june24-chapter}.\\}}
\def\vesperspsalmslabel{\label{june24-2vespers}}
\def\prevesperspsalmsalt{\noindent\printnote{II Vespers psalms and antiphons, page \pageref{june24-2vespers}.}\medskip\label{june24-1vespers}}
\def\prechapter{\label{june24-chapter}}
%\def\premagnificat{\pagebreak}

\printvespers[../June24-BirthOfJohnTheBaptist]{inc-BirthOfJohnTheBaptist}

\ifthenelse{\boolean{birmingham}}{
  \printnote{If today is Sunday, 23 June then \emph{Second Vespers of the Dedication of a Church} is commemorated as follows.
  \ifthenelse{\boolean{includebenedicamusdominoreferences}}{
    %Otherwise \benedicamusdominoreference{}%
  }{}
  }
  \medskip
  \hrule
  \medskip

  % commemoration of Second Vespers of Dedication of Church
  {
  %\def\beginvrcols{\begin{parcolumns}[rulebetween,colwidths={1=0.45\linewidth}]{2}}
  \printcommemoration[../CommonOfDedicationOfChurch]{commemorationDedicationOfChurch-Vespers2}

  \medskip
  \hrule
  \medskip
  %\bigskip
  \benedicamusdomino{}
  }
  \renewcommand{\sundaycommemnote}{
    If today is Sunday, 24 June then \emph{Vespers of the Sunday} is commemorated with \emph{Magnificat antiphon}, \emph{\Vbar{}~Dirigátur}, and \emph{Collect}.
  }

}{}

\printnote{\sundaycommemnote{}
%\vspace{-\baselineskip}
\begin{multicols}{2}
\noindent\emph{\nth{2} Sunday after Pentecost}, p.~\pageref{pentecost2-mag}.\\
\emph{\nth{3} Sunday after Pentecost}, p.~\pageref{pentecost3-mag}.\\
\emph{\nth{4} Sunday after Pentecost}, p.~\pageref{pentecost4-mag}.\\
\emph{\nth{5} Sunday after Pentecost}, p.~\pageref{pentecost5-mag}.\\
\emph{\nth{6} Sunday after Pentecost}, p.~\pageref{pentecost6-mag}.
\end{multicols}
}
\benedicamusdomino{}
}

%June 29: Ss Peter \& Paul (1st class)
{
\section{June 29: Sts Peter \& Paul}
\subtitle{\nth{1} Class, Red}
\subtitle{I \& II Vespers}
\medskip

\def\definevesperspropers{\import{../CommonOfApostles/}{inc-CommonOfApostles-2Vespers-psalms}
\edef\antonetex{../CommonOfApostles/\antonetex}
\edef\anttwotex{../CommonOfApostles/\anttwotex}
\edef\antthreetex{../CommonOfApostles/\antthreetex}
\edef\antfourtex{../CommonOfApostles/\antfourtex}
\edef\antfivetex{../CommonOfApostles/\antfivetex}

\newcommand{\vrtex}{vrAnnuntiaverunt}
\newcommand{\vtranslation}{They declared the works of God.}
\newcommand{\rtranslation}{And understood His doings.}

\newcommand{\maganttex}{MagnificatAntiphon2-Hodie}
\newcommand{\magantinitial}{H}
\newcommand{\maganttranslation}{Today, Simon Peter went up upon the gibbet of the cross, alleluia; today, he that holdeth the keys of the kingdom, departed with joy to be with Christ; today, the Apostle Paul, the light of the world, bowing his head, for Christ's sake was crowned with martyrdom, alleluia.}
\def\magsolemn{T}
\definemag{1}{D}

  \def\preanttranslationone{\vspace{-0.5\baselineskip}}
  \def\prepsalmtitleone{\vspace{-0.3\baselineskip}}
  \def\prerepeatantiphonone{}
  \def\preantfive{\bigskip}
}
\def\definevesperspropersalt{\newcommand{\antonetex}{Ant1-PetrusEtJoannes}
\newcommand{\antoneinitial}{P}
\newcommand{\antonetranslation}{Peter and John went up together into the Temple at the hour of prayer, being the ninth hour.}
\definepsalm{1}{109}{8}{G}

\newcommand{\anttwotex}{Ant2-Argentum}
\newcommand{\anttwoinitial}{A}
\newcommand{\anttwotranslation}{Silver and gold have I none, but such as I have, give I thee.}
\definepsalm{2}{110}{7}{b}

\newcommand{\antthreetex}{Ant3-DixitAngelusAdPetrum}
\newcommand{\antthreeinitial}{D}
\newcommand{\antthreetranslation}{The Angel said unto Peter: Cast thy garment about thee, and follow me.}
\definepsalm{3}{111}{8}{c}

\newcommand{\antfourtex}{Ant4-MisitDominus}
\newcommand{\antfourinitial}{M}
\newcommand{\antfourtranslation}{The Lord hath sent His Angel, and hath delivered me out of the hand of Herod.  Alleluia.}
\definepsalm{4}{112}{7}{c2}

\newcommand{\antfivetex}{Ant5-TuEsPetrus}
\newcommand{\antfiveinitial}{T}
\newcommand{\antfivetranslation}{Thou art Peter and upon this Rock I will build My Church.}
\definepsalm{5}{116}{7}{c}

\newcommand{\vrtex}{vrInOmnemTerram}
\newcommand{\vtranslation}{Their sound hath gone forth into all the earth.}
\newcommand{\rtranslation}{And their words unto the ends of the world.}

\newcommand{\maganttex}{MagnificatAntiphon1}
\newcommand{\magantinitial}{T}
\newcommand{\maganttranslation}{Thou art the Shepherd of the sheep and the Prince of the Apostles, and unto thee are given the keys of the kingdom of heaven.}
\def\magsolemn{T}
\definemag{1}{f}

  \def\postpsalmtitleone{\oldneedspace{10\baselineskip}}
  \def\postpsalmtitletwo{\oldneedspace{10\baselineskip}}
  \def\preanttwo{\vspace{-0.4\baselineskip}}
  \def\prerepeatantiphontwo{}
}
\def\vesperspropersnote{At II Vespers:}
\def\vesperspropersaltnote{At I Vespers:}
\def\prevesperspsalms{\noindent\printnote{Chapter and following, page \pageref{june29-chapter}.\\}}
\def\vesperspsalmslabel{\label{june29-2vespers}}
\def\prevesperspsalmsalt{\noindent\printnote{II Vespers psalms and antiphons, page \pageref{june29-2vespers}.}\medskip}
\def\prechapter{\label{june29-chapter}}
%\def\precollect{\vspace{-0.5\baselineskip}}

\def\begincollectcols{\begin{parcolumns}[rulebetween,colwidths={1=0.45\linewidth}]{2}}
\printvespers[../June29-StsPeterAndPaul]{inc-StsPeterAndPaul}

\smallskip
\printnote{\sundaycommemnote{}
\vspace{-0.5\baselineskip}
\begin{multicols}{2}
\noindent\emph{\nth{3} Sunday after Pentecost}, p.~\pageref{pentecost3}.\\
\emph{\nth{4} Sunday after Pentecost}, p.~\pageref{pentecost4}.\\
\emph{\nth{5} Sunday after Pentecost}, p.~\pageref{pentecost5}.\\
\emph{\nth{6} Sunday after Pentecost}, p.~\pageref{pentecost6}.\\
\emph{\nth{7} Sunday after Pentecost}, p.~\pageref{pentecost7}.
\end{multicols}
}
\benedicamusdomino{}
}

%\cleardoublepage
%\cleartoleftpage
%July 1: Most Precious Blood (1st class)
{
\global\let\psalmclefthree=\undefined
\section{1 July: The Precious Blood of Our Lord Jesus Christ}
\subtitle{\nth{1} Class}
\subtitle{I \& II Vespers}
\medskip

\def\deusinadjutoriumsolemn{T}
\def\prepsalmtitlethree{\vspace{-0.5\baselineskip}}
\def\premagtitle{\oldneedspace{8\baselineskip}}
\ifthenelse{\boolean{birmingham}}{
	\def\prepsalmtwoverses{\bigskip\bigskip}
	\def\postpsalmtitlefour{\needspace{10\baselineskip}}
	%\def\postpsalmtitlefive{\medskip}
    %\def\chapterreplacement{}
	%\def\prechapter{}
	%\def\prehymn{\needspace{5\baselineskip}}
	\def\begincollectcols{\begin{parcolumns}[rulebetween,colwidths={1=0.43\linewidth}]{2}}
}{
	\def\prepsalmtitleone{\vspace{-0.5\baselineskip}}
	\def\prerepeatantiphonthree{}
	\def\preantfour{\vspace{-0.5\baselineskip}}
	\def\prepsalmtitlefour{\vspace{-0.5\baselineskip}}
	\def\begincollectcols{\begin{parcolumns}[rulebetween,colwidths={1=0.45\linewidth}]{2}}
}
\def\definevesperspropers{\definepsalm{5}{147}{2}{D}

\newcommand{\vrtex}{vrTeErgo}
\newcommand{\vtranslation}{We therefore pray Thee, help Thy servants.}
\newcommand{\rtranslation}{Whom Thou hast redeemed with Thy precious Blood.}

\newcommand{\maganttex}{MagnificatAntiphon2}
\newcommand{\magantinitial}{H}
\newcommand{\maganttranslation}{Ye shall observe this day for a memorial: and ye shall keep it holy unto the Lord, in your generations with an everlasting worship.}
\newcommand{\magsolemn}{F}
\definemag{1}{D2}

	\ifthenelse{\boolean{birmingham}}{
		\def\prepsalmtitlefive{\medskip}
		\def\postpsalmtitlefive{\bigskip}
	}{
		%
	}
}
\def\definevesperspropersalt{\definepsalm{5}{116}{2}{D}

\newcommand{\vrtex}{vrRedemistiNos}
\newcommand{\vtranslation}{Thou hast redeemed us, O Lord, in Thy Blood.}
\newcommand{\rtranslation}{And hast made of us a kingdom unto our God.}

\newcommand{\maganttex}{MagnificatAntiphon1}
\newcommand{\magantinitial}{A}
\newcommand{\maganttranslation}{Ye are come to Mount Sion, to the city of the living God, the heavenly Jerusalem, and to Jesus the Mediator of the new Testament, and to the sprinkling of blood which speaketh better than that of Abel.}
\newcommand{\magsolemn}{F}
\definemag{3}{a}
%
	\ifthenelse{\boolean{birmingham}}{
		\def\prepsalmtitlefive{\medskip}
		\def\postpsalmtitlefive{\bigskip}
	}{}
  \def\postmagtitle{\vspace{-0.5\baselineskip}}%
}
\def\vesperspropersnote{At II Vespers:}
\def\vesperspropersaltnote{At I Vespers:}

\printvespers[../July1-MostPreciousBloodOfChrist]{inc-MostPreciousBloodOfChrist}
\medskip
\benedicamusdomino{}
}
}{}

\ifthenelse{\boolean{includejuly9}}{
	%July 9: Ss Thomas More & John Fisher
	{
\section{9 July: Ss John Fisher \& Thomas More}
\subtitle{\nth{1} Class}
\subtitle{I \& II Vespers}
\medskip

\def\deusinadjutoriumsolemn{F}
\def\vrlinebreak{F}
\def\preanttwo{\vspace{-0.5\baselineskip}}
%\def\prepsalmtitletwo{\vspace{-0.5\baselineskip}}
\ifthenelse{\boolean{birmingham}}{
	%
}{
	% I'm not sure why these are defined for both, since there are different 1st and 2nd vespers psalms
	\def\prepsalmthreeverses{\vspace{-0.1\baselineskip}}
	\def\prerepeatantiphonthree{}
	\def\prepsalmtitlefour{\vspace{-0.3\baselineskip}}
	\def\prepsalmtitlefive{\vspace{-0.5\baselineskip}}
}
\def\begincollectcols{\begin{parcolumns}[rulebetween,colwidths={1=0.44\linewidth}]{2}}
\def\definevesperspropers{%\input{\gabcfolder/inc-StsThomasMoreAndJohnFisher-2Vespers}
  \newcommand{\antonetex}{an--isti_sunt_sancti--solesmes}
\newcommand{\antoneinitial}{I}
\newcommand{\antonetranslation}{These are the Saints who for God's covenant gave up their bodies, and washed their robes in the Blood of the Lamb.}
\definepsalm{1}{109}{2}{D}

\newcommand{\anttwotex}{an--sancti_per_fidem--solesmes}
\newcommand{\anttwoinitial}{S}
\newcommand{\anttwotranslation}{By faith the Saints overcame kingdoms; they performed what was right, they obtained the promises.}
\definepsalm{2}{110}{2}{D}

\newcommand{\antthreetex}{an--sanctorum_velut_aquilae--solesmes}
\newcommand{\antthreeinitial}{S}
\newcommand{\antthreetranslation}{The youth of the Saints shall be renewed like the eagle's; they shall flower like the lily in the City of God.}
\definepsalm{3}{111}{8}{G}

\newcommand{\antfourtex}{an--absterget_deus--solesmes}
\newcommand{\antfourinitial}{A}
\newcommand{\antfourtranslation}{God shall wipe away every tear from the eyes of the Saints; and no longer shall be any sorrow, or crying, nor any pain; for the former things have passed away.}
\definepsalm{4}{112}{7}{a}

\newcommand{\antfivetex}{an--in_caelestibus_regnis--solesmes}
\newcommand{\antfiveinitial}{I}
\newcommand{\antfivetranslation}{The dwelling of the Saints is in the heavenly kingdoms, and their rest for ever.}
\def\psalmclef{2}
\definepsalm{5}{115}{8}{G}
\let\psalmclef=\undefined

  \ifthenelse{\boolean{birmingham}}{
  	\def\prepsalmtitletwo{\bigskip\bigskip\bigskip}
  	\def\postpsalmtitletwo{\bigskip}
  	\def\postpsalmtitlethree{\needspace{10\baselineskip}}
  	\def\postpsalmtitlefive{\bigskip}
  }{
  	%
  }
  %\let\oldthing=\maganttranslation
  %\def\maganttranslation{\oldthing\needspace{10\baselineskip}}
}
\def\definevesperspropersalt{
	%\newcommand{\chaptertext}{\dropcap{latin}{Justórum ánimæ in manu Dei sunt, et non tanget illos torméntum} \textbf{mor}\-tis.~\dag{} Visi sunt óculis insipién\-\emph{ti\-um} \textbf{mo}\-ri~:~* illi autem sunt in \textbf{pa}ce.}
\newcommand{\chaptertranslation}{The souls of the just are in the hand of God~: and the torment of death shall not touch them.  In the sight of the unwise they seemed to die~: but they are in peace.}

\newcommand{\hymnlinetwo}{2.}
\newcommand{\hymntex}{hy--sanctorum_meritis--solesmes}
\newcommand{\hymninitial}{S}
\newcommand{\hymntranslation}{
\item By help of Saints, come let our tongues relate
their famous joys and their courageous deeds;
our mind desires in songs to celebrate
their conquest, which all gain exceeds.

\item While here they lived, the world these men abhorred,
for they this withered soil did much despise
whose flowers are barren, and with thee, their Lord,
up to thy heavenly joys did rise.

\item They for thy sake with stout contempt have borne
the causeless rage of men, and torment fierce,
and cruel hooks, which have their bodies torn,
but had not power their soul to pierce.

\item They like mild sheep to slaughter are assigned,
at which they never murmur nor complain,
but with a silent heart and guiltless mind
their constant patience they maintain.

\item What voice, what tongue, those gifts can fitly show
which thou prepar'st for martyrs? Who, once stained
with streams of blood, which from their wounds did flow,
have now bright crown of laurel gained.

\item We thee beseech, one highest Deity,
to wash our sins, to drive our harms away,
to give thy servants peace, that we
to thee may everlasting praise repay.
Amen.\grechangestaffsize{15}}

%\newcommand{\vrtex}{vrAnnuntiaveruntOperaDei}
%\newcommand{\vtranslation}{They declared the works of God.}
%\newcommand{\rtranslation}{And understood His doings.}

%\newcommand{\maganttex}{MagnificatAntiphon}
%\newcommand{\magantinitial}{E}
%\newcommand{\maganttranslation}{Be ye valiant in warfare, and fight with the old serpent: and ye shall receive an everlasting kingdom.}
%\def\magsolemn{T}
%\definemag{1}{g2}

	\input{../CommonOfMartyrs/inc-CommonOfMartyrs-1Vespers-psalms}

\renewcommand{\maganttex}{an--astiterunt_justi--solesmes}
\renewcommand{\magantinitial}{A}
\renewcommand{\maganttranslation}{The just stood before the Lord, and from one another they were not divided; they drank the cup of the Lord, and were called the friends of God.}
\def\magsolemn{F}
\let\magant=\undefined
\let\magantlinetwo=\undefined
\let\magverses=\undefined
\definemag{1}{g}

	\def\vrlinebreak{T}
	\ifthenelse{\boolean{birmingham}}{
		\def\prepsalmtitletwo{\bigskip\bigskip}
		\def\postpsalmtitletwo{\bigskip}
		\def\postpsalmtitlethree{\needspace{10\baselineskip}}
	}{
		%
	}
}
\def\vesperspropersnote{At II Vespers:}
\def\vesperspropersaltnote{At I Vespers:}
\def\prevesperspsalms{\noindent\printnote{Chapter and following, page \pageref{july9-chapter}.\\}
  \medskip
  \hrule
  \medskip
}
\def\vesperspsalmslabel{\label{july9-2vespers}}
\def\prevesperspsalmsalt{\noindent\printnote{II Vespers psalms and antiphons, page \pageref{july9-2vespers}.}\medskip}
\def\prechapter{\label{july9-chapter}}
% \def\definevesperspropersalt{\newcommand{\vrtex}{vrSolemnitasEstHodie}
\newcommand{\vtranslation}{Today is the solemnity of St Joseph, alleluia.}
\newcommand{\rtranslation}{Who ministered with his hands to the Son of God, alleluia.}

\newcommand{\maganttex}{an--christus_dominus--solesmes}
\newcommand{\magantinitial}{C}
\newcommand{\maganttranslation}{Christ the Lord deigned to be thought the son of a carpenter, alleluia.}
\definemag{7}{c2}

%   \def\vrlinebreak{T}
%   \let\oldthing=\maganttranslation
%   \def\maganttranslation{\oldthing\needspace{10\baselineskip}}
% }
% \def\prepsalmtitleone{\vspace{-0.75\baselineskip}}
%\def\prepsalmthreeverses{\vspace{-0.1\baselineskip}}
%\def\prerepeatontiphonthree{}
% \def\prepsalmtitlefour{\needspace{8\baselineskip}}
% \def\premagtitle{\bigskip}

\printvespers[../CommonOfMartyrs]{inc-sts-thomas-more-and-john-fisher-2vespers}
%if feast of St Joseph the worker falls from 2nd through 5th Sunday after Easter, it outranks the Sunday and the Sunday is commemorated
\bigskip
\benedicamusdomino{}
}
}{}


\ifthenelse{\boolean{testrun}}{}{
%Aug 6: Transfiguration (2nd class)
{
\section{6 August: Transfiguration of Our Lord Jesus Christ}
\subtitle{\nth{2} Class}
\subtitle{I \& II Vespers}
\medskip

%\def\prepsalmtitleone{\needspace{4\baselineskip}}
\ifthenelse{\boolean{birmingham}}{
	\def\postpsalmtitleone{\medskip}
	\def\prepsalmoneverses{\bigskip}
	\def\psepsalmtitletwo{\bigskip}
	\def\prepsalmtitlefour{\oldneedspace{4\baselineskip}}
	\def\prepsalmtitlefive{\needspace{4\baselineskip}}
	\def\premagtitle{\needspace{10\baselineskip}}
}{
	\def\postpsalmtitleone{\needspace{4\baselineskip}}
	\def\prepsalmoneverses{}
	\def\prerepeatantiphonone{}	
	\def\preanttwo{\vspace{-0.5\baselineskip}}
	\def\preanttranslationtwo{\vspace{-0.2\baselineskip}}
	\def\prerepeatantiphontwo{}
	\def\prepsalmtitletwo{\vspace{-0.5\baselineskip}}
	\def\prepsalmtwoverses{\vspace{-0.01\baselineskip}}
}
\def\definevesperspropers{\newcommand{\maganttex}{MagnificatAntiphon2}
\newcommand{\magantinitial}{E}
\newcommand{\maganttranslation}{And the disciples hearing, fell on their faces, and were sore afraid; and Jesus came, and touched them, and said to them, Arise, and fear not, alleluia.}
\newcommand{\magsolemn}{F}
\definemag{1}{f}

  \def\premagverses{\vspace{-0.5\baselineskip}}
}
\def\definevesperspropersalt{\newcommand{\maganttex}{an--christus_jesus--solesmes}
\newcommand{\magantinitial}{C}
\newcommand{\maganttranslation}{Christ Jesus, radiance of the Father and image of His Being, upholding all things by the word of His power; making atonement for sins, has deigned to appear today in glory on the high mountain.}
\newcommand{\magsolemn}{F}
\definemag{4}{E}

  \def\premagverses{\vspace{-0.5\baselineskip}}
}
\def\vesperspropersnote{At II Vespers:}
\def\vesperspropersaltnote{At I Vespers:}
\def\postmagtitle{\vspace{-0.5\baselineskip}}

\def\begincollectcols{\begin{parcolumns}[rulebetween,colwidths={1=0.44\linewidth}]{2}}
\printvespers[../August6-TransfigurationOfOurLord]{inc-Transfiguration}
\smallskip
\benedicamusdomino[2]{}
}

%Aug 15: Assumption (1st class)
{
\section{August 15: Assumption of the B.~V.~M.}
\subtitle{\nth{1} Class}
\subtitle{I \& II Vespers}
\medskip

%\def\prepsalmfivetitle{\vspace{-0.5\baselineskip}}
%\def\preanttwo{\oldneedspace{12\baselineskip}}
\def\deusinadjutoriumsolemn{T}
\ifthenelse{\boolean{birmingham}}{
  \def\prepsalmtitletwo{\needspace{8\baselineskip}}
}{
  \def\prepsalmoneverses{}
  \def\prerepeatantiphonone{}
  \def\preanttwo{\vspace{-0.5\baselineskip}}
  \def\preanttranslationtwo{\vspace{-0.2\baselineskip}}
  \def\prepsalmtitletwo{\vspace{-0.5\baselineskip}}  
}
\def\precollect{\needspace{8\baselineskip}}

\def\definevesperspropers{% hymn is ave maris stella
%\input{inc-hymn-avemarisstella}

\newcommand{\maganttex}{an--hodie_maria_virgo--solesmes}
\newcommand{\magantinitial}{H}
\newcommand{\maganttranslation}{Today the Virgin Mary has gone up to heaven: rejoice, for with Christ she reigns forever.}
\newcommand{\magsolemn}{T}
\definemag{8}{G*}
}
\def\definevesperspropersalt{\newcommand{\hymnlinetwo}{2.}
\newcommand{\hymntex}{Hymn-OPrimaVirgoProdita}
\newcommand{\hymninitial}{O}
\newcommand{\hymntranslation}{
\item O Virgin who was first to receive
The Creator’s grace by the spirit,
Who was predestined by the Most High
To bear in her womb the Son.

\item O woman, who was foretold to be
The perpetual enemy of the demon;
Who alone was filled with grace,
Undefiled from conception.

\item Thou who conceives Life itself in thy womb,
Life that was lost by Adam;
Furnishing the divine Victim,
A body for his sacrifice.

\item Death, the recompense for sin,
Had no victory over thee, and now departs;
And then thou hastened bodily to heaven
To be thy loving Son’s companion.

\item Illuminated by so great a Glory,
All nature is raised up;
And in thee is called to reach
The pinnacle of all glory and splendour.

\item In thy triumph O our Queen,
Turn thine eyes to us exiles;
That through thy patronage,
We may come to heaven, our blessed homeland.

\item Praise to the Father! praise to Him,
The Virgin’s holy Son!
Praise to the Spirit Paraclete,
While endless ages run! 
Amen.
}

\newcommand{\maganttex}{MagAntiphon-VirgoPrudentissima}
\newcommand{\magantinitial}{V}
\newcommand{\maganttranslation}{O Virgin most prudent, whither goest thou, like the golden dawn?  Daughter of Sion, thou art all beautiful and sweet; fair as the moon, bright as the sun.}
\newcommand{\magsolemn}{T}
\definemag{1}{f}

  \let\oldthing=\maganttranslation
  \def\maganttranslation{\oldthing\needspace{10\baselineskip}}
}
\def\premagtitle{\needspace{12\baselineskip}}
\def\vesperspropersnote{At II Vespers:}
\def\vesperspropersaltnote{At I Vespers:}
\def\printfullhymn{
  {
    \oldneedspace{3\baselineskip}
    \printnote{At II Vespers: Hymn.~\emph{Ave Maris Stella}, p.~\pageref{hymn-avemarisstella}. \Vbar{} \emph{Exaltata.} p.~\pageref{vr-assumption}.\\}

    \printnote{\vesperspropersaltnote}
    \definevesperspropersalt
    \printhymn{\oldstylenums{\hymnlinetwo}}{\hymninitial}{\hymntex}{\hymntranslation}
  }
  {
    \def\vrlinebreak{T}
    \oldneedspace{3\baselineskip}
    \label{vr-assumption}
    \printvr[\greseteolcustos{manual}]{\vrtex}{\vtranslation}{\rtranslation}
  }
}

\printvespers[../August15-AssumptionOfTheBlessedVirginMary]{inc-Assumption}

\medskip
\printnote{
  If today is Saturday, \emph{Vespers of the Saturday before 3rd Sunday of August} is commemorated as follows:\par
}
  \oldneedspace{4\baselineskip}
  {
  \def\noeuouae{T}
  \printgabc{At Magn.}{\oldstylenums{Ant.~8.~G}}{O}{../August15-AssumptionOfTheBlessedVirginMary/an--omnis_sapientia--solesmes}
  }
  \translation[]{All wisdom is from the Lord God, and with Him it will always be, and has been before all time.}
  \smallskip
  {
    \newcommand{\commvlatin}{Vespertína orátio ascéndat ad te \textbf{Dó}mine.}
    \newcommand{\commrlatin}{Et descéndat super nos misericórdia \textbf{tú}a.}
    \newcommand{\commvtranslation}{May the evening prayer ascend to Thee, O Lord.}
    \newcommand{\commrtranslation}{And may Thy mercy descend upon us.}
    \printvrcommem{}
  }
  \medskip
  \printnote{The \emph{Collect} is taken from the respective Sunday, found below.}
  \medskip
  \hrule
  \bigskip

\printnote{\sundaycommemnote{}
\begin{multicols}{2}
\noindent\emph{\nth{9} Sunday after Pentecost}, p.~\pageref{pentecost9-mag}.\\
\emph{\nth{10} Sunday after Pentecost}, p.~\pageref{pentecost10-mag}.\\
\emph{\nth{11} Sunday after Pentecost}, p.~\pageref{pentecost11-mag}.\\
\emph{\nth{12} Sunday after Pentecost}, p.~\pageref{pentecost12-mag}.\\
\emph{\nth{13} Sunday after Pentecost}, p.~\pageref{pentecost13-mag}.
\end{multicols}
}
\bigskip
\benedicamusdomino{}
}

%\cleardoublepage
%Sep 14: Exaltation of Holy Cross (2nd class)
{
\section{14 September: Exaltation of the Holy Cross}
\subtitle{\nth{2} Class}
\subtitle{I \& II Vespers}
\medskip

\def\definevesperspropers{\newcommand{\maganttex}{MagnificatAntiphon2-OCrux}
\newcommand{\magantinitial}{O}
\newcommand{\maganttranslation}{O blessed art thou, O Cross which wast counted the only tree worthy to bear the Lord and King of heaven. Alleluia.}
\def\magsolemn{F}
\definemag{1}{D2}
}
\def\definevesperspropersalt{\newcommand{\maganttex}{MagnificatAntiphon1-OCrux}
\newcommand{\magantinitial}{T}
\newcommand{\maganttranslation}{O Cross, brighter than all the stars thy name is honourable upon earth; exceeding lovely to mankind; holier than all things; thou alone wast worthy to carry the ransom of the world; sweet wood, sweet nails, bearing a burden sweeter still; save this people gathered here to praise thee.}
\def\magsolemn{F}
\definemag{1}{D}
%
%\def\premagverses{\oldneedspace{4\baselineskip}}%
}
\def\vesperspropersnote{At II Vespers:}
\def\vesperspropersaltnote{At I Vespers:}
\ifthenelse{\boolean{birmingham}}{
	\def\prepsalmtitleone{\vspace{-0.7\baselineskip}}
	\def\prepsalmtwoverses{\bigskip\bigskip}
}{
	\def\prepsalmtitleone{\vspace{-0.3\baselineskip}}
	\def\prerepeatantiphonthree{}
	\def\prepsalmthreeverses{\vspace{-0.1\baselineskip}}
}
\def\postpsalmtitlethree{\oldneedspace{9\baselineskip}}
\def\prepsalmtitlefour{\needspace{8\baselineskip}}
%\def\prepsalmfourverses{\oldneedspace{2\baselineskip}}
\def\prepsalmtitlefive{\needspace{8\baselineskip}}
\def\prehymn{\printnote{All kneel for the sixth verse of the following hymn.}}
\def\prehymntranslation{\oldneedspace{3\baselineskip}}
%\def\precollect{\vspace{-0.5\baselineskip}}

\def\begincollectcols{\begin{parcolumns}[rulebetween,colwidths={1=0.42\linewidth}]{2}}
\printvespers[../September14-ExaltationOfTheHolyCross]{inc-ExaltationOfTheHolyCross}
\bigskip
\benedicamusdomino[2]{}
}

%Sep 29: Dedication of St Michael (1st class)
{
\section{September 29: Dedication of St Michael the Archangel}
\subtitle{\nth{1} Class, White or Gold}
\subtitle{I \& II Vespers}
\medskip

\def\postdeusinadjutorium{\needspace{10\baselineskip}}

\def\definevesperspropers{\definepsalm{5}{137}{7}{c}

\newcommand{\vrtex}{vrInConspectuAngelorum}
\newcommand{\vtranslation}{In the sight of the Angels, I will sing praise to Thee, O my God.}
\newcommand{\rtranslation}{I will worship towards Thy holy temple, and I will give glory to Thy name.}

\newcommand{\maganttex}{MagnificatAntiphon2}
\definemag{1}{D2}
\newcommand{\magantinitial}{P}
\newcommand{\maganttranslation}{O most glorious prince, Michael Archangel, be mindful of us, and here and everywhere entreat the Son of God for us, alleluia, alleluia.}

  \def\vrlinebreak{F}
}
\def\definevesperspropersalt{\definepsalm{5}{116}{7}{c}

\newcommand{\vrtex}{vrStetitAngelus}
\newcommand{\vtranslation}{The Angel stood by the altar of the temple.}
\newcommand{\rtranslation}{Holding in his hand a censer of gold.}

\newcommand{\maganttex}{MagnificatAntiphon1}
\newcommand{\magantinitial}{D}
\definemag{8}{G}
\newcommand{\maganttranslation}{While John was beholding the sacred Mystery, the Archangel Michael sounded a trumpet.  Forgive us, O Lord our God, Thou who openest the book, and loosest the seals thereof.  Alleluia.\vspace{-1ex}}
}
\def\vesperspropersnote{At II Vespers:}
\def\vesperspropersaltnote{At I Vespers:}
%\def\prepsalmtitleone{\vspace{-0.5\baselineskip}}
\def\preanttwo{\bigskip\bigskip}
\def\prepsalmtitletwo{\bigskip}
\def\postpsalmtitlethree{\oldneedspace{9\baselineskip}}
\def\prerepeatantiphonthree{}
\def\prepsalmthreeverses{\vspace{-0.1\baselineskip}}
%\def\prepsalmtitlefour{\needspace{8\baselineskip}}
%\def\prepsalmfour{\needspace{8\baselineskip}}
\def\prechapter{\vspace{-\baselineskip}}
\def\postmagtitle{\vspace{-\baselineskip}}
\def\premagverses{\oldneedspace{8\baselineskip}}

\printvespers[../September29-DedicationOfChurchOfStMichaelArchangel]{inc-DedicationStMichael}
%an--domine_rex_omipotens--solesmes is for the commemoration of the saturday

\medskip
\printnote{\sundaycommemnote{}

\begin{multicols}{2}
\noindent\emph{\nth{16} Sunday after Pentecost}, p.~\pageref{pentecost16}.\\
\emph{\nth{17} Sunday after Pentecost}, p.~\pageref{pentecost17}.\\
\emph{\nth{18} Sunday after Pentecost}, p.~\pageref{pentecost18}.\\
\emph{\nth{19} Sunday after Pentecost}, p.~\pageref{pentecost19}.\\
\emph{\nth{20} Sunday after Pentecost}, p.~\pageref{pentecost20}.
\end{multicols}
}
\medskip
\benedicamusdomino{}
}

%\cleardoublepage
\cleartoleftpage

}

\ifthenelse{\boolean{includeoct7}}{
	%October 7: Our Lady of the Rosary
	{
\ifthenelse{\boolean{birmingham}}{
	\section{First Sunday in October: Rosary Sunday}	
}{
	\section{October 7: Our Lady of the Rosary}
}
%\subtitle{\nth{1} Class}
\subtitle{II Vespers}
\medskip

\def\deusinadjutoriumsolemn{F}
\def\definevesperspropers{%\input{\gabcfolder/inc-StsThomasMoreAndJohnFisher-2Vespers}
  \newcommand{\maganttex}{an--beata_mater--solesmes}
\newcommand{\magantinitial}{B}
\newcommand{\maganttranslation}{Blessed Mother and unspotted Virgin, glorious Queen of the world, may all be conscious of thine aid who keep thy feast of the most holy Rosary.}
\def\magpsalmclef{c3}
\definemag{8}{G}

  \def\vrlinebreak{T}
  %\let\oldthing=\maganttranslation
  %\def\maganttranslation{\oldthing\needspace{10\baselineskip}}
}
%\def\prepsalmtitleone{\vspace{-0.5\baselineskip}}
%\def\preanttwo{\vspace{-0.3\baselineskip}}
%\def\prepsalmtitletwo{\vspace{-0.1\baselineskip}}
%\def\prepsalmtwoverses{\vspace{-0.1\baselineskip}}
%\def\prepsalmtitlethree{\vspace{-0.5\baselineskip}}
% \def\definevesperspropersalt{\newcommand{\vrtex}{vrAmavitEum}
\newcommand{\vtranslation}{The Lord loved him and adorned him.}
\newcommand{\rtranslation}{He clothed him with a robe of glory.}

\newcommand{\magantinitial}{S}
\newcommand{\maganttranslation}{I will liken him to a wise man, that build his house upon a rock.}
\newcommand{\maganttex}{mag-ant-1-similabo-eum}
\def\magsolemn{T}
\definemag{1}{D}

%   \def\vrlinebreak{T}
%   \let\oldthing=\maganttranslation
%   \def\maganttranslation{\oldthing\needspace{10\baselineskip}}
% }
% \def\vesperspropersnote{At II Vespers:}
% \def\vesperspropersaltnote{At I Vespers:}
% \def\prepsalmtitleone{\vspace{-0.75\baselineskip}}
%\def\prepsalmthreeverses{\vspace{-0.1\baselineskip}}
%\def\prerepeatontiphonthree{}
% \def\prepsalmtitlefour{\needspace{8\baselineskip}}
% \def\premagtitle{\bigskip}

\printvespers[../October7-OurLadyOfTheRosary]{inc-OurLadyOfTheRosary}
\bigskip
\benedicamusdomino{}
}
}{}

\ifthenelse{\boolean{testrun}}{}{
\ifthenelse{\boolean{includeoct9}}{
	%October 9: Blessed John Henry Newman
	{
\section{October 9: Blessed John Henry Newman}
\subtitle{\nth{2} Class}
\subtitle{I \& II Vespers}
\medskip

\def\deusinadjutoriumsolemn{F}
\def\definevesperspropers{%\input{\gabcfolder/inc-StsThomasMoreAndJohnFisher-2Vespers}
  \newcommand{\vrtex}{vrJustumDeduxit}
\newcommand{\vtranslation}{The Lord led the just by right ways.}
\newcommand{\rtranslation}{And showed him the kingdom of God.}

\newcommand{\magantinitial}{H}
\newcommand{\maganttranslation}{This man, despising the world and earthly things, by word and deed has laid up treasures in heaven.}
\newcommand{\maganttex}{mag-ant-2-hic-vir-despiciens-mundum}
\def\magsolemn{T}
\def\magpsalmclef{c3}
\definemag{8}{G}

  \def\vrlinebreak{T}
  %\let\oldthing=\maganttranslation
  %\def\maganttranslation{\oldthing\needspace{10\baselineskip}}
}
\def\prepsalmtitleone{\vspace{-0.5\baselineskip}}
\def\preanttwo{\vspace{-0.3\baselineskip}}
\def\prepsalmtitletwo{\vspace{-0.1\baselineskip}}
\def\prepsalmtwoverses{\vspace{-0.1\baselineskip}}
\def\prepsalmtitlethree{\vspace{-0.5\baselineskip}}
\def\definevesperspropersalt{\newcommand{\vrtex}{vrAmavitEum}
\newcommand{\vtranslation}{The Lord loved him and adorned him.}
\newcommand{\rtranslation}{He clothed him with a robe of glory.}

\newcommand{\magantinitial}{S}
\newcommand{\maganttranslation}{I will liken him to a wise man, that build his house upon a rock.}
\newcommand{\maganttex}{mag-ant-1-similabo-eum}
\def\magsolemn{T}
\definemag{1}{D}

  \def\vrlinebreak{T}
  \let\oldthing=\maganttranslation
  \def\maganttranslation{\oldthing\needspace{10\baselineskip}}
}
% \def\vesperspropersnote{At II Vespers:}
% \def\vesperspropersaltnote{At I Vespers:}
% \def\prepsalmtitleone{\vspace{-0.75\baselineskip}}
%\def\prepsalmthreeverses{\vspace{-0.1\baselineskip}}
%\def\prerepeatontiphonthree{}
% \def\prepsalmtitlefour{\needspace{8\baselineskip}}
% \def\premagtitle{\bigskip}

\printvespers[../CommonOfConfessorNotBishop]{inc-BlessedJohnHenryNewman}
%if feast of St Joseph the worker falls from 2nd through 5th Sunday after Easter, it outranks the Sunday and the Sunday is commemorated
\bigskip
\benedicamusdomino{}
}
}{}

\ifthenelse{\boolean{includeoct11}}{
	%Oct 11: Maternity of the BVM (2nd class)
	{
\section{October 11: Maternity of the B.~V.~M.}
\subtitle{\nth{2} Class, White}
\subtitle{I \& II Vespers}
\medskip

\printnote{This feast is outranked by any Sunday and no commemoration is made.}

\def\definevesperspropers{\newcommand{\maganttex}{MagnificatAntiphon2}
\newcommand{\magantinitial}{M}
\newcommand{\maganttranslation}{Thy Motherhood, O Virgin Mother of God, heralded joy to the whole world: for out of thee has arisen the sun of justice, Christ our God.}
\definemag{4}{E}

}
\def\definevesperspropersalt{\newcommand{\maganttex}{MagnificatAntiphon1}
\newcommand{\magantinitial}{C}
\newcommand{\maganttranslation}{Let us celebrate with joy the Motherhood of blessed Mary ever Virgin.}
\definemag{7}{a}
}
\def\vesperspropersnote{At II Vespers:}
\def\vesperspropersaltnote{At I Vespers:}
\def\printhymnnote{
  {
    \oldneedspace{3\baselineskip}
    \printnote{Hymn.~\emph{Ave Maris Stella}, p.~\pageref{hymn-avemarisstella}.\\}
  }
}
%\def\prepsalmtitleone{\vspace{-0.4\baselineskip}}
\def\prepsalmtitletwo{}
\def\prepsalmtitlefive{\vspace{-0.7\baselineskip}}
\def\prevespers{
  \let\oldthing=\anttwotranslation
  \def\anttwotranslation{\vspace{-0.3\baselineskip}\oldthing\vspace{-0.7\baselineskip}}
}
\def\vrlabel{vr-october11}
\printvespers[../October11-MaternityOfBlessedVirginMary]{inc-MaternityOfBVM}

%an--refulsit_sol--solesmes is for commem of saturday
\medskip
\benedicamusdomino[mary]{}
}
}{}

%Last Sunday in October: Christ the King (1st class)
{
\ifthenelse{\boolean{birmingham}}{
  \def\mytitle{Last Sunday in October: Kingship of Our Lord, Jesus Christ}
}{
  \def\mytitle{Last Sunday in October: Jesus Christ, King}
}
\section{\mytitle}
\subtitle{\nth{1} Class}
\ifthenelse{\boolean{includechristthekingfirstvespers}}{% true
	\def\definevesperspropersalt{\newcommand{\vrtex}{vrDataEstMihi}
\newcommand{\vtranslation}{All power is given to me.}
\newcommand{\rtranslation}{In heaven and in earth.}

\newcommand{\magantinitial}{D}
\newcommand{\maganttex}{an--dabit_illi--solesmes}
\newcommand{\maganttranslation}{The Lord God shall give unto him the throne of David his father~: and he shall reign in the house of Jacob forever, and of his kingdom there shall be no end.}
\def\magsolemn{T}
\definemag{1}{f}
}
	\def\vesperspropersnote{At II Vespers:}
	\def\vesperspropersaltnote{At I Vespers:}
	\subtitle{I \& II Vespers}
}{% false
	\subtitle{II Vespers}
}
\medskip

\def\deusinadjutoriumsolemn{T}
\ifthenelse{\boolean{birmingham}}{
	%\def\prepsalmtitleone{\vspace{-0.5\baselineskip}}
	%\def\prepsalmtitletwo{\vspace{-0.5\baselineskip}}
	%\def\prerepeatantiphontwo{}
	%\def\preantthree{\vspace{-0.2\baselineskip}}
	%\def\prepsalmtitlethree{\vspace{-0.1\baselineskip}}
	\def\prepsalmtitlefour{\vspace{-0.5\baselineskip}}
	\def\prechapter{\vspace{-1\baselineskip}}
	\def\beginchaptercols{\begin{parcolumns}[rulebetween,colwidths={1=0.43\linewidth}]{2}}
}{
	\def\postpsalmtitleone{\needspace{8\baselineskip}}
	\def\prepsalmoneverses{}
	\def\prerepeatantiphonone{}
	\def\preanttwo{\vspace{-0.5\baselineskip}}
	\def\preanttranslationtwo{\vspace{-0.5\baselineskip}}
	\def\prepsalmtitletwo{\vspace{-0.5\baselineskip}}
	\def\postpsalmtitletwo{\needspace{12\baselineskip}}
	\def\prerepeatantiphontwo{}
	\def\postpsalmtitlefour{\bigskip}
	\def\beginchaptercols{\begin{parcolumns}[rulebetween,colwidths={1=0.46\linewidth}]{2}}
}
\def\prehymn{\vfill}
\def\definevesperspropers{\newcommand{\vrtex}{vr}
\newcommand{\vtranslation}{His dominion shall be increased.}
\newcommand{\rtranslation}{And of peace there shall be no end.}

\newcommand{\magantinitial}{H}
\newcommand{\maganttex}{MagnificatAntiphon}
\newcommand{\maganttranslation}{He hath on His garment and on His thigh written: King of kings and Lord of lords.  To Him be glory and empire for ever and ever.}
\def\magsolemn{T}
\definemag{7}{a}
}
%\def\begincollectcols{\begin{parcolumns}[rulebetween,colwidths={1=0.45\linewidth}]{2}}

\printvespers[../OctoberLastSunday-ChristTheKing]{inc-ChristTheKing}
\noindent
\printnote{If today is 31 October, \emph{First Vespers of All Saints} is commemorated with \emph{Magnificat antiphon}, p.~\pageref{allsaints1-magnificat}; \emph{\Vbar{}~Lætámini}.~in simple commemoration tone, p.~\pageref{allsaints1-vr}; and \emph{Collect}, p.~\pageref{allsaints-collect}.}

\bigskip
\benedicamusdomino[1]{}
}

%Nov 1: All Saints (1st class)
{
\section{November 1: All Saints}
\subtitle{\nth{1} Class, White or Gold}
\subtitle{I \& II Vespers}
\medskip

\def\deusinadjutoriumsolemn{T}
\ifthenelse{\boolean{birmingham}}{
	\def\prerepeatantiphonone{}	
	\def\preanttwo{\vspace{-0.2\baselineskip}}
	\def\prepsalmtitletwo{\vspace{-0.1\baselineskip}}
	\def\postpsalmtitlethree{\needspace{8\baselineskip}}
	\def\prepsalmthreeverses{\vspace{-0.05\baselineskip}}
	\def\prerepeatantiphonthree{}
}{
	\def\postdeusinadjutorium{\pagebreak}
	\def\postpsalmtitleone{\oldneedspace{12\baselineskip}}
	%\def\prepsalmoneverses{\vspace{-0.2\baselineskip}}
	%\def\presalmtitleone{\vspace{-0.8\baselineskip}}
	\def\prerepeatantiphonone{}
	%\def\preanttwo{\vspace{-0.5\baselineskip}}
	%\def\prepsalmtitletwo{\vspace{-0.3\baselineskip}}
	\def\postpsalmtitletwo{\needspace{12\baselineskip}}
	\def\prepsalmtitlethree{\vspace{-\baselineskip}}
	\def\prerepeatantiphontwo{}
	\def\postpsalmtitlefour{\needspace{12\baselineskip}}
}
\def\prevespers{
  \let\oldthinga=\antonetranslation
  \def\antonetranslation{\vspace{-0.2\baselineskip}\oldthinga\vspace{-0.5\baselineskip}}
  %\let\oldthing=\anttwotranslation
  %\def\anttwotranslation{\vspace{-0.5\baselineskip}\oldthing}
}
\def\definevesperspropers{\definepsalm{5}{115}{8}{G}

\newcommand{\vrtex}{vrExsultabunt}
\newcommand{\vtranslation}{The Saints will rejoice in glory.}
\newcommand{\rtranslation}{They will be joyful upon their beds.}

\newcommand{\maganttex}{an--o_quam_gloriosum--solesmes}
\newcommand{\magantinitial}{O}
\newcommand{\maganttranslation}{Oh! how glorious is the kingdom where all the Saints rejoice with Christ; clothed in white robes, they follow the Lamb whithersoever he goeth!}
\def\magsolemn{T}
\definemag{6}{F}

  %\def\prepsalmtitlefive{\bigskip}
  \def\prepsalmtitlefive{\vspace{-0.5\baselineskip}}
}
\def\definevesperspropersalt{\definepsalm{5}{116}{8}{G}

\newcommand{\vrtex}{vrLaetamini}
\newcommand{\vtranslation}{Be glad in the Lord, and rejoice ye righteous.}
\newcommand{\rtranslation}{And shout for joy, all ye that are upright in heart.}

\newcommand{\maganttex}{an--angeli_archangeli_all_saints--solesmes}
\newcommand{\magantinitial}{A}
\newcommand{\maganttranslation}{O ye Angels, Archangels, Thrones and Dominions, Principalities and Powers, Virtues, Cherubim and Seraphim, Patriarchs and Prophets, holy Teachers of the Law, all Apostles, Martyrs of Christ, holy Confessors, Virgins of the Lord, Hermits, and all Saints, intercede for us.}
\def\magsolemn{T}
\definemag{1}{D}

  \def\prepsalmtitlefive{\vspace{-0.5\baselineskip}}
  \def\prerepeatantiphonfive{}
  \def\premag{\label{allsaints1-magnificat}}
}
\def\vraltlabel{allsaints1-vr}
\def\vesperspropersnote{%
	\ifthenelse{\boolean{birmingham}}{%
	}{%
		\oldneedspace{15\baselineskip}At II Vespers:%
	}%
}
\def\vesperspropersaltnote{At I Vespers:}
\def\prechapter{\vspace{-\baselineskip}}
\def\begincollectcols{\label{allsaints-collect}\begin{parcolumns}[rulebetween]{2}}

\printvespers[../November1-AllSaints]{inc-AllSaints}

\medskip
%an--vidi_dominum--solesmes is for commem of saturday
\printnote{\sundaycommemnote{}

\begin{multicols}{2}
\noindent\emph{\nth{21} Sunday after Pentecost}, p.~\pageref{pentecost21-mag}.\\
\emph{\nth{22} Sunday after Pentecost}, p.~\pageref{pentecost22-mag}.\\
\emph{\nth{23} Sunday after Pentecost}, p.~\pageref{pentecost23-mag}.\\
\emph{\nth{4} Sunday after Epiphany}, p.~\pageref{epiphany4-mag}.
\end{multicols}
}
\bigskip{}
\benedicamusdomino{}
}

%Nov 9: Dedication of Archbasilica of Holy Saviour (2nd class)
{
\section{November 9: Dedication of Archbasilica of Holy Savior}
\subtitle{\nth{2} Class, White}
\subtitle{I \& II Vespers}
\medskip

\printnote{All from the Common of the Dedication of a Church.}

\def\prepsalmtitleone{\needspace{8\baselineskip}}
\def\postpsalmtitletwo{\needspace{8\baselineskip}}
\def\prepsalmtwoverses{\vspace{-0.05\baselineskip}}
\def\prerepeatantiphontwo{}
\def\postpsalmtitlethree{\medskip\needspace{10\baselineskip}}
\def\prepsalmthreeverses{\medskip}
%\def\prepsalmthreeverses{\vspace{-0.1\baselineskip}}
%\def\prerepeatantiphonthree{}
%\def\prerepeatantiphonfive{}
%\def\prepsalmtitlefive{\bigskip}
\def\prepsalmfiveverses{\smallskip}
%\def\prevespers{
%  \let\oldthing=\antfivetranslation
%  \def\antfivetranslation{\vspace{-0.6\baselineskip}\oldthing\vspace{-1\baselineskip}}
%}
\def\prevr{\needspace{10\baselineskip}}
\def\definevesperspropers{\newcommand{\vrtex}{vrDomumTuam}
\newcommand{\vtranslation}{Holiness becometh thy house, O Lord.}
\newcommand{\rtranslation}{Forever.}

\newcommand{\maganttex}{MagnificatAntiphon2-OQuamMetuendusEst}
\newcommand{\magantinitial}{O}
\newcommand{\maganttranslation}{How dreadful is this place. Surely this is none other but the house of God, and the gate of heaven.}
\def\magsolemn{T}
\def\magoneline{T}
\definemag{6}{F}

  \def\premagverses{\oldneedspace{15\baselineskip}}
}
\def\definevesperspropersalt{\newcommand{\vrtex}{vrHaecEstDomusDomini}
\newcommand{\vtranslation}{This is the house of the Lord, strongly built.}
\newcommand{\rtranslation}{It is well founded upon strong rock.}

\newcommand{\maganttex}{an--sanctificavit_dominus--solesmes}
\newcommand{\magantinitial}{S}
\newcommand{\maganttranslation}{The Lord has hallowed His dwelling; for this is the house of God; there they call on His Name, of which it is written: And My Name shall be there, saith the Lord.}
\def\magsolemn{T}
\definemag{1}{g}

}
\def\vesperspropersnote{At II Vespers:}
\def\vesperspropersaltnote{At I Vespers:}

\printvespers[../CommonOfDedicationOfChurch]{inc-DedicationOfChurch}
\bigskip
\benedicamusdomino[2]{}
}
}
}
}{

	\clearpage

	\makeatletter  
\newcounter{score}
\newcounter{tabstop}[score]
\newcommand{\grealign}{%
  \@bsphack%
  \ifgre@boxing\else%
    \kern\gre@dimen@begindifference%
    \stepcounter{tabstop}%
    \expandafter\zsavepos{stop-\thescore-\thetabstop}%
    \kern-\gre@dimen@begindifference%
  \fi%
  \@esphack%
}

\newcommand{\setstops}{%
  \gdef\nstabbing@stops{%
    \checkoddpage
    \hspace*{-\ifoddpage\oddsidemargin\else\evensidemargin\fi}\hspace{-1in}%
    \hspace*{\zposx{stop-\thescore-1} sp}\=%
  }%
  \count@=\@ne
  \loop\ifnum\count@<\value{tabstop}%
    \begingroup\edef\x{\endgroup
      \noexpand\g@addto@macro\noexpand\nstabbing@stops{%
        \noexpand\hspace{-\noexpand\zposx{stop-\thescore-\the\count@} sp}%
        \noexpand\hspace{\noexpand\zposx{stop-\thescore-\the\numexpr\count@+1} sp}\noexpand\=%
      }%
    }\x
    \advance\count@\@ne
  \repeat
  \nstabbing@stops\kill
}
\makeatother

\newenvironment{nstabbing}
  {\setlength{\topsep}{0pt}%
   \setlength{\partopsep}{0pt}%
   \tabbing%
   \setstops}
  {\endtabbing\stepcounter{score}}

\def\beginpsalmcols{\begin{parcolumns}[rulebetween]{2}}
\def\endpsalmcols{\end{parcolumns}}
\newcommand{\printpsalmtitle}[1]{%
    {\addfontfeature{Numbers=Lining}\centering Psalm #1\par}
  }

\newcommand{\printpsalmwithtranslation}[3]{
	% #1 psalm number
	% #2 ending
	% #3 0 to start on 1st verse, 1 for second
	\setlength{\columnsep}{18pt}
	\setlength{\columnseprule}{.4pt}
	\beginpsalmcols
	\colchunk{\vspace{-12pt}%
	\begin{psalmverses}[#3]
	\vspace{-\baselineskip}%
	\input{psalms/Psalm#1-#2-verses}
	\end{psalmverses}
	}

	\selectlanguage{american}
	\colchunk{\vspace{-12pt}%
	\sloppy
	\begin{psalmverses}
	\vspace{-\baselineskip}%
	\input{psalms/Psalm#1-verses-en}
	\end{psalmverses}
	}
	\selectlanguage{latin}
	\endpsalmcols
}

\renewcommand{\printpsalm}[3]{
  % #1 psalm number
  % #2 ending
  % #3 0 to start on 1st verse, 1 for second
  \setlength{\columnsep}{18pt}
  \setlength{\columnseprule}{.4pt}
  \begin{multicols}{2}
  \begin{psalmverses}[#3]
  \input{psalms/Psalm#1-#2-verses}
  \end{psalmverses}
  \end{multicols}
}



\newcommand{\printpsalmtranslation}[3]{
  % #1 psalm number
  % #2 ending
  % #3 0 to start on 1st verse, 1 for second
  \setlength{\columnsep}{18pt}
  \setlength{\columnseprule}{.4pt}
  \begin{multicols}{2}
  \begin{psalmverses}[#3]
  \input{psalms/Psalm#1-verses-en}
  \end{psalmverses}
  \end{multicols}
}

\def\beginchaptercols{\begin{parcolumns}[rulebetween]{2}}
\newcommand{\printchapternew}[4]{
	% #1 ref
	% #2 latin first words
	% #3 latin remaining
	% #4 english text
	{\noindent\hspace{3em}Chapter.\hfill\emph{#1}\hspace{3em}}
    \beginchaptercols{}
    \colchunk{\sloppy\dropcap{latin}{#2} #3\hfill{}\Rbar{}~Deo~grátias.}
    \colchunk{\sloppy \dropcap{american}{#4}}
    \end{parcolumns}
}
\newcommand{\printvrwithtranslation}{
    {\normalsize
    \ifx\beginvrcols\undefined\def\beginvrcols{\begin{parcolumns}[rulebetween]{2}}\fi
    \beginvrcols
    \colchunk{
      \par\vspace{-\baselineskip}\noindent\selectlanguage{latin}%
      \Vbar{}~\vlatin{}
    }
    \colchunk{%
      \par\vspace{-\baselineskip}\noindent\selectlanguage{american}%
      \Vbar{}~\vtranslation{}
    }%
    \colplacechunks%
    \ifx\rlatin\undefined\else
    \colchunk{
      \par\vspace{-\baselineskip}\noindent\selectlanguage{latin}%
      \Rbar{}~\rlatin{}%
    }
    \colchunk{%
      \par\vspace{-\baselineskip}\noindent\selectlanguage{american}%
      \Rbar{}~\rtranslation{}%
    }%
    \colplacechunks%
    \fi
    \end{parcolumns}
    }
}
\newcommand{\printtercepsalms}[1]{
  % #1 is psalm tone like 2D
  \printpsalmtitle{118. III.}

  %print first verse in chant
  {
    \gresetinitiallines{0}
    \grechangedim{interwordspacetext}{0.15 cm plus 0.15 cm minus 0.10 cm}{scalable}%
    \gregorioscore{psalms/Psalm118.3-#1}
    % the additional width of the additional lines (compared to the width of the glyph they're associated with)
\grechangedim{additionallineswidth}{0.14584 cm}{scalable}%
% width of the additional lines, used only for the custos (maybe should depend on the width of the custos...)
% the width is the one for the custos at end of lines, the line for custos in the middle of a score is the same
% multiplied by 2.
\grechangedim{additionalcustoslineswidth}{0.09114 cm}{scalable}%
% null space
\grechangedim{zerowidthspace}{0 cm}{scalable}%
% space between glyphs in the same element
\grechangedim{interglyphspace}{0.06927 cm plus 0.00363 cm minus 0.00363 cm}{scalable}%
% space between an alteration (flat or natural) and the next glyph
\grechangedim{alterationspace}{0.07747 cm plus 0.01276 cm minus 0.00455 cm}{scalable}%
% space between a clef and a flat (for clefs with flat)
\grechangedim{clefflatspace}{0.05469 cm plus 0.00638 cm minus 0.00638 cm}{scalable}%
% space before a choral sign
\grechangedim{beforelowchoralsignspace}{0.04556 cm plus 0.00638 cm minus 0.00638 cm}{scalable}%
% when bolshifts are enabled, minimal space between a clef at the beginning of the line and a leading alteration glyph (should be larger than clefflatspace so that a flatted clef can be distinguished from a flat which is part of the first glyph on a line, but also smaller than spaceafterlineclef, the distance from the clef to the first notes)
\grechangedim{beforealterationspace}{0.1 cm}{scalable}%
% space between elements
\grechangedim{interelementspace}{0.06927 cm plus 0.00182 cm minus 0.00363 cm}{scalable}%
% larger space between elements
\grechangedim{largerspace}{0.10938 cm plus 0.01822 cm minus 0.00911 cm}{scalable}%
% space between elements in ancient notation
\grechangedim{nabcinterelementspace}{0.06927 cm plus 0.00182 cm minus 0.00363 cm}{scalable}%
% larger space between elements in ancient notation
\grechangedim{nabclargerspace}{0.10938 cm plus 0.01822 cm minus 0.00911 cm}{scalable}%
% space between elements which has the size of a note
\grechangedim{glyphspace}{0.21877 cm plus 0.01822 cm minus 0.01822 cm}{scalable}%
% space before custos
\grechangedim{spacebeforecustos}{0.1823 cm plus 0.31903 cm minus 0.0638 cm}{scalable}%
% space before punctum mora and augmentum duplex
\grechangedim{spacebeforesigns}{0.05469 cm plus 0.00455 cm minus 0.00455 cm}{scalable}%
% space after punctum mora and augmentum duplex
\grechangedim{spaceaftersigns}{0.08203 cm plus 0.0082 cm minus 0.0082 cm}{scalable}%
% space after a clef at the beginning of a line
\grechangedim{spaceafterlineclef}{0.27345 cm plus 0.14584 cm minus 0.01367 cm}{scalable}%
% minimal space between notes of different words
%\grechangedim{interwordspacenotes}{0.27 cm plus 0.15 cm minus 0.05 cm}{scalable}%
\grechangedim{interwordspacenotes}{0.27 cm plus 0.08 cm minus 0.05 cm}{scalable}%
% minimal space between notes of the same syllable.
% Warning: always keep minus to 0; also keep plus very low, or some words won't be hyphenated
%\grechangedim{intersyllablespacenotes}{0.24 cm plus 0.04cm minus 0cm}{scalable}%
\grechangedim{intersyllablespacenotes}{0.24 cm plus 0.04cm minus 0cm}{scalable}%
% minimal space between letters of different words. Makes sense to have
% the same plus and minus as interwordspacenotes.
%\grechangedim{interwordspacetext}{0.38 cm plus 0.15 cm minus 0.05 cm}{scalable}%
\grechangedim{interwordspacetext}{0.18 cm plus 0.08 cm minus 0.05 cm}{scalable}%
% Versions of interword spaces for euouae blocks
%\grechangedim{interwordspacenotes@euouae}{0.19 cm plus 0.1 cm minus 0.05 cm}{scalable}%
\grechangedim{interwordspacenotes@euouae}{0.13 cm plus 0.1 cm minus 0.05 cm}{1}%
%\grechangedim{interwordspacetext@euouae}{0.27 cm plus 0.1 cm minus 0.05 cm}{scalable}%
\grechangedim{interwordspacetext@euouae}{0.13 cm plus 0.1 cm minus 0.05 cm}{1}%
% space between notes of a bivirga or trivirga
\grechangedim{bitrivirspace}{0.06927 cm plus 0.00182 cm minus 0.00546 cm}{scalable}%
% space between notes of a bistropha or tristrophae
\grechangedim{bitristrospace}{0.06927 cm plus 0.00182 cm minus 0.00546 cm}{scalable}%
% space between two punctum inclinatum
\grechangedim{punctuminclinatumshift}{-0.03918 cm plus 0.0009 cm minus 0.0009 cm}{scalable}%
% space before puncta inclinata
\grechangedim{beforepunctainclinatashift}{0.05286 cm plus 0.00728 cm minus 0.00455 cm}{scalable}%
% space between a punctum inclinatum and a punctum inclinatum deminutus
\grechangedim{punctuminclinatumanddebilisshift}{-0.02278 cm plus 0.0009 cm minus 0.0009 cm}{scalable}%
% space between two punctum inclinatum deminutus
\grechangedim{punctuminclinatumdebilisshift}{-0.00728 cm plus 0.0009 cm minus 0.0009 cm}{scalable}%
% space between puncta inclinata, larger ambitus (range=3rd)
\grechangedim{punctuminclinatumbigshift}{0.07565 cm plus 0.0009 cm minus 0.0009 cm}{scalable}%
% space between puncta inclinata, larger ambitus (range=4th -or more?-)
\grechangedim{punctuminclinatummaxshift}{0.17865 cm plus 0.0009 cm minus 0.0009 cm}{scalable}%
% space for the bars (inside syllables)
%first for virgula and divisio minima
\grechangedim{spacearoundsmallbar}{0.1823 cm plus 0.22787 cm minus 0.00469 cm}{scalable}%
%then divisio minor
\grechangedim{spacearoundminor}{0.1823 cm plus 0.22787 cm minus 0.00469 cm}{scalable}%
%divisio major
\grechangedim{spacearoundmaior}{0.1823 cm plus 0.22787 cm minus 0.00469 cm}{scalable}%
%divisio finalis
\grechangedim{spacearoundfinalis}{0.1823 cm plus 0.22787 cm minus 0.00469 cm}{scalable}%
%a special space for finalis, for when it is the last glyph
\grechangedim{spacebeforefinalfinalis}{0.29169 cm plus 0.07292 cm minus 0.27345 cm}{scalable}%
% additional space that will appear around bars that are preceded by a custos and followed by a key.
\grechangedim{spacearoundclefbars}{0.03645 cm plus 0.00455 cm minus 0.0009 cm}{scalable}%
% space between the text and the text of the bar
\grechangedim{textbartextspace}{0.24611 cm plus 0.13672 cm minus 0.04921 cm}{scalable}%
% minimal space between a note and a bar
\grechangedim{notebarspace}{0.31903 cm plus 0.27345 cm minus 0.02824 cm}{scalable}%
% maximal space between two syllables for which we consider a dash is not needed
\grechangedim{maximumspacewithoutdash}{0.00 cm}{scalable}%
% an extensible space for the beginning of lines
\grechangedim{afterclefnospace}{0 cm plus 0.27345 cm minus 0 cm}{scalable}%
% space between the initial and the beginning of the score
\grechangedim{afterinitialshift}{0.2457 cm}{scalable}%
% space before the initial
\grechangedim{beforeinitialshift}{0.2457 cm}{scalable}%
% when bolshifts are enabled, minimum space between beginning of line and first syllable text
\grechangedim{minimalspaceatlinebeginning}{0.05 cm}{scalable}%
% space to force the initial width to.  Ignored when 0.
\grechangedim{manualinitialwidth}{0 cm}{scalable}%
% distance to move the initial up by
\grechangedim{initialraise}{0 cm}{scalable}%
% Space between lines in the annotation
\grechangedim{annotationseparation}{0.05cm}{scalable}%
% Amount to raise (positive) or lower (negative) the annotations from the default position (base line of top annotation aligned with top line of staff)
\grechangedim{annotationraise}{0cm}{scalable}%
% space at the beginning of the lines if there is no clef
\grechangedim{noclefspace}{0.1 cm}{scalable}%
% space around a clef change
\grechangedim{clefchangespace}{0.01768 cm plus 0.00175 cm minus 0.01768 cm}{scalable}%
%When \gre@clivisalignment is 2, this distance is the maximum length of the consonants after vowels for which the clivis will be aligned on its center.
\grechangedim{clivisalignmentmin}{0.3 cm}{scalable}%



%%%%%%%%%%%%%%%%%%
% vertical spaces
%%%%%%%%%%%%%%%%%%

% first, we have two spaces for the chironomic signs
\grechangedim{abovesignsspace}{0.8 cm}{scalable}%
\grechangedim{belowsignsspace}{0 cm}{scalable}%
% the amount to shift down:
% (a) low choral signs that are not lower than the note, regardless of whether
%     it's on a line or in a space
% (b) high choral signs and low choral signs that are lower than the note which
%     are in a space
\grechangedim{choralsigndownshift}{0.00911 cm}{scalable}%
% the amount to shift up:
% (a) high choral signs and low choral signs that are lower than the note which
%     are on a line
\grechangedim{choralsignupshift}{0.04556 cm}{scalable}%
% the space for the translation
\grechangedim{translationheight}{0.5 cm}{scalable}%
%the space above the lines
\grechangedim{spaceabovelines}{0.45576 cm plus 0.36461 cm minus 0.09114 cm}{scalable}%
%the space between the lines and the bottom of the text
\grechangedim{spacelinestext}{0.60617 cm}{scalable}%
%the space beneath the text
\grechangedim{spacebeneathtext}{0 cm}{scalable}%
% height of the text above the note line
\grechangedim{abovelinestextraise}{-0.1 cm}{scalable}%
% height that is added at the top of the lines if there is text above the lines (it must be bigger than the text for it to be taken into consideration)
\grechangedim{abovelinestextheight}{0.3 cm}{scalable}%
% an additional shift you can give to the brace above the bars if you don't like it
\grechangedim{braceshift}{0 cm}{scalable}%
% a shift you can give to the accentus above the curly brace
\grechangedim{curlybraceaccentusshift}{-0.05 cm}{scalable}%


    \greseteolcustos{auto}
  }

  \setlength{\columnsep}{18pt}
  \setlength{\columnseprule}{.4pt}
  \selectlanguage{latin}
  \begin{multicols}{2}
  \begin{psalmverses}[1]
  \input{psalms/Psalm118.3-#1-verses}
  \end{psalmverses}
  \printpsalmtitle{118. IV.}
  \vspace{-0.8\baselineskip}
  \begin{psalmverses}
  \input{psalms/Psalm118.4-#1-verses}
  \end{psalmverses}
  \printpsalmtitle{118. V.}
  \vspace{-0.8\baselineskip}
  \begin{psalmverses}
  \input{psalms/Psalm118.5-#1-verses}
  \end{psalmverses}
  \end{multicols}
}
\newcommand{\printtercepsalmstranslation}{
  \setlength{\columnsep}{18pt}
  \setlength{\columnseprule}{.4pt}
  \selectlanguage{american}
  \begin{multicols}{2}
  \printpsalmtitle{118. III.}
  \vspace{-0.5\baselineskip}
  \begin{psalmverses}
  \item Set before me for a law the way of thy justifications, O Lord: * and I will always seek after it.
\item Give me understanding, and I will search thy law; * and I will keep it with my whole heart.
\item Lead me into the path of thy commandments; * for this same I have desired.
\item Incline my heart into thy testimonies * and not to covetousness.
\item Turn away my eyes that they may not behold vanity: * quicken me in thy way.
\item Establish thy word to thy servant, * in thy fear.
\item Turn away my reproach, which I have apprehended: * for thy judgments are delightful.
\item Behold I have longed after thy precepts: * quicken me in thy justice.
\item Let thy mercy also come upon me, O Lord: * thy salvation according to thy word.
\item So shall I answer them that reproach me in any thing; * that I have trusted in thy words.
\item And take not thou the word of truth utterly out of my mouth: * for in thy words, I have hoped exceedingly.
\item So shall I always keep thy law, * for ever and ever.
\item And I walked at large: * because I have sought after thy commandments.
\item And I spoke of thy testimonies before kings: * and I was not ashamed.
\item I meditated also on thy commandments, * which I loved.
\item And I lifted up my hands to thy commandments, which I loved: * and I was exercised in thy justifications.
  \end{psalmverses}
  \printpsalmtitle{118. IV.}
  \vspace{-0.5\baselineskip}
  \begin{psalmverses}
  \item Be thou mindful of thy word to thy servant, * in which thou hast given me hope.
\item This hath comforted me in my humiliation: * because thy word hath enlivened me.
\item The proud did iniquitously altogether: * but I declined not from thy law.
\item I remembered, O Lord, thy judgments of old: * and I was comforted.
\item A fainting hath taken hold of me, * because of the wicked that forsake thy law.
\item Thy justifications were the subject of my song, * in the place of my pilgrimage.
\item In the night I have remembered thy name, O Lord: * and have kept thy law.
\item This happened to me: * because I sought after thy justifications.
\item O Lord, my portion, * I have said, I would keep thy law.
\item I entreated thy face with all my heart: * have mercy on me according to thy word.
\item I have thought on my ways: * and turned my feet unto thy testimonies.
\item I am ready, and am not troubled: * that I may keep thy commandments.
\item The cords of the wicked have encompassed me: * but I have not forgotten thy law.
\item I rose at midnight to give praise to thee; * for the judgments of thy justification.
\item I am a partaker with all them that fear thee, * and that keep thy commandments.
\item The earth, O Lord, is full of thy mercy: * teach me thy justifications.
  \end{psalmverses}
  \printpsalmtitle{118. V.}
  \vspace{-0.5\baselineskip}
  \begin{psalmverses}
  \item Thou hast done well with thy servant, O Lord, * according to thy word.
\item Teach me goodness and discipline and knowledge; * for I have believed thy commandments.
\item Before I was humbled I offended; * therefore have I kept thy word.
\item Thou art good; * and in thy goodness teach me thy justifications.
\item The iniquity of the proud hath been multiplied over me: * but I will seek thy commandments with my whole heart.
\item Their heart is curdled like milk: * but I have meditated on thy law.
\item It is good for me that thou hast humbled me, * that I may learn thy justifications.
\item The law of thy mouth is good to me, * above thousands of gold and silver.
\item Thy hands have made me and formed me: * give me understanding, and I will learn thy commandments.
\item They that fear thee shall see me, and shall be glad: * because I have greatly hoped in thy words.
\item I know, O Lord, that thy judgments are equity: * and in thy truth thou hast humbled me.
\item O! let thy mercy be for my comfort, * according to thy word unto thy servant.
\item Let thy tender mercies come unto me, and I shall live: * for thy law is my meditation.
\item Let the proud be ashamed, because they have done unjustly towards me: * but I will be employed in thy commandments.
\item Let them that fear thee turn to me: * and they that know thy testimonies.
\item Let my heart be undefiled in thy justifications, * that I may not be confounded.
  \end{psalmverses}
  \end{multicols}
}

{
  \label{advent}
  \chapter{Sunday at Terce}
  \section{Sunday at Terce throughout the year}

  \printgabc{\Vbar}{}{D}{deus-in-adjutorium}

  \bigskip

  {\centering Hymn.\par}

  {\centering 1.~On Ordinary Sundays.\par}

  \printgabc{2.}{}{N}{hy--nunc_sancte_nobis_(in_dominicis_per_annum)--solesmes_1961}

  \oldneedspace{10\baselineskip}
  {\centering 2.~On Solemn Feasts.\par}

  \printgabc{8.}{}{N}{hy--nunc_sancte_nobis_(in_festis)--solesmes_1961}

  \begin{multicols}{2}
  \begin{psalmverses}
\item Come Holy Ghost, who ever One
Art with the Father and the Son,
It is the hour, our souls possess
With thy full flood of holiness.

\item Let flesh and heart and lips and mind
Sound forth our witness to mankind;
And love light up our mortal frame,
Till others catch the living flame.

\item Almighty Father, hear our cry,
Through Jesus Christ, our Lord most High,
Who, with the Holy Ghost and thee,
Doth live and reign eternally.
Amen.
  \end{psalmverses}
  \end{multicols}

\bigskip



% print antiphon
%    \printgabc[\preant]{\antlineone}{\antlinetwo}{\antinitial}{\anttex}
  \bigskip
  \printgabc{Ant.}{2. D}{A}{an--alleluia._(sund._at_terce)--solesmes}
% print translation
  \translation[]{Alleluia, Lead me into the path of thy commandments, alleluia, alleluia.}

\bigskip
  \printtercepsalms{2D}

  \bigskip
  \printtercepsalmstranslation{}

% \printpsalm{118.3}{2D}{1}
% \printpsalmtitle{118. IV.}
% \printpsalm{118.4}{2D}{0}
% \printpsalmtitle{118. V.}
% \printpsalm{118.5}{2D}{0}

% print antiphon
%    \printgabc[\preant]{\antlineone}{\antlinetwo}{\antinitial}{\anttex}
  \printgabc{Ant.}{2. D}{A}{an--alleluia._(sund._at_terce)--solesmes}
% print translation
  \translation[]{Alleluia, Lead me into the path of thy commandments, alleluia, alleluia.}

\bigskip
\printchapternew{1. John 4.}{Deus cáritas}{est~:~\dag{} et qui manet in caritáte, in Deo manet,~* et Deus in eo.}{God is charity: and he that abideth in charity, abideth in God, and God in him.}

\bigskip
\printgabc{Short}{Resp.}{I}{re--inclina_cor_meum--solesmes}

\translation[]{\Vbar{}~Incline my heart into thy testimonies.
\Rbar{}~Incline\dots{}
\Vbar{}~Turn away my eyes that they may not behold vanity: quicken me in thy way.
\Rbar{}~Into thy testimonies.
\Vbar{}~Glory be to the Father, and to the Son, and to the Holy Ghost.
\Vbar{}~Incline\dots{}}

\bigskip
\gresetinitiallines{0}
\gregorioscore{vr-ego-dixi}
\newlength{\myhwidth}
\settowidth{\myhwidth}{tib}
\begin{nstabbing}
\>\Rbar{}~Sana ánimam meam, quia peccávi \>\hspace{-\myhwidth}tibi.\\
\end{nstabbing}

\translation[]{\Vbar{}~I said: O Lord, be thou merciful to me.\\
\Rbar{}~Heal my soul, for I have sinned against thee.}

\bigskip
\def\beginvrcols{\begin{parcolumns}[rulebetween,colwidths={1=0.48\linewidth}]{2}}
\def\vlatin{Dóminus vobíscum.}
\def\rlatin{Et cum spíritu tuo.}
\def\vtranslation{The Lord be with you.}
\def\rtranslation{And with thy spirit.}
{\centering Collect.\par}
\printvrwithtranslation{}

\bigskip
\emph{See proper of time.}

\bigskip
\printvrwithtranslation{}
\bigskip

\gresetinitiallines{0}
\gregorioscore{vr-benedicamus-domino}
\bigskip

\def\beginvrcols{\begin{parcolumns}[rulebetween,colwidths={1=0.44\linewidth}]{2}}
\def\vlatin{Fidélium ánimæ per misericórdiam Dei requiéscant in pace.\hfill{}Amen.}
\let\rlatin=\undefined
\def\vtranslation{May the souls of the faithful departed through the mercy of God rest in peace.\hfill{}Amen.}
\let\rtranslation=\undefined
\printvrwithtranslation{} 
}

	\chapter{Proper of the Time -- Advent Season}
{
\newcommand{\benedicamusdomino}[1][advent]{
	\benedicamusdominomaster{#1}
}
\newcommand{\printhymnnote}{
	\noindent\printnote{Hymn.~\emph{Creátor alme síderum}, page \pageref{hymn-creatoralmesiderum}.
	\Vbar~\emph{Roráte}, page \pageref{vr-rorate}.}
}
\newcommand{\printoant}[2]{
%	#1 number
%	#2 translation
{
\oldneedspace{5\baselineskip}
\subtitle{December #1.}
\printgabc{\small At Magn.}{\small \oldstylenums{Ant.~2.~D}}{O}{December#1-MagAntiphon}
\translation[]{#2}
\medskip
\emph{\emph{Magnificat}.~page \pageref{oantiphon-magnificat}.}
}
}

\normalsize
%\def\nogloriapatri{T}
%\def\breakbeforeEuouae{T}
%1st sunday of advent
{
\section{First Sunday of Advent}
\subtitle{\nth{1} Class, Violet}

\def\premag{\def\noeuouae{T}}
\def\premagverses{\greseteolcustos{manual}}
\def\printfullhymn{
	\label{hymn-creatoralmesiderum}
	{
	\def\gabcfolder{../Advent}
	\def\prehymntranslation{\vspace{-\baselineskip}}
	\printhymn{\oldstylenums{4.}}{C}{hymn-CreatorAlmeSiderum}
	{\item Creator of the stars of night
	Thy people's everlasting light,
	Jesu, Redeemer, save us all,
	And hear Thy servants when they call.

	\item Thou, lest the demon's ancient curse
	Should doom to death a universe,
	In love wast made, Thyself alone,
	The means to save a world undone.

	\item Towards the Cross Thou wentest forth,
	That Thou might'st heal the crimes of earth;
	Proceeding from a virgin shrine,
	The spotless Victim all divine.

	\item At whose dread Name, majestic now,
	All knees must bend, all hearts must bow;
	And things celestial Thee shall own,
	And things terrestrial, Lord alone.

	\item O Thou, whose coming is with dread,
	To judge and doom the quick and dead.
	Thy heavenly grace on us bestow,
	To shield us from our ghostly foe.

	\item To God the Father, God the Son,
	And God the Spirit, Three in One,
	Laud, honour, might, and glory be
	From age to age eternally.
	Amen.}

	{
		\def\vrlinebreak{T}
		\label{vr-rorate}
		\printvr[\greseteolcustos{manual}]{vr}
		{Ye heavens, drop down dew from above, and let the clouds rain down the Just One.}
		{Let the earth open and bud forth the Saviour.}
	}
	}
}
\def\prepsalmfive{\greseteolcustos{manual}}
\def\prevespers{%
	\let\oldthing\antfourtranslation
	\def\antfourtranslation{\oldthing\vspace{-\baselineskip}}%
	\let\oldthingb\antfivetranslation
	\def\antfivetranslation{\oldthingb\vspace{-\baselineskip}}%
}
\printvespers[../Advent1]{inc-Advent1}
\bigskip
\benedicamusdomino{}
}



% Advent 2
{
\section{Second Sunday of Advent}
\subtitle{\nth{1} Class, Violet}
\printnote{If today is December 8, the Second Vespers of the Immaculate Conception is sung with a Commemoration of this Sunday, page \pageref{immaculateconception}.}
\medskip
\def\preanttwo{\needspace{5\baselineskip}}
\def\prepsalmthree{\needspace{5\baselineskip}}
\def\prepsalmfive{\greseteolcustos{manual}}
\def\premag{\def\noeuouae{T}}
\def\premagverses{\greseteolcustos{manual}}
%\def\prechapter{\vspace{-\baselineskip}}
\printvespers[../Advent2]{inc-Advent2-Vespers2}

\noindent
\printnote{If today is December 7, the First Vespers of the Immaculate Conception is commemorated as follows.  Otherwise \benedicamusdominoreference{advent}}
\bigskip
\hrule
}



% commemoration of First Vespers of Immaculate Conception
{
\def\beginvrcols{\begin{parcolumns}[rulebetween,colwidths={1=0.45\linewidth}]{2}}
\printcommemoration[../December8-ImmaculateConception]{commemorationImmaculateConception-Vespers1}

\bigskip
\benedicamusdomino{}
}


% 3rd Sunday of Advent
{
\section{Third Sunday of Advent}
\subtitle{\nth{1} Class, Violet or Rose}

\def\prepsalmfive{\greseteolcustos{manual}}
\def\preantfour{\needspace{15\baselineskip}}
\def\premag{\def\noeuouae{T}
\printnote{When the third Sunday of Advent falls on December 17, the Antiphon \emph{O Sapiéntia} on page \pageref{osapientia} is sung in place of \emph{Beáta es}.}
\medskip

\label{magant-beataes}
}
\def\premagverses{\greseteolcustos{manual}}
\def\postmag{
	\def\nomagtitle{T}
	\def\magsolemn{T}
	%\definemag{2}{D}
	\def\gabcfolder{../Advent}
	\def\noeuouae{T}
		%print o sapientia
	%\printoant{17}{O Wisdom, which camest out of the mouth of the Most High, reaching from end to end and ordering all things mightily and sweetly: come and teach us the way of prudence.}
	\renewcommand{\anttranslation}{O Wisdom, which camest out of the mouth of the Most High, reaching from end to end and ordering all things mightily and sweetly: come and teach us the way of prudence.}
	\oldneedspace{5\baselineskip}
	\subtitle{December 17.}
	\label{osapientia}
	\def\preverses{\greseteolcustos{manual}}
	\printmag{2D}{O}{December17-MagAntiphon}
}
\def\prechapter{\vspace{-\baselineskip}}
\def\beginchaptercols{\begin{parcolumns}[rulebetween,colwidths={1=0.48\linewidth}]{2}}
\printvespers[../Advent3]{inc-Advent3}
\bigskip
\benedicamusdomino{}
}



%Advent 4
{
\section{Fourth Sunday of Advent}
\subtitle{\nth{1} Class, Violet}
\printnote{If today is December 24, First Vespers of Christmas is used on page \pageref{christmas}.}
\medskip

\def\prepsalmtwoverses{\oldneedspace{2\baselineskip}}
\def\prepsalmfive{\greseteolcustos{manual}}
\def\magreplacement{
	%\pagebreak
	\bigskip
	{
	\def\magsolemn{T}
	\definemag{2}{D}
	\def\gabcfolder{../Advent}
	\def\noeuouae{T}
	\printoant{18}{O Adonai, and Leader of the house of Israël, who didst appear to Moses in the flame of the burning bush and didst give unto him the law on Sinai: come and with an outstretched arm redeem us.}
	\printoant{19}{O Root of Jesse, which standest for an ensign of the people, before whom kings shall keep silence, whom the Gentiles shall beseech: come and deliver us, and tarry not.}
	\printoant{20}{O Key of David, and Sceptre of the house of Israël, that openest and no man shutteth, and shuttest and no man openeth: come and bring the prisoner forth from the prison-house, and him that sitteth in darkness and in the shadow of death.}
	\printoant{21}{O Day-spring, Brightness of light eternal, and Sun of Justice, come and enlighten them that sit in darkness and in the shadow of death.}
	\printoant{22}{O King of the Gentiles and the desire thereof, Thou cornerstone that makest both one, come and deliver mankind, whom Thou didst form out of clay.}
	\printoant{23}{O Emmanuel, our King and Lawgiver, the desire of the nations and the Saviour thereof, come to save us, O Lord our God.}

	\bigskip
	\pagebreak
	\vfil
	{\greseteolcustos{manual}
	\label{oantiphon-magnificat}
	\begin{magnificat}{\magtex}
	\magverses
	\end{magnificat}
	\noindent\emph{Repeat antiphon.}}
	\vfil
	}
}
\printvespers[../Advent4]{inc-Advent4}
\bigskip
\benedicamusdomino{}
}
}

	\chapter{Proper of the Time -- Christmas Season}
{
\newcommand{\benedicamusdomino}[1][1]{
  \benedicamusdominomaster{#1}
}

{
\label{christmas}
\section{December 24: Nativity of our Lord -- I Vespers}
\subtitle{\nth{1} Class, White or Gold}

\def\deusinadjutoriumsolemn{T}
\newcommand{\printhymnnote}{
	\noindent\printnote{Hymn.~\emph{Jesu Redémptor ómnium}, page \pageref{hymn-jesuredemptoromnium}.\\}
}
\def\definevesperspropers{\newcommand{\chaptertext}{\dropcap{latin}{Appáruit benígnitas et humánitas Salvatóris nostri} \textbf{Dé}\-i~\dag{} non ex opéribus justítiæ, quæ \emph{fé\-ci}\-\textbf{mus} nos,~* sed secúndum suam misericórdiam salvos nos \textbf{fé}cit.}
\newcommand{\chaptertranslation}{The goodness and kindness of God our Saviour appeared; not by the works of justice which we have done, but according to His mercy He saved us.}

\newcommand{\vrtex}{vrCrastinaDie}
\newcommand{\vtranslation}{Tomorrow shall the iniquity of the land be blotted out.}
\newcommand{\rtranslation}{And the Saviour of the world shall reign over us.}

\newcommand{\maganttex}{an--cum_ortus_fuerit--solesmes}
\newcommand{\magantinitial}{C}
\newcommand{\maganttranslation}{When the sun has risen in the sky, you shall see the King of kings coming forth from the Father, like a bridegroom from his chamber.}
\def\magsolemn{T}
\definemag{8}{G}

\newcommand{\collect}{Concéde, quǽsumus omnípotens Deus~:~\dag{} ut nos Unigéniti tui nova per carnem Natívitas líberet;~* quos sub peccáti jugo vetústa sérvitus tenet. Per eúm\-dem Dóminum.}
\newcommand{\collecttranslation}{Grant, we beseech Thee, almighty God, that the new birth of Thine only-begotten Son in the flesh may set us free, who are held by the old bondage under the yoke of sin.  Through the same our Lord.}
%
}
\def\prepsalmtitleone{\vspace{-0.5\baselineskip}}
\def\prepsalmtitletwo{\vspace{-0.5\baselineskip}}
\def\prepsalmtitlethree{\vspace{-0.5\baselineskip}}
\def\preantfour{\needspace{10\baselineskip}}
%\def\prepsalmtitlefour{\vspace{-\baselineskip}}
\def\preantfive{\oldneedspace{10\baselineskip}}
%\def\prechapter{\vspace{-\baselineskip}}
%\def\precollect{\vspace{-0.2\baselineskip}}
\def\begincollectcols{\begin{parcolumns}[rulebetween,colwidths={1=0.44\linewidth}]{2}}
\def\prevr{\label{vrcrastina}}
\printvespers[../Christmas-Vespers1]{inc-Christmas-Vespers1-psalms}

\benedicamusdomino{}
}

{
\section[December 25: Nativity of our Lord -- II Vespers]{December 25: Nativity of our Lord -- II Vespers\sectionmark{December 25: Nativity of our Lord \& Sunday within the Octave}}
\sectionmark{December 25: Nativity of our Lord \& Sunday within the Octave}
\subtitle{\nth{1} Class, White or Gold}
\medskip
\subtitle{\&}
\vspace{-0.5\baselineskip}
\section[Sunday within the Octave]{Sunday within the Octave\sectionmark{December 25: Nativity of our Lord \& Sunday within the Octave}}
\sectionmark{December 25: Nativity of our Lord \& Sunday within the Octave}
\subtitle{\nth{2} Class, White}
\bigskip

\def\deusinadjutoriumsolemn{T}
\def\definevesperspropersalt{\newcommand{\chaptertext}{\dropcap{latin}{Multifáriam multísque modis olim Deus loquens pátribus in} pro\-\textbf{phé}\-tis~:~\dag{} novíssime diébus istis locútus est nobis in Fílio, quem constítuit hærédem ú\-\emph{ni\-ver}\-\textbf{só}\-rum,~* per quem fecit et \textbf{saé}\-cu\-la.}
\newcommand{\chaptertranslation}{God, who at sundry times and in divers manners spoke in times past to the fathers by the prophets, last of all in these days hath spoken to us by His Son, whom He hath appointed heir of all things, by whom also He made the world.}

\newcommand{\vrtex}{vrNotumFecitDominusSolemn}
\newcommand{\vtranslation}{The Lord hath made known, alleluia.}
\newcommand{\rtranslation}{His salvation, alleluia.}

\newcommand{\maganttex}{MagnificatAntiphon}
\newcommand{\magantinitial}{H}
\newcommand{\maganttranslation}{This day Christ is born: this day the Saviour hath appeared: this day the Angels sing on earth, and the Archangels rejoice: this day the just exult, saying: Glory to God in the highest, alleluia.}
\def\magsolemn{T}
\definemag{1}{g2}

\newcommand{\collect}{Concéde, quaésumus omnípotens Deus~:~\dag{} ut nos Unigéniti tui nova per carnem Natívitas líberet;~* quos sub peccáti jugo vetústa sérvitus tenet. Per eúmdem Dóminum.}
\newcommand{\collecttranslation}{Grant, we beseech Thee, almighty God, that the new birth of Thine only-begotten Son in the flesh may set us free, who are held by the old bondage under the yoke of sin.  Through the same Jesus Christ our Lord.}

  \let\oldthing\maganttranslation
  \def\maganttranslation{\oldthing\vspace{-0.5\baselineskip}}
}
\def\definevesperspropers{\newcommand{\chaptertext}{\dropcap{latin}{Fratres~: Quanto témpore hæres párvulus est, nihil differt a servo, cum sit dóminus} {\textbf{óm}\-ni\-um~:~\dag{} sed sub tutóribus et ac\-\emph{tó\-ri}\-\textbf{bus} est,~* usque ad præfinítum tempus a \textbf{pá}tre.}}
\newcommand{\chaptertranslation}{Brethren, as long as the heir is a child, he differeth nothing from a servant, though he be lord of all: but is under tutors and governors until the time appointed by the father.}

\newcommand{\vrtex}{vrVerbumCaroFactumEst}
\newcommand{\vtranslation}{The Word was made flesh, alleluia.}
\newcommand{\rtranslation}{And dwelt among us, alleluia.}

\newcommand{\maganttex}{MagAnt-PuerJesus}
\newcommand{\magantinitial}{P}
\newcommand{\maganttranslation}{The Child Jesus advanced in age and wisdom before God and men.}
\def\magsolemn{F}
\definemag{6}{F}

\newcommand{\collect}{Omnípotens sempitérne Deus, dírige actus nostros in beneplácito tuo~:~\dag{} ut in nómine dilécti Fílii tui,~* mereámur bonis opéribus abundáre.  Qui tecum vivit et regnat.}
\newcommand{\collecttranslation}{O almighty and everlasting God, direct our actions according to Thy good pleasure; that in the Name of Thy beloved Son we may deserve to abound in good works: Who with Thee liveth and reigneth.}


  \let\oldthing\maganttranslation
  \def\maganttranslation{\oldthing\vspace{-0.5\baselineskip}}
}
\def\vesperspropersaltnote{At II Vespers of Christmas:}
\def\vesperspropersnote{\needspace{4\baselineskip}At II Vespers of the Sunday within the Octave of Christmas:}
\def\prepsalmthree{\needspace{8\baselineskip}}
\def\prepsalmthreeverses{\vspace{-0.1\baselineskip}}
\def\prerepeatantiphonthree{\vspace{-0.85\baselineskip}}
\def\prechapter{\vspace{-1\baselineskip}}
\def\premag{\def\noeuouae{T}}
\def\premagtitle{\vspace{-0.3\baselineskip}}
\def\premagverses{\greseteolcustos{manual}}
\def\prevr{On \emph{Dec.~24, \emph{\Vbar{}~Crástina die}, p.~\pageref{vrcrastina}}.

\emph{On Dec.~31, \emph{\Vbar{}~Verbum} \& Jan.~1, \emph{\Vbar{}~Notum fecit}, p.~\pageref{vrverbumcarofactumest}}.
\medskip
}
\def\hymnlabel{hymn-jesuredemptoromnium}
\def\hymnlinetwo{\oldstylenums{1.}}
\def\hymntex{hymn-JesuRedemptorOmnium}
\def\hymninitial{J}
\def\hymntranslation{\item Jesus, Redeemer of the world,
Begotten ere the dawn of light,
Wast of the Father's glory born,
Immense in glory as in might.

\item Thou art the Father's splendid light,
Thou art th'eternal hope of all
Throughout the world to Thee we pray,
O, hear Thy servants as they call.

\item Remember, O creator Lord!
That from the Virgin's sacred womb
Thou didst come forth, and of her flesh
Thou didst our mortal form assume.

\item This day, recurring, year by year,
Bears witness true that all alone
To save the world Thou camest forth,
Proceeding from the Father's throne.

\item This day, the stars, the earth, and sea,
And all creation welcome sing.
This day which brought our liberty.
When came our Lord, our Saviour, King.

\item And we, too, Lord, who have been washed
In Thine own font of Blood divine,
Offer the tribute of our praise
On this blest natal day of Thine.

\item O Jesu, of the Virgin born,
Unceasing glory be to Thee;
And to the Father infinite,
And Holy Ghost eternally.  Amen.
}

\def\prehymn{%
  \grechangedim{spaceabovelines}{0.40 cm plus 0.36461 cm minus 0.09114 cm}{scalable}%
}
\def\begincollectcols{\begin{parcolumns}[rulebetween,colwidths={1=0.42\linewidth}]{2}}
\printvespers[../Christmas]{inc-Christmas-Vespers2-psalms}

\grechangedim{spaceabovelines}{0.45576 cm plus 0.36461 cm minus 0.09114 cm}{scalable}%
\benedicamusdomino{}
}

{
\section{January 1: Octave of the Nativity of our Lord}
\vspace{-0.5\baselineskip}
\subtitle{\nth{1} Class, White or Gold}

\subtitle{I \& II Vespers}
\label{circumcision}
\medskip

\def\deusinadjutoriumsolemn{T}
\def\definevesperspropers{\newcommand{\vrtex}{vrNotumFecitDominus}
\newcommand{\vtranslation}{The Lord hath made known, alleluia.}
\newcommand{\rtranslation}{His salvation, alleluia.}

\newcommand{\maganttex}{MagnificatAntiphon}
\newcommand{\magantinitial}{M}
\newcommand{\maganttranslation}{Great the mystery of our inheritance: the womb that knew not man is become the temple of God: taking flesh from her, he is not defiled: all nations shall come, and say: Glory to Thee, O Lord.}
\def\magsolemn{T}
\definemag{2}{A}
}
\def\preantthree{\needspace{10\baselineskip}}
 \newcommand{\printhymnnote}{
 	\noindent\printnote{Hymn.~\emph{Jesu Redémptor ómnium}, page \pageref{hymn-jesuredemptoromnium}.\\}
 }
\def\definevesperspropersalt{\newcommand{\vrtex}{vrVerbumCaro}
\newcommand{\vtranslation}{The Word was made flesh, alleluia.}
\newcommand{\rtranslation}{And dwelt amongst us, alleluia.}

\newcommand{\maganttex}{an--propter_nimiam--solesmes}
\newcommand{\magantinitial}{P}
\newcommand{\maganttranslation}{For His exceeding charity wherewith He loved us, God sent His own Son, in the likeness of sinful flesh, alleluia.}
\def\magsolemn{T}
\definemag{8}{G}
}
\def\vesperspropersnote{At II Vespers:}
\def\vesperspropersaltnote{At I Vespers:}
\def\prepsalmtitletwo{\needspace{6\baselineskip}}
\def\preantfive{\bigskip}
%\def\precollect{\vspace{-2\baselineskip}}
\def\prechapter{\noindent\printnote{For the Feast of the Purification \& Presentation, continue to \emph{Chapter}, p.~\pageref{purification-chapter}.}\par}
\def\premagtitle{\oldneedspace{6\baselineskip}}
\def\prevr{\label{vrverbumcarofactumest}}
\printvespers[../ChristmasOctave-Circumcision]{inc-Circumcision-Vespers-common}

\benedicamusdomino{}
}

{
\needspace{5\baselineskip}
\section{The Most Holy Name of Jesus}
\vspace{-0.5\baselineskip}
\subtitle{\nth{2} Class, White}
\medskip
%\subtitle{II Vespers}

\printnote{This feast is celebrated on the Sunday between the Octave Day of the Nativity and the Epiphany of Our Lord.  If no Sunday occurs within that time, this feast is then celebrated on January 2.}

\printnote{When the Feast of the Most Holy Name of Jesus occurs on January 5, \emph{First Vespers of the Epiphany} on page \pageref{epiphany} is sung without a commemoration of the Holy Name of Jesus.}

\medskip{}
%\def\definevesperspropersalt{\newcommand{\maganttex}{MagnificatAntiphon1}
\newcommand{\magantinitial}{T}
\newcommand{\maganttranslation}{Thou art the Shepherd of the sheep and the Prince of the Apostles, and unto thee are given the keys of the kingdom of heaven.}
\def\magsolemn{T}
\definemag{1}{f}
}
\def\definevesperspropers{\newcommand{\maganttex}{an--vocabis_nomen_ejus--solesmes}
\newcommand{\magantinitial}{V}
\newcommand{\maganttranslation}{Thou shalt call His Name Jesus, for He shall save His people from their sins, alleluia.}
\def\magsolemn{T}
\definemag{1}{g}
}
%\def\vesperspropersaltnote{At I Vespers of the Most Holy Name of Jesus:}
%\def\vesperspropersnote{At II Vespers of the Most Holy Name of Jesus:}
\def\preantone{\bigskip}
\def\prepsalmtitleone{\needspace{5\baselineskip}}
\def\preanttranslationthree{\vspace{-0.2\baselineskip}}
\def\prepsalmtitlethree{\vspace{-0.1\baselineskip}}
\def\prepsalmthreeverses{}
\def\prerepeatantiphonthree{}
\def\prepsalmtitlefour{\vspace{-0.5\baselineskip}}
\def\hymnlabel{hymn-jesudulcismemoria}
\def\begincollectcols{\begin{parcolumns}[rulebetween,colwidths={1=0.425\linewidth}]{2}}
%\def\prepsalmtitlefive{\vspace{\baselineskip}}
\def\prechapter{\vspace{-0.5\baselineskip}}
\def\prehymn{\needspace{5\baselineskip}}
\printvespers[../HolyName]{inc-HolyName-common}

\benedicamusdomino[2]{}
}

{
\chapter{Proper of the Time -- Epiphany Season}
\vspace{-0.5\baselineskip}
\label{epiphany}
\section{January 6: The Epiphany of Our Lord}
\subtitle{\nth{1} Class, White or Gold}

\subtitle{I \& II Vespers}

\def\deusinadjutoriumsolemn{T}
\def\definevesperspropersalt{\definepsalm{5}{116}{7}{c2}

\newcommand{\maganttex}{MagnificatAntiphon1}
\newcommand{\magantinitial}{M}
\newcommand{\maganttranslation}{When the Wise Men saw the star, they said one to another: This is the sign of the great King: let us go and search for Him, and offer Him gifts, gold, frankincense, and myrrh.}
\definemag{8}{G}

  %\def\prepsalmtitlefive{\medskip}
}
\def\definevesperspropers{\definepsalm{5}{113}{7}{c2}

\newcommand{\maganttex}{MagnificatAntiphon2}
\newcommand{\magantinitial}{T}
\newcommand{\maganttranslation}{This day we keep a holiday in honour of three wonders: today a star led the wise men to the manger; today water was made wine at the marriage; today Christ was pleased to be baptised in the Jordan by John for our salvation, alleluia.}
\definemag{1}{D}
%
  %\def\prepsalmtitlefive{\medskip}
  \let\oldthing\maganttranslation
  \def\maganttranslation{\oldthing\vspace{-0.25\baselineskip}}
}
\def\vesperspropersaltnote{At I Vespers:}
\def\vesperspropersnote{At II Vespers:}
\def\prepsalmthree{\needspace{5\baselineskip}}
\def\prepsalmthreeverses{\vspace{-0.05\baselineskip}}
\def\prerepeatantiphonthree{}
\def\prepsalmtitlefour{\vspace{-0.5\baselineskip}}
%\def\prechapter{\vspace{-1\baselineskip}}
\def\beginchaptercols{\begin{parcolumns}[rulebetween,colwidths={1=0.42\linewidth}]{2}}
\def\prevr{\needspace{8\baselineskip}}

\printvespers[../Epiphany]{inc-Epiphany-Vespers}

\benedicamusdomino{}
}

{
\section{Holy Family of Jesus, Mary \& Joseph}
\subtitle{First Sunday after the Epiphany.}
\subtitle{\nth{2} Class, White}

\subtitle{I \& II Vespers}
%\def\begincollectcols{\begin{parcolumns}[rulebetween,colwidths={1=0.43825\linewidth}]{2}}

\def\preanttwo{\oldneedspace{12\baselineskip}}
%\def\prepsalmtitletwo{\vspace{-0.1\baselineskip}}
%\def\prerepeatantiphontwo{}
%\def\prerepeatantiphonthree{}
%\def\prepsalmtitlethree{\vspace{-1.5\baselineskip}}
%\def\prepsalmtitlefive{\vspace{-1.5\baselineskip}}
\def\prepsalmfiveverses{\vspace{-0.05\baselineskip}}
%\def\prerepeatantiphonfive{\vspace{-0.3\baselineskip}}
\def\prevespers{%
  \let\oldthingb\antthreetranslation
  \def\antthreetranslation{\vspace{-0.1\baselineskip}\oldthingb\vspace{-1\baselineskip}}%
  \let\oldthingc\antfivetranslation
  \def\antfivetranslation{\vspace{-0.4\baselineskip}\oldthingc\vspace{-0.8\baselineskip}}%
  %\def\prehymntranslation{\vspace{-0.3\baselineskip}}
  \def\prehymn{\vfil}
}
\printvespers[../HolyFamily]{inc-HolyFamily-Vespers2}

\benedicamusdomino[2]{}
}

}

	\chapter{Proper of the Time -- After Epiphany}
{
\newcommand{\benedicamusdomino}[1][sunday]{
  \benedicamusdominomaster{#1}
}
\def\printhymnnote{}
\def\printcommonvespers{
  \vspace{-0.5\baselineskip}
	\subtitle{\nth{2} Class, Green}
  \smallskip
  \deusinadjutorium{}
  \hfill
	\printnote{\emph{Vespers of Sundays throughout the year}, p.~\pageref{sundayvespers}.\par}

}

{
\section{\nth{2} Sunday after Epiphany}
\label{epiphany2}
\printcommonvespers{}
\printvespersmag[../TimeAfterEpiphany]{inc-VespersMagnificatEpiphany2}

\bigskip
\benedicamusdomino{}
}

{
\section{\nth{3} Sunday after Epiphany}
\label{epiphany3}
\printcommonvespers{}
\printvespersmag[../TimeAfterEpiphany]{inc-VespersMagnificatEpiphany3}

\bigskip
\benedicamusdomino{}
}

{
\section{\nth{4} Sunday after Epiphany}
\label{epiphany4}
\def\postmagtitle{\label{epiphany4-mag}}
\def\precollect{\printvrdirigatur}
\printcommonvespers{}
\printvespersmag[../TimeAfterEpiphany]{inc-VespersMagnificatEpiphany4}

\bigskip
\benedicamusdomino{}
}

{
\needspace{18\baselineskip}
\section{\nth{5} Sunday after Epiphany}
\label{epiphany5}
\printcommonvespers{}
\printvespersmag[../TimeAfterEpiphany]{inc-VespersMagnificatEpiphany5}

\bigskip
\benedicamusdomino{}
}

{
\section{\nth{6} Sunday after Epiphany}
\label{epiphany6}
\printcommonvespers{}
\def\prevespers{%
%  \let\oldthing=\englishmagantiphon
%  \def\englishmagantiphon{\oldthing\pagebreak}
}
\def\premagverses{\needspace{16\baselineskip}}
\printvespersmag[../TimeAfterEpiphany]{inc-VespersMagnificatEpiphany6}

\bigskip
\benedicamusdomino{}
}
}

	{
	\label{septuagesima}
	\chapter{Proper of the Time -- Septuagesima \& Lent}
  \def\printbenedicamusdomino{
    \noindent\emph{Benedicamus Domino}, p. \pageref{benedicamus-domino-sunday}.
  }
  \section{Septuagesima}
  \printterce{inc-septuagesima}{septuagesima}

  \section{Sexagesima}
  \printterce{inc-sexagesima}{sexagesima}

  \section{Quinquagesima}
  \printterce{inc-quinquagesima}{quinquagesima}

  {
    \newcommand{\printrefs}[1]{%
      \def\dotting{\hfill%\leaders\hbox to 1em{\hfil.\hfil}\hfill
        }%
      \begin{multicols}{2}%
      \noindent{}1st Sunday of Lent,\dotting \emph{below}\\
      2nd Sunday of Lent,\dotting \emph{p.~\pageref{lent2-#1}}\\
      3rd Sunday of Lent,\dotting \emph{p.~\pageref{lent3-#1}}\\
      4th Sunday of Lent,\dotting \emph{p.~\pageref{lent4-#1}}
      \end{multicols}%
      \smallskip
    }

	  \def\printhymn{

		  {\centering Hymn.\par}\label{hymn-lent}

	  	{\def\gabcfolder{.}
	  	\printgabc{1.}{}{N}{hy--nunc_sancte_nobis_(in_quadragesima)--solesmes_1961}}

      \printrefs{ant}
		  \bigskip
	  }

    \def\printshortresp{%
    \label{shortresp-lent}%
    {\def\gabcfolder{.}
    \printgabc{Short}{Resp.}{I}{re--ipse_liberavit--solesmes}}

    \translation[]{\Vbar{}~For he hath delivered me from the snare of the hunters.
    \Rbar{}~For he hath\dots{}
    \Vbar{}~And from the sharp word.
    \Rbar{}~From the snare of the hunters.
    \Vbar{}~Glory be to the Father, and to the Son, and to the Holy Ghost.
    \Rbar{}~For he hath\dots{}
    }

    \bigskip
    \gresetinitiallines{0}\label{vr-lent}
    \gregorioscore{vr-scapulis-suis}
    \let\myhwidth\relax
    \let\myhhwidth\relax
    \newlength{\myhwidth}
    \settowidth{\myhwidth}{speráb} % text in last word before last vowel of response
    \newlength{\myhhwidth}
    \settowidth{\myhhwidth}{b} % text in last syllable before vowel of versicle
    \addtolength{\myhhwidth}{-\myhwidth}
    \def\myhspace{\hspace{1ex}}
    \begin{nstabbing}
    %\>\Rbar{}~Et \myhspace{} omnes \myhspace{} reges \myhspace{} terræ \myhspace{} glóriam \>\hspace{\myhhwidth}tuam.
    \>\Rbar{}~Et \myhspace{} sub \myhspace{} pennis \myhspace{} ejus \>\hspace{\myhhwidth}sperábis.
    \end{nstabbing}

    \translation[]{\Vbar{}~He will overshadow thee with his shoulders.\\
    \Rbar{}~And under his wings thou shalt trust.}

    \printrefs{collect}
    \bigskip
    }

    \section{First Sunday of Lent}
	  \printterce{inc-Lent1}{lent1}
  }

  \def\printhymn{%
  	, \emph{Hymn}, page \pageref{hymn-lent}

  	\medskip
  }
  \def\printshortresp{
    \noindent\emph{Short Resp. \emph{Ipse liberavit me}}, p. \pageref{shortresp-lent}, \Vbar{}~Scapulis suis, p. \pageref{vr-lent}.
  }
  \section{Second Sunday of Lent}
  \printterce{inc-Lent2}{lent2}

  \section{Third Sunday of Lent}
  \printterce{inc-Lent3}{lent3}

  \section{Fourth Sunday of Lent}
  \printterce{inc-Lent4}{lent4}

  {
    \newcommand{\printrefs}[1]{%
      \def\dotting{\hfill%\leaders\hbox to 1em{\hfil.\hfil}\hfill
        }%
      \begin{multicols}{2}%
      \noindent{}Passion Sunday,\dotting \emph{below}\\
      Palm Sunday,\dotting \emph{p.~\pageref{passion2-#1}}
      \end{multicols}%
      \smallskip
    }

    \def\printhymn{

      {\centering Hymn.\par}\label{hymn-passiontide}

      {\def\gabcfolder{.}
      \printgabc{2.}{}{N}{hy--nunc_sancte_nobis_(in_tempore_passionis)--solesmes_1961}}

      \printrefs{ant}
      \bigskip
    }

    \def\printshortresp{%
    \label{shortresp-passiontide}%
    {\def\gabcfolder{.}
    \printgabc{Short}{Resp.}{E}{re--erue_a_frama--solesmes}}

    \translation[]{\Vbar{}~Deliver from the sword, O God, my soul.
    \Rbar{}~Deliver\dots{}
    \Vbar{}~My only one from the hand of the dog.
    \Rbar{}~O God, my soul.
    \Rbar{}~Deliver, * O God, my soul from the sword.For he hath delivered me from the snare of the hunters.
    }

    \bigskip
    \gresetinitiallines{0}\label{vr-passiontide}
    \gregorioscore{vr-de-ore-leonis}
    \let\myhwidth\relax
    \let\myhhwidth\relax
    \newlength{\myhwidth}
    \settowidth{\myhwidth}{me} % text in last word before last vowel of response
    \newlength{\myhhwidth}
    \settowidth{\myhhwidth}{n} % text in last syllable before vowel of versicle
    \addtolength{\myhhwidth}{-\myhwidth}
    \def\myhspace{\hspace{1.4ex}}
    \begin{nstabbing}
    %\>\Rbar{}~Et \myhspace{} omnes \myhspace{} reges \myhspace{} terræ \myhspace{} glóriam \>\hspace{\myhhwidth}tuam.
    \>\Rbar{}~Et a córnibus unicórnium humilitátem \>\hspace{\myhhwidth}meam.
    \end{nstabbing}

    \translation[]{\Vbar{}~From the lion's mouth, O Lord, save me.
    \Rbar{}~And my lowness from the horns of the unicorns.}

    \printrefs{collect}
    \bigskip
    }

    \section{Passion Sunday}
    \printterce{inc-Passion1}{passion1}
  }
 
  \def\printhymn{%
    , \emph{Hymn}, page \pageref{hymn-passiontide}

    \medskip
  }
  \def\printshortresp{
    \noindent\emph{Short Resp. \emph{Erue a framea}}, p. \pageref{shortresp-passiontide}, \Vbar{}~De ore leonis, p. \pageref{vr-passiontide}.
  }
  \section{Palm Sunday}
  \printterce{inc-Passion2}{passion2}
}

	{
\def\printcommemnote{\smallskip
\noindent
\printnote{\commemorations{}  Otherwise \Vbar~\emph{Bendicámus Dómino}, page \pageref{benedicamusdomino-easter}.}
}
{
\chapter{Proper of the Time -- Easter}
\section{Easter Sunday}
\subtitle{1st Class}
\def\printfullhymn{
    \emph{Chapter, Hymn, and Versicle are all omitted, but the following Antiphon is said :}

    \bigskip
    \def\annot{\small{Ant.}}
    \def\annottwo{\small{\chapterhymnversicleantiphonmode.}}
    %\setinitialspacing{\chapterhymnversicleantiphoninitial}
    \gregorioscore{\gabcfolder/\chapterhymnversicleantiphontex}
    \translation[]{\chapterhymnversicleantiphontranslation}
    \bigskip
}
\def\chapterreplacement{\bigskip}
\def\begincollectcols{\begin{parcolumns}[rulebetween,colwidths={1=0.45\linewidth}]{2}}
\def\postmag{\vspace{-0.05\baselineskip}}
\printvespers[../Easter]{inc-EasterVespers}
\newcommand{\printbenedicamusdomino}[2]{
	\def\annot{\small{#1}}
	\def\annottwo{}
	%\setinitialspacing{B}
    \greseteolcustos{manual}
	\gregorioscore{#2}
    \bigskip
    \hrule
}
\def\breakbeforeresp{T}
\printbenedicamusdomino{\Vbar}{../BenedicamusDomino/BenedicamusDomino_Easter}
}

\newcommand{\printcommonvespers}[1][2nd]{
    \subtitle{#1 Class}
    \printnote{From Vespers of Sundays in Eastertide on page \pageref{sundayvespers-easter}.}
}
{
\newcommand{\benedicamusdomino}[1][easter]{
    \noindent\printnote{\Vbar~\emph{Benedicámus Dómino}, page \pageref{benedicamusdomino-#1}.}
    \bigskip
    \hrule
}

{
\section{Low Sunday}
\printcommonvespers[1st]
\def\printfullhymn{
    \label{hymn-adregiasagnidapes}
    {
        \def\hymnlinetwo{\oldstylenums{8.}}
\def\hymntex{hymn-AdRegiasAgniDapes}
\def\hymninitial{A}
\def\hymntranslation{\item At the Lamb's high feast we sing
praise to our victorious King,
Who hath washed us in the tide
flowing from His pierced side.
\item Praise we Him Whose love divine
gives the guests His Blood for wine,
gives His Body for the feast,
Love the victim, Love the priest.
\item Where the Paschal blood is poured,
Death's dark Angel sheathes his sword;
Israel's hosts triumphant go
through the wave that drowns the foe.
\item Christ, the Lamb Whose Blood was shed,
Paschal victim, Paschal bread;
with sincerity and love
eat we manna from above.
\item Mighty Victim from the sky,
powers of hell beneath Thee lie;
Death is conquered in the fight;
Thou hast brought us life and light.
\item Now Thy banner Thou dost wave;
vanquished Satan and the grave;
see the prince of darkness quelled;
heaven's bright gates are open held.
\item Paschal triumph, Paschal joy,
only sin can this destroy;
from sin's death do Thou set free
souls re-born, dear Lord, in Thee.
\item Hymns of glory, songs of praise,
Father, unto Thee we raise;
risen Lord, all praise to Thee,
ever with the Spirit be.
Amen.}

\def\vrtex{vrManeNobiscum}
\def\vtranslation{Stay with us O Lord, alleluia.}
\def\rtranslation{Because it is towards evening, alleluia.}

        \printhymn{\oldstylenums{\hymnlinetwo}}{\hymninitial}{\hymntex}{\hymntranslation}
        \def\vrlinebreak{T}
        \label{vr-manenobiscum}
        \printvr[\greseteolcustos{manual}]{\vrtex}{\vtranslation}{\rtranslation}
        \bigskip
    }
}
%\def\begincollectcols{\begin{parcolumns}[rulebetween,colwidths={1=0.42\linewidth}]{2}}
\printvespersmag[../TimeAfterEaster]{inc-VespersMagnificatEaster1}

\def\commemorations{If the Feast of the Annunciation has been transferred to the Monday following Low Sunday, First Vespers is commemorated as on page \pageref{annunciation-commem}.  If today is April 30, May 1, or May 2, then First Vespers of St Joseph the Worker is commemorated as follows.}
\printcommemnote{}
}

{
\label{stjoseph-worker-commem}
\def\vrlinebreak{T}
%\printcommemoration[../May1-StJosephWorker]{commemorationStJosephWorker-Vespers1}

\bigskip
\benedicamusdomino{}
}

{
    TODO Feast of St Joseph the worker could have commem of 2nd through 5th Sunday after Easter
    probably just give them a page number and say to use the simple tone for the versicle and response
}

\newcommand{\printhymnnote}{
    \noindent\printnote{Hymn. \emph{Ad Régias Agni Dapes}, page \pageref{hymn-adregiasagnidapes}.
    \Vbar~\emph{Mane nobíscum}, page \pageref{vr-manenobiscum}.}
}
{
\section{2nd Sunday after Easter}
\printcommonvespers{}
\printvespersmag[../TimeAfterEaster]{inc-VespersMagnificatEaster2}
\benedicamusdomino{}
}
{
\section{3rd Sunday after Easter}
\printcommonvespers{}
\printvespersmag[../TimeAfterEaster]{inc-VespersMagnificatEaster3}
\benedicamusdomino{}
}

{
\section{4th Sunday after Easter}
\printcommonvespers{}
\printvespersmag[../TimeAfterEaster]{inc-VespersMagnificatEaster4}
\benedicamusdomino{}
}

{
\section{5th Sunday after Easter}
\printcommonvespers{}
\printvespersmag[../TimeAfterEaster]{inc-VespersMagnificatEaster5}
\benedicamusdomino{}
}

{
\section{Ascension of Our Lord}
\subtitle{1st Class}
TODO

TODO could be commem of 1st vespers of St Joseph the worker if today is April 30 or May 1

}

{
\section{Sunday after the Ascension}
\printcommonvespers{}
\def\printfullhymn{
    {
        \printhymn{\oldstylenums{\hymnlinetwo}}{\hymninitial}{\hymntex}{\hymntranslation}
        \def\vrlinebreak{T}
        \printvr[\greseteolcustos{manual}]{\vrtex}{\vtranslation}{\rtranslation}
    }
}
\printvespersmag[../TimeAfterEaster]{inc-VespersMagnificatSundayAfterAscension}
\benedicamusdomino{}
}
}
}

	\chapter{Proper of the Time -- Pentecost Octave}
{
\newcommand{\benedicamusdomino}[1][1]{
	\noindent\printnote{\Vbar~\emph{Benedicámus Dómino}, page \pageref{benedicamusdomino-#1}.}
	\bigskip
	\hrule
}

{
\section{Pentecost Sunday}
\subtitle{1st Class}
\subtitle{I \& II Vespers}

\def\definevesperspropers{\definepsalm{5}{113}{7}{c2}

\newcommand{\vrtex}{vrLoquebantur}
\newcommand{\vtranslation}{The apostles spoke in divers tongues.}
\newcommand{\rtranslation}{The wonderful works of God.}

\newcommand{\maganttex}{MagnificatAntiphon2}
\newcommand{\magantinitial}{H}
\newcommand{\maganttranslation}{Today the days of Pentecost are complete, alleluia; today the Holy Ghost appeared in fire to the disciples, gave them gifts and graces, sent them into all the world to preach and to bear witness; whoever believes and is baptised shall be saved, alleluia.}

	\def\prepsalmfive{\greseteolcustos{manual}}
}
\def\definevesperspropersalt{\definepsalm{5}{116}{7}{c2}

\newcommand{\vrtex}{vrRepletiSunt}
\newcommand{\vtranslation}{They were all filled with the Holy Ghost, alleluia.}
\newcommand{\rtranslation}{And they began to speak, alleluia.}

\newcommand{\maganttex}{MagnificatAntiphon1}
\newcommand{\magantinitial}{N}
\newcommand{\maganttranslation}{I will not leave you orphans, alleluia; I go away, and I come unto you, alleluia; and your heart shall rejoice, alleluia.}
}
\def\vesperspropersnote{At II Vespers:}
\def\vesperspropersaltnote{At I Vespers:}
%\def\premag{\def\noeuouae{T}}
\def\premagverses{\greseteolcustos{manual}}
\def\printfullhymn{
	{\printhymn{\oldstylenums{\hymnlinetwo}}{\hymninitial}{\hymntex}{\hymntranslation}}
		% print all versicles that could follow hymn?
	{
		\def\vrlinebreak{T}
		%\label{vr-rorate}
		\printnote{\vesperspropersaltnote}
		\definevesperspropersalt
		\printvr[\greseteolcustos{manual}]{\vrtex}{\vtranslation}{\rtranslation}
	}
	\bigskip
	{
		\def\vrlinebreak{T}
		%\label{vr-rorate}
		\printnote{\vesperspropersnote}
		\definevesperspropers
		\printvr[\greseteolcustos{manual}]{\vrtex}{\vtranslation}{\rtranslation}
	}
}
\printvespers[../Pentecost]{inc-PentecostVespers}
\bigskip
\benedicamusdomino{}
}

{
\section{Trinity Sunday}
\subtitle{1st Class}

%\def\premag{\def\noeuouae{T}}
\def\premagverses{\greseteolcustos{manual}}
\def\printfullhymn{
	{
		\printhymn{\oldstylenums{\hymnlinetwo}}{\hymninitial}{\hymntex}{\hymntranslation}
		\def\vrlinebreak{T}
		\printvr[\greseteolcustos{manual}]{\vrtex}{\vtranslation}{\rtranslation}
	}
}
\def\begincollectcols{\vspace{-0.5\baselineskip}\begin{parcolumns}[rulebetween,colwidths={1=0.44\linewidth}]{2}}
\printvespers[../TrinitySunday]{inc-TrinitySunday-Vespers}
\bigskip
\benedicamusdomino{}
}

}


{
Easter is March 22 to April 25
    TODO Feast of Nativity of St John the Baptist (6/24) could have commem of 2nd through 6th Sunday after Pentecost
    TODO Sts Peter and Paul (6/29) could have commem of 3rd through 7th Sunday after Pentecost
    probably just give them a page number and say to use the simple tone for the versicle and response

    TODO Feast of Assumption (8/15) could have commem of Saturday before 3rd Sunday of August or 9th to 13th Sunday after Pentecost

    TODO Feast of St Michael the Archangel (9/29) could have commem of 16th to 20th Sunday after Pentecost

    TODO Feast of Christ the King if on October 31, commem for 1st vespers of All Saints

    TODO Feast of All Saints (11/1) commem of Sunday, Saturday before 1st Sunday of November, 21st - 23rd Sunday after Pentecost or 4th after Epiphany
}

\chapter{Proper of the Time -- Time After Pentecost}
{
\def\printcommonvespers{
	\subtitle{2nd Class}
	\printnote{From Vespers of Sundays throughout the year on page \pageref{sundayvespers}.}
}
\newcommand{\benedicamusdomino}[1][sunday]{
	\noindent\printnote{\Vbar~\emph{Benedicámus Dómino}, page \pageref{benedicamusdomino-#1}.}
	\bigskip
	\hrule
}
\newcommand{\printhymnnote}{}
\newcommand{\printvespersafterpentecost}[1]{
	{
	\def\ordinalending{th}
	\ifnum#1=2\def\ordinalending{nd}\fi
	\ifnum#1=3\def\ordinalending{rd}\fi
	\ifnum#1=21\def\ordinalending{st}\fi
	\ifnum#1=22\def\ordinalending{nd}\fi
	\ifnum#1=23\def\ordinalending{rd}\fi
	\ifnum#1=24\def\ordinalending{th or Last}\fi

	\section{#1\ordinalending{}	Sunday after Pentecost}
	\printcommonvespers{}
	\printvespersmag[../TimeAfterPentecost]{inc-VespersMagnificatPentecost#1}
	\smallskip
	\benedicamusdomino{}
	}
	
}

\printvespersafterpentecost{2}
\printvespersafterpentecost{3}
\printvespersafterpentecost{4}
\printvespersafterpentecost{5}
\printvespersafterpentecost{6}
\printvespersafterpentecost{7}
\printvespersafterpentecost{8}
\printvespersafterpentecost{9}
\printvespersafterpentecost{10}
\printvespersafterpentecost{11}
\printvespersafterpentecost{12}
\printvespersafterpentecost{13}
\printvespersafterpentecost{14}
\printvespersafterpentecost{15}
\printvespersafterpentecost{16}
\printvespersafterpentecost{17}
\printvespersafterpentecost{18}
\printvespersafterpentecost{19}
\printvespersafterpentecost{20}
\printvespersafterpentecost{21}
\printvespersafterpentecost{22}
\printvespersafterpentecost{23}
\printvespersafterpentecost{24}

}

	{
	\chapter{Proper of the Time -- Time After Pentecost}
  \def\deusinadjutorium{%
    \noindent\printnote{All from \emph{Sunday at Terce throughout the year}, p.~\pageref{deusinadjutorium}}
  }
  \def\printshortresp{}
  \def\printhymn{\medskip}
  \newcommand{\printterceafterpentecost}[1]{
    \section{\nth{#1} \ifnum#1=24{or Last }\fi Sunday after Pentecost}
    \printterce[../TimeAfterPentecost]{inc-VespersMagnificatPentecost#1}{pentecost#1}
  }
%\tracingmacros=1
  \printterceafterpentecost{2}
  \printterceafterpentecost{3}
  \printterceafterpentecost{4}
  \printterceafterpentecost{5}
  \printterceafterpentecost{6}
  \printterceafterpentecost{7}
  \printterceafterpentecost{8}
  \printterceafterpentecost{9}
  \printterceafterpentecost{10}
  \printterceafterpentecost{11}
  \printterceafterpentecost{12}
  \printterceafterpentecost{13}
  \printterceafterpentecost{14}
  \printterceafterpentecost{15}
  \printterceafterpentecost{16}
  \printterceafterpentecost{17}
  \printterceafterpentecost{18}
  \printterceafterpentecost{19}
  \printterceafterpentecost{20}
  \printterceafterpentecost{21}
  \printterceafterpentecost{22}
  \printterceafterpentecost{23}
  \printterceafterpentecost{24}
}

	\clearpage
\chapter{Proper of the Saints}
\vspace{-0.75\baselineskip}
{
\let\oldsection=\section
\renewcommand{\section}[1]{
  \oldsection{#1}
  \vspace{-0.35\baselineskip}
}
\newcommand{\benedicamusdomino}[1][1]{
  \benedicamusdominomaster{#1}
}
\newcommand{\sundaycommemnote}{%
  If today is Sunday, \emph{Vespers of the Sunday} is commemorated with \emph{Magnificat antiphon}, \emph{\Vbar{}~Dirigátur}, and \emph{Collect}.
}
\newcommand{\sundaycommemnoteeaster}{%
  If today is Sunday, \emph{Vespers of the Sunday} is commemorated with \emph{Magnificat antiphon}, \emph{\Vbar{}~Mane nobíscum}, and \emph{Collect}.
}
\newcommand{\sundaycommemnoteeasterpentecost}{%
  If today is Sunday, \emph{Vespers of the Sunday} is commemorated with \emph{Magnificat antiphon}, \emph{\Vbar{}}, and \emph{Collect}.
}

\ifthenelse{\boolean{testrun}}{}{
%December 8: Immaculate Conception
%Second Vespers of Immaculate Conception
{
\label{immaculateconception}
\section{December 8: Immaculate Conception}
\subtitle{\nth{1} Class, White or Gold}
\subtitle{I \& II Vespers}
\smallskip

\def\preanttranslationone{\vspace{-0.35\baselineskip}}
\def\prepsalmtitleone{\vspace{-0.15\baselineskip}}
\def\prepsalmone{\vspace{-0.1\baselineskip}}
\def\prepsalmoneverses{\vspace{-0.05\baselineskip}}
\def\prerepeatantiphonone{}

\def\prepsalmtitlethree{\vspace{-0.75\baselineskip}}

\def\preanttranslationfive{\vspace{-0.35\baselineskip}}
\def\prepsalmtitlefive{\vspace{-\baselineskip}}
\def\prepsalmfive{\vspace{-0.2\baselineskip}}
\def\prepsalmfiveverses{\vspace{-0.025\baselineskip}}
\def\prerepeatantiphonfive{}
\def\premagverses{\greseteolcustos{manual}}
\def\definevesperspropersalt{\def\noeuouae{T}\newcommand{\maganttex}{MagnificatAntiphon1}
\newcommand{\magantinitial}{B}
\newcommand{\maganttranslation}{All generations shall call me blessed, because he that is mighty, hath done great things for me, alleluia.}
\def\magpsalmclef{c3}
\definemag{8}{G}
}
\def\definevesperspropers{\def\noeuouae{T}\newcommand{\maganttex}{MagnificatAntiphon2}
\newcommand{\magantinitial}{H}
\newcommand{\maganttranslation}{This day a rod came forth from the root of Jesse: this day Mary was conceived without any stain of sin: this day the head of the old serpent was crushed by her.  Alleluia.}
\definemag{1}{f}
}
\def\vesperspropersaltnote{At I Vespers:}
\def\vesperspropersnote{At II Vespers:}
\def\prehymn{\printnote{All kneel for the first stanza of the following hymn.}}
\def\hymnlabel{hymn-avemarisstella}
\def\vrlinebreak{F}
%\def\prechapter{\vspace{-\baselineskip}}
\printvespers[../December8-ImmaculateConception]{inc-ImmaculateConceptionVespers}
}

{
\bigskip

\bigskip
\noindent
\printnote{Then follows a Commemoration of the Advent Sunday or Feria according to the day of the week.  The antiphon is followed by \emph{\Vbar{} Roráte cæli}, page \pageref{vr-rorate-dec8}, and then the appropriate Collect, page \pageref{collect-dec8}.}
\bigskip
}

{
  \oldneedspace{5\baselineskip}
  \subtitle{First Week of Advent.}
  \subtitle{\small{Thursday.}}
  {
  \def\noeuouae{T}
  \printgabc{At Magn.}{\oldstylenums{Ant.~4.}}{E}{../Advent1/MagAntThursday-Exspectabo}
  }
  \translation[]{I will look for the Lord my Saviour, and await Him, while He is near, alleluia.}
  \medskip

  \oldneedspace{5\baselineskip}
  \subtitle{\small{Friday.}}
  {
  \def\noeuouae{T}
  \printgabc{At Magn.}{\oldstylenums{Ant.~4.}}{E}{../Advent1/MagAntFriday-ExAegypto}
  }
  \translation[]{Out of Egypt I have called my Son; he shall come to save His people.}
  \medskip

  \oldneedspace{5\baselineskip}
  \subtitle{Second Week of Advent.}
  \subtitle{\small{Saturday.}}
  {
  \def\noeuouae{T}
  \printgabc{At Magn.}{\oldstylenums{Ant.~7.}}{V}{../Advent1/MagAntSaturday-VeniDomine}
  }
  \translation[]{Come, Lord, to visit us in peace, that we may rejoice before Thee with a perfect heart.}
  \medskip

  \oldneedspace{5\baselineskip}
  \subtitle{\small{Sunday.}}
  {
  \def\noeuouae{T}
  \printgabc{At Magn.}{\oldstylenums{Ant.~8.~G}}{T}{../Advent2/MagnificatAntiphon-noEuouae}
  }
  \translation[]{Art Thou He that art to come, or look we for another? Relate to John what you have seen: The blind recover their sight, the dead rise again, the poor have the Gospel preached to them, alleluia.}
  \medskip

  \oldneedspace{5\baselineskip}
  \subtitle{\small{Monday.}}
  {
  \def\noeuouae{T}
  \printgabc{At Magn.}{\oldstylenums{Ant.~4.}}{E}{../Advent2/MagAntMonday-EcceRexVeniet}
  }
  \translation[]{Behold, the King shall come, the Lord of the land; and He shall take away the yoke of our captivity.}
  \medskip

  \oldneedspace{5\baselineskip}
  \subtitle{\small{Tuesday.}}
  {
  \def\noeuouae{T}
  \printgabc{At Magn.}{\oldstylenums{Ant.~5.}}{V}{../Advent2/MagAntTuesday-VoxClamantis}
  }
  \translation[]{A voice of one crying in the desert, Prepare ye the way of the Lord, make straight His paths}
  \medskip

  \oldneedspace{5\baselineskip}
  \subtitle{\small{Wednesday.}}
  {
  \def\noeuouae{T}
  \printgabc{At Magn.}{\oldstylenums{Ant.~4.}}{S}{../Advent2/MagAntWednesday-Sion}
  }
  \translation[]{Sion, thou shalt be restored, and shalt see the Just One who shall appear in thee.}
  \medskip

  \oldneedspace{5\baselineskip}
  \subtitle{\small{Thursday.}}
  {
  \def\noeuouae{T}
  \printgabc{At Magn.}{\oldstylenums{Ant.~4.}}{Q}{../Advent2/MagAntThursday-QuiPostMeVenit}
  }
  \translation[]{He that shall come after me is preferred before me; whose shoes I am not worthy to loose.}
  \medskip
  \hrule
  \medskip
  {
      \label{vr-rorate-dec8}
      \newcommand{\commvlatin}{Roráte cæli désuper, et nubes pluant \textbf{ju}stum.}
      \newcommand{\commrlatin}{Aperiátur terra, et gérminet Salva\textbf{tó}rem.}
      \newcommand{\commvtranslation}{Ye heavens, drop down dew from above, and let the clouds rain down the Just One.}
      \newcommand{\commrtranslation}{Let the earth open and bud forth the Saviour.}
  \printvrcommem{}
  }

  \oldneedspace{3\baselineskip}
  \label{collect-dec8}
  \begin{center}{\large Collect.}\end{center}
  \vspace{-1.5\baselineskip}
  \def\printcollectheading{F}
  {
  \begin{center}{First Week of Advent.}\end{center}
  \def\gabcfolder{../Advent1}
  \newcommand{\antonetex}{Ant1-InIllaDie}
\newcommand{\antoneinitial}{I}
\newcommand{\antonetranslation}{In that day the mountains shall drop down sweetness, and the hills shall flow with milk and honey, alleluia.}
\definepsalm{1}{109}{8}{G}

\newcommand{\anttwotex}{Ant2-Jucundare}
\newcommand{\anttwoinitial}{J}
\newcommand{\anttwotranslation}{Shout for joy, O daughter of Sion, rejoice greatly, O daughter of Jerusalem, alleluia.}
\definepsalm{2}{110}{8}{G*}

\newcommand{\antthreetex}{Ant3-EcceDominusVeniet}
\newcommand{\antthreeinitial}{E}
\newcommand{\antthreetranslation}{Behold, the Lord shall come, and all His Saints with Him: and there shall be in that day a great light, alleluia.}
\definepsalm{3}{111}{5}{a}

\newcommand{\antfourtex}{Ant4-Omnes}
\newcommand{\antfourinitial}{O}
\newcommand{\antfourtranslation}{All ye that thirst come to the waters: seek the Lord while He can be found, alleluia.}
\definepsalm{4}{112}{7}{c}

\newcommand{\antfivetex}{Ant5-EcceVeniet}
\newcommand{\antfiveinitial}{E}
\newcommand{\antfivetranslation}{Behold there shall come a great Prophet, and He shall renew Jerusalem, alleluia.\vspace{1ex}}
\definepsalm{5}{113}{4}{A*}

\newcommand{\chaptertext}{\dropcap{latin}{Fratres~: Hora est jam nos de somno} \textbf{súr}\-ge\-re~:~\gredagger{} nunc enim própior est \emph{no\-stra} \textbf{sá}\-lus,~* quam cum cre\-\textbf{dí}\-dimus.}
\newcommand{\chaptertranslation}{Brethren: it is now the hour for us to rise from sleep.  For now our salvation is nearer than when we believed.}

\newcommand{\magantinitial}{N}
\newcommand{\maganttex}{MagnificatAntiphon}
\newcommand{\maganttranslation}{Fear not, Mary, for thou hast found grace with the Lord: behold thou shalt conceive and bring forth a son, alleluia.}
\def\magsolemn{F}
\definemag{8}{G}

\newcommand{\collect}{Excita, quaésumus Dómine, poténtiam tuam, et veni~:~† ut ab imminéntibus peccatórum nostrórum perículis, te mereámur protegénte éripi,~* te liberánte salvári.  Qui vivis et regnas cum Deo Patre in unitáte Spíritus Sancti Deus~:~* per ómnia saécula sæculórum.}
\newcommand{\collecttranslation}{Stir up Thy power, we beseech Thee, O Lord, and come: that from the threatening dangers of our sins we may deserve to be rescued by Thy protection, and to be saved by Thy deliverance: Who livest and reignest with God the Father in the unity of the Holy Ghost, world without end.}

  \printcollect{\collect}{\collecttranslation}
  }
  {
  \begin{center}{Second Week of Advent.}\end{center}
  \def\gabcfolder{../Advent2}
  % !TEX TS-program = lualatex
% !TEX encoding = UTF-8

% This is a simple template for a LuaLaTeX document using gregorio scores.

\newcommand{\comheadingtext}{Commemoration of 2nd Sunday of Advent}

\newcommand{\latincomcollect}{Excita Dómine corda nostra ad præparándas Unigéniti tui vias~:~† ut per ejus advéntum,~* purificátis tibi méntibus servíre mereámur. Qui tecum.}
\newcommand{\englishcomcollect}{Stir up our hearts, O Lord, to prepare the ways of Thine only-begotten Son: that through His coming we may deserve to serve Thee with purified minds: Who with Thee.}

\newcommand{\englishcommagantiphon}{Art Thou He that art to come, or look we for another? Relate to John what you have seen: The blind recover their sight, the dead rise again, the poor have the Gospel preached to them, alleluia.}

\newcommand{\commagantlinetwo}{Ant. 8. G}
\newcommand{\commaganttex}{MagnificatAntiphon-noEuouae}
\newcommand{\commagantinitial}{T}
\newcommand{\commagantinitialsize}{35}

\newcommand{\commvrtex}{../Advent/vr-commemoration}
\newcommand{\commvtranslation}{Ye heavens, drop down dew from above, and let the clouds rain down the Just One.}
\newcommand{\commrtranslation}{Let the earth open and bud forth the Saviour.}

\grechangestaffsize{17}
  \printcollect{\latincomcollect}{\englishcomcollect}
  }

  \bigskip
  \benedicamusdomino{}
}


\ifthenelse{\boolean{includejan29}}{
	%January 29, St Francis de Sales
	{
\section{January 29: St Francis de Sales}
\subtitle{\nth{1} Class (patron of the Archdiocese of Cincinnati), White or Gold}
\subtitle{From Common of Confessor Bishop \& Common of Doctors}
\subtitle{I \& II Vespers}
\medskip

\def\deusinadjutoriumsolemn{T}
\def\definevesperspropers{\definepsalm{5}{131}{3}{g}

\newcommand{\vrtex}{vrJustumDeduxit}
\newcommand{\vtranslation}{The Lord led the just by right ways.}
\newcommand{\rtranslation}{And showed him the kingdom of God.}

\def\prepsalmtitlefive{\vspace{-\baselineskip}}
\def\prerepeatantiphonfive{}
}
\def\definevesperspropersalt{\definepsalm{5}{116}{3}{g}

\newcommand{\vrtex}{vrAmavitEum}
\newcommand{\vtranslation}{The Lord lloved him and adorned him.}
\newcommand{\rtranslation}{He clothed him with a robe of glory.}

\def\prepsalmtitlefive{\vspace{-\baselineskip}}
}
\def\prepsalmtitleone{\vspace{-0.5\baselineskip}}
\def\prepsalmtitletwo{\vspace{-0.5\baselineskip}}
\def\vesperspropersnote{At II Vespers:}
\def\vesperspropersaltnote{At I Vespers:}
%\def\premagverses{\vspace{-0.1\baselineskip}}
\def\prevespers{%
  %\let\oldthing=\maganttranslation
  %\def\maganttranslation{\vspace{-0.25\baselineskip}\oldthing\vspace{-0.25\baselineskip}}
  %\def\premagtexverses{\smallskip}
}
\def\premagtitle{\oldneedspace{8\baselineskip}}
%\def\premaganttranslation{\vspace{-2ex}}
%\def\precollect{\pagebreak}%\vspace{-0.5\baselineskip}}
\printvespers[../CommonOfConfessorBishop]{inc-StFrancisDeSales}
\medskip
\printnote{\sundaycommemnote{}

\medskip{}

\emph{\nth{4} Sunday after Epiphany}, p.~\pageref{epiphany4}.

\emph{Septuagesima}, p.~\pageref{septuagesima}.

\emph{Sexagesima}, p.~\pageref{sexagesima}.
}

\medskip{}
\benedicamusdomino{}
}
}{}

%Purification & Presentation (2nd class no commem of Sunday because it is a feast of the Lord)
{
\section{February 2: Purification \& Presentation}
\subtitle{\nth{2} Class, White}
\subtitle{I \& II Vespers}
\medskip
\printnote{At I Vespers, Psalms and Antiphons of the Circumcision, p.~\pageref{circumcision}, continuing with the chapter on p.~\pageref{purification-chapter}.}

\def\definevesperspropers{\newcommand{\maganttex}{an--hodie_beata_virgo--solesmes}
\newcommand{\magantinitial}{H}
\newcommand{\maganttranslation}{Today did the Blessed Virgin Mary present the Child Jesus in the temple; and Simeon, filled with the Holy Ghost, took Him up in his arms, and blessed God for ever and ever.\vspace{-2ex}}
\definemag{8}{G*}
}
\def\definevesperspropersalt{\newcommand{\maganttex}{an--senex_puerum--solesmes}
\newcommand{\magantinitial}{S}
\newcommand{\maganttranslation}{This day did the Blessed Virgin Mary present the Child Jesus in the temple; and Simeon, filled with the Holy Ghost, took Him up in his arms, and blessed God for ever and ever.}
\definemag{1}{D}
}
%\def\prepsalmtitletwo{\needspace{12\baselineskip}}
\def\vesperspropersnote{At II Vespers:}
\def\vesperspropersaltnote{At I Vespers:}
\def\prepsalmtitlefour{\vspace{-0.5\baselineskip}}
\def\prerepeatantiphonfour{}
\def\prepsalmfourverses{}
\def\preantfive{\vspace{-0.5\baselineskip}}
\def\prechapter{\vspace{-\baselineskip}\label{purification-chapter}}
\def\vrlabel{vr-feb2}
\def\printhymnnote{
  {
    \oldneedspace{3\baselineskip}
    \printnote{Hymn.~\emph{Ave Maris Stella}, p.~\pageref{hymn-avemarisstella}.\\}
    %
    % \def\vrlinebreak{T}
    % \oldneedspace{3\baselineskip}
    % \printvr[\greseteolcustos{manual}]{\vrtex}{\vtranslation}{\rtranslation}
  }
}
\printvespers[../February2-PurificationOfBlessedVirginMary]{inc-purification}
\bigskip{}
\benedicamusdomino[2]{}
}

%March 25: Annunciation 
{
\def\vrlinebreak{F}
\section{March 25: Annunciation of the Blessed Virgin Mary}
\subtitle{\nth{1} Class}
\subtitle{I \& II Vespers}
\medskip

\def\deusinadjutoriumsolemn{T}
\def\definevesperspropers{\newcommand{\maganttex}{an--gabriel_angelus_ave--solesmes}
\newcommand{\magantinitial}{G}
\newcommand{\maganttranslation}{The Angel Gabriel spoke to Mary, saying: Hail, full of grace, the Lord is with thee; blessed art thou amongst women.}
\def\magsolemn{T}
\def\psalmclef{c3}
\definemag{7}{d}

  \def\prepsalmfive{\greseteolcustos{manual}}
}
\def\definevesperspropersalt{\newcommand{\maganttex}{an--spiritus_sanctus--solesmes}
\newcommand{\magantinitial}{S}
\newcommand{\maganttranslation}{The Holy Ghost shall come down upon thee, Mary, and the power of the Most High shall overshadow thee.}
\def\magsolemn{T}
\definemag{8}{G}
}
\def\vesperspropersnote{At II Vespers:}
\def\vesperspropersaltnote{At I Vespers:}
%\def\premag{\def\noeuouae{T}}
\ifthenelse{\boolean{birmingham}}{
  \def\preantthree{\vspace{-0.75\baselineskip}}
  \def\prepsalmtitlethree{\vspace{-0.25\baselineskip}}
  \def\preantfive{\vspace{-0.75\baselineskip}}
  \def\prepsalmtitlefive{\vspace{-0.75\baselineskip}}
  \def\premagtitle{\oldneedspace{2\baselineskip}}
  \def\precollect{\vspace{-0.5\baselineskip}}
}{
  \def\prepsalmtitlefive{\vspace{-0.5\baselineskip}}
}
\def\prepsalmtitletwo{\vspace{-0.5\baselineskip}}
\def\premagverses{\greseteolcustos{manual}}
\def\printhymnnote{
  {
    \oldneedspace{3\baselineskip}
    \printnote{Hymn.~\emph{Ave Maris Stella}, p.~\pageref{hymn-avemarisstella}.\\}

    \printnote{At II Vespers in Paschal time, add Alleluia to the versicle and response.\\}
    %
    % \def\vrlinebreak{T}
    % \oldneedspace{3\baselineskip}
    % \printvr[\greseteolcustos{manual}]{\vrtex}{\vtranslation}{\rtranslation}
  }
}
\def\beginchaptercols{\begin{parcolumns}[rulebetween,colwidths={1=0.44\linewidth}]{2}}
\def\begincollectcols{\begin{parcolumns}[rulebetween,colwidths={1=0.43\linewidth}]{2}}
%\def\prepsalmthreeverses{\pagebreak}
\def\prevespers{
  %\let\oldthing=\anttwotranslation  
  %\def\anttwotranslation{\oldthing\vspace{-\baselineskip}}
  %\let\oldthingb=\antfourtranslation  
  %\def\antfourtranslation{\oldthing\pagebreak}
}
\def\vrlabel{vr-march25}

\printvespers[../March25-Annunciation]{inc-Annunciation}
}

{
\bigskip
\noindent
\printnote{Then follows a Commemoration of the Lenten Feria.  Finally, \benedicamusdominoreference{1}}
\bigskip
}


% Easter can be as early as March 22 or as late as April 25, so we basically need to include all the lenten ferias
% I think if it falls on or after Palm Sunday, it is translated to Monday after Low Sunday
% so I should only need to go from Wednesday of the second week unless I want to add St Joseph
{
\newcommand{\printcommem}[8][../Lent]{
% #1 Folder (../Lent)
% #2 name of the day of the week (Thursday)
% #3 Tone (2)
% #4 First letter of antiphon (Q)
% #5 gabc (an--qui_me_sanum_fecit--solesmes)
% #6 ant translation (He who healed me commanded me: Take up thy mat and walk in peace.)
% #7 collect
% #8 collect translation
  \oldneedspace{4\baselineskip}
  \sectionmark{#2 in \beforecommemweek{}\commemweek{}.}
  \subtitle{\small{#2.}}
  \smallskip
  {
  \def\noeuouae{T}
  \printgabc{At Magn.}{\oldstylenums{Ant.~#3.}}{#4}{#1/#5}
  }
  \translation[]{#6}
  \smallskip
  \ifx\vrtitle\undefined%
  {
    \label{\vrlabel}
    \printvrcommem{}
  }\else%
  \printnote{\emph{\Vbar{}~\vrtitle},
  \ifnum\getpagerefnumber{\vrlabel}=\thepage
  below%
  \else
  page \pageref{\vrlabel}%
  \fi.}
  \fi
  \ifx\precollect\undefined\else\precollect\fi
  \printcollect{#7}{#8}
  \bigskip
  \hrule
  \bigskip
}
\newcommand{\printweektitle}[2][the ]{
  \edef\commemweek{#2}%
  \edef\beforecommemweek{#1}%
  \subtitle{#2.}%
}
{
  \newcommand{\vrtitle}{Roráte cæli}
  \newcommand{\vrlabel}{commvrrorate}
  \newcommand{\commvlatin}{Roráte cæli désuper, et nubes pluant \textbf{ju}stum.}
  \newcommand{\commrlatin}{Aperiátur terra, et gérminet Salva\textbf{tó}rem.}
  \newcommand{\commvtranslation}{Ye heavens, drop down dew from above, and let the clouds rain down the Just One.}
  \newcommand{\commrtranslation}{Let the earth open and bud forth the Saviour.}
  % \oldneedspace{5\baselineskip}
  % \subtitle{First Week of Lent.}\vspace{-0.5\baselineskip}
  % \printcommem{Thursday}{4}{O}{an--o_mulier--solesmes}{O woman, great is thy faith: let it be unto thee as thou hast asked.}
  % {Da, quǽsumus Dómine, pópulis christiánis, et quæ profiténtur agnóscere~:~* et cæléste munus dilígere, quod frequéntant. Per Dóminum.}{Grant, Lord, we beseech Thee, unto all Christian people, that what they now believe they may one day know and see in love unchecked, that heavenly gift whereof now they are the worshippers and the partakers.}

  % \printcommem{Friday}{1}{Q}{an--qui_me_sanum_fecit--solesmes}{He who healed me commanded me: Take up thy bed and walk in peace.}
  % {Exáudi nos miséricors Deus~:~* et méntibus nostris grátiæ tuæ lumen osténde. Per Dóminum nostrum.}
  % {Hear us, O merciful God, and cause the bright beams of thy grace to shine upon our souls.}
  
  \oldneedspace{5\baselineskip}
  \printweektitle{Second Week of Lent}
  % \printcommem{Monday}{1}{Q}{an--qui_me_misit--solesmes}
  % {He who sent Me is with Me and hath not left Me alone: for I always do those things that please Him.}
  % {Adésto supplicatiónibus nostris omnípotesn Deus~:~* et quibus fidúciam sperándæ pietátis indúlges, consuétæ misericórdiæ tríbue benígnus efféctum. Per Dóminum.}
  % {Graciously hear our prayers, O Almighty God, and as Thou dost give us to look with confidence for Thy favour for which we hope, so grant us, in Thy goodness, the manifestation of Thine accustomed mercy.}

  % \printcommem{Tuesday}{4}{O}{an--omnes_autem_vos--solesmes}
  % {And all ye are brethren, and call no man your father upon earth for One is your Father, Which is in heaven neither be ye called masters, for One is your Master, even Christ.}
  % {Propitiáre Dómine supplicatiónibus nostris, et animárum nostrárum medére languóribus~:~* ut remissióne percépta, in tua semper benedictióne lætémur. Per Dóminum.}
  % {Tend thy merciful ears, O Lord, we beseech thee, unto our supplications, and heal the sickness of our souls, that we, receiving thy pardon, may rejoice forever in thy blessing.}
  
  \printcommem{Wednesday}{1}{T}{an--tradetur_enim_gentibus--solesmes}
  {He will be handed over to the gentiles to be mocked and scourged and crucified.}
  {Deus innocéntiæ restitútor et amátor, dírige ad te tuórum corda servórum~:~* ut spíritus tui fervóre concépto, et in fide inveniántur stábiles, et in ópere efficáces. Per Dóminum.}
  {God, the Renewer and Lover of innocence, turn the hearts of Thy servants to Thyself, so that they, enkindled with the fire of Thy Spirit, may be found ever rooted in faith, and fruitful in works.}

  %\newcommand{\vrtitle}{Roráte cæli}
  \printcommem{Thursday}{7}{D}{an--dives_ille_guttam--solesmes}
  {That rich man who had refused Lazarus bread-crumbs, cried for a drop of water.}
  {Adésto Dómine fámulis tuis, et perpétuam benignitátem largíre poscéntibus~:~* ut iis, qui te auctóre et gubernatóre gloriántur, et congregáta restáures, et restauráta consérves. Per Dóminum.}
  {Lord, be present to Thy servants, and grant unto those asking an abiding mercy; that as they boast in Thee, their creator and governer, so Thou wilt renew in them the gifts bestowed, and preserve what Thou hast renewed.}

  \printcommem{Friday}{3}{Q}{an--quaerentes_eum_tenere--solesmes}
  {Seeking to lay hands on Him, they feared the multitude, because they took Him for a Prophet.}
  {Da, quǽsumus Dómine, pópulo tuo salútem mentis et córporis~:~* ut bonis opéribus inhæréndo, tuæ semper virtútis mereátur protectióne deféndi. Per Dóminum.}
  {Grant unto Thy people, O Lord, we beseech Thee, soundness of mind and body, that they, cleaving unto good works, may evermore worthily be defended by the protection of Thy might.}

  \oldneedspace{5\baselineskip}
  \printweektitle{Third Week of Lent}
  \printcommem{Monday}{1}{J}{an--jesus_autem_transiens--solesmes}
  {But Jesus passing through their midst, went His way.}
  {Subvéniat nobis Dómine misericórdia tua~:~* ut ab imminéntibus peccatórum nostrórum perículis te mereámur protegénte éripi, te liberánte salvári. Per Dóminum.}
  {Let our help, O Lord, be in Thy mercy, that we over whom Thy wrath doth most justly hang because of our sins, may in all dangers worthily be shielded by Thy protection and delivered by Thy salvation.}

  %\let\vrtitle=\undefined
  \printcommem{Tuesday}{4}{U}{an--ubi_duo_vel_tres--solesmes}
  {Where two or three are gathered in my name, I am in their midst, saith the Lord.}
  {Tua nos Dómine protectióne defénde~:~* et ab omni semper iniquitáte costódi. Per Dóminum.}
  {O Lord, shield us by Thy protection, and keep us ever from all iniquity.}

  %\newcommand{\vrtitle}{Roráte cæli}
  \printcommem{Wednesday}{7}{N}{an--non_lotis_manibus--solesmes}
  {To eat with unwashen hands, defileth not a man.}
  {Concéde, quǽsumus omnípotens Deus~: ut qui protectiónis tuæ grátiam quǽrimus,~* liberáti a malis ómnibus, secúra tibi mente serviámus. Per Dóminum.}
  {Grant, we beseech Thee, Almighty God, that we who seek the proctection of Thy grace, freed from all evils, may serve Thee in peace and quietness of spirit.}

  %\let\vrtitle=\undefined
  \printcommem{Thursday}{1}{O}{an--omnes_qui_habebant--solesmes}
  {All who had any sick brought them to Jesus, and they were healed.}
  {Subjéctum tibi pópulum, quǽsumus Dómine, propitiátio cæléstis amplíficet~:~* et tuis semper fáciat servíre mandátis. Per Dóminum.}
  {Lord, we beseech Thee that Thine heavenly Peace-Offering may so effectually work for all Thy people, bowing down before Thee, that they may ever continue to keep Thy commandments.}

  %\newcommand{\vrtitle}{Roráte cæli}
  \printcommem{Friday}{3}{D}{an--domine_ut_video--solesmes}
  {Lord, I see that Thou art a prophet: our fathers worshipped on this mountain.}
  {Præsta, quǽsumus omnípotens Deus~:~\dag{} ut qui in tua protectióne confídimus,~* cuncta nobis adversántia te adjuvánte vincámus. Per Dóminum.}
  {Grant, we beseech Thee, Almighty God, that we who trust is Thy protection, may, with Thy help, overcome all evils that rise up against us.}

  \oldneedspace{5\baselineskip}
  \printweektitle{Fourth Week of Lent}
  \printcommem{Monday}{5}{S}{an--solvite_templum_hoc--solesmes}
  {Thus saith the Lord: Destroy this temple, and in three days I will rebuild it. But He was speaking of the temple of His Body.}
  {Deprecatiónem nostram, quǽ\-su\-mus Dómine, benígnus exáudi~:~* et quibus supplicándi præstas afféctum, tríbue defensiónis auxílium. Per Dóminum.}
  {Lord, we beseech Thee, graciously hear our supplication, and evermore help and defend all those to whom Thou hast given the mind to pray.}

  \printcommem{Tuesday}{1}{N}{an--nemo_in_eum_misit--solesmes}
  {No man laid hands on Him; because His hour was not yet come.}
  {Miserére Dómine pópulo tuo~:~* et contínuis tribulatiónibus laborántem, propítius respiráre concéde. Per Dóminum.}
  {Lord, have mercy upon Thy people, and be graciously pleased to grant relief unto the same, who are ever toiling amid the storms of diverse tribulations.}

  %\newcommand{\vrtitle}{Roráte cæli}
  \printcommem{Wednesday}{1}{I}{an--ille_homo--solesmes}
  {The man that is called Jesus, made clay of spittle, and anointed mine eyes, and now I see.}
  {Páteant aures misericórdiæ tuæ Dómine précibus supplicántium~:~* et ut peténtibus desideráta concédas, fac eos quæ tibi sunt plácita postuláre. Per Dóminum.}
  {Let Thy merciful ears, Lord, be open unto the prayers of those entreating Thee, and that Thou mayest grant what we ask, teach us ever to ask what is pleasing to Thee.}

  \printcommem{Thursday}{4}{P}{an--propheta_magnus--solesmes}
  {A great prophet is risen up among us, and God hath visited His people.}
  {Pópuli tui Deus institútor et rector, peccáta quibus impugnátur, expélle~:~* ut semper tibi plácitus, et tuo munímine sit secúrus. Per Dóminum.}
  {O God, Teacher and Shepherd of Thy people, free the same from all sins assailing them, that they may be ever pleasing in Thy sight and safe under Thy shelter.}

  \let\vrtitle=\undefined
  \printcommem{Friday}{1}{D}{an--domine_si_hic_fuisses--solesmes}
  {Lord, if Thou hadst been here, Lazarus had not died; behold already he stinketh, for he hath lain in the grave four days.}
  {Da nobis, quǽsumus omnípotens Deus~:~* ut qui infirmitátis nostræ cónscii, de tua virtúte confídimus, sub tua semper pietáte gaudeámus. Per Dóminum.}
  {Grant, we beseech Thee, Almighty God, unto us who know our weakness and who trust in Thy strength, to ever rejoice in Thy loving kindness.}
}
  
{
  \newcommand{\commvlatin}{Eripe me, Dómine, ab hómine \textbf{ma}lo.}
  \newcommand{\commrlatin}{A viro iníquo éri\textbf{pe} me.}
  \newcommand{\commvtranslation}{Deliver me, O Lord, from the wicked man.}
  \newcommand{\commrtranslation}{And save me from the evil-doer.}
  \newcommand{\vrtitle}{Eripe me}
  \newcommand{\vrlabel}{commvreripeme}
  \oldneedspace{7\baselineskip}
  \printweektitle[]{Passiontide}
  {
  \ifthenelse{\boolean{birmingham}}{
    \def\precollect{\vspace{-0.75\baselineskip}}
  }{
    \def\precollect{\vspace{-0.5\baselineskip}}
  }
  \printcommem{Monday}{4}{S}{an--si_quis_sitit--solesmes}
    {If anyone thirst, let him come and drink; and from his belly will flow living water.}
    {Da, quǽsumus Dómine, pópulo tuo salútem mentis et córporis~:~* ut bonis opéribus inhæréndo, tua semper mereátur protectióne deféndi. Per Dóminum.}
    {Grant unto Thy people, O Lord, we beseech Thee, soundness of mind and body, that they, cleaving unto good works, may evermore worthily be defended by Thy protection.}
  }
  {%\newcommand{\vrtitle}{Eripe me}
  \ifthenelse{\boolean{birmingham}}{
    \def\precollect{\vspace{-0.75\baselineskip}}
  }{}
  \printcommem{Tuesday}{1}{V}{an--vos_ascendite--solesmes}
  {Go ye up unto this Feast; I go not, for My time is not yet full come.}
  {Da nobis, quǽsumus Dómine, perseverántem in tua voluntáte famulátum~:~* ut in diébus nostris, et mérito et número pópulus tibi sérviens augeátur. Per Dóminum.}
  {Grant, we beseech Thee, Lord, to give us grace to endure to the end in doing Thy will, that in our days Thy people which serve Thee may have increase, both in merit and number.}
  }

  {
  \def\precollect{\vspace{-0.5\baselineskip}}
  \printcommem{Wednesday}{4}{M}{an--multa_bona_opera--solesmes}
    {I have wrought many good works among you: on account of which work do you want to kill me?}
    {Adésto supplicatiónibus nostris omnípotens Deus~:~* et quibus fidúciam sperándæ pietátis indúlges, consuétæ misericórdiæ tríbue benígnus efféctum. Per Dóminum.}
    {Gratiously hear our prayers, Almighty God, and as Thou dost give us to look with confidence for Thy favour for which we hope, so grant us, in Thy goodness, Thine accustomed mercy.}
  }
  %\newcommand{\vrtitle}{Eripe me}
  \printcommem{Thursday}{4}{D}{an--desiderio_desideravi--solesmes}
  {With desire I have desired to eat this Pasch with you before I suffer.}
  {Esto, quǽsumus Dómine, propítius plebi tuæ~:~* ut quæ tibi non placent respuéntes, tuórum pótius repleántur delectatiónibus mandatórum. Per Dóminum.}
  {Lord, we beseech Thee, deal mercifully with Thy people, and fill plentifully with the rich things of Thy commandments those who shun that which displeaseth Thee.}

  \let\vrtitle=\undefined
  \printcommem{Friday}{1}{P}{an--principes_consilium--solesmes}
  {The chief Priests consulted that they might kill Jesus, but they said: Not on the Feast-day, lest there be an uproar among the people.}
  {Concéde, quǽsumus omnípotens Deus~:~* ut qui protectiónis tuæ grátiam quǽrimus, liberáti a malis ómnibus, secúra tibi mente serviámus. Per Dóminum.}
  {Grant, we beseech Thee, Almighty God, that we who seek the proctection of Thy grace, freed from all evils, may serve Thee in peace and quietness of spirit.}
}
\vspace{-\baselineskip}
  % \bigskip
  % \benedicamusdomino{}
}


\ifthenelse{\boolean{includeapril23}}{
	%April 23: St George
	{
\section{April 23: St George}
\subtitle{? Class}
\subtitle{II Vespers}
\medskip

%\def\deusinadjutoriumsolemn{T}
%\def\prepsalmtitleone{\vspace{-0.75\baselineskip}}
%\def\prepsalmthreeverses{\vspace{-0.1\baselineskip}}
%\def\prerepeatontiphonthree{}
%\def\prepsalmtitlefour{\needspace{8\baselineskip}}
%\def\premagtitle{\bigskip}

%\printvespers[../April23-StGeorge]{inc-StGeorge-2Vespers}
\printvespers[../CommonOfApostles]{inc-StGeorge-2Vespers}
\medskip
\printnote{\sundaycommemnoteeaster{}

\begin{multicols}{2}
\noindent\emph{\nth{2} Sunday after Easter}, p.~\pageref{easter2-mag}.\\
\emph{\nth{3} Sunday after Easter}, p.~\pageref{easter3-mag}.\\
\emph{\nth{4} Sunday after Easter}, p.~\pageref{easter4-mag}.
\end{multicols}
}
\bigskip
\benedicamusdomino{}
}
}{}

\ifthenelse{\boolean{includemay1}}{
	%May 1: St Joseph the Worker (1st class)
	{
\section{1 May: St Joseph the Worker}
\subtitle{\nth{1} Class}
\subtitle{I \& II Vespers}
\medskip

\def\deusinadjutoriumsolemn{T}
\def\definevesperspropers{\newcommand{\vrtex}{vrOraProNobis}
\newcommand{\vtranslation}{Pray for us, St Joseph, alleluia.}
\newcommand{\rtranslation}{Faithful protector of our labors, alleluia.}

\newcommand{\maganttex}{an--et_ipse_jesus--solesmes}
\newcommand{\magantinitial}{E}
\newcommand{\maganttranslation}{}
\definemag{7}{d}

  \let\oldthing=\maganttranslation
  \def\maganttranslation{\oldthing\needspace{10\baselineskip}}
}
\def\definevesperspropersalt{\newcommand{\vrtex}{vrSolemnitasEstHodie}
\newcommand{\vtranslation}{Today is the solemnity of St Joseph, alleluia.}
\newcommand{\rtranslation}{Who ministered with his hands to the Son of God, alleluia.}

\newcommand{\maganttex}{an--christus_dominus--solesmes}
\newcommand{\magantinitial}{C}
\newcommand{\maganttranslation}{Christ the Lord deigned to be thought the son of a carpenter, alleluia.}
\definemag{7}{c2}

  \def\vrlinebreak{T}
  \let\oldthing=\maganttranslation
  \def\maganttranslation{\oldthing\needspace{10\baselineskip}}
}
\def\vesperspropersnote{At II Vespers:}
\def\vesperspropersaltnote{At I Vespers:}
\ifthenelse{\boolean{birmingham}}{
	%\def\postpsalmtitletwo{\needspace{12\baselineskip}}
	\def\preanttwo{\vspace{-0.3\baselineskip}}
	\def\preantfive{\vspace{-0.4\baselineskip}}
	\def\postpsalmtitlethree{\needspace{15\baselineskip}}
	\def\prepsalmtitlefive{\vspace{-0.1\baselineskip}}
}{
	\def\premagtitle{\bigskip}
}
\def\prepsalmtitleone{\vspace{-0.75\baselineskip}}
%\def\prepsalmthreeverses{\vspace{-0.1\baselineskip}}
%\def\prerepeatontiphonthree{}
\def\prepsalmtitlefour{\needspace{8\baselineskip}}

\printvespers[../May1-StJosephWorker]{inc-StJosephWorker}
%if feast of St Joseph the worker falls from 2nd through 5th Sunday after Easter, it outranks the Sunday and the Sunday is commemorated
\medskip
\printnote{\sundaycommemnoteeaster{}

\begin{multicols}{2}
\noindent\emph{\nth{2} Sunday after Easter}, p.~\pageref{easter2-mag}.\\
\emph{\nth{3} Sunday after Easter}, p.~\pageref{easter3-mag}.\\
\emph{\nth{4} Sunday after Easter}, p.~\pageref{easter4-mag}.\\
\emph{\nth{5} Sunday after Easter}, p.~\pageref{easter5-mag}.
\end{multicols}
}
\ifthenelse{\boolean{birmingham}}{
	%
}{
	\bigskip
}
\benedicamusdomino{}
}
}{}

%May 26: St Philip Neri (1)
{
\section{May 26: St Philip Neri}
\subtitle{\nth{1} Class (proper to the Oratory), White or Gold}
\subtitle{I \& II Vespers}
\medskip

\def\prepsalmtitleone{\needspace{8\baselineskip}}
\def\prepsalmtwoverses{}
%\def\prerepeatantiphonfour{}
%\def\preantfive{\vspace{-0.2\baselineskip}}
%\def\prepsalmtitlefive{\vspace{-0.3\baselineskip}}
%\def\prerepeatantiphonfive{}
\def\definevesperspropers{\newcommand{\maganttex}{MagnificatAntiphon2}
\newcommand{\magantinitial}{H}
\newcommand{\maganttranslation}{Come, children, hearken to me: I will teach you the fear of the Lord.}
\definemag{4}{A}

}
\def\definevesperspropersalt{\newcommand{\maganttex}{MagnificatAntiphon1}
\newcommand{\magantinitial}{D}
\newcommand{\maganttranslation}{My house shall be called a house of prayer, says the Lord.}

}
\def\vesperspropersnote{
\oldneedspace{8\baselineskip}
At II Vespers:}
\def\vesperspropersaltnote{At I Vespers:}
\def\prehymn{\vfill}

\printvespers[../StPhilipNeri]{inc-StPhilipNeri}
% feast can fall from 5th sunday after easter to 2nd sunday after pentecost
% pentecost and trinity would outrank it
\medskip
\printnote{\sundaycommemnoteeasterpentecost{}

\bigskip{}
\emph{\nth{5} Sunday after Easter}, p.~\pageref{easter5}.

\emph{Sunday after the Ascension}, p.~\pageref{easter6}.

% pentecost and trinity sunday would outrank this feast
\emph{\nth{2} Sunday after Pentecost}, p.~\pageref{pentecost2}.
}

\bigskip{}
\benedicamusdomino{}
%\vfil
%\pagebreak
}

\ifthenelse{\boolean{birmingham}}{
	%June 23: Dedication of Oratory Church
	{
\section{23 June: Dedication of Birmingham Oratory Church}
\subtitle{\nth{1} Class}
\medskip
\printnote{If today is Sunday, 22 June then \emph{First Vespers of the Dedication}, on page \pageref{commondedicationofchurch}, is sung without a commemoration of the Sunday.

If today is Sunday, 23 June then \emph{Dedication Vespers} will not be sung but \emph{First Vespers of the Nativity of St John the Baptist}, which follows, takes precedence with a commemoration of \emph{Second Vespers of the Dedication}.
}

\bigskip
\hrule
\bigskip
}
}{}

%June 24: Nativity of St John the Baptist (1st class)
{
\section{June 24: Nativity of St John the Baptist}
\subtitle{\nth{1} Class}
\subtitle{I \& II Vespers}
\medskip

\def\deusinadjutoriumsolemn{T}
\def\definevesperspropers{\newcommand{\antonetex}{Ant1-ElisabethZachariae}
\newcommand{\antoneinitial}{E}
\newcommand{\antonetranslation}{Elizabeth, the wife of Zacharias, gave birth to a great man, John the Baptist, the forerunner of the Lord.}
\definepsalm{1}{109}{3}{a}

\newcommand{\anttwotex}{Ant2-Innuebant}
\newcommand{\anttwoinitial}{I}
\newcommand{\anttwotranslation}{They made signs unto his father, by what name he should be called: and he wrote, saying: His name is John.}
\definepsalm{2}{110}{4}{E}

\newcommand{\antthreetex}{Ant3-JoannesVocabitur}
\newcommand{\antthreeinitial}{J}
\newcommand{\antthreetranslation}{His name shall be called John, and many shall rejoice in his birth.}
\definepsalm{3}{111}{1}{f}

\newcommand{\antfourtex}{Ant4-InterNatos}
\newcommand{\antfourinitial}{I}
\newcommand{\antfourtranslation}{Among those born of women, there hath not risen a greater than John the Baptist.}
\definepsalm{4}{112}{3}{b}

\newcommand{\antfivetex}{Ant5-TuPuer}
\newcommand{\antfiveinitial}{T}
\newcommand{\antfivetranslation}{Thou, child, shalt be called the Prophet of the Most High: thou shalt go before the Lord to prepare His ways.}
\definepsalm{5}{116}{3}{b}

\newcommand{\vrtex}{vrIstePuerMagnus}
\newcommand{\vtranslation}{This child is great before the Lord.}
\newcommand{\rtranslation}{For in truth His hand is with him.}

\newcommand{\maganttex}{MagnificatAntiphon2}
\newcommand{\magantinitial}{P}
\newcommand{\maganttranslation}{The child that is born to us is more than a prophet; for this is he of whom the Saviour said: Among those born of women there hath not risen a greater than John the Baptist.}
\def\magsolemn{T}
\definemag{7}{d}

  %  \def\preantone{\bigskip}
  %  \def\prepsalmtitleone{\bigskip}
  \def\preantfour{\bigskip}
  \def\prepsalmtitlefour{\needspace{14\baselineskip}}
  \ifthenelse{\boolean{birmingham}}{
    \def\prepsalmtitleone{\bigskip}
    \def\postpsalmtitlethree{\needspace{10\baselineskip}}
    \def\prepsalmthreeverses{\vspace{-0.1\baselineskip}}
    \def\prerepeatantiphonthree{}
  }{}
}
\def\definevesperspropersalt{\newcommand{\antonetex}{Ant1-IpsePraeibit}
\newcommand{\antoneinitial}{I}
\newcommand{\antonetranslation}{He shall go before Him in the spirit and power of Elias, to prepare unto the Lord a perfect people.\vspace{0ex plus 0ex minus 3ex}}
\definepsalm{1}{109}{7}{a}

\newcommand{\anttwotex}{Ant2-Joannes}
\newcommand{\anttwoinitial}{J}
\newcommand{\anttwotranslation}{John is his name.  Wine and strong drink he shall not drink, and many shall rejoice in his birth.}
\definepsalm{2}{110}{8}{G}

\newcommand{\antthreetex}{Ant3-ExUteroSenectutis}
\newcommand{\antthreeinitial}{E}
\newcommand{\antthreetranslation}{From the barren womb of age was born John, the forerunner of the Lord.}
\definepsalm{3}{111}{1}{f}

\newcommand{\antfourtex}{Ant4-IstePuer}
\newcommand{\antfourinitial}{I}
\newcommand{\antfourtranslation}{This child is great before the Lord, for the hand of God is with him.}
\definepsalm{4}{112}{4}{A*}

\newcommand{\antfivetex}{Ant5-Nazaraeus}
\newcommand{\antfiveinitial}{N}
\newcommand{\antfivetranslation}{This child shall be called a Nazarite; wine and strong drink he shall not drink, and from his mother's womb he shall eat nothing unclean.}
\definepsalm{5}{116}{5}{a}

\newcommand{\vrtex}{vrFuitHomo}
\newcommand{\vtranslation}{There was a man sent from God.}
\newcommand{\rtranslation}{Whose name was John.}

\newcommand{\maganttex}{MagnificatAntiphon1}
\newcommand{\magantinitial}{I}
\newcommand{\maganttranslation}{When Zacharias had entered the temple of the Lord, there appeared to him the Angel Gabriel, standing at the right hand of the altar of incense.\vspace{-4pt plus 4pt}}
\def\magpsalmclef{c3}
\def\magsolemn{T}
\definemag{8}{G}

  \def\preantthree{\needspace{8\baselineskip}}
  \ifthenelse{\boolean{birmingham}}
  {}
  {
    \def\prepsalmtitletwo{\vspace{-0.5\baselineskip}}
    \def\preantfour{\bigskip}
  }
}
\def\vesperspropersnote{At II Vespers:}
\def\vesperspropersaltnote{At I Vespers:}
\def\prevesperspsalms{\noindent\printnote{Chapter and following, page \pageref{june24-chapter}.\\}}
\def\vesperspsalmslabel{\label{june24-2vespers}}
\def\prevesperspsalmsalt{\noindent\printnote{II Vespers psalms and antiphons, page \pageref{june24-2vespers}.}\medskip\label{june24-1vespers}}
\def\prechapter{\label{june24-chapter}}
%\def\premagnificat{\pagebreak}

\printvespers[../June24-BirthOfJohnTheBaptist]{inc-BirthOfJohnTheBaptist}

\ifthenelse{\boolean{birmingham}}{
  \printnote{If today is Sunday, 23 June then \emph{Second Vespers of the Dedication of a Church} is commemorated as follows.
  \ifthenelse{\boolean{includebenedicamusdominoreferences}}{
    %Otherwise \benedicamusdominoreference{}%
  }{}
  }
  \medskip
  \hrule
  \medskip

  % commemoration of Second Vespers of Dedication of Church
  {
  %\def\beginvrcols{\begin{parcolumns}[rulebetween,colwidths={1=0.45\linewidth}]{2}}
  \printcommemoration[../CommonOfDedicationOfChurch]{commemorationDedicationOfChurch-Vespers2}

  \medskip
  \hrule
  \medskip
  %\bigskip
  \benedicamusdomino{}
  }
  \renewcommand{\sundaycommemnote}{
    If today is Sunday, 24 June then \emph{Vespers of the Sunday} is commemorated with \emph{Magnificat antiphon}, \emph{\Vbar{}~Dirigátur}, and \emph{Collect}.
  }

}{}

\printnote{\sundaycommemnote{}
%\vspace{-\baselineskip}
\begin{multicols}{2}
\noindent\emph{\nth{2} Sunday after Pentecost}, p.~\pageref{pentecost2-mag}.\\
\emph{\nth{3} Sunday after Pentecost}, p.~\pageref{pentecost3-mag}.\\
\emph{\nth{4} Sunday after Pentecost}, p.~\pageref{pentecost4-mag}.\\
\emph{\nth{5} Sunday after Pentecost}, p.~\pageref{pentecost5-mag}.\\
\emph{\nth{6} Sunday after Pentecost}, p.~\pageref{pentecost6-mag}.
\end{multicols}
}
\benedicamusdomino{}
}

%June 29: Ss Peter \& Paul (1st class)
{
\section{June 29: Sts Peter \& Paul}
\subtitle{\nth{1} Class, Red}
\subtitle{I \& II Vespers}
\medskip

\def\definevesperspropers{\import{../CommonOfApostles/}{inc-CommonOfApostles-2Vespers-psalms}
\edef\antonetex{../CommonOfApostles/\antonetex}
\edef\anttwotex{../CommonOfApostles/\anttwotex}
\edef\antthreetex{../CommonOfApostles/\antthreetex}
\edef\antfourtex{../CommonOfApostles/\antfourtex}
\edef\antfivetex{../CommonOfApostles/\antfivetex}

\newcommand{\vrtex}{vrAnnuntiaverunt}
\newcommand{\vtranslation}{They declared the works of God.}
\newcommand{\rtranslation}{And understood His doings.}

\newcommand{\maganttex}{MagnificatAntiphon2-Hodie}
\newcommand{\magantinitial}{H}
\newcommand{\maganttranslation}{Today, Simon Peter went up upon the gibbet of the cross, alleluia; today, he that holdeth the keys of the kingdom, departed with joy to be with Christ; today, the Apostle Paul, the light of the world, bowing his head, for Christ's sake was crowned with martyrdom, alleluia.}
\def\magsolemn{T}
\definemag{1}{D}

  \def\preanttranslationone{\vspace{-0.5\baselineskip}}
  \def\prepsalmtitleone{\vspace{-0.3\baselineskip}}
  \def\prerepeatantiphonone{}
  \def\preantfive{\bigskip}
}
\def\definevesperspropersalt{\newcommand{\antonetex}{Ant1-PetrusEtJoannes}
\newcommand{\antoneinitial}{P}
\newcommand{\antonetranslation}{Peter and John went up together into the Temple at the hour of prayer, being the ninth hour.}
\definepsalm{1}{109}{8}{G}

\newcommand{\anttwotex}{Ant2-Argentum}
\newcommand{\anttwoinitial}{A}
\newcommand{\anttwotranslation}{Silver and gold have I none, but such as I have, give I thee.}
\definepsalm{2}{110}{7}{b}

\newcommand{\antthreetex}{Ant3-DixitAngelusAdPetrum}
\newcommand{\antthreeinitial}{D}
\newcommand{\antthreetranslation}{The Angel said unto Peter: Cast thy garment about thee, and follow me.}
\definepsalm{3}{111}{8}{c}

\newcommand{\antfourtex}{Ant4-MisitDominus}
\newcommand{\antfourinitial}{M}
\newcommand{\antfourtranslation}{The Lord hath sent His Angel, and hath delivered me out of the hand of Herod.  Alleluia.}
\definepsalm{4}{112}{7}{c2}

\newcommand{\antfivetex}{Ant5-TuEsPetrus}
\newcommand{\antfiveinitial}{T}
\newcommand{\antfivetranslation}{Thou art Peter and upon this Rock I will build My Church.}
\definepsalm{5}{116}{7}{c}

\newcommand{\vrtex}{vrInOmnemTerram}
\newcommand{\vtranslation}{Their sound hath gone forth into all the earth.}
\newcommand{\rtranslation}{And their words unto the ends of the world.}

\newcommand{\maganttex}{MagnificatAntiphon1}
\newcommand{\magantinitial}{T}
\newcommand{\maganttranslation}{Thou art the Shepherd of the sheep and the Prince of the Apostles, and unto thee are given the keys of the kingdom of heaven.}
\def\magsolemn{T}
\definemag{1}{f}

  \def\postpsalmtitleone{\oldneedspace{10\baselineskip}}
  \def\postpsalmtitletwo{\oldneedspace{10\baselineskip}}
  \def\preanttwo{\vspace{-0.4\baselineskip}}
  \def\prerepeatantiphontwo{}
}
\def\vesperspropersnote{At II Vespers:}
\def\vesperspropersaltnote{At I Vespers:}
\def\prevesperspsalms{\noindent\printnote{Chapter and following, page \pageref{june29-chapter}.\\}}
\def\vesperspsalmslabel{\label{june29-2vespers}}
\def\prevesperspsalmsalt{\noindent\printnote{II Vespers psalms and antiphons, page \pageref{june29-2vespers}.}\medskip}
\def\prechapter{\label{june29-chapter}}
%\def\precollect{\vspace{-0.5\baselineskip}}

\def\begincollectcols{\begin{parcolumns}[rulebetween,colwidths={1=0.45\linewidth}]{2}}
\printvespers[../June29-StsPeterAndPaul]{inc-StsPeterAndPaul}

\smallskip
\printnote{\sundaycommemnote{}
\vspace{-0.5\baselineskip}
\begin{multicols}{2}
\noindent\emph{\nth{3} Sunday after Pentecost}, p.~\pageref{pentecost3}.\\
\emph{\nth{4} Sunday after Pentecost}, p.~\pageref{pentecost4}.\\
\emph{\nth{5} Sunday after Pentecost}, p.~\pageref{pentecost5}.\\
\emph{\nth{6} Sunday after Pentecost}, p.~\pageref{pentecost6}.\\
\emph{\nth{7} Sunday after Pentecost}, p.~\pageref{pentecost7}.
\end{multicols}
}
\benedicamusdomino{}
}

%\cleardoublepage
%\cleartoleftpage
%July 1: Most Precious Blood (1st class)
{
\global\let\psalmclefthree=\undefined
\section{1 July: The Precious Blood of Our Lord Jesus Christ}
\subtitle{\nth{1} Class}
\subtitle{I \& II Vespers}
\medskip

\def\deusinadjutoriumsolemn{T}
\def\prepsalmtitlethree{\vspace{-0.5\baselineskip}}
\def\premagtitle{\oldneedspace{8\baselineskip}}
\ifthenelse{\boolean{birmingham}}{
	\def\prepsalmtwoverses{\bigskip\bigskip}
	\def\postpsalmtitlefour{\needspace{10\baselineskip}}
	%\def\postpsalmtitlefive{\medskip}
    %\def\chapterreplacement{}
	%\def\prechapter{}
	%\def\prehymn{\needspace{5\baselineskip}}
	\def\begincollectcols{\begin{parcolumns}[rulebetween,colwidths={1=0.43\linewidth}]{2}}
}{
	\def\prepsalmtitleone{\vspace{-0.5\baselineskip}}
	\def\prerepeatantiphonthree{}
	\def\preantfour{\vspace{-0.5\baselineskip}}
	\def\prepsalmtitlefour{\vspace{-0.5\baselineskip}}
	\def\begincollectcols{\begin{parcolumns}[rulebetween,colwidths={1=0.45\linewidth}]{2}}
}
\def\definevesperspropers{\definepsalm{5}{147}{2}{D}

\newcommand{\vrtex}{vrTeErgo}
\newcommand{\vtranslation}{We therefore pray Thee, help Thy servants.}
\newcommand{\rtranslation}{Whom Thou hast redeemed with Thy precious Blood.}

\newcommand{\maganttex}{MagnificatAntiphon2}
\newcommand{\magantinitial}{H}
\newcommand{\maganttranslation}{Ye shall observe this day for a memorial: and ye shall keep it holy unto the Lord, in your generations with an everlasting worship.}
\newcommand{\magsolemn}{F}
\definemag{1}{D2}

	\ifthenelse{\boolean{birmingham}}{
		\def\prepsalmtitlefive{\medskip}
		\def\postpsalmtitlefive{\bigskip}
	}{
		%
	}
}
\def\definevesperspropersalt{\definepsalm{5}{116}{2}{D}

\newcommand{\vrtex}{vrRedemistiNos}
\newcommand{\vtranslation}{Thou hast redeemed us, O Lord, in Thy Blood.}
\newcommand{\rtranslation}{And hast made of us a kingdom unto our God.}

\newcommand{\maganttex}{MagnificatAntiphon1}
\newcommand{\magantinitial}{A}
\newcommand{\maganttranslation}{Ye are come to Mount Sion, to the city of the living God, the heavenly Jerusalem, and to Jesus the Mediator of the new Testament, and to the sprinkling of blood which speaketh better than that of Abel.}
\newcommand{\magsolemn}{F}
\definemag{3}{a}
%
	\ifthenelse{\boolean{birmingham}}{
		\def\prepsalmtitlefive{\medskip}
		\def\postpsalmtitlefive{\bigskip}
	}{}
  \def\postmagtitle{\vspace{-0.5\baselineskip}}%
}
\def\vesperspropersnote{At II Vespers:}
\def\vesperspropersaltnote{At I Vespers:}

\printvespers[../July1-MostPreciousBloodOfChrist]{inc-MostPreciousBloodOfChrist}
\medskip
\benedicamusdomino{}
}
}{}

\ifthenelse{\boolean{includejuly9}}{
	%July 9: Ss Thomas More & John Fisher
	{
\section{9 July: Ss John Fisher \& Thomas More}
\subtitle{\nth{1} Class}
\subtitle{I \& II Vespers}
\medskip

\def\deusinadjutoriumsolemn{F}
\def\vrlinebreak{F}
\def\preanttwo{\vspace{-0.5\baselineskip}}
%\def\prepsalmtitletwo{\vspace{-0.5\baselineskip}}
\ifthenelse{\boolean{birmingham}}{
	%
}{
	% I'm not sure why these are defined for both, since there are different 1st and 2nd vespers psalms
	\def\prepsalmthreeverses{\vspace{-0.1\baselineskip}}
	\def\prerepeatantiphonthree{}
	\def\prepsalmtitlefour{\vspace{-0.3\baselineskip}}
	\def\prepsalmtitlefive{\vspace{-0.5\baselineskip}}
}
\def\begincollectcols{\begin{parcolumns}[rulebetween,colwidths={1=0.44\linewidth}]{2}}
\def\definevesperspropers{%\input{\gabcfolder/inc-StsThomasMoreAndJohnFisher-2Vespers}
  \newcommand{\antonetex}{an--isti_sunt_sancti--solesmes}
\newcommand{\antoneinitial}{I}
\newcommand{\antonetranslation}{These are the Saints who for God's covenant gave up their bodies, and washed their robes in the Blood of the Lamb.}
\definepsalm{1}{109}{2}{D}

\newcommand{\anttwotex}{an--sancti_per_fidem--solesmes}
\newcommand{\anttwoinitial}{S}
\newcommand{\anttwotranslation}{By faith the Saints overcame kingdoms; they performed what was right, they obtained the promises.}
\definepsalm{2}{110}{2}{D}

\newcommand{\antthreetex}{an--sanctorum_velut_aquilae--solesmes}
\newcommand{\antthreeinitial}{S}
\newcommand{\antthreetranslation}{The youth of the Saints shall be renewed like the eagle's; they shall flower like the lily in the City of God.}
\definepsalm{3}{111}{8}{G}

\newcommand{\antfourtex}{an--absterget_deus--solesmes}
\newcommand{\antfourinitial}{A}
\newcommand{\antfourtranslation}{God shall wipe away every tear from the eyes of the Saints; and no longer shall be any sorrow, or crying, nor any pain; for the former things have passed away.}
\definepsalm{4}{112}{7}{a}

\newcommand{\antfivetex}{an--in_caelestibus_regnis--solesmes}
\newcommand{\antfiveinitial}{I}
\newcommand{\antfivetranslation}{The dwelling of the Saints is in the heavenly kingdoms, and their rest for ever.}
\def\psalmclef{2}
\definepsalm{5}{115}{8}{G}
\let\psalmclef=\undefined

  \ifthenelse{\boolean{birmingham}}{
  	\def\prepsalmtitletwo{\bigskip\bigskip\bigskip}
  	\def\postpsalmtitletwo{\bigskip}
  	\def\postpsalmtitlethree{\needspace{10\baselineskip}}
  	\def\postpsalmtitlefive{\bigskip}
  }{
  	%
  }
  %\let\oldthing=\maganttranslation
  %\def\maganttranslation{\oldthing\needspace{10\baselineskip}}
}
\def\definevesperspropersalt{
	%\newcommand{\chaptertext}{\dropcap{latin}{Justórum ánimæ in manu Dei sunt, et non tanget illos torméntum} \textbf{mor}\-tis.~\dag{} Visi sunt óculis insipién\-\emph{ti\-um} \textbf{mo}\-ri~:~* illi autem sunt in \textbf{pa}ce.}
\newcommand{\chaptertranslation}{The souls of the just are in the hand of God~: and the torment of death shall not touch them.  In the sight of the unwise they seemed to die~: but they are in peace.}

\newcommand{\hymnlinetwo}{2.}
\newcommand{\hymntex}{hy--sanctorum_meritis--solesmes}
\newcommand{\hymninitial}{S}
\newcommand{\hymntranslation}{
\item By help of Saints, come let our tongues relate
their famous joys and their courageous deeds;
our mind desires in songs to celebrate
their conquest, which all gain exceeds.

\item While here they lived, the world these men abhorred,
for they this withered soil did much despise
whose flowers are barren, and with thee, their Lord,
up to thy heavenly joys did rise.

\item They for thy sake with stout contempt have borne
the causeless rage of men, and torment fierce,
and cruel hooks, which have their bodies torn,
but had not power their soul to pierce.

\item They like mild sheep to slaughter are assigned,
at which they never murmur nor complain,
but with a silent heart and guiltless mind
their constant patience they maintain.

\item What voice, what tongue, those gifts can fitly show
which thou prepar'st for martyrs? Who, once stained
with streams of blood, which from their wounds did flow,
have now bright crown of laurel gained.

\item We thee beseech, one highest Deity,
to wash our sins, to drive our harms away,
to give thy servants peace, that we
to thee may everlasting praise repay.
Amen.\grechangestaffsize{15}}

%\newcommand{\vrtex}{vrAnnuntiaveruntOperaDei}
%\newcommand{\vtranslation}{They declared the works of God.}
%\newcommand{\rtranslation}{And understood His doings.}

%\newcommand{\maganttex}{MagnificatAntiphon}
%\newcommand{\magantinitial}{E}
%\newcommand{\maganttranslation}{Be ye valiant in warfare, and fight with the old serpent: and ye shall receive an everlasting kingdom.}
%\def\magsolemn{T}
%\definemag{1}{g2}

	\input{../CommonOfMartyrs/inc-CommonOfMartyrs-1Vespers-psalms}

\renewcommand{\maganttex}{an--astiterunt_justi--solesmes}
\renewcommand{\magantinitial}{A}
\renewcommand{\maganttranslation}{The just stood before the Lord, and from one another they were not divided; they drank the cup of the Lord, and were called the friends of God.}
\def\magsolemn{F}
\let\magant=\undefined
\let\magantlinetwo=\undefined
\let\magverses=\undefined
\definemag{1}{g}

	\def\vrlinebreak{T}
	\ifthenelse{\boolean{birmingham}}{
		\def\prepsalmtitletwo{\bigskip\bigskip}
		\def\postpsalmtitletwo{\bigskip}
		\def\postpsalmtitlethree{\needspace{10\baselineskip}}
	}{
		%
	}
}
\def\vesperspropersnote{At II Vespers:}
\def\vesperspropersaltnote{At I Vespers:}
\def\prevesperspsalms{\noindent\printnote{Chapter and following, page \pageref{july9-chapter}.\\}
  \medskip
  \hrule
  \medskip
}
\def\vesperspsalmslabel{\label{july9-2vespers}}
\def\prevesperspsalmsalt{\noindent\printnote{II Vespers psalms and antiphons, page \pageref{july9-2vespers}.}\medskip}
\def\prechapter{\label{july9-chapter}}
% \def\definevesperspropersalt{\newcommand{\vrtex}{vrSolemnitasEstHodie}
\newcommand{\vtranslation}{Today is the solemnity of St Joseph, alleluia.}
\newcommand{\rtranslation}{Who ministered with his hands to the Son of God, alleluia.}

\newcommand{\maganttex}{an--christus_dominus--solesmes}
\newcommand{\magantinitial}{C}
\newcommand{\maganttranslation}{Christ the Lord deigned to be thought the son of a carpenter, alleluia.}
\definemag{7}{c2}

%   \def\vrlinebreak{T}
%   \let\oldthing=\maganttranslation
%   \def\maganttranslation{\oldthing\needspace{10\baselineskip}}
% }
% \def\prepsalmtitleone{\vspace{-0.75\baselineskip}}
%\def\prepsalmthreeverses{\vspace{-0.1\baselineskip}}
%\def\prerepeatontiphonthree{}
% \def\prepsalmtitlefour{\needspace{8\baselineskip}}
% \def\premagtitle{\bigskip}

\printvespers[../CommonOfMartyrs]{inc-sts-thomas-more-and-john-fisher-2vespers}
%if feast of St Joseph the worker falls from 2nd through 5th Sunday after Easter, it outranks the Sunday and the Sunday is commemorated
\bigskip
\benedicamusdomino{}
}
}{}


\ifthenelse{\boolean{testrun}}{}{
%Aug 6: Transfiguration (2nd class)
{
\section{6 August: Transfiguration of Our Lord Jesus Christ}
\subtitle{\nth{2} Class}
\subtitle{I \& II Vespers}
\medskip

%\def\prepsalmtitleone{\needspace{4\baselineskip}}
\ifthenelse{\boolean{birmingham}}{
	\def\postpsalmtitleone{\medskip}
	\def\prepsalmoneverses{\bigskip}
	\def\psepsalmtitletwo{\bigskip}
	\def\prepsalmtitlefour{\oldneedspace{4\baselineskip}}
	\def\prepsalmtitlefive{\needspace{4\baselineskip}}
	\def\premagtitle{\needspace{10\baselineskip}}
}{
	\def\postpsalmtitleone{\needspace{4\baselineskip}}
	\def\prepsalmoneverses{}
	\def\prerepeatantiphonone{}	
	\def\preanttwo{\vspace{-0.5\baselineskip}}
	\def\preanttranslationtwo{\vspace{-0.2\baselineskip}}
	\def\prerepeatantiphontwo{}
	\def\prepsalmtitletwo{\vspace{-0.5\baselineskip}}
	\def\prepsalmtwoverses{\vspace{-0.01\baselineskip}}
}
\def\definevesperspropers{\newcommand{\maganttex}{MagnificatAntiphon2}
\newcommand{\magantinitial}{E}
\newcommand{\maganttranslation}{And the disciples hearing, fell on their faces, and were sore afraid; and Jesus came, and touched them, and said to them, Arise, and fear not, alleluia.}
\newcommand{\magsolemn}{F}
\definemag{1}{f}

  \def\premagverses{\vspace{-0.5\baselineskip}}
}
\def\definevesperspropersalt{\newcommand{\maganttex}{an--christus_jesus--solesmes}
\newcommand{\magantinitial}{C}
\newcommand{\maganttranslation}{Christ Jesus, radiance of the Father and image of His Being, upholding all things by the word of His power; making atonement for sins, has deigned to appear today in glory on the high mountain.}
\newcommand{\magsolemn}{F}
\definemag{4}{E}

  \def\premagverses{\vspace{-0.5\baselineskip}}
}
\def\vesperspropersnote{At II Vespers:}
\def\vesperspropersaltnote{At I Vespers:}
\def\postmagtitle{\vspace{-0.5\baselineskip}}

\def\begincollectcols{\begin{parcolumns}[rulebetween,colwidths={1=0.44\linewidth}]{2}}
\printvespers[../August6-TransfigurationOfOurLord]{inc-Transfiguration}
\smallskip
\benedicamusdomino[2]{}
}

%Aug 15: Assumption (1st class)
{
\section{August 15: Assumption of the B.~V.~M.}
\subtitle{\nth{1} Class}
\subtitle{I \& II Vespers}
\medskip

%\def\prepsalmfivetitle{\vspace{-0.5\baselineskip}}
%\def\preanttwo{\oldneedspace{12\baselineskip}}
\def\deusinadjutoriumsolemn{T}
\ifthenelse{\boolean{birmingham}}{
  \def\prepsalmtitletwo{\needspace{8\baselineskip}}
}{
  \def\prepsalmoneverses{}
  \def\prerepeatantiphonone{}
  \def\preanttwo{\vspace{-0.5\baselineskip}}
  \def\preanttranslationtwo{\vspace{-0.2\baselineskip}}
  \def\prepsalmtitletwo{\vspace{-0.5\baselineskip}}  
}
\def\precollect{\needspace{8\baselineskip}}

\def\definevesperspropers{% hymn is ave maris stella
%\input{inc-hymn-avemarisstella}

\newcommand{\maganttex}{an--hodie_maria_virgo--solesmes}
\newcommand{\magantinitial}{H}
\newcommand{\maganttranslation}{Today the Virgin Mary has gone up to heaven: rejoice, for with Christ she reigns forever.}
\newcommand{\magsolemn}{T}
\definemag{8}{G*}
}
\def\definevesperspropersalt{\newcommand{\hymnlinetwo}{2.}
\newcommand{\hymntex}{Hymn-OPrimaVirgoProdita}
\newcommand{\hymninitial}{O}
\newcommand{\hymntranslation}{
\item O Virgin who was first to receive
The Creator’s grace by the spirit,
Who was predestined by the Most High
To bear in her womb the Son.

\item O woman, who was foretold to be
The perpetual enemy of the demon;
Who alone was filled with grace,
Undefiled from conception.

\item Thou who conceives Life itself in thy womb,
Life that was lost by Adam;
Furnishing the divine Victim,
A body for his sacrifice.

\item Death, the recompense for sin,
Had no victory over thee, and now departs;
And then thou hastened bodily to heaven
To be thy loving Son’s companion.

\item Illuminated by so great a Glory,
All nature is raised up;
And in thee is called to reach
The pinnacle of all glory and splendour.

\item In thy triumph O our Queen,
Turn thine eyes to us exiles;
That through thy patronage,
We may come to heaven, our blessed homeland.

\item Praise to the Father! praise to Him,
The Virgin’s holy Son!
Praise to the Spirit Paraclete,
While endless ages run! 
Amen.
}

\newcommand{\maganttex}{MagAntiphon-VirgoPrudentissima}
\newcommand{\magantinitial}{V}
\newcommand{\maganttranslation}{O Virgin most prudent, whither goest thou, like the golden dawn?  Daughter of Sion, thou art all beautiful and sweet; fair as the moon, bright as the sun.}
\newcommand{\magsolemn}{T}
\definemag{1}{f}

  \let\oldthing=\maganttranslation
  \def\maganttranslation{\oldthing\needspace{10\baselineskip}}
}
\def\premagtitle{\needspace{12\baselineskip}}
\def\vesperspropersnote{At II Vespers:}
\def\vesperspropersaltnote{At I Vespers:}
\def\printfullhymn{
  {
    \oldneedspace{3\baselineskip}
    \printnote{At II Vespers: Hymn.~\emph{Ave Maris Stella}, p.~\pageref{hymn-avemarisstella}. \Vbar{} \emph{Exaltata.} p.~\pageref{vr-assumption}.\\}

    \printnote{\vesperspropersaltnote}
    \definevesperspropersalt
    \printhymn{\oldstylenums{\hymnlinetwo}}{\hymninitial}{\hymntex}{\hymntranslation}
  }
  {
    \def\vrlinebreak{T}
    \oldneedspace{3\baselineskip}
    \label{vr-assumption}
    \printvr[\greseteolcustos{manual}]{\vrtex}{\vtranslation}{\rtranslation}
  }
}

\printvespers[../August15-AssumptionOfTheBlessedVirginMary]{inc-Assumption}

\medskip
\printnote{
  If today is Saturday, \emph{Vespers of the Saturday before 3rd Sunday of August} is commemorated as follows:\par
}
  \oldneedspace{4\baselineskip}
  {
  \def\noeuouae{T}
  \printgabc{At Magn.}{\oldstylenums{Ant.~8.~G}}{O}{../August15-AssumptionOfTheBlessedVirginMary/an--omnis_sapientia--solesmes}
  }
  \translation[]{All wisdom is from the Lord God, and with Him it will always be, and has been before all time.}
  \smallskip
  {
    \newcommand{\commvlatin}{Vespertína orátio ascéndat ad te \textbf{Dó}mine.}
    \newcommand{\commrlatin}{Et descéndat super nos misericórdia \textbf{tú}a.}
    \newcommand{\commvtranslation}{May the evening prayer ascend to Thee, O Lord.}
    \newcommand{\commrtranslation}{And may Thy mercy descend upon us.}
    \printvrcommem{}
  }
  \medskip
  \printnote{The \emph{Collect} is taken from the respective Sunday, found below.}
  \medskip
  \hrule
  \bigskip

\printnote{\sundaycommemnote{}
\begin{multicols}{2}
\noindent\emph{\nth{9} Sunday after Pentecost}, p.~\pageref{pentecost9-mag}.\\
\emph{\nth{10} Sunday after Pentecost}, p.~\pageref{pentecost10-mag}.\\
\emph{\nth{11} Sunday after Pentecost}, p.~\pageref{pentecost11-mag}.\\
\emph{\nth{12} Sunday after Pentecost}, p.~\pageref{pentecost12-mag}.\\
\emph{\nth{13} Sunday after Pentecost}, p.~\pageref{pentecost13-mag}.
\end{multicols}
}
\bigskip
\benedicamusdomino{}
}

%\cleardoublepage
%Sep 14: Exaltation of Holy Cross (2nd class)
{
\section{14 September: Exaltation of the Holy Cross}
\subtitle{\nth{2} Class}
\subtitle{I \& II Vespers}
\medskip

\def\definevesperspropers{\newcommand{\maganttex}{MagnificatAntiphon2-OCrux}
\newcommand{\magantinitial}{O}
\newcommand{\maganttranslation}{O blessed art thou, O Cross which wast counted the only tree worthy to bear the Lord and King of heaven. Alleluia.}
\def\magsolemn{F}
\definemag{1}{D2}
}
\def\definevesperspropersalt{\newcommand{\maganttex}{MagnificatAntiphon1-OCrux}
\newcommand{\magantinitial}{T}
\newcommand{\maganttranslation}{O Cross, brighter than all the stars thy name is honourable upon earth; exceeding lovely to mankind; holier than all things; thou alone wast worthy to carry the ransom of the world; sweet wood, sweet nails, bearing a burden sweeter still; save this people gathered here to praise thee.}
\def\magsolemn{F}
\definemag{1}{D}
%
%\def\premagverses{\oldneedspace{4\baselineskip}}%
}
\def\vesperspropersnote{At II Vespers:}
\def\vesperspropersaltnote{At I Vespers:}
\ifthenelse{\boolean{birmingham}}{
	\def\prepsalmtitleone{\vspace{-0.7\baselineskip}}
	\def\prepsalmtwoverses{\bigskip\bigskip}
}{
	\def\prepsalmtitleone{\vspace{-0.3\baselineskip}}
	\def\prerepeatantiphonthree{}
	\def\prepsalmthreeverses{\vspace{-0.1\baselineskip}}
}
\def\postpsalmtitlethree{\oldneedspace{9\baselineskip}}
\def\prepsalmtitlefour{\needspace{8\baselineskip}}
%\def\prepsalmfourverses{\oldneedspace{2\baselineskip}}
\def\prepsalmtitlefive{\needspace{8\baselineskip}}
\def\prehymn{\printnote{All kneel for the sixth verse of the following hymn.}}
\def\prehymntranslation{\oldneedspace{3\baselineskip}}
%\def\precollect{\vspace{-0.5\baselineskip}}

\def\begincollectcols{\begin{parcolumns}[rulebetween,colwidths={1=0.42\linewidth}]{2}}
\printvespers[../September14-ExaltationOfTheHolyCross]{inc-ExaltationOfTheHolyCross}
\bigskip
\benedicamusdomino[2]{}
}

%Sep 29: Dedication of St Michael (1st class)
{
\section{September 29: Dedication of St Michael the Archangel}
\subtitle{\nth{1} Class, White or Gold}
\subtitle{I \& II Vespers}
\medskip

\def\postdeusinadjutorium{\needspace{10\baselineskip}}

\def\definevesperspropers{\definepsalm{5}{137}{7}{c}

\newcommand{\vrtex}{vrInConspectuAngelorum}
\newcommand{\vtranslation}{In the sight of the Angels, I will sing praise to Thee, O my God.}
\newcommand{\rtranslation}{I will worship towards Thy holy temple, and I will give glory to Thy name.}

\newcommand{\maganttex}{MagnificatAntiphon2}
\definemag{1}{D2}
\newcommand{\magantinitial}{P}
\newcommand{\maganttranslation}{O most glorious prince, Michael Archangel, be mindful of us, and here and everywhere entreat the Son of God for us, alleluia, alleluia.}

  \def\vrlinebreak{F}
}
\def\definevesperspropersalt{\definepsalm{5}{116}{7}{c}

\newcommand{\vrtex}{vrStetitAngelus}
\newcommand{\vtranslation}{The Angel stood by the altar of the temple.}
\newcommand{\rtranslation}{Holding in his hand a censer of gold.}

\newcommand{\maganttex}{MagnificatAntiphon1}
\newcommand{\magantinitial}{D}
\definemag{8}{G}
\newcommand{\maganttranslation}{While John was beholding the sacred Mystery, the Archangel Michael sounded a trumpet.  Forgive us, O Lord our God, Thou who openest the book, and loosest the seals thereof.  Alleluia.\vspace{-1ex}}
}
\def\vesperspropersnote{At II Vespers:}
\def\vesperspropersaltnote{At I Vespers:}
%\def\prepsalmtitleone{\vspace{-0.5\baselineskip}}
\def\preanttwo{\bigskip\bigskip}
\def\prepsalmtitletwo{\bigskip}
\def\postpsalmtitlethree{\oldneedspace{9\baselineskip}}
\def\prerepeatantiphonthree{}
\def\prepsalmthreeverses{\vspace{-0.1\baselineskip}}
%\def\prepsalmtitlefour{\needspace{8\baselineskip}}
%\def\prepsalmfour{\needspace{8\baselineskip}}
\def\prechapter{\vspace{-\baselineskip}}
\def\postmagtitle{\vspace{-\baselineskip}}
\def\premagverses{\oldneedspace{8\baselineskip}}

\printvespers[../September29-DedicationOfChurchOfStMichaelArchangel]{inc-DedicationStMichael}
%an--domine_rex_omipotens--solesmes is for the commemoration of the saturday

\medskip
\printnote{\sundaycommemnote{}

\begin{multicols}{2}
\noindent\emph{\nth{16} Sunday after Pentecost}, p.~\pageref{pentecost16}.\\
\emph{\nth{17} Sunday after Pentecost}, p.~\pageref{pentecost17}.\\
\emph{\nth{18} Sunday after Pentecost}, p.~\pageref{pentecost18}.\\
\emph{\nth{19} Sunday after Pentecost}, p.~\pageref{pentecost19}.\\
\emph{\nth{20} Sunday after Pentecost}, p.~\pageref{pentecost20}.
\end{multicols}
}
\medskip
\benedicamusdomino{}
}

%\cleardoublepage
\cleartoleftpage

}

\ifthenelse{\boolean{includeoct7}}{
	%October 7: Our Lady of the Rosary
	{
\ifthenelse{\boolean{birmingham}}{
	\section{First Sunday in October: Rosary Sunday}	
}{
	\section{October 7: Our Lady of the Rosary}
}
%\subtitle{\nth{1} Class}
\subtitle{II Vespers}
\medskip

\def\deusinadjutoriumsolemn{F}
\def\definevesperspropers{%\input{\gabcfolder/inc-StsThomasMoreAndJohnFisher-2Vespers}
  \newcommand{\maganttex}{an--beata_mater--solesmes}
\newcommand{\magantinitial}{B}
\newcommand{\maganttranslation}{Blessed Mother and unspotted Virgin, glorious Queen of the world, may all be conscious of thine aid who keep thy feast of the most holy Rosary.}
\def\magpsalmclef{c3}
\definemag{8}{G}

  \def\vrlinebreak{T}
  %\let\oldthing=\maganttranslation
  %\def\maganttranslation{\oldthing\needspace{10\baselineskip}}
}
%\def\prepsalmtitleone{\vspace{-0.5\baselineskip}}
%\def\preanttwo{\vspace{-0.3\baselineskip}}
%\def\prepsalmtitletwo{\vspace{-0.1\baselineskip}}
%\def\prepsalmtwoverses{\vspace{-0.1\baselineskip}}
%\def\prepsalmtitlethree{\vspace{-0.5\baselineskip}}
% \def\definevesperspropersalt{\newcommand{\vrtex}{vrAmavitEum}
\newcommand{\vtranslation}{The Lord loved him and adorned him.}
\newcommand{\rtranslation}{He clothed him with a robe of glory.}

\newcommand{\magantinitial}{S}
\newcommand{\maganttranslation}{I will liken him to a wise man, that build his house upon a rock.}
\newcommand{\maganttex}{mag-ant-1-similabo-eum}
\def\magsolemn{T}
\definemag{1}{D}

%   \def\vrlinebreak{T}
%   \let\oldthing=\maganttranslation
%   \def\maganttranslation{\oldthing\needspace{10\baselineskip}}
% }
% \def\vesperspropersnote{At II Vespers:}
% \def\vesperspropersaltnote{At I Vespers:}
% \def\prepsalmtitleone{\vspace{-0.75\baselineskip}}
%\def\prepsalmthreeverses{\vspace{-0.1\baselineskip}}
%\def\prerepeatontiphonthree{}
% \def\prepsalmtitlefour{\needspace{8\baselineskip}}
% \def\premagtitle{\bigskip}

\printvespers[../October7-OurLadyOfTheRosary]{inc-OurLadyOfTheRosary}
\bigskip
\benedicamusdomino{}
}
}{}

\ifthenelse{\boolean{testrun}}{}{
\ifthenelse{\boolean{includeoct9}}{
	%October 9: Blessed John Henry Newman
	{
\section{October 9: Blessed John Henry Newman}
\subtitle{\nth{2} Class}
\subtitle{I \& II Vespers}
\medskip

\def\deusinadjutoriumsolemn{F}
\def\definevesperspropers{%\input{\gabcfolder/inc-StsThomasMoreAndJohnFisher-2Vespers}
  \newcommand{\vrtex}{vrJustumDeduxit}
\newcommand{\vtranslation}{The Lord led the just by right ways.}
\newcommand{\rtranslation}{And showed him the kingdom of God.}

\newcommand{\magantinitial}{H}
\newcommand{\maganttranslation}{This man, despising the world and earthly things, by word and deed has laid up treasures in heaven.}
\newcommand{\maganttex}{mag-ant-2-hic-vir-despiciens-mundum}
\def\magsolemn{T}
\def\magpsalmclef{c3}
\definemag{8}{G}

  \def\vrlinebreak{T}
  %\let\oldthing=\maganttranslation
  %\def\maganttranslation{\oldthing\needspace{10\baselineskip}}
}
\def\prepsalmtitleone{\vspace{-0.5\baselineskip}}
\def\preanttwo{\vspace{-0.3\baselineskip}}
\def\prepsalmtitletwo{\vspace{-0.1\baselineskip}}
\def\prepsalmtwoverses{\vspace{-0.1\baselineskip}}
\def\prepsalmtitlethree{\vspace{-0.5\baselineskip}}
\def\definevesperspropersalt{\newcommand{\vrtex}{vrAmavitEum}
\newcommand{\vtranslation}{The Lord loved him and adorned him.}
\newcommand{\rtranslation}{He clothed him with a robe of glory.}

\newcommand{\magantinitial}{S}
\newcommand{\maganttranslation}{I will liken him to a wise man, that build his house upon a rock.}
\newcommand{\maganttex}{mag-ant-1-similabo-eum}
\def\magsolemn{T}
\definemag{1}{D}

  \def\vrlinebreak{T}
  \let\oldthing=\maganttranslation
  \def\maganttranslation{\oldthing\needspace{10\baselineskip}}
}
% \def\vesperspropersnote{At II Vespers:}
% \def\vesperspropersaltnote{At I Vespers:}
% \def\prepsalmtitleone{\vspace{-0.75\baselineskip}}
%\def\prepsalmthreeverses{\vspace{-0.1\baselineskip}}
%\def\prerepeatontiphonthree{}
% \def\prepsalmtitlefour{\needspace{8\baselineskip}}
% \def\premagtitle{\bigskip}

\printvespers[../CommonOfConfessorNotBishop]{inc-BlessedJohnHenryNewman}
%if feast of St Joseph the worker falls from 2nd through 5th Sunday after Easter, it outranks the Sunday and the Sunday is commemorated
\bigskip
\benedicamusdomino{}
}
}{}

\ifthenelse{\boolean{includeoct11}}{
	%Oct 11: Maternity of the BVM (2nd class)
	{
\section{October 11: Maternity of the B.~V.~M.}
\subtitle{\nth{2} Class, White}
\subtitle{I \& II Vespers}
\medskip

\printnote{This feast is outranked by any Sunday and no commemoration is made.}

\def\definevesperspropers{\newcommand{\maganttex}{MagnificatAntiphon2}
\newcommand{\magantinitial}{M}
\newcommand{\maganttranslation}{Thy Motherhood, O Virgin Mother of God, heralded joy to the whole world: for out of thee has arisen the sun of justice, Christ our God.}
\definemag{4}{E}

}
\def\definevesperspropersalt{\newcommand{\maganttex}{MagnificatAntiphon1}
\newcommand{\magantinitial}{C}
\newcommand{\maganttranslation}{Let us celebrate with joy the Motherhood of blessed Mary ever Virgin.}
\definemag{7}{a}
}
\def\vesperspropersnote{At II Vespers:}
\def\vesperspropersaltnote{At I Vespers:}
\def\printhymnnote{
  {
    \oldneedspace{3\baselineskip}
    \printnote{Hymn.~\emph{Ave Maris Stella}, p.~\pageref{hymn-avemarisstella}.\\}
  }
}
%\def\prepsalmtitleone{\vspace{-0.4\baselineskip}}
\def\prepsalmtitletwo{}
\def\prepsalmtitlefive{\vspace{-0.7\baselineskip}}
\def\prevespers{
  \let\oldthing=\anttwotranslation
  \def\anttwotranslation{\vspace{-0.3\baselineskip}\oldthing\vspace{-0.7\baselineskip}}
}
\def\vrlabel{vr-october11}
\printvespers[../October11-MaternityOfBlessedVirginMary]{inc-MaternityOfBVM}

%an--refulsit_sol--solesmes is for commem of saturday
\medskip
\benedicamusdomino[mary]{}
}
}{}

%Last Sunday in October: Christ the King (1st class)
{
\ifthenelse{\boolean{birmingham}}{
  \def\mytitle{Last Sunday in October: Kingship of Our Lord, Jesus Christ}
}{
  \def\mytitle{Last Sunday in October: Jesus Christ, King}
}
\section{\mytitle}
\subtitle{\nth{1} Class}
\ifthenelse{\boolean{includechristthekingfirstvespers}}{% true
	\def\definevesperspropersalt{\newcommand{\vrtex}{vrDataEstMihi}
\newcommand{\vtranslation}{All power is given to me.}
\newcommand{\rtranslation}{In heaven and in earth.}

\newcommand{\magantinitial}{D}
\newcommand{\maganttex}{an--dabit_illi--solesmes}
\newcommand{\maganttranslation}{The Lord God shall give unto him the throne of David his father~: and he shall reign in the house of Jacob forever, and of his kingdom there shall be no end.}
\def\magsolemn{T}
\definemag{1}{f}
}
	\def\vesperspropersnote{At II Vespers:}
	\def\vesperspropersaltnote{At I Vespers:}
	\subtitle{I \& II Vespers}
}{% false
	\subtitle{II Vespers}
}
\medskip

\def\deusinadjutoriumsolemn{T}
\ifthenelse{\boolean{birmingham}}{
	%\def\prepsalmtitleone{\vspace{-0.5\baselineskip}}
	%\def\prepsalmtitletwo{\vspace{-0.5\baselineskip}}
	%\def\prerepeatantiphontwo{}
	%\def\preantthree{\vspace{-0.2\baselineskip}}
	%\def\prepsalmtitlethree{\vspace{-0.1\baselineskip}}
	\def\prepsalmtitlefour{\vspace{-0.5\baselineskip}}
	\def\prechapter{\vspace{-1\baselineskip}}
	\def\beginchaptercols{\begin{parcolumns}[rulebetween,colwidths={1=0.43\linewidth}]{2}}
}{
	\def\postpsalmtitleone{\needspace{8\baselineskip}}
	\def\prepsalmoneverses{}
	\def\prerepeatantiphonone{}
	\def\preanttwo{\vspace{-0.5\baselineskip}}
	\def\preanttranslationtwo{\vspace{-0.5\baselineskip}}
	\def\prepsalmtitletwo{\vspace{-0.5\baselineskip}}
	\def\postpsalmtitletwo{\needspace{12\baselineskip}}
	\def\prerepeatantiphontwo{}
	\def\postpsalmtitlefour{\bigskip}
	\def\beginchaptercols{\begin{parcolumns}[rulebetween,colwidths={1=0.46\linewidth}]{2}}
}
\def\prehymn{\vfill}
\def\definevesperspropers{\newcommand{\vrtex}{vr}
\newcommand{\vtranslation}{His dominion shall be increased.}
\newcommand{\rtranslation}{And of peace there shall be no end.}

\newcommand{\magantinitial}{H}
\newcommand{\maganttex}{MagnificatAntiphon}
\newcommand{\maganttranslation}{He hath on His garment and on His thigh written: King of kings and Lord of lords.  To Him be glory and empire for ever and ever.}
\def\magsolemn{T}
\definemag{7}{a}
}
%\def\begincollectcols{\begin{parcolumns}[rulebetween,colwidths={1=0.45\linewidth}]{2}}

\printvespers[../OctoberLastSunday-ChristTheKing]{inc-ChristTheKing}
\noindent
\printnote{If today is 31 October, \emph{First Vespers of All Saints} is commemorated with \emph{Magnificat antiphon}, p.~\pageref{allsaints1-magnificat}; \emph{\Vbar{}~Lætámini}.~in simple commemoration tone, p.~\pageref{allsaints1-vr}; and \emph{Collect}, p.~\pageref{allsaints-collect}.}

\bigskip
\benedicamusdomino[1]{}
}

%Nov 1: All Saints (1st class)
{
\section{November 1: All Saints}
\subtitle{\nth{1} Class, White or Gold}
\subtitle{I \& II Vespers}
\medskip

\def\deusinadjutoriumsolemn{T}
\ifthenelse{\boolean{birmingham}}{
	\def\prerepeatantiphonone{}	
	\def\preanttwo{\vspace{-0.2\baselineskip}}
	\def\prepsalmtitletwo{\vspace{-0.1\baselineskip}}
	\def\postpsalmtitlethree{\needspace{8\baselineskip}}
	\def\prepsalmthreeverses{\vspace{-0.05\baselineskip}}
	\def\prerepeatantiphonthree{}
}{
	\def\postdeusinadjutorium{\pagebreak}
	\def\postpsalmtitleone{\oldneedspace{12\baselineskip}}
	%\def\prepsalmoneverses{\vspace{-0.2\baselineskip}}
	%\def\presalmtitleone{\vspace{-0.8\baselineskip}}
	\def\prerepeatantiphonone{}
	%\def\preanttwo{\vspace{-0.5\baselineskip}}
	%\def\prepsalmtitletwo{\vspace{-0.3\baselineskip}}
	\def\postpsalmtitletwo{\needspace{12\baselineskip}}
	\def\prepsalmtitlethree{\vspace{-\baselineskip}}
	\def\prerepeatantiphontwo{}
	\def\postpsalmtitlefour{\needspace{12\baselineskip}}
}
\def\prevespers{
  \let\oldthinga=\antonetranslation
  \def\antonetranslation{\vspace{-0.2\baselineskip}\oldthinga\vspace{-0.5\baselineskip}}
  %\let\oldthing=\anttwotranslation
  %\def\anttwotranslation{\vspace{-0.5\baselineskip}\oldthing}
}
\def\definevesperspropers{\definepsalm{5}{115}{8}{G}

\newcommand{\vrtex}{vrExsultabunt}
\newcommand{\vtranslation}{The Saints will rejoice in glory.}
\newcommand{\rtranslation}{They will be joyful upon their beds.}

\newcommand{\maganttex}{an--o_quam_gloriosum--solesmes}
\newcommand{\magantinitial}{O}
\newcommand{\maganttranslation}{Oh! how glorious is the kingdom where all the Saints rejoice with Christ; clothed in white robes, they follow the Lamb whithersoever he goeth!}
\def\magsolemn{T}
\definemag{6}{F}

  %\def\prepsalmtitlefive{\bigskip}
  \def\prepsalmtitlefive{\vspace{-0.5\baselineskip}}
}
\def\definevesperspropersalt{\definepsalm{5}{116}{8}{G}

\newcommand{\vrtex}{vrLaetamini}
\newcommand{\vtranslation}{Be glad in the Lord, and rejoice ye righteous.}
\newcommand{\rtranslation}{And shout for joy, all ye that are upright in heart.}

\newcommand{\maganttex}{an--angeli_archangeli_all_saints--solesmes}
\newcommand{\magantinitial}{A}
\newcommand{\maganttranslation}{O ye Angels, Archangels, Thrones and Dominions, Principalities and Powers, Virtues, Cherubim and Seraphim, Patriarchs and Prophets, holy Teachers of the Law, all Apostles, Martyrs of Christ, holy Confessors, Virgins of the Lord, Hermits, and all Saints, intercede for us.}
\def\magsolemn{T}
\definemag{1}{D}

  \def\prepsalmtitlefive{\vspace{-0.5\baselineskip}}
  \def\prerepeatantiphonfive{}
  \def\premag{\label{allsaints1-magnificat}}
}
\def\vraltlabel{allsaints1-vr}
\def\vesperspropersnote{%
	\ifthenelse{\boolean{birmingham}}{%
	}{%
		\oldneedspace{15\baselineskip}At II Vespers:%
	}%
}
\def\vesperspropersaltnote{At I Vespers:}
\def\prechapter{\vspace{-\baselineskip}}
\def\begincollectcols{\label{allsaints-collect}\begin{parcolumns}[rulebetween]{2}}

\printvespers[../November1-AllSaints]{inc-AllSaints}

\medskip
%an--vidi_dominum--solesmes is for commem of saturday
\printnote{\sundaycommemnote{}

\begin{multicols}{2}
\noindent\emph{\nth{21} Sunday after Pentecost}, p.~\pageref{pentecost21-mag}.\\
\emph{\nth{22} Sunday after Pentecost}, p.~\pageref{pentecost22-mag}.\\
\emph{\nth{23} Sunday after Pentecost}, p.~\pageref{pentecost23-mag}.\\
\emph{\nth{4} Sunday after Epiphany}, p.~\pageref{epiphany4-mag}.
\end{multicols}
}
\bigskip{}
\benedicamusdomino{}
}

%Nov 9: Dedication of Archbasilica of Holy Saviour (2nd class)
{
\section{November 9: Dedication of Archbasilica of Holy Savior}
\subtitle{\nth{2} Class, White}
\subtitle{I \& II Vespers}
\medskip

\printnote{All from the Common of the Dedication of a Church.}

\def\prepsalmtitleone{\needspace{8\baselineskip}}
\def\postpsalmtitletwo{\needspace{8\baselineskip}}
\def\prepsalmtwoverses{\vspace{-0.05\baselineskip}}
\def\prerepeatantiphontwo{}
\def\postpsalmtitlethree{\medskip\needspace{10\baselineskip}}
\def\prepsalmthreeverses{\medskip}
%\def\prepsalmthreeverses{\vspace{-0.1\baselineskip}}
%\def\prerepeatantiphonthree{}
%\def\prerepeatantiphonfive{}
%\def\prepsalmtitlefive{\bigskip}
\def\prepsalmfiveverses{\smallskip}
%\def\prevespers{
%  \let\oldthing=\antfivetranslation
%  \def\antfivetranslation{\vspace{-0.6\baselineskip}\oldthing\vspace{-1\baselineskip}}
%}
\def\prevr{\needspace{10\baselineskip}}
\def\definevesperspropers{\newcommand{\vrtex}{vrDomumTuam}
\newcommand{\vtranslation}{Holiness becometh thy house, O Lord.}
\newcommand{\rtranslation}{Forever.}

\newcommand{\maganttex}{MagnificatAntiphon2-OQuamMetuendusEst}
\newcommand{\magantinitial}{O}
\newcommand{\maganttranslation}{How dreadful is this place. Surely this is none other but the house of God, and the gate of heaven.}
\def\magsolemn{T}
\def\magoneline{T}
\definemag{6}{F}

  \def\premagverses{\oldneedspace{15\baselineskip}}
}
\def\definevesperspropersalt{\newcommand{\vrtex}{vrHaecEstDomusDomini}
\newcommand{\vtranslation}{This is the house of the Lord, strongly built.}
\newcommand{\rtranslation}{It is well founded upon strong rock.}

\newcommand{\maganttex}{an--sanctificavit_dominus--solesmes}
\newcommand{\magantinitial}{S}
\newcommand{\maganttranslation}{The Lord has hallowed His dwelling; for this is the house of God; there they call on His Name, of which it is written: And My Name shall be there, saith the Lord.}
\def\magsolemn{T}
\definemag{1}{g}

}
\def\vesperspropersnote{At II Vespers:}
\def\vesperspropersaltnote{At I Vespers:}

\printvespers[../CommonOfDedicationOfChurch]{inc-DedicationOfChurch}
\bigskip
\benedicamusdomino[2]{}
}
}
}

	{
	\label{appendix}
  \cleardoublepage  
  \chapter{Appendix}
  \section{Common of the Dedication of a Church}
  \label{commondedicationofchurch}
  {
    \let\printhymn=\printhymnfeastref
    \def\printshortresp{%
      \label{shortresp-dedication}%
      {\def\gabcfolder{.}
      \printgabc{Short}{Resp.}{D}{rb--domum_tuam--solesmes_1961}
      \translation[]{\Vbar{}~Holiness becometh thine house, O Lord.
        \Rbar{}~Holiness\dots{}
        \Vbar{}~For ever.
        \Rbar{}~Holiness becometh.
        \Vbar{}~Glory be\dots{}
        \Rbar{}~Holiness\dots{}
      }
      \medskip
      {\centering In Paschal Time.\par}
      \printgabc{Short}{Resp.}{D}{rb--domum_tuam_(paschal_time)--solesmes_1961}
      }

      \bigskip
      \gresetinitiallines{0}\label{vr-dedication}
      \gregorioscore{vr-locus-iste}
      \let\myhwidth\relax
      \let\myhhwidth\relax
      \let\myhwidthb\relax
      \let\myhhwidthb\relax
      \newlength{\myhwidth}
      \settowidth{\myhwidth}{pópul}
      \newlength{\myhhwidth}
      \settowidth{\myhhwidth}{do}
      \addtolength{\myhhwidth}{-\myhwidth}
      \newlength{\myhwidthb}
      \settowidth{\myhwidthb}{\emph{T. P.} Allelui}
      \newlength{\myhhwidthb}
      \settowidth{\myhhwidthb}{i}
      \addtolength{\myhhwidthb}{-\myhwidthb}
      \def\myhspace{\hspace{3.5ex}}
      \begin{nstabbing}
      %\>\Rbar{}~Et \myhspace{} omnes \myhspace{} reges \myhspace{} terræ \myhspace{} glóriam \>\hspace{\myhhwidth}tuam.
      \>\Rbar{}~Pro\myhspace{} delíctis\myhspace{} et\myhspace{} peccátis \>\hspace{\myhhwidth}pópuli. \>\hspace{\myhhwidthb}\emph{T. P.} Alleluia.
      \end{nstabbing}

      \translation[]{\Vbar{}~This place is holy, wherein the Priest prayeth.\\
      \Rbar{}~For the pardon of the transgressions and offences of the people.}

      \bigskip
    }

    \printterce[../CommonOfDedicationOfChurch]{inc-DedicationOfChurch}{dedication}
  }

}


}
\end{document}

