% !TEX TS-program = lualatex
% !TEX encoding = UTF-8

% This is a simple template for a LuaLaTeX document using gregorio scores.

% easter can be from march 22 to april 25

\usepackage{../definepsalms}
\usepackage{titlesec}
\usepackage{titletoc}
\usepackage{titleps}
\usepackage{letltxmacro}
\usepackage{changepage} % gives us \ifoddpage use [strict]
\usepackage[super]{nth}
\usepackage[savepos]{zref}
\usepackage{xparse}
\usepackage{setspace}
\usepackage{amsfonts}
\usepackage{refcount}
\usepackage{metalogo}
\usepackage{enumitem}
\usepackage{zref-titleref}
\makeatletter
\newcommand*{\currentname}{\zref@getcurrent{title}}
% or \newcommand*{\currentname}{\zref@titleref@current}
\makeatother
%\nofiles
%\includeonly{inc-grassi}
\LetLtxMacro{\oldnth}{\nth}
\renewcommand{\nth}[1]{{\addfontfeature{Numbers=Lining}\oldnth{#1}}}
\LetLtxMacro{\oldneedspace}{\needspace}
\renewcommand{\needspace}[1]{
	\checkoddpage\ifoddpage\oldneedspace{#1}\else\fi
}
\let\gredagger=\dag
\newcommand*\cleartoleftpage{%
  \clearpage
  \ifodd\value{page}\hbox{}\newpage\fi
}
\hyphenation{GregoBase}

%\usepackage{hyperref}
\ifx\phantomsection\undefined%
	\newcommand{\phantomsection}{}
\fi

\setcounter{secnumdepth}{-1}

\ifthenelse{\boolean{lettersize}}{
	\def\mywidth{8.5in}
	\def\myheight{11in}
}{
	\def\mywidth{6in}
	\def\myheight{9in}
}

\usepackage{../definepsalms}
\usepackage{adjustbox}
\sloppy
\usepackage{../definepsalms}
\usepackage{adjustbox}
\sloppy
\usepackage{../definepsalms}
\usepackage{adjustbox}
\sloppy
\input{../inc_header} %


\let\oldVbar=\Vbar
\def\Vbar{\oldVbar\hspace{-2pt}}
\let\oldRbar=\Rbar
\def\Rbar{\oldRbar\hspace{-2pt}}

\ifbook{
\lfoot[\thepage]{}
\rfoot[]{\thepage}
}
\ifnotbook{
\cfoot{%
  \ifnum\thepage>1
    \thepage
  \fi
}
}
%\renewcommand\headrulewidth{\oldheadrulewidth}
\chead{
  \ifnum\thepage>1
    \addfontfeature{Numbers=Lining}%
    \heading{} \ifx\matinsnocturn\undefined\else(\matinsnocturn)\fi
  \fi
}

\newcommand{\writeheading}[1]{
  \begin{center}{
  \addfontfeature{Numbers=Lining}
  \textsc{#1}
  }\end{center}
  \medskip
}
\newcommand{\printseparation}{
  %\hfil\rule{3in}{0.5pt}\hfil
  \bigskip
  \bigskip
}

\def\nogloriapatri{T}
 %


\let\oldVbar=\Vbar
\def\Vbar{\oldVbar\hspace{-2pt}}
\let\oldRbar=\Rbar
\def\Rbar{\oldRbar\hspace{-2pt}}

\ifbook{
\lfoot[\thepage]{}
\rfoot[]{\thepage}
}
\ifnotbook{
\cfoot{%
  \ifnum\thepage>1
    \thepage
  \fi
}
}
%\renewcommand\headrulewidth{\oldheadrulewidth}
\chead{
  \ifnum\thepage>1
    \addfontfeature{Numbers=Lining}%
    \heading{} \ifx\matinsnocturn\undefined\else(\matinsnocturn)\fi
  \fi
}

\newcommand{\writeheading}[1]{
  \begin{center}{
  \addfontfeature{Numbers=Lining}
  \textsc{#1}
  }\end{center}
  \medskip
}
\newcommand{\printseparation}{
  %\hfil\rule{3in}{0.5pt}\hfil
  \bigskip
  \bigskip
}

\def\nogloriapatri{T}
 %


\let\oldVbar=\Vbar
\def\Vbar{\oldVbar\hspace{-2pt}}
\let\oldRbar=\Rbar
\def\Rbar{\oldRbar\hspace{-2pt}}

\ifbook{
\lfoot[\thepage]{}
\rfoot[]{\thepage}
}
\ifnotbook{
\cfoot{%
  \ifnum\thepage>1
    \thepage
  \fi
}
}
%\renewcommand\headrulewidth{\oldheadrulewidth}
\chead{
  \ifnum\thepage>1
    \addfontfeature{Numbers=Lining}%
    \heading{} \ifx\matinsnocturn\undefined\else(\matinsnocturn)\fi
  \fi
}

\newcommand{\writeheading}[1]{
  \begin{center}{
  \addfontfeature{Numbers=Lining}
  \textsc{#1}
  }\end{center}
  \medskip
}
\newcommand{\printseparation}{
  %\hfil\rule{3in}{0.5pt}\hfil
  \bigskip
  \bigskip
}

\def\nogloriapatri{T}

\gresetbarspacing{new}
%\gresetlastline{justified}
\setlength\headheight{0.25in+15pt}
\setlength\headsep{1pc}
\setlength\topskip{0pc}
\setlength\footskip{1pc}
\geometry{outer=0.4in,inner=0.85in,top=0pc+\headheight+\headsep,bottom=0.4in,twoside=true}
\newpagestyle{main}{
%\setheadrule{0pt}
\sethead[\garamond{\thepage}][\garamond{\chaptertitle}][] % even
{}{\garamond{\sectiontitle}}{\garamond{\thepage}} % odd
\setfoot[][][] % even
{}{}{} % odd
}
\pagestyle{main}

%\grechangeglyph{Porrectus*}{*}{.alt}
%\grechangeglyph{TorculusResupinus*}{*}{.alt}

\titleformat
{\section} % command
[block] % shape
{\phantomsection\large\addfontfeature{Numbers=Lining}} % format
{} % label
{} % sep
{
    % \rule{\textwidth}{1pt}
    % \vspace{1ex}
    \centering
} % before-code
%[
% \vspace{-0.5ex}%
% \rule{\textwidth}{0.3pt}
%] % after-code
 
 
\titleformat{\chapter}[block]
{\thispagestyle{empty}\phantomsection\Large\scshape\addfontfeature{Numbers=Lining}}
{}{0.5em}{\centering}
 
\titlespacing{\chapter}{0pt}{6pt-\headheight}{1pc}
\titlespacing{\section}{0pt}{*2.5}{*1}
\titleclass{\chapter}{top}
\newcommand{\chapterbreak}{\clearpage}
%\titleclass{\section}{top}

\contentsmargin{2pc}
%\dottedcontents{chapter}[2.3em]{}{2.3em}{1pc}
\titlecontents{chapter}[2.3em]{}{\contentslabel{2.3em}}{\hspace*{-2.3em}}{}
\dottedcontents{section}[5.5em]{}{3.2em}{1pc}

\newcommand{\printnote}[1]{
	{\normalsize \emph{#1}}%
}
\newcommand{\subtitle}[1]{
{
	\centering
	{\addfontfeature{Numbers=Lining} \normalsize \emph{#1}}\par
}
}

\newcommand{\deusinadjutorium}{\noindent\printnote{\Vbar~\emph{Deus in adjutórium}, p.~\pageref{deusinadjutorium}}}
%\newcommand{\deusinadjutoriumsolemn}{\noindent\printnote{\Vbar~\emph{Deus in adjutórium}, p.~\pageref{deusinadjutoriumsolemn}.}}
\newcommand{\printcollect}[2]{
	\ifx\undefined\begincollectcols\def\begincollectcols{\begin{parcolumns}[rulebetween]{2}}\fi
	\ifx\printcollectheading\undefined\def\printcollectheading{T}\fi
	\if\printcollectheading T
	\oldneedspace{3\baselineskip}
	%\smallskip
	{\centering\large Collect.\par}
	%\smallskip
	\fi
	\begincollectcols
	\sloppy
	\prayer{#1}{#2}
	\end{parcolumns}
	\let\begincollectcols=\undefined
}
\newcommand{\benedicamusdominoreference}[1]{%
	\ifthenelse{\boolean{includebenedicamusdominoreferences}}{
		\Vbar~\emph{Benedicámus Dómino \IfInteger{#1}{#1}{\csname benedicamusdominoname#1\endcsname}}, page \pageref{benedicamusdomino-#1}.
	}{}
}
\newcommand{\benedicamusdominoreferencelentoreaster}{%
	\ifthenelse{\boolean{includebenedicamusdominoreferences}}{
		In Lent, \Vbar~\emph{Benedicámus Dómino \benedicamusdominonamelent}, page \pageref{benedicamusdomino-lent}, or in Easter, \emph{\benedicamusdominonameeaster}, page \pageref{benedicamusdomino-easter}.
	}{}
}
\newcommand{\benedicamusdominomaster}[1]{
	\ifthenelse{\boolean{includebenedicamusdominoreferences}}{
		\noindent\printnote{\benedicamusdominoreference{#1}}
		\ifx\postbenedicamusdomino\undefined\else\postbenedicamusdomino\fi
	}{}
	\bigskip
	\hrule
}
\newcommand{\benedicamusdominolentoreaster}{
	\ifthenelse{\boolean{includebenedicamusdominoreferences}}{
		\noindent\printnote{\benedicamusdominoreferencelentoreaster}
		\ifx\postbenedicamusdomino\undefined\else\postbenedicamusdomino\fi
	}{}
	\bigskip
	\hrule
}

\ifthenelse{\boolean{lettersize}}{
	\newcommand{\psalmcolsoverride}[1][0]{
	}	
}{
	\newcommand{\psalmcolsoverride}[1][0]{
		\def\beginpsalmcols{\begin{parcolumns}[rulebetween,colwidths={1=0.45\linewidth}]{2}}
		\ifnum#1=110
		\def\beginpsalmcols{\begin{parcolumns}[rulebetween,colwidths={1=0.465\linewidth}]{2}}
		\fi
		\ifnum#1=111
		\def\beginpsalmcols{\begin{parcolumns}[rulebetween,colwidths={1=0.475\linewidth}]{2}}
		\fi
		%\ifnum#1=112
		%\def\beginpsalmcols{\begin{parcolumns}[rulebetween,colwidths={1=0.445\linewidth}]{2}}
		%\fi
		\ifnum#1=116
		\def\beginpsalmcols{\begin{parcolumns}[rulebetween,colwidths={1=0.5655\linewidth}]{2}}
		\fi
		\ifnum#1=121
		\def\beginpsalmcols{\begin{parcolumns}[rulebetween,colwidths={1=0.47\linewidth}]{2}}
		\fi
		\ifnum#1=125
		\def\beginpsalmcols{\begin{parcolumns}[rulebetween,colwidths={1=0.49\linewidth}]{2}}
		\fi
		\ifnum#1=129
		\def\beginpsalmcols{\begin{parcolumns}[rulebetween,colwidths={1=0.475\linewidth}]{2}}
		\fi
		\ifnum#1=131
		\def\beginpsalmcols{\begin{parcolumns}[rulebetween,colwidths={1=0.4875\linewidth},distance=1em]{2}}
		\fi
		\ifnum#1=137
		\def\beginpsalmcols{\begin{parcolumns}[rulebetween]{2}}
		\fi
		\ifnum#1=147
		%\def\beginpsalmcols{\begin{parcolumns}[rulebetween,colwidths={1=0.465\linewidth}]{2}}
		\def\beginpsalmcols{\begin{parcolumns}[rulebetween,colwidths={1=0.475\linewidth}]{2}}
		\fi
	}
}
\newcommand{\printvrcommem}{
		{\normalsize
		\ifx\beginvrcols\undefined\def\beginvrcols{\begin{parcolumns}[rulebetween]{2}}\fi
		\beginvrcols
		\colchunk{%
			\selectlanguage{latin}%
	    \Vbar{}~\commvlatin{}%
    }
		\colchunk{%
			\selectlanguage{american}%
	    \Vbar{}~\commvtranslation{}%
		}%
		\colplacechunks%
    \colchunk{%
			\selectlanguage{latin}%
	    \Rbar{}~\commrlatin{}%
		}%
		\colchunk{%
			\selectlanguage{american}%
	    \Rbar{}~\commrtranslation{}%
		}
		\end{parcolumns}
		}
}
\newcommand{\printvrdirigatur}{
	\smallskip{}
	\oldneedspace{3\baselineskip}
	\noindent\printnote{If the Sunday is being commemorated:\\}\vspace{-0.5\baselineskip}
	{
		\ifthenelse{\boolean{birmingham}}{
			\def\beginvrcols{\begin{parcolumns}[rulebetween,colwidths={1=0.53\linewidth}]{2}}
		}{
			\def\beginvrcols{\begin{parcolumns}[rulebetween,colwidths={1=0.51\linewidth}]{2}}
		}
		\def\commvlatin{Dirigátur Dómine orátio \textbf{mé}a.}
		\def\commrlatin{Sicut incénsum in conspéctu \textbf{tú}o.}
		\def\commvtranslation{\noindent{}Let my prayer be directed, O Lord.}
		\def\commrtranslation{\noindent{}As incense in Thy sight.}
		\printvrcommem{}
	}
}
\newcommand{\printvrmanenobiscum}{
	\smallskip{}
	\oldneedspace{3\baselineskip}
	\noindent\printnote{If the Sunday is being commemorated:\\}\vspace{-0.5\baselineskip}
	{
		\def\commvlatin{Mane nobíscum Dómine, alle\textbf{lú}ia.}
		\def\commrlatin{Quóniam advesperáscit, alle\textbf{lú}ia.}
		\def\commvtranslation{Stay with us O Lord, alleluia.}
		\def\commrtranslation{Because it is towards evening, alleluia.}
		\printvrcommem{}
	}
}
\newcommand{\printvrdominusincaelo}{
	\smallskip{}
	\oldneedspace{3\baselineskip}
	\noindent\printnote{If the Sunday is being commemorated:\\}\vspace{-0.5\baselineskip}
	{
		\def\commvlatin{Dóminus in cælo, alle\textbf{lú}ia.}
		\def\commrlatin{Parávit sedem suam, alle\textbf{lú}ia.}
		\def\commvtranslation{The Lord in heaven, alleluia.}
		\def\commrtranslation{Has prepared His throne, alleluia.}
		\printvrcommem{}
	}
}
\newcommand{\printcommemoration}[2][.]{
	\def\gabcfolder{#1}
	\input{\gabcfolder/#2}
	\subtitle{\comheadingtext}
	\sectionmark{\comheadingtext}
	{
	\def\noeuouae{F}
	%\printgabc{At Magn.}{\oldstylenums{\commagantlinetwo}}{\commagantinitial}{\commaganttex}
		{%
			\gresetinitiallines{0}
			\ifx\gabcfolder\undefined
		        \gregorioscore{\commaganttex}
		    \else
		        \gregorioscore{\gabcfolder/\commaganttex}
		    \fi

			\greseteolcustos{auto}%
		}
	}
	\translation[]{\englishcommagantiphon}

	\smallskip
	\ifx\commvlatin\undefined
		\printvr[\greseteolcustos{manual}]{\commvrtex}{\commvtranslation}{\commrtranslation}
			\greseteolcustos{auto}
	\else
	{
		\oldneedspace{3\baselineskip}
		\printvrcommem{}
	}
	\fi

	\printcollect{\latincomcollect}{\englishcomcollect}
}
\DeclareDocumentCommand{\printbothversions}{ O{\undefined} O{\undefined} m m }{
% #1 & #2 are the label
% #3 is what to check for
% #4 is the body of what to print
	\ifx#3\undefined
		\ifx\definevesperspropersalt\undefined\else
	  {
			\ifx\vesperspropersaltnote\undefined\else
		    \oldneedspace{3\baselineskip}
				\printnote{\vesperspropersaltnote}
			\fi
			\definevesperspropersalt
			\ifx#2\undefined\else\oldneedspace{5\baselineskip}\label{#2}\fi
	  	#4
		}
		\medskip
		\fi
		\ifx\definevesperspropers\undefined\else
	  {
			\ifx\vesperspropersnote\undefined\else
	    	\oldneedspace{3\baselineskip}
				\printnote{\vesperspropersnote}
			\fi
			\definevesperspropers
			\ifx#1\undefined\else\oldneedspace{5\baselineskip}\label{#1}\fi
	  	#4
		}
		\fi
	\else
  {
		\ifx#1\undefined\else\label{#1}\fi
  	#4
	}
	\fi
}
\newcommand{\printpsalms}{
	\printpsalm{1}{\psalmonenum}{\psalmonetone\psalmoneend}{\antonetex}{\antoneinitial}
	\medskip
	\needspace{3\baselineskip}
	\printpsalm{2}{\psalmtwonum}{\psalmtwotone\psalmtwoend}{\anttwotex}{\anttwoinitial}
	\medskip
	\needspace{3\baselineskip}
	\printpsalm{3}{\psalmthreenum}{\psalmthreetone\psalmthreeend}{\antthreetex}{\antthreeinitial}
	\medskip
	\needspace{3\baselineskip}
	\printpsalm{4}{\psalmfournum}{\psalmfourtone\psalmfourend}{\antfourtex}{\antfourinitial}
	\medskip
	\ifx\psalmfivenum\undefined{
		\ifx\antfivetex\undefined{%then it is a different antiphon for each
			\ifx\definevesperspropersalt\undefined\else{
				\definevesperspropersalt
				\ifx\vesperspropersaltnote\undefined\else
					\oldneedspace{3\baselineskip}
					\noindent\printnote{\vesperspropersaltnote}
				\fi
				\printpsalm{5}{\psalmfivenum}{\psalmfivetone\psalmfiveend}{\antfivetex}{\antfiveinitial}
			}\fi
			\ifx\definevesperspropers\undefined\else{
				\definevesperspropers
				\ifx\vesperspropersnote\undefined\else
					\oldneedspace{3\baselineskip}
					\noindent\printnote{\vesperspropersnote}
				\fi
				\printpsalm{5}{\psalmfivenum}{\psalmfivetone\psalmfiveend}{\antfivetex}{\antfiveinitial}
			}\fi
		}\else{%they share the same antiphon
			\ifx\definevesperspropersalt\undefined\else{
				\definevesperspropersalt
				\ifx\vesperspropersaltnote\undefined\else
					\let\oldprepsalmtitle=\prepsalmtitlefive
					\def\prepsalmtitlefive{
						\smallskip
						\oldneedspace{3\baselineskip}
						\noindent\printnote{\vesperspropersaltnote}

						\ifx\oldprepsalmtitle\undefined\else\oldprepsalmtitle\fi
					}
				\fi
				\printpsalm{5}{\psalmfivenum}{\psalmfivetone\psalmfiveend}{\antfivetex}{\antfiveinitial}
			}\fi
			\def\onlyoneant{T}
			\ifx\definevesperspropers\undefined\else{
				\definevesperspropers
				\ifx\vesperspropersnote\undefined\else
					\oldneedspace{3\baselineskip}
					\noindent\printnote{\vesperspropersnote}
				\fi
				\let\oldprepsalmtitle=\prepsalmtitlefive
				\def\prepsalmtitlefive{\def\onlyoneant{F}\ifx\oldprepsalmtitle\undefined\else\oldprepsalmtitle\fi}
				\def\printrepeatantiphon{T}
				\printpsalm{5}{\psalmfivenum}{\psalmfivetone\psalmfiveend}{\antfivetex}{\antfiveinitial}
			}\fi
		}\fi
	}\else{
		\printpsalm{5}{\psalmfivenum}{\psalmfivetone\psalmfiveend}{\antfivetex}{\antfiveinitial}
	}\fi
}
\let\printhymn=\undefined
\newcommand{\printterce}[3][.]{
	{
		\grechangestaffsize{15}
		\ifx\gabcfolder\undefined
			\def\gabcfolder{#1}
			\input{\gabcfolder/#2}
		\fi

		\deusinadjutorium{}%
		\ifx\printhymn\undefined%
	      , \emph{Hymn}, page \pageref{hymn-sunday}

    	  \medskip
		\else%
		  \printhymn
		\fi

		\ifx\anttwotex\undefined\else%
			\setlabel{#3}{ant}
			\printgabc{Ant.}{\anttwolinetwo}{\anttwoinitial}{\anttwotex .noEuouae}

			\ifx\anttwotranslation\undefined\else%
				\translation[]{\anttwotranslation}
			\fi

			\printtercepsalms{\anttwotoneendexpanded}
		\fi
		\ifx\chaptertext\undefined\else
			\ifx\prechapter\undefined\else%
				\prechapter
			\fi
			\setlabel{#3}{chapter}
			\printchapter{\chaptertext}{\chaptertranslation}

			\medskip
		\fi

		\ifx\printshortresp\undefined%
			%\noindent\emph{Short Resp. \emph{Inclina cor meum}}, p. \pageref{shortresp-sunday}, \Vbar{}~Ego dixi, p. \pageref{vr-sunday}.
		\else
			\printshortresp

			\medskip
		\fi

		\ifx\collect\undefined%
			\let\collect=\latincollect
			\let\collecttranslation=\englishcollect
		\fi
		\setlabel{#3}{collect}
		\printcollect{\collect}{\collecttranslation}

		\medskip

		\ifx\printbenedicamusdomino\undefined\else\printbenedicamusdomino\fi
	}
}

\sloppy
%\nofiles
\begin{document}
\normalsize
\grechangestaffsize{15}
{
	\thispagestyle{empty}
	% title page
	% general
	\vspace*{5\baselineskip}

	{\centering

	{\Huge
	Terce with Gregorian Chant

	}

	{\Large\medskip
	\emph{for}

	\medskip}

	{\Huge
	Sundays \& Holy Days

	}
	\vfill}
	\pagebreak
	% copyright page
	\thispagestyle{empty}
	\noindent{}Terce with Gregorian Chant for Sundays \& Holy Days: \emph{newly typeset, based on \emph{The Liber Usualis}, edited by the Benedictines of Solesmes (Desclee Company, 1961).}

	\bigskip{}\noindent{}Many thanks to Elie Roux for creating Gregorio, to all the people answering questions on the Gregorio forums, to Olivier Berton for creating GregoBase (http://gregobase.selapa.net) which makes it much easier to find all the chant that Andrew Hinkley and others have contributed to the public domain.  Also many thanks to my brother Benjamin, whose website (http://bbloomf.github.io/jgabc) makes it incredibly easy to typeset the psalms and substantially easier than otherwise to typeset Gregorian chant with Gregorio.

	\begin{flushright}
	\emph{Albert Bloomfield}\\
	Cincinnati, Ohio
	\end{flushright}

	\bigskip{}\noindent{}http://asbloomf.github.io/gabc-chants

	\vfill
%editions
	\bigskip{}\noindent{}%
	First edition, 9 December 2020.

	\hangindent=1em % indent all subsequent lines
    \bigskip

\makeatletter
	\noindent{}Typeset using \LuaLaTeX{} and Gregorio version \gre@gregoriotexversion{}
\makeatother

	\bigskip{}\noindent{}This work is free of known copyright restrictions.

	\bigskip{}\noindent{}%CreateSpace
	ISBN: \isbn{}
	%
  %\noindent{}Lulu ISBN: 978-1-329-59990-1
}
%\frontmatter
%\pagenumbering{roman}

\tableofcontents%

%\pagenumbering{arabic}
%\mainmatter

% redefine the  label command
\newcommand{\setlabel}[2]{
% #1 label part 1 (like easter2)
% #2 label part 2 (any of these 3: ant/chapter/collect)
	\label{#1-#2}% now execute the original label command
}

%\include{inc-common-terce}
\ifthenelse{\boolean{testrun}}{

}{

	\clearpage

	\newcommand{\printpsalmwithtranslation}[3]{
	% #1 psalm number
	% #2 ending
	% #3 0 to start on 1st verse, 1 for second
	\setlength{\columnsep}{18pt}
	\setlength{\columnseprule}{.4pt}
	\beginpsalmcols
	\colchunk{\vspace{-12pt}%
	\begin{psalmverses}[#3]
	\vspace{-\baselineskip}%
	\input{psalms/Psalm#1-#2-verses}
	\end{psalmverses}
	}

	\selectlanguage{american}
	\colchunk{\vspace{-12pt}%
	\sloppy
	\begin{psalmverses}
	\vspace{-\baselineskip}%
	\input{psalms/Psalm#1-verses-en}
	\end{psalmverses}
	}
	\selectlanguage{latin}
	\endpsalmcols
}

\renewcommand{\printpsalm}[3]{
  % #1 psalm number
  % #2 ending
  % #3 0 to start on 1st verse, 1 for second
  \setlength{\columnsep}{18pt}
  \setlength{\columnseprule}{.4pt}
  \setlength\multicolsep{0pt}%
  \begin{multicols*}{2}
  \begin{psalmverses}[#3]
  \input{psalms/Psalm#1-#2-verses}
  \end{psalmverses}
  \end{multicols*}
}



\newcommand{\printpsalmtranslation}[3]{
  % #1 psalm number
  % #2 ending
  % #3 0 to start on 1st verse, 1 for second
  \setlength{\columnsep}{18pt}
  \setlength{\columnseprule}{.4pt}
  \setlength\multicolsep{0pt}%
  \begin{multicols*}{2}
  \begin{psalmverses}[#3]
  \input{psalms/Psalm#1-verses-en}
  \end{psalmverses}
  \end{multicols*}
}

\def\beginchaptercols{\begin{parcolumns}[rulebetween]{2}}
\newcommand{\printchapternew}[4]{
	% #1 ref
	% #2 latin first words
	% #3 latin remaining
	% #4 english text
	{\noindent\hspace{3em}Chapter.\hfill\emph{#1}\hspace{3em}}
    \beginchaptercols{}
    \colchunk{\sloppy\dropcap{latin}{#2} #3\hfill{}\Rbar{}~Deo~grátias.}
    \colchunk{\sloppy \dropcap{american}{#4}}
    \end{parcolumns}
}
\newcommand{\printvrwithtranslation}{
    {\normalsize
    \ifx\beginvrcols\undefined\def\beginvrcols{\begin{parcolumns}[rulebetween]{2}}\fi
    \beginvrcols
    \colchunk{
      \par\vspace{-\baselineskip}\noindent\selectlanguage{latin}%
      \Vbar{}~\vlatin{}
    }
    \colchunk{%
      \par\vspace{-\baselineskip}\noindent\selectlanguage{american}%
      \Vbar{}~\vtranslation{}
    }%
    \colplacechunks%
    \ifx\rlatin\undefined\else
    \colchunk{
      \par\vspace{-\baselineskip}\noindent\selectlanguage{latin}%
      \Rbar{}~\rlatin{}%
    }
    \colchunk{%
      \par\vspace{-\baselineskip}\noindent\selectlanguage{american}%
      \Rbar{}~\rtranslation{}%
    }%
    \colplacechunks%
    \fi
    \end{parcolumns}
    }
}
{
  \label{sunday}
  \chapter{Sunday at Terce}
  \phantomsection
  \sectionmark{Sunday at Terce throughout the year}
  \addcontentsline{toc}{section}{Sunday at Terce throughout the year}

  {\centering Oratio ante officium.\par}
  {\centering \emph{Aperi. Pater. Ave.}\par}
  \begin{columns}
  \colchunk{\selectlanguage{latin}Aperi, Domine, os meum ad benedicendum nomen
sanctum tuum: munda quoque cor meum ab omnibus
vanis, perversis et alienis cogitationibus; intellectum
illumina, affectum inflamma, ut digne, attente ac
devote hoc Officium recitare valeam, et exaudiri
merear ante conspectum divinae Majestatis tuae. Per
Christum Dóminum nostrum. \Rbar{}~Amen.}
  \colchunk{\selectlanguage{american}Open, O Lord, my mouth to bless Thy holy name;
cleanse my heart from all vain, evil, and
wandering thoughts; enlighten my understanding
and kindle my affections; that I may worthily,
attentively, and devoutly say this Office, and so
deserve to be heard before the presence of Thy
divine Majesty. Through Christ our Lord. \Rbar{}~Amen.}
  \colplacechunks{}
  \colchunk{\selectlanguage{latin}Domine, in unione illius divinae intentionis, qua ipse
in terries laudes Deo persolvisti, hanc tibi
Horam persolvo.}
  \colchunk{\selectlanguage{american}O Lord, in union with that divine intention, with
which Thou didst praise God upon earth, I render
this Hour to Thee.}
  \colplacechunks{}
  \end{columns}
  \medskip

  %\oldneedspace{6\baselineskip}
  \emph{The Ferial Tone is used except before Pontifical Mass.}

  \printgabc{\Vbar}{}{D}{deus-in-adjutorium}\label{deusinadjutorium}
  \medskip

  %\oldneedspace{6\baselineskip}
  \emph{The Festal Tone is used only before Pontifical Mass.}

  {\def\gabcfolder{..}
    \printgabc{\Vbar}{}{D}{DeusInAdjutorium_laustibi}\label{deusinadjutorium-festal}
  }

  \medskip\vfill
  {\centering Hymn.\par}%
  %\vspace{-0.5\baselineskip}
  \def\dotting{\leaders\hbox to 1em{\hfil.\hfil}\hfill}
  %\def\dotting{\hfill}
  \setlength\parsep{0pt}%
  \setlength\topsep{0pt}%
  \setlength\partopsep{0pt}%
  \setlength\multicolsep{0pt}%
  \begin{multicols}{2}%
  \noindent{}\emph{In Advent},\dotting p.~\pageref{hymn-advent}\\
  \emph{Christmas},\dotting p.~\pageref{hymn-christmas}\\
  \emph{Epiphany},\dotting p.~\pageref{hymn-epiphany}\\
  \emph{Holy Family},\dotting p.~\pageref{hymn-holyfamily}\\
  \emph{In Lent},\dotting p.~\pageref{hymn-lent}\\
  \emph{In Passiontide},\dotting p.~\pageref{hymn-passiontide}\\
  \emph{In Paschaltide},\dotting p.~\pageref{hymn-paschaltide}\\
  \emph{Ascension},\dotting p.~\pageref{hymn-ascension}\\
  \emph{Pentecost},\dotting p.~\pageref{hymn-pentecost}\\
  \emph{Christ the King},\dotting p.~\pageref{hymn-oct-last-sunday}%\\
  %\emph{Corpus Christi},\dotting p.~\pageref{hymn-corpus-christi}\\
  %\emph{Sacred Heart},\dotting p.~\pageref{hymn-sacred-heart}
  \end{multicols}%

  \bigskip
  %\oldneedspace{4\baselineskip}
  {\centering 1.~On Ordinary Sundays.\par}\label{hymn-sunday}

  \printgabc{Hymn.}{2.}{N}{hy--nunc_sancte_nobis_(in_dominicis_per_annum)--solesmes_1961}
  \bigskip

  \oldneedspace{4\baselineskip}
  {\centering 2.~On Solemn Feasts.\par}\label{hymn-feast}

  \printgabc{Hymn.}{8.}{N}{hy--nunc_sancte_nobis_(in_festis)--solesmes_1961}
  \bigskip

  \oldneedspace{4\baselineskip}\vfill
  {\centering 3.~On Feasts of the Blessed Virgin Mary.\par}\label{hymn-bvm}

  \printgabc{Hymn.}{2.}{N}{hy--nunc_sancte_nobis_(in_festis_bmv)--solesmes_1961}

  \begin{multicols}{2}
  \begin{psalmverses}
\item Come Holy Ghost, who ever One
Art with the Father and the Son,
It is the hour, our souls possess
With thy full flood of holiness.

\item Let flesh and heart and lips and mind
Sound forth our witness to mankind;
And love light up our mortal frame,
Till others catch the living flame.

\item Almighty Father, hear our cry,
Through Jesus Christ, our Lord most High,
Who, with the Holy Ghost and thee,
Doth live and reign eternally.
Amen.
  \end{psalmverses}
  \end{multicols}

\bigskip

% print antiphon
%    \printgabc[\preant]{\antlineone}{\antlinetwo}{\antinitial}{\anttex}
  \printgabc{Ant.}{2. D}{A}{an--alleluia._(sund._at_terce)--solesmes}
% print translation
  \translation[]{Alleluia, Lead me into the path of thy commandments, alleluia, alleluia.}

  { \def\printrepeatantiphon{\emph{Repeat antiphon, \emph{p.~\pageref{ant2-sunday}.}}}
    \printtercepsalms{2D}%
  }

  \printtercepsalmstranslation{}

% \printpsalm{118.3}{2D}{1}
% \printpsalmtitle{118. IV.}
% \printpsalm{118.4}{2D}{0}
% \printpsalmtitle{118. V.}
% \printpsalm{118.5}{2D}{0}

% print antiphon
%    \printgabc[\preant]{\antlineone}{\antlinetwo}{\antinitial}{\anttex}
  \bigskip\oldneedspace{8\baselineskip}\label{ant2-sunday}
  \printgabc{Ant.}{2. D}{A}{an--alleluia._(sund._at_terce)--solesmes}
% print translation
  %\translation[]{Alleluia, Lead me into the path of thy commandments, alleluia, alleluia.}

\medskip\oldneedspace{6\baselineskip}
\printchapternew{1. John 4.}{Deus cáritas}{est~:~\dag{} et qui manet in caritáte, in Deo manet,~* et Deus in eo.}{God is charity: and he that abideth in charity, abideth in God, and God in him.}

\medskip\label{shortresp-sunday}
\printgabc{Short}{Resp.}{I}{re--inclina_cor_meum--solesmes}

\translation[]{\Vbar{}~Incline my heart into thy testimonies.
\Rbar{}~Incline\dots{}
\Vbar{}~Turn away my eyes that they may not behold vanity: quicken me in thy way.
\Rbar{}~Into thy testimonies.
\Vbar{}~Glory be\dots{}
\Vbar{}~Incline\dots{}}

\bigskip\label{vr-sunday}
\gresetinitiallines{0}
\gregorioscore{vr-ego-dixi}
\newlength{\myhwidth}
\settowidth{\myhwidth}{tib}
\begin{nstabbing}
\>\Rbar{}~Sana ánimam meam, quia peccávi \>\hspace{-\myhwidth}tibi.
\end{nstabbing}

\translation[]{\Vbar{}~I said: O Lord, be thou merciful to me.\\
\Rbar{}~Heal my soul, for I have sinned against thee.}

\bigskip
%\def\beginvrcols{\begin{parcolumns}[rulebetween,colwidths={1=0.48\linewidth}]{2}}
\def\vlatin{Dóminus vobíscum.}
\def\rlatin{Et cum spíritu tuo.}
\def\vtranslation{The Lord be with you.}
\def\rtranslation{And with thy spirit.}
{\centering Collect.\par}
\printvrwithtranslation{}

\bigskip
    %\def\dotting{\leaders\hbox to 1em{\hfil.\hfil}\hfill}
    \def\dotting{\hfill}
    \begin{multicols}{2}%
    %\printnote{Advent,\dotting \emph{p.~\pageref{advent1-collect} to \pageref{advent4-collect}}}

    %\printnote{Christmas,\dotting \emph{p.~\pageref{christmas-eve-collect} to \pageref{holyfamily-collect}}}

    \noindent\emph{Time after Epiphany},\dotting p.~\pageref{epiphany2-collect} to \pageref{epiphany6-collect}\\
    \emph{Septuagesima},\dotting p.~\pageref{septuagesima-collect} to \pageref{quinquagesima-collect}\\
    %\emph{Easter to Pentecost},\dotting p.~\pageref{easter1-collect} to \pageref{pentecost-collect}\\
    \emph{Time after Pentecost},\dotting p.~\pageref{pentecost2-collect} to \pageref{pentecost24-collect}
    \end{multicols}

\bigskip
\printvrwithtranslation{}
\bigskip

\printnote{Except before Pontifical Mass, the Officiant chants the \emph{\Vbar{}~Benedicámus Dómino} to the following tone.}
\medskip
\gresetinitiallines{0}\label{benedicamus-domino-sunday}
\gregorioscore{vr-benedicamus-domino}
\bigskip

%\def\beginvrcols{\begin{parcolumns}[rulebetween,colwidths={1=0.44\linewidth}]{2}}
\def\vlatin{Fidélium ánimæ per misericórdiam Dei requiéscant in pace.\hfill{}Amen.}
\let\rlatin=\undefined
\def\vtranslation{May the souls of the faithful departed through the mercy of God rest in peace.\hfill{}Amen.}
\let\rtranslation=\undefined
\printvrwithtranslation{}

\bigskip
\printnote{If Pontifical Mass is to follow, the \emph{\Vbar{}~Benedicámus Dómino} is sung by the Cantors in one of the following tones; and the \emph{\Vbar{}~Fidelium} is not said.}
\medskip
{
\newcommand{\printbenedicamusdomino}[2]{
	\greseteolcustos{manual}
	\gresetinitiallines{1}
	\def\annot{\small{#1}}
	\alsetinitialspacing{B}
	\gregorioscore{#2}
  \greseteolcustos{auto}
}

\vfil
\grechangestaffsize{15}
\oldneedspace{4\baselineskip}\label{benedicamusdomino-1}
{{\centering \bfseries 1.~On feasts of the I class.\\}
\smallskip
\def\breakbeforeresp{T}
\printbenedicamusdomino{5.}{../BenedicamusDomino/or--benedicamus_(in_festis_i_classis_ad_laudes)--solesmes}
\vfil
}
\oldneedspace{4\baselineskip}\label{benedicamusdomino-2}
{{\centering \bfseries 2.~On feasts of the II class.\\}
\smallskip
{\def\breakbeforeresp{T}
\printbenedicamusdomino{2.}{../BenedicamusDomino/or--benedicamus_domino_laudes_--solesmes_1961}
}
}
\vfil

\oldneedspace{4\baselineskip}\label{benedicamusdomino-mary}
\gdef\benedicamusdominonamemary{3}
{{\centering \bfseries 3.~On feasts of the Blessed Virgin.\\}
\smallskip
%\def\breakbeforeresp{T}
\printbenedicamusdomino{1.}{../BenedicamusDomino/BenedicamusDomino_blessedVirgin}}
\vfil

\needspace{6\baselineskip}
\label{benedicamusdomino-sunday}
\gdef\benedicamusdominonamesunday{4}
{{\centering \bfseries 4.~On Sundays during the Year\\and Septuagesima, Sexagesima, and Quinquagesima.\\}
\smallskip
%\def\breakbeforeresp{T}
\printbenedicamusdomino{1.}{../BenedicamusDomino/BenedicamusDomino_Sundays}}

\vfil
\label{benedicamusdomino-lent}\label{benedicamusdomino-advent}
\gdef\benedicamusdominonamelent{5}
\gdef\benedicamusdominonameadvent{\benedicamusdominonamelent}
{{\centering \bfseries 5.~On Sundays of Advent and Lent.\\}
\smallskip
%\def\breakbeforeresp{T}
\printbenedicamusdomino{6.}{../BenedicamusDomino/BenedicamusDomino_SundaysOfAdventAndLent}
\ifthenelse{\boolean{birmingham}}{
	\medskip
	\emph{\normalsize or from \emph{Mass XVII}:}

	\smallskip
	\printbenedicamusdomino{6.}{../BenedicamusDomino/ky--benedicamus_xviia--solesmes}
}{}
}
\vfil

\label{benedicamusdomino-easter}
\gdef\benedicamusdominonameeaster{6}
{{\centering \bfseries 6.~On Sundays of Paschal Time.\\}
\smallskip
%\def\breakbeforeresp{T}
\printbenedicamusdomino{7.}{../BenedicamusDomino/BenedicamusDomino_SundaysOfPaschalTime}}

}

}

	{
	\label{advent}
	\chapter{Advent}
	\section{First Sunday of Advent}

  {
	  \def\printhymn{%
		{\centering Hymn.\par}\label{hymn-advent}

	  	{\def\gabcfolder{.}
	  	\printgabc{1.}{}{N}{hy--nunc_sancte_nobis_(in_adventu)--solesmes_1961}}

		\bigskip
	  }
    \def\printshortresp{%
    \label{shortresp-advent}%
    {\def\gabcfolder{.}
    \printgabc{Short}{Resp.}{V}{re--veni_ad_liberandum--solesmes}}

    \translation[]{\Vbar{}~Come to my rescue O God, Lord of hosts.
    \Rbar{}~Come\dots{}
    \Vbar{}~Show us thy face, and we shall be saved.
    \Rbar{}~O God, Lord of hosts.
    \Vbar{}~Glory be to the Father, and to the Son, and to the Holy Ghost.
    \Rbar{}~Come\dots{}
    }

    \bigskip
    \gresetinitiallines{0}
    \gregorioscore{vr-timebunt-gentes}
    \let\myhwidth\relax
    \let\myhhwidth\relax
    \newlength{\myhwidth}
    \settowidth{\myhwidth}{tu}
    \newlength{\myhhwidth}
    \settowidth{\myhhwidth}{n}
    \addtolength{\myhhwidth}{-\myhwidth}
    \def\myhspace{\hspace{1.4ex}}
    \begin{nstabbing}
    \>\Rbar{}~Et \myhspace{} omnes \myhspace{} reges \myhspace{} terræ \myhspace{} glóriam \>\hspace{\myhhwidth}tuam.
    \end{nstabbing}

    \translation[]{\Vbar{}~And the Gentiles shall fear thy name, O Lord.\\
    \Rbar{}~And all the kings of the earth thy glory.}

    \bigskip
    }

	  \printterce[../Advent1]{inc-Advent1}
  }

  \def\printhymn{
  	\noindent\emph{Hymn}, page \pageref{hymn-advent}

  	\medskip
  }
  \def\printshortresp{
    \noindent\emph{Short Resp. \emph{Veni ad liberándum nos}, p.} \pageref{shortresp-advent}, \Vbar{}~Timébunt gentes. 
  }
  \section{Second Sunday of Advent}
  \printterce[../Advent2]{inc-Advent2-Vespers2}

  \section{Third Sunday of Advent}
  \printterce[../Advent3]{inc-Advent3}

  \section{Fourth Sunday of Advent}
  \printterce[../Advent4]{inc-Advent4}

}

	{
	\label{christmas}
	\chapter{Christmas}
	\section{Christmas Eve}
  \def\printbenedicamusdomino{
    \noindent\emph{Benedicamus Domino}, p. \pageref{benedicamus-domino-sunday}.
  }
  {
    \def\printhymn{%
    	, \emph{Hymn}, page \pageref{hymn-advent}

    	\medskip
    }
    \def\printshortresp{%
      \label{shortresp-christmas-eve}%
      {\def\gabcfolder{.}
      \printgabc{Short}{Resp.}{H}{re--hodie_scietis--solesmes_1961}}

      \translation[]{\Vbar{}~This day ye shall know that the Lord cometh.
      \Rbar{}~This\dots{}
      \Vbar{}~And in the morning, then ye shall see His glory.
      \Rbar{}~That the Lord cometh.
      \Vbar{}~Glory be to the Father, and to the Son, and to the Holy Ghost.
      \Rbar{}~This\dots{}
      }

      \bigskip
      \gresetinitiallines{0}\label{vr-christmas-eve}
      \gregorioscore{vr-constantes-estote}
      \let\myhwidth\relax
      \let\myhhwidth\relax
      \newlength{\myhwidth}
      \settowidth{\myhwidth}{v} %this is what precedes the last vowel in the response
      \newlength{\myhhwidth}
      \settowidth{\myhhwidth}{t} %this is what preceded the last vowel in the verse
      \addtolength{\myhhwidth}{-\myhwidth}
      \def\myhspace{\hspace{1.4ex}}
      \begin{nstabbing}
      \>\Rbar{}~Vidébitis auxílium Dómini super \>\hspace{\myhhwidth}vos.
      \end{nstabbing}

      \translation[]{\Vbar{}~Stand ye still.\\
      \Rbar{}~And ye shall see the salvation of the Lord with you.}

      \bigskip
    }
    \printterce{inc-christmas-eve}{christmas-eve}
  }
  {
    \newcommand{\printrefshymn}[1]{%
      \def\dotting{\hfill%\leaders\hbox to 1em{\hfil.\hfil}\hfill
        }%
      \begin{multicols}{2}%
      \noindent{}Christmas \& Sunday,\dotting \emph{below}\\
      Circumcision,\dotting \emph{p.~\pageref{circumcision-#1}}\\
      Holy Name,\dotting \emph{p.~\pageref{holyname-#1}}
      \end{multicols}%
      \smallskip
    }
    \def\printhymn{

      {\centering Hymn.\par}\label{hymn-christmas}

      {\def\gabcfolder{.}
       \printgabc{8.}{}{N}{hy--nunc_sancte_nobis_(in_nativitate_domini)--solesmes_1961}
      }

      \printrefshymn{ant}
      \bigskip
    }
    \newcommand{\printrefs}[1]{%
      \def\dotting{\hfill%\leaders\hbox to 1em{\hfil.\hfil}\hfill
        }%
      \begin{multicols}{2}%
      \noindent{}Christmas,\dotting \emph{below}\\
      Sunday within the octave,\dotting \emph{p.~\pageref{christmas-sunday-#1}}
      \end{multicols}%
      \smallskip
    }
    \newcommand{\anttwotex}{an--genuit_puerpera_regem--solesmes_1961}
    \newcommand{\anttwoinitial}{G}
    \newcommand{\anttwotranslation}{The Mother brought forth the King, Whose name is called The Eternal; the joy of a Mother was hers, remaining a Virgin unsullied; neither before nor henceforth hath there been or shall be such another, alleluia.}
    \definepsalm{2}{109}{2}{D}

    \def\printshortresp{%
      \label{shortresp-christmas}%
      {\def\gabcfolder{.}
      \printgabc{Short}{Resp.}{V}{rb--verbum_caro_factum_est--solesmes_1961}}

      \translation[]{\Vbar{}~The Word was made flesh. Alleluia, Alleluia.
      \Rbar{}~The Word\dots{}
      \Vbar{}~And dwelt among us.
      \Rbar{}~Alleluia, Alleluia.
      \Vbar{}~Glory be to the Father, and to the Son, and to the Holy Ghost.
      \Rbar{}~The Word\dots{}
      }

      \bigskip
      \gresetinitiallines{0}\label{vr-christmas}
      \gregorioscore{vr-ipse-invocabit}
      \let\myhwidth\relax
      \let\myhhwidth\relax
      \newlength{\myhwidth}
      \settowidth{\myhwidth}{allelui} %this is what precedes the last vowel in the response
      \newlength{\myhhwidth}
      \settowidth{\myhhwidth}{i} %this is what preceded the last vowel in the verse
      \addtolength{\myhhwidth}{-\myhwidth}
      \def\myhspace{\hspace{0.3ex}}
      \begin{nstabbing}
      \>\Rbar{}~Pater \myhspace{} meus \myhspace{} es \myhspace{} tu, \>\hspace{\myhhwidth}allelúia.
      \end{nstabbing}

      \translation[]{\Vbar{}~He shall cry unto Me, Alleluia.\\
      \Rbar{}~Thou art My Father, Alleluia.}

      \bigskip
    }
    \def\prechapter{\printrefs{chapter}}
    \section{Christmas}
    \printterce{../Christmas/inc-Christmas-Vespers2}{christmas}
  }

  \def\printshortresp{
    \noindent\emph{Short Resp. \emph{Verbum caro}}, p. \pageref{shortresp-christmas}, \Vbar{}~Ipse invocabit, p. \pageref{vr-christmas}.
  }
  {
    \def\printhymn{%
      , \emph{Hymn}, page \pageref{hymn-sunday}%
      , Ant. \emph{Genuit puerpera Regem}, \pageref{christmas-ant}

      \medskip
    }

    \section{Sunday within the octave}
    \printterce{../Christmas/inc-SundayWithinOctaveOfChristmas-Vespers2}{christmas-sunday}
  }
  \def\printhymn{%
    , \emph{Hymn}, page \pageref{hymn-christmas}%

    \medskip
  }
  {
    \section{Octave of the Nativity}
    \printterce[../ChristmasOctave-Circumcision]{inc-Circumcision-Vespers-common}{circumcision}
  }
  {
    % holy name
    % proper ant, chapter from vespers, short resp
    \newcommand{\antonetex}{an--omnis_qui_invocaverit--solesmes}
\newcommand{\antoneinitial}{O}
\newcommand{\antonetranslation}{All who shall call on the name of the Lord shall be saved.}
\definepsalm{1}{109}{8}{G}

\newcommand{\anttwotex}{an--sanctum_et_terribile--solesmes}
\newcommand{\anttwoinitial}{S}
\newcommand{\anttwotranslation}{Holy and terrible is His name; the fear of the Lord is the beginning of wisdom.}
\definepsalm{2}{110}{5}{a}

\newcommand{\antthreetex}{an--ego_autem_in_domino--solesmes}
\newcommand{\antthreeinitial}{E}
\newcommand{\antthreetranslation}{Yet I will rejoice in the Lord and exult in the God of my salvation.}
\definepsalm{3}{111}{3}{a2}

\newcommand{\antfourtex}{an--a_solis_ortu--solesmes}
\newcommand{\antfourinitial}{A}
\newcommand{\antfourtranslation}{From the rising of the sun to its setting, the Lord's Name is to be praised.}
\definepsalm{4}{112}{4}{E}

\newcommand{\antfivetex}{an--sacrificabo_hostiam--solesmes}
\newcommand{\antfiveinitial}{S}
\newcommand{\antfivetranslation}{I will offer the sacrifice of praise, and will call upon the Name of the Lord.}
\def\psalmclef{2} %if I didn't use definepsalm, I would define this as psalmcleffive
\definepsalm{5}{115}{8}{c}
\let\psalmclef=\undefined % but then I have to undefine it!
%\renewcommand{\psalmcleffive}{2}

\newcommand{\chaptertext}{\dropcap{latin}{Fratres~: Christus humiliávit semetípsum, factus obédiens usque ad mortem, mortem autem} \textbf{crú}\-cis.~\dag{} Propter quod et Deus exaltávit illum, et donávit illi nomen quod est super \emph{om\-ne} \textbf{nó}\-men~:~* ut in nómine Jesu omne genu flec\textbf{tá}tur.}
\newcommand{\chaptertranslation}{Christ humbled Himself, becoming obedient unto death, even the death of the cross. For which cause God also hath exalted Him, and given Him a name which is above all names: That in the name of Jesus every knee should bow.}

\newcommand{\hymnlinetwo}{1.}
\newcommand{\hymntex}{hy--jesu_dulcis_memoria--solesmes}
\newcommand{\hymninitial}{J}
\newcommand{\hymntranslation}{
\item Jesu, the very thought of Thee
With sweetness fills my breast;
But sweeter far Thy face to see,
And in Thy presence rest.

\item Nor voice can sing, nor heart can frame,
Nor can the memory find,
A sweeter sound than Thy blest Name,
O Saviour of mankind!

\item O Hope of every contrite heart,
O Joy of all the meek,
To those who fall, how kind Thou art!
How good to those who seek!

\item But what to those who find? Ah! this
Nor tongue nor pen can show:
The love of Jesus, what it is
None but His loved ones know.

\item Jesu, our only joy be Thou,
As Thou our prize wilt be;
Jesu, be Thou our glory now,
And through eternity.
Amen.%\grechangestaffsize{15}
}

\newcommand{\vrtex}{vrSitNomenDominiBenedictum}
\newcommand{\vtranslation}{Blessed be the Name of the Lord, alleluia.}
\newcommand{\rtranslation}{From this time forth, and for evermore, alleluia.}

\newcommand{\collect}{Deus, qui unigénitum Fílium tuum constituísti humáni géneris Salvatórem, et Jesum vocári jussísti~:~\dag{} concéde propítius; ut cujus sanctum nomen venerámur in terris,~* ejus quoque aspéctu perfruámur in cælis. Per eúmdem Dóminum.}
\newcommand{\collecttranslation}{O God, Who hast appointed Thine Only-begotten Son to be the Saviour of mankind, and commanded that He should be called Jesus, mercifully grant that we, who here on earth do worship His Holy Name, may be made glad in heaven by His Presence. Through the same our Lord.}

    \renewcommand{\anttwotex}{an--scitote_quia_dominus--solesmes_1961}
    \renewcommand{\anttwoinitial}{S}
    \renewcommand{\anttwotranslation}{}
    \definepsalm{2}{109}{3}{a}

    \def\printshortresp{%
      \label{shortresp-holyname}%
      {\def\gabcfolder{.}
      \printgabc{Short}{Resp.}{S}{rb--sit_nomen--solesmes}}

      \translation[]{\Vbar{}~Blessed be the Name of the Lord, Alleluia, alleluia.
        \Rbar{}~Blessed\dots{}
        \Vbar{}~From this time forth, and for evermore.
        \Rbar{}~Alleluia, alleluia.
        \Vbar{}~Glory be to the Father, and to the Son, and to the Holy Ghost.
        \Rbar{}~Blessed\dots{}
      }

      \bigskip
      \gresetinitiallines{0}\label{vr-holyname}
      \gregorioscore{vr-afferte-domino}
      \let\myhwidth\relax
      \let\myhhwidth\relax
      \newlength{\myhwidth}
      \settowidth{\myhwidth}{allelui} %this is what precedes the last vowel in the response
      \newlength{\myhhwidth}
      \settowidth{\myhhwidth}{i} %this is what preceded the last vowel in the verse
      \addtolength{\myhhwidth}{-\myhwidth}
      \def\myhspace{\hspace{0.5ex}}
      \begin{nstabbing}
      %\>\Rbar{}~Pater \myhspace{} meus \myhspace{} es \myhspace{} tu, \>\hspace{\myhhwidth}allelúia.
      \>\Rbar{}~Afférte \myhspace{}Dómino\myhspace{} glóriam\myhspace{} nómini\myhspace{} ejus, \>\hspace{\myhhwidth}allelúia.
      \end{nstabbing}

      \translation[]{\Vbar{}~Give unto the Lord glory and honour, alleluia.\\
        \Rbar{}~Give unto the Lord the glory due unto His Name, alleluia.}

      \bigskip
    }
    \section{Holy Name}
    \def\gabcfolder{.}
    \printterce{}{holyname}
  }
  {
    % epiphany
    % proper hymn, ant & chapter from vespers, short resp
    \def\printhymn{

      {\centering Hymn.\par}\label{hymn-epiphany}

      {\def\gabcfolder{.}
       \printgabc{8.}{}{N}{hy--nunc_sancte_nobis_(in_epiphania_domini)--solesmes_1961}
      }

      %\printrefshymn{ant}
      \bigskip
    }
    \def\printshortresp{%
      \label{shortresp-epiphany}%
      {\def\gabcfolder{.}
      \printgabc{Short}{Resp.}{R}{re--reges_tharsis--solesmes_1961}}

      \translation[]{\Vbar{}~The kings of Tarshish and of the isles shall bring presents. Alleluia, Alleluia.
        \Rbar{}~The kings\dots{}
        \Vbar{}~The kings of Arabia and Saba shall offer gifts.
        \Rbar{}~Alleluia, Alleluia.
        \Vbar{}~Glory be to the Father, and to the Son, and to the Holy Ghost.
        \Rbar{}~The kings\dots{}
      }

      \bigskip
      \gresetinitiallines{0}\label{vr-epiphany}
      \gregorioscore{vr-omnes-de-saba}
      \let\myhwidth\relax
      \let\myhhwidth\relax
      \newlength{\myhwidth}
      \settowidth{\myhwidth}{allelui} %this is what precedes the last vowel in the response
      \newlength{\myhhwidth}
      \settowidth{\myhhwidth}{i} %this is what preceded the last vowel in the verse
      \addtolength{\myhhwidth}{-\myhwidth}
      \def\myhspace{\hspace{0.1ex}}
      \begin{nstabbing}
      %\>\Rbar{}~Pater \myhspace{} meus \myhspace{} es \myhspace{} tu, \>\hspace{\myhhwidth}allelúia.
      \>\Rbar{}~Aurum et thus deferéntes, \>\hspace{\myhhwidth}allelúia.
      \end{nstabbing}

      \translation[]{\Vbar{}~All they from Saba shall come. Alleluia.\\
        \Rbar{}~They shall bring gold and incense. Alleluia.}

      \bigskip
    }
    \section{Epiphany}
    \printterce[../Epiphany]{inc-Epiphany-vespers}{epiphany}
  }
  {
    % holy family
    % ant & chapter from vespers, proper short resp
    % slightly proper hymn (rest from epiphany)
    \def\printhymn{

      {\centering Hymn.\par}\label{hymn-holyfamily}

      {\def\gabcfolder{.} %TODO: this has the correct verse 3 but first 2 verses are wrong!
       \printgabc{8.}{}{N}{hy--te_lucis_ante_terminum_(holy_family)--solesmes_1961}
      }

      %\printrefshymn{ant}
      \bigskip
    }
    \def\printshortresp{%
      \label{shortresp-holyfamily}%
      {\def\gabcfolder{.}
      \printgabc{Short}{Resp.}{P}{re--propter_nos--solesmes_1961}}

      \translation[]{\Vbar{}~For our sake he became poor, Being rich.
        \Rbar{}~For our sake\dots{}
        \Vbar{}~That through his poverty we might be rich.
        \Rbar{}~Being rich.
        \Vbar{}~Glory be to the Father, and to the Son, and to the Holy Ghost.
        \Rbar{}~For our sake\dots{}
      }

      \bigskip
      \gresetinitiallines{0}\label{vr-holyfamily}
      \gregorioscore{vr-dominus-vias-suas}
      \let\myhwidth\relax
      \let\myhhwidth\relax
      \newlength{\myhwidth}
      \settowidth{\myhwidth}{ej} %this is what precedes the last vowel in the response
      \newlength{\myhhwidth}
      \settowidth{\myhhwidth}{n} %this is what preceded the last vowel in the verse
      \addtolength{\myhhwidth}{-\myhwidth}
      \def\myhspace{\hspace{0.1ex}}
      \begin{nstabbing}
      %\>\Rbar{}~Pater \myhspace{} meus \myhspace{} es \myhspace{} tu, \>\hspace{\myhhwidth}allelúia.
      \>\Rbar{}~Et \myhspace{} ambulábimus \myhspace{} in \myhspace{} sémitis \>\hspace{\myhhwidth}ejus.
      \end{nstabbing}

      \translation[]{\Vbar{}~The Lord will teach us his ways.\\
        \Rbar{}~And we will walk in his paths.}

      \bigskip
    }
    \section{Holy Family}
    \printterce[../HolyFamily]{inc-HolyFamily-Vespers2}{holyfamily}
  }
}

	\chapter{Proper of the Time -- After Epiphany}
{
\newcommand{\benedicamusdomino}[1][sunday]{
  \benedicamusdominomaster{#1}
}
\def\printhymnnote{}
\def\printcommonvespers{
	\subtitle{\nth{2} Class, Green}
	\printnote{From \emph{Vespers of Sundays throughout the year}, page \pageref{sundayvespers}.\\}
}

{
\section{\nth{2} Sunday after Epiphany}
\label{epiphany2}
\printcommonvespers{}
\printvespersmag[../TimeAfterEpiphany]{inc-VespersMagnificatEpiphany2}

\bigskip
\benedicamusdomino{}
}

{
\section{\nth{3} Sunday after Epiphany}
\label{epiphany3}
\printcommonvespers{}
\printvespersmag[../TimeAfterEpiphany]{inc-VespersMagnificatEpiphany3}

\bigskip
\benedicamusdomino{}
}

{
\section{\nth{4} Sunday after Epiphany}
\label{epiphany4}
\def\precollect{\printvrdirigatur}
\printcommonvespers{}
\printvespersmag[../TimeAfterEpiphany]{inc-VespersMagnificatEpiphany4}

\bigskip
\benedicamusdomino{}
}

{
\section{\nth{5} Sunday after Epiphany}
\label{epiphany5}
\printcommonvespers{}
\printvespersmag[../TimeAfterEpiphany]{inc-VespersMagnificatEpiphany5}

\bigskip
\benedicamusdomino{}
}

{
\section{\nth{6} Sunday after Epiphany}
\label{epiphany6}
\printcommonvespers{}
\def\prevespers{%
  \let\oldthing=\englishmagantiphon
  \def\englishmagantiphon{\oldthing\pagebreak}
}
\printvespersmag[../TimeAfterEpiphany]{inc-VespersMagnificatEpiphany6}

\bigskip
\benedicamusdomino{}
}
}

	{
	\label{septuagesima}
	\chapter{Septuagesima \& Lent}
  \def\printbenedicamusdomino{
    \noindent\emph{Benedicamus Domino}, p. \pageref{benedicamus-domino-sunday}.
  }
  \section{Septuagesima}
  \printterce{inc-septuagesima}{septuagesima}

  \section{Sexagesima}
  \printterce{inc-sexagesima}{sexagesima}

  \section{Quinquagesima}
  \printterce{inc-quinquagesima}{quinquagesima}

  {
    \newcommand{\printrefs}[1]{%
      \def\dotting{\hfill%\leaders\hbox to 1em{\hfil.\hfil}\hfill
        }%
      \begin{multicols}{2}%
      \noindent{}1st Sunday of Lent,\dotting \emph{below}\\
      2nd Sunday of Lent,\dotting \emph{p.~\pageref{lent2-#1}}\\
      3rd Sunday of Lent,\dotting \emph{p.~\pageref{lent3-#1}}\\
      4th Sunday of Lent,\dotting \emph{p.~\pageref{lent4-#1}}
      \end{multicols}%
      \smallskip
    }

	  \def\printhymn{

		  {\centering Hymn.\par}\label{hymn-lent}

	  	{\def\gabcfolder{.}
	  	\printgabc{1.}{}{N}{hy--nunc_sancte_nobis_(in_quadragesima)--solesmes_1961}}

      \printrefs{ant}
		  \bigskip
	  }

    \def\printshortresp{%
    \label{shortresp-lent}%
    {\def\gabcfolder{.}
    \printgabc{Short}{Resp.}{I}{re--ipse_liberavit--solesmes}}

    \translation[]{\Vbar{}~For he hath delivered me from the snare of the hunters.
    \Rbar{}~For he hath\dots{}
    \Vbar{}~And from the sharp word.
    \Rbar{}~From the snare of the hunters.
    \Vbar{}~Glory be to the Father, and to the Son, and to the Holy Ghost.
    \Rbar{}~For he hath\dots{}
    }

    \bigskip
    \gresetinitiallines{0}\label{vr-lent}
    \gregorioscore{vr-scapulis-suis}
    \let\myhwidth\relax
    \let\myhhwidth\relax
    \newlength{\myhwidth}
    \settowidth{\myhwidth}{speráb} % text in last word before last vowel of response
    \newlength{\myhhwidth}
    \settowidth{\myhhwidth}{b} % text in last syllable before vowel of versicle
    \addtolength{\myhhwidth}{-\myhwidth}
    \def\myhspace{\hspace{1ex}}
    \begin{nstabbing}
    %\>\Rbar{}~Et \myhspace{} omnes \myhspace{} reges \myhspace{} terræ \myhspace{} glóriam \>\hspace{\myhhwidth}tuam.
    \>\Rbar{}~Et \myhspace{} sub \myhspace{} pennis \myhspace{} ejus \>\hspace{\myhhwidth}sperábis.
    \end{nstabbing}

    \translation[]{\Vbar{}~He will overshadow thee with his shoulders.\\
    \Rbar{}~And under his wings thou shalt trust.}

    \printrefs{collect}
    \bigskip
    }

    \section{First Sunday of Lent}
	  \printterce{inc-Lent1}{lent1}
  }

  \def\printhymn{%
  	, \emph{Hymn}, page \pageref{hymn-lent}

  	\medskip
  }
  \def\printshortresp{
    \noindent\emph{Short Resp. \emph{Ipse liberavit me}}, p. \pageref{shortresp-lent}, \Vbar{}~Scapulis suis, p. \pageref{vr-lent}.
  }
  \section{Second Sunday of Lent}
  \printterce{inc-Lent2}{lent2}

  \section{Third Sunday of Lent}
  \printterce{inc-Lent3}{lent3}

  \section{Fourth Sunday of Lent}
  \printterce{inc-Lent4}{lent4}

  {
    \newcommand{\printrefs}[1]{%
      \def\dotting{\hfill%\leaders\hbox to 1em{\hfil.\hfil}\hfill
        }%
      \begin{multicols}{2}%
      \noindent{}1st Sunday of the Passion,\dotting \emph{below}\\
      2nd Sunday of the Passion,\dotting \emph{p.~\pageref{passion2-#1}}
      \end{multicols}%
      \smallskip
    }

    \def\printhymn{

      {\centering Hymn.\par}\label{hymn-passiontide}

      {\def\gabcfolder{.}
      \printgabc{2.}{}{N}{hy--nunc_sancte_nobis_(in_tempore_passionis)--solesmes_1961}}

      \printrefs{ant}
      \bigskip
    }

    \def\printshortresp{%
    \label{shortresp-passiontide}%
    {\def\gabcfolder{.}
    \printgabc{Short}{Resp.}{E}{re--erue_a_frama--solesmes}}

    \translation[]{\Vbar{}~Deliver from the sword, O God, my soul.
    \Rbar{}~Deliver\dots{}
    \Vbar{}~My only one from the hand of the dog.
    \Rbar{}~O God, my soul.
    \Rbar{}~Deliver, * O God, my soul from the sword.For he hath delivered me from the snare of the hunters.
    }

    \bigskip
    \gresetinitiallines{0}\label{vr-passiontide}
    \gregorioscore{vr-de-ore-leonis}
    \let\myhwidth\relax
    \let\myhhwidth\relax
    \newlength{\myhwidth}
    \settowidth{\myhwidth}{me} % text in last word before last vowel of response
    \newlength{\myhhwidth}
    \settowidth{\myhhwidth}{n} % text in last syllable before vowel of versicle
    \addtolength{\myhhwidth}{-\myhwidth}
    \def\myhspace{\hspace{1.4ex}}
    \begin{nstabbing}
    %\>\Rbar{}~Et \myhspace{} omnes \myhspace{} reges \myhspace{} terræ \myhspace{} glóriam \>\hspace{\myhhwidth}tuam.
    \>\Rbar{}~Et a córnibus unicórnium humilitátem \>\hspace{\myhhwidth}meam.
    \end{nstabbing}

    \translation[]{\Vbar{}~From the lion's mouth, O Lord, save me.
    \Rbar{}~And my lowness from the horns of the unicorns.}

    \printrefs{collect}
    \bigskip
    }

    \section{First Sunday of the Passion}
    \printterce{inc-Passion1}{passion1}
  }
 
  \def\printhymn{%
    , \emph{Hymn}, page \pageref{hymn-passiontide}

    \medskip
  }
  \def\printshortresp{
    \noindent\emph{Short Resp. \emph{Erue a framea}}, p. \pageref{shortresp-passiontide}, \Vbar{}~De ore leonis, p. \pageref{vr-passiontide}.
  }
  \section{Second Sunday of the Passion}
  \printterce{inc-Passion2}{passion2}
}

	{
\newcommand{\printcommemnote}[1][easter]{\smallskip
\noindent
\printnote{\commemorations{}  Otherwise \benedicamusdominoreference{#1}}
}
{
\chapter{Proper of the Time -- Easter}
\section{Easter Sunday}
\subtitle{\nth{1} Class, White or Gold}
\def\printfullhymn{
    \vspace{-0.5\baselineskip}
    \emph{Chapter, Hymn, and Versicle are all omitted, but the following Antiphon is said :}

    \bigskip
    \def\annot{\small{Ant.}}
    \def\annottwo{\small{\chapterhymnversicleantiphonmode.}}
    \alsetinitialspacing{\chapterhymnversicleantiphoninitial}
    \gregorioscore{\gabcfolder/\chapterhymnversicleantiphontex}
    \translation[]{\chapterhymnversicleantiphontranslation}
    \vspace{-0.5\baselineskip}
    \bigskip
}
\def\chapterreplacement{\bigskip}
\def\begincollectcols{\begin{parcolumns}[rulebetween,colwidths={1=0.45\linewidth}]{2}}
\def\postmag{\vspace{-0.05\baselineskip}}
\printvespers[../Easter]{inc-EasterVespers}
\newcommand{\printbenedicamusdomino}[2]{
	\def\annot{\small{#1}}
	\def\annottwo{}
	\alsetinitialspacing{B}
    \greseteolcustos{manual}
	\gregorioscore{#2}
    \greseteolcustos{auto}
    \bigskip
    \hrule
}
\def\breakbeforeresp{T}
\printbenedicamusdomino{\Vbar}{../BenedicamusDomino/BenedicamusDomino_Easter}
}

\newcommand{\printcommonvespers}[1][2]{
    \subtitle{\nth{#1} Class, \liturgicalcolor{}}
    \deusinadjutorium{}
    \printnote{\emph{Vespers of Sundays in Paschaltide}, p.~\pageref{sundayvespers-easter}.\\}
}
\def\liturgicalcolor{White}
{
\newcommand{\benedicamusdomino}[1][easter]{
  \benedicamusdominomaster{#1}
}

\newcommand{\printhymnnote}{
    \noindent\printnote{Hymn.~\emph{Ad Régias Agni Dapes}, page \pageref{hymn-adregiasagnidapes}.
    \Vbar~\emph{Mane nobíscum}, page \pageref{vr-manenobiscum}.}
}

{
\section{Low Sunday}
\label{easter1}
\printcommonvespers[1]
\def\begincollectcols{\begin{parcolumns}[rulebetween,colwidths={1=0.42\linewidth}]{2}}
\printvespersmag[../TimeAfterEaster]{inc-VespersMagnificatEaster1}

\def\commemorations{If the Feast of the Annunciation has been transferred to the Monday following Low Sunday, First Vespers is commemorated as on page \pageref{annunciation-commem}.  If today is April 30, May 1, or May 2, First Vespers of St Joseph the Worker is commemorated as follows.}
\printcommemnote{}
}

\medskip
\hrule
{
\label{stjoseph-worker-commem}
\def\begincollectcols{\begin{parcolumns}[rulebetween,colwidths={1=0.43\linewidth}]{2}}
\def\vrlinebreak{T}
\printcommemoration[../May1-StJosephWorker]{commemorationStJosephWorker-Vespers1}

\bigskip
\benedicamusdomino{}
}

{
\section{\nth{2} Sunday after Easter}
\label{easter2}
\printcommonvespers{}
\def\precollect{\printvrmanenobiscum}
\printvespersmag[../TimeAfterEaster]{inc-VespersMagnificatEaster2}
\benedicamusdomino{}
}
{
\section{\nth{3} Sunday after Easter}
\label{easter3}
\printcommonvespers{}
\def\begincollectcols{\begin{parcolumns}[rulebetween,colwidths={1=0.44\linewidth}]{2}}
\def\precollect{\printvrmanenobiscum}
\printvespersmag[../TimeAfterEaster]{inc-VespersMagnificatEaster3}
\benedicamusdomino{}
}

{
\section{\nth{4} Sunday after Easter}
\label{easter4}
\printcommonvespers{}
\def\precollect{\printvrmanenobiscum}
\printvespersmag[../TimeAfterEaster]{inc-VespersMagnificatEaster4}
\benedicamusdomino{}
}

{
\section{\nth{5} Sunday after Easter}
\label{easter5}
\printcommonvespers{}
\def\precollect{\printvrmanenobiscum}
\printvespersmag[../TimeAfterEaster]{inc-VespersMagnificatEaster5}
\benedicamusdomino{}
}

{
\section{Ascension of Our Lord}
\label{ascension}
\subtitle{\nth{1} Class, White or Gold}
\vspace{-0.5\baselineskip}
\subtitle{I \& II Vespers}
\vspace{-0.5\baselineskip}

\def\premagverses{\greseteolcustos{manual}}
\def\begincollectcols{%\vspace{-0.5\baselineskip}
\begin{parcolumns}[rulebetween,colwidths={1=0.46\linewidth}]{2}}
\def\definevesperspropers{\newcommand{\maganttex}{an--o_rex_gloriae--solesmes}
\newcommand{\magantinitial}{O}
\newcommand{\maganttranslation}{O King of glory, Lord of hosts, who today didst ascend in triumph above all heavens, leave us not orphans, but send us the promise of the Father, the Spirit of truth, alleluia.}
\def\magsolemn{T}
\definemag{2}{D}

    \def\prepsalmfive{\greseteolcustos{manual}}
}
\def\definevesperspropersalt{\newcommand{\maganttex}{MagnificatAntiphon1}
\newcommand{\magantinitial}{P}
\newcommand{\maganttranslation}{Father, I have manifested Thy Name unto the men whom Thou gavest Me; and now I pray for them, not for the world, because I come to Thee, alleluia.}
\def\magsolemn{T}
\definemag{6}{F}

    %\def\premagverses{\pagebreak}
}
\def\vesperspropersnote{At II Vespers:}
\def\vesperspropersaltnote{At I Vespers:}
\let\printhymnnote=\undefined
\def\hymnlabel{hymn-salutishumanaesator}
\printvespers[../Ascension]{inc-Ascension}

\def\commemorations{If today is April 30 or May 1, First Vespers of St Joseph the Worker is commemorated as on page \pageref{stjoseph-worker-commem}.}
\printcommemnote[1]{}
\medskip
\hrule
}

{
\section{Sunday after the Ascension}
\label{easter6}\label{sundayafterascension}
\printcommonvespers{}
%\let\printhymnnote=\undefined
\renewcommand{\printhymnnote}{
    \noindent\printnote{Hymn.~\emph{Salútis humánæ Sator}, page \pageref{hymn-salutishumanaesator}.}
    \def\vrlinebreak{T}
    \printvr[\greseteolcustos{manual}]{\vrtex}{\vtranslation}{\rtranslation}
}
\def\precollect{\printvrmanenobiscum}
%\def\premagverses{\pagebreak}
\printvespersmag[../TimeAfterEaster]{inc-VespersMagnificatSundayAfterAscension}
\benedicamusdomino{}
}
}
}

	\chapter{Proper of the Time -- Pentecost Octave}
{
\newcommand{\benedicamusdomino}[1][1]{
  \benedicamusdominomaster{#1}
}
{
\section{Pentecost Sunday}
\subtitle{\nth{1} Class}
\subtitle{I \& II Vespers}

\def\deusinadjutoriumsolemn{T}
\def\definevesperspropers{\definepsalm{5}{113}{7}{c2}

\newcommand{\vrtex}{vrLoquebantur}
\newcommand{\vtranslation}{The apostles spoke in divers tongues.}
\newcommand{\rtranslation}{The wonderful works of God.}

\newcommand{\maganttex}{MagnificatAntiphon2}
\newcommand{\magantinitial}{H}
\newcommand{\maganttranslation}{Today the days of Pentecost are complete, alleluia; today the Holy Ghost appeared in fire to the disciples, gave them gifts and graces, sent them into all the world to preach and to bear witness; whoever believes and is baptised shall be saved, alleluia.}

	%\def\prepsalmfivetitle{\medskip}
	\def\prepsalmfive{\greseteolcustos{manual}}
}
\def\definevesperspropersalt{\definepsalm{5}{116}{7}{c2}

\newcommand{\vrtex}{vrRepletiSunt}
\newcommand{\vtranslation}{They were all filled with the Holy Ghost.}
\newcommand{\rtranslation}{And they began to speak.}

\newcommand{\maganttex}{MagnificatAntiphon1}
\newcommand{\magantinitial}{N}
\newcommand{\maganttranslation}{I will not leave you orphans, alleluia; I go away, and I come unto you, alleluia; and your heart shall rejoice, alleluia.}
%
%\let\oldthing=\maganttranslation
%\def\maganttranslation{\oldthing\vspace{-0.5\baselineskip}}
%\def\postmagtitle{\vspace{-\baselineskip}}%
}%
%
%\def\prepsalmtitleone{\vspace{-0.5\baselineskip}}
%\def\prepsalmtitlethree{\vspace{-0.5\baselineskip}}
\ifthenelse{\boolean{birmingham}}{}{
	\def\prepsalmtitleone{\bigskip{}}
	\def\postpsalmtitleone{\bigskip{}}
}
\def\preantthree{\bigskip}
\def\prepsalmtitlefour{\needspace{4\baselineskip}}
\def\vesperspropersnote{At II Vespers:}
\def\vesperspropersaltnote{At I Vespers:}

%\def\premag{\def\noeuouae{T}}
\def\premagverses{\greseteolcustos{manual}}
\def\prehymn{\printnote{All kneel for the first stanza of the following hymn.}}
\def\prehymntranslation{\vspace{-0.3\baselineskip}}
\def\prevespers{%
%	\let\oldthing\antthreetranslation
%	\def\antthreetranslation{\oldthing\vspace{-\baselineskip}}%
	%\let\oldthingb\antfivetranslation
	%\def\antfivetranslation{\oldthingb\vspace{-\baselineskip}}%
}
%\def\prevr{\vspace{-0.7\baselineskip}}
\printvespers[../Pentecost]{inc-PentecostVespers}
\bigskip
\benedicamusdomino{}
}

{
\section{Trinity Sunday}
\subtitle{\nth{1} Class}

\def\deusinadjutoriumsolemn{T}
%\def\premag{\def\noeuouae{T}}
%\def\prepsalmtwoverses{\vspace{-0.05\baselineskip}}
%\def\prepsalmtitleone{\needspace{10\baselineskip}}
\def\prepsalmone{\needspace{10\baselineskip}}

\def\prerepeatantiphontwo{}
\def\preantfive{\bigskip}
\newcommand{\psalmcolsoverrideoverride}[1][0]{
	\psalmcolsoverride[#1]
	\ifnum#1=110
		\def\beginpsalmcols{\begin{parcolumns}[rulebetween,colwidths={1=0.49\linewidth},distance=1.3em]{2}}
	\fi
}
%\def\prepsalmtitlethree{\vspace{-0.5\baselineskip}}
%\def\prerepeatantiphonthree{\vspace{-0.25\baselineskip}}
\def\premagnificat{\needspace{12\baselineskip}}
\def\premagverses{\greseteolcustos{manual}}
\def\prehymn{\oldneedspace{16\baselineskip}}
\def\begincollectcols{\begin{parcolumns}[rulebetween,colwidths={1=0.44\linewidth}]{2}}
\printvespers[../TrinitySunday]{inc-TrinitySunday-Vespers}
\bigskip
\noindent
\printnote{If the feast of St Philip Neri is celebrated tomorrow, then \emph{First Vespers} is commemorated as follows.
\ifthenelse{\boolean{includebenedicamusdominoreferences}}{
	Otherwise \benedicamusdomino{}%
}{}
}
\medskip
%\hrule
%\medskip
}

% commemoration of First Vespers of Immaculate Conception
{
\def\beginvrcols{\begin{parcolumns}[rulebetween,colwidths={1=0.45\linewidth}]{2}}
\def\vrlinebreak{T}
\printcommemoration[../StPhilipNeri]{Commemoration-StPhilipNeri-Vespers1}

%\bigskip
\benedicamusdomino{}
}

}


	\chapter{Proper of the Time -- Time After Pentecost}
{
\def\printcommonvespers{
  \vspace{-0.5\baselineskip}
  \subtitle{\nth{2} Class, Green}
  \medskip
  \deusinadjutorium{}
  \hfill
  \emph{\emph{Vespers of Sundays throughout the year,} p.~\pageref{sundayvespers}.\par}
}
\newcommand{\benedicamusdomino}[1][sunday]{
  \benedicamusdominomaster{#1}
}
\def\premagtitle{
  \needspace{8\baselineskip}  
}

\newcommand{\printhymnnote}{}
\newcommand{\printvespersafterpentecost}[1]{
  {
  \section{\nth{#1} \ifnum#1=24{or Last }\fi Sunday after Pentecost}
  \label{pentecost#1}
\printcommonvespers{}
\ifx\printaftercommonvespers\undefined\else\printaftercommonvespers{}\fi
  \ifx\nocommemoration\undefined%
    \def\precollect{\printvrdirigatur\smallskip}%
  \fi
  \ifx\postmagtitle\undefined\def\oldpostmagtitle{}\else\let\oldpostmagtitle=\postmagtitle\fi
  \def\postmagtitle{\oldpostmagtitle\label{pentecost#1-mag}}
  \printvespersmag[../TimeAfterPentecost]{inc-VespersMagnificatPentecost#1}
  \smallskip
  \benedicamusdomino{}
  }
  
}


\ifthenelse{\boolean{includecorpuschristi}}{% true
{
  \section{Corpus Christi}

  \def\definevesperspropers{\newcommand{\maganttex}{an--o_sacrum_convivium--solesmes}
\newcommand{\magantinitial}{O}
\newcommand{\maganttranslation}{O sacred banquet, in which Christ is received; the memory of His Passion is renewed; the mind is filled with grace; and a pledge of future glory is given to us, alleluia.}
\newcommand{\magsolemn}{T}
\definemag{5}{a}
}

  \printvespers[../CorpusChristi]{inc-CorpusChristi}
  \medskip
  \benedicamusdomino{}
}
}{}% false

{\def\begincollectcols{\begin{parcolumns}[rulebetween,colwidths={1=0.455\linewidth}]{2}}
\printvespersafterpentecost{2}}
\ifthenelse{\boolean{includesacredheart}}{% true
{
  \section{Sacred Heart of Jesus}

  \def\definevesperspropers{\newcommand{\antonetex}{an--unus_militum--solesmes}
\newcommand{\antoneinitial}{U}
\newcommand{\antonetranslation}{A soldier with a spear opened his side, and immediately there came forth blood and water.}
\definepsalm{1}{109}{1}{f}

\newcommand{\anttwotex}{an--stans_jesus--solesmes}
\newcommand{\anttwoinitial}{S}
\newcommand{\anttwotranslation}{Jesus stood and cried, saying~: If any man thirst, let him come to me and drink.}
\definepsalm{2}{110}{7}{c}

\newcommand{\antthreetex}{an--in_caritate--solesmes}
\newcommand{\antthreeinitial}{I}
\newcommand{\antthreetranslation}{With an everlasting love has God loved us; lifted up, therefore, from the earth, he has drawn us to his Heart, taking pity on us.}
\definepsalm{3}{115}{3}{a2}

\newcommand{\antfourtex}{an--venite_ad_me--solesmes}
\newcommand{\antfourinitial}{V}
\newcommand{\antfourtranslation}{Come to me, all you that labour and are burdened~: and I will refresh you.}
\definepsalm{4}{127}{4}{E}

\newcommand{\antfivetex}{an--fili_praebe--solesmes}
\newcommand{\antfiveinitial}{F}
\newcommand{\antfivetranslation}{My son, give me thy heart~: and let thine eyes keep my ways.}
\definepsalm{5}{147}{5}{a}

\newcommand{\maganttex}{an--ad_jesum_autem--solesmes}
\newcommand{\magantinitial}{A}
\newcommand{\maganttranslation}{But after they were come to Jesus, when they saw that he was already dead, they did not break his legs.  But one of the soldiers with a spear opened his side~: and immediately there came out blood and water.}
\newcommand{\magsolemn}{T}
\definemag{1}{f}

\newcommand{\vrtex}{vrHaurietis}
\newcommand{\vtranslation}{Ye shall draw waters with joy.}
\newcommand{\rtranslation}{Out of the Saviour's fountains.}

}

  \printvespers[../SacredHeart]{inc-SacredHeart}
  \medskip
  \benedicamusdomino{}
}
}{}% false
\ifthenelse{\boolean{testrun}}{}{
\printvespersafterpentecost{3}
\printvespersafterpentecost{4}
{\def\begincollectcols{\begin{parcolumns}[rulebetween,colwidths={1=0.44\linewidth}]{2}}
\printvespersafterpentecost{5}}
\printvespersafterpentecost{6}
\printvespersafterpentecost{7}
{
  \def\premagtitle{}
  \def\postmagtitle{
    \pagebreak[2]
  }
\printvespersafterpentecost{8}}
\printvespersafterpentecost{9}
\printvespersafterpentecost{10}
\printvespersafterpentecost{11}
{\def\begincollectcols{\begin{parcolumns}[rulebetween,colwidths={1=0.455\linewidth}]{2}}
%\def\printaftercommonvespers{\needspace{6\baselineskip}}
\printvespersafterpentecost{12}}
{
\def\premagnificat{\bigskip}
\printvespersafterpentecost{13}}
\printvespersafterpentecost{14}
\printvespersafterpentecost{15}
\printvespersafterpentecost{16}
{
  \def\premagtitle{}
  \def\postmagtitle{
    \pagebreak[2]
  }
\printvespersafterpentecost{17}}
\printvespersafterpentecost{18}
{\def\begincollectcols{\begin{parcolumns}[rulebetween,colwidths={1=0.47\linewidth}]{2}}
\def\postbenedicamusdomino{\bigskip}
%\def\premagverses{\vspace{-0.7\baselineskip}}
\printvespersafterpentecost{19}}
\bigskip\bigskip
\printvespersafterpentecost{20}
\printvespersafterpentecost{21}
{
\def\postmagtitle{\medskip}
\printvespersafterpentecost{22}
}
{
\def\postbenedicamusdomino{\bigskip}
\printvespersafterpentecost{23}
}
\bigskip
{%\def\prevespers{
 % \let\oldthing=\englishmagantiphon
 % \def\englishmagantiphon{\oldthing\pagebreak}  
%}
\def\begincollectcols{\begin{parcolumns}[rulebetween,colwidths={1=0.455\linewidth}]{2}}
\printvespersafterpentecost{24}}
}
}

}
\end{document}

