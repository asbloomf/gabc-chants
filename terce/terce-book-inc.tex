% !TEX TS-program = lualatex
% !TEX encoding = UTF-8

% This is a simple template for a LuaLaTeX document using gregorio scores.

% easter can be from march 22 to april 25

\usepackage{../definepsalms}
\usepackage{titlesec}
\usepackage{titletoc}
\usepackage{titleps}
\usepackage{letltxmacro}
\usepackage{changepage} % gives us \ifoddpage use [strict]
\usepackage[super]{nth}
\usepackage[savepos]{zref}
\usepackage{xparse}
\usepackage{setspace}
\usepackage{amsfonts}
\usepackage{refcount}
\usepackage{metalogo}
\usepackage{enumitem}
\usepackage{zref-titleref}
\makeatletter
\newcommand*{\currentname}{\zref@getcurrent{title}}
% or \newcommand*{\currentname}{\zref@titleref@current}
\makeatother
%\nofiles
%\includeonly{inc-grassi}
\LetLtxMacro{\oldnth}{\nth}
\renewcommand{\nth}[1]{{\addfontfeature{Numbers=Lining}\oldnth{#1}}}
\LetLtxMacro{\oldneedspace}{\needspace}
\renewcommand{\needspace}[1]{
	\checkoddpage\ifoddpage\oldneedspace{#1}\else\fi
}
\let\gredagger=\dag
\newcommand*\cleartoleftpage{%
  \clearpage
  \ifodd\value{page}\hbox{}\newpage\fi
}
\hyphenation{GregoBase}

%\usepackage{hyperref}
\ifx\phantomsection\undefined%
	\newcommand{\phantomsection}{}
\fi

\setcounter{secnumdepth}{-1}

\ifthenelse{\boolean{lettersize}}{
	\def\mywidth{8.5in}
	\def\myheight{11in}
}{
	\def\mywidth{6in}
	\def\myheight{9in}
}

\usepackage{../definepsalms}
\usepackage{adjustbox}
\sloppy
\usepackage{../definepsalms}
\usepackage{adjustbox}
\sloppy
\usepackage{../definepsalms}
\usepackage{adjustbox}
\sloppy
\input{../inc_header} %


\let\oldVbar=\Vbar
\def\Vbar{\oldVbar\hspace{-2pt}}
\let\oldRbar=\Rbar
\def\Rbar{\oldRbar\hspace{-2pt}}

\ifbook{
\lfoot[\thepage]{}
\rfoot[]{\thepage}
}
\ifnotbook{
\cfoot{%
  \ifnum\thepage>1
    \thepage
  \fi
}
}
%\renewcommand\headrulewidth{\oldheadrulewidth}
\chead{
  \ifnum\thepage>1
    \addfontfeature{Numbers=Lining}%
    \heading{} \ifx\matinsnocturn\undefined\else(\matinsnocturn)\fi
  \fi
}

\newcommand{\writeheading}[1]{
  \begin{center}{
  \addfontfeature{Numbers=Lining}
  \textsc{#1}
  }\end{center}
  \medskip
}
\newcommand{\printseparation}{
  %\hfil\rule{3in}{0.5pt}\hfil
  \bigskip
  \bigskip
}

\def\nogloriapatri{T}
 %


\let\oldVbar=\Vbar
\def\Vbar{\oldVbar\hspace{-2pt}}
\let\oldRbar=\Rbar
\def\Rbar{\oldRbar\hspace{-2pt}}

\ifbook{
\lfoot[\thepage]{}
\rfoot[]{\thepage}
}
\ifnotbook{
\cfoot{%
  \ifnum\thepage>1
    \thepage
  \fi
}
}
%\renewcommand\headrulewidth{\oldheadrulewidth}
\chead{
  \ifnum\thepage>1
    \addfontfeature{Numbers=Lining}%
    \heading{} \ifx\matinsnocturn\undefined\else(\matinsnocturn)\fi
  \fi
}

\newcommand{\writeheading}[1]{
  \begin{center}{
  \addfontfeature{Numbers=Lining}
  \textsc{#1}
  }\end{center}
  \medskip
}
\newcommand{\printseparation}{
  %\hfil\rule{3in}{0.5pt}\hfil
  \bigskip
  \bigskip
}

\def\nogloriapatri{T}
 %


\let\oldVbar=\Vbar
\def\Vbar{\oldVbar\hspace{-2pt}}
\let\oldRbar=\Rbar
\def\Rbar{\oldRbar\hspace{-2pt}}

\ifbook{
\lfoot[\thepage]{}
\rfoot[]{\thepage}
}
\ifnotbook{
\cfoot{%
  \ifnum\thepage>1
    \thepage
  \fi
}
}
%\renewcommand\headrulewidth{\oldheadrulewidth}
\chead{
  \ifnum\thepage>1
    \addfontfeature{Numbers=Lining}%
    \heading{} \ifx\matinsnocturn\undefined\else(\matinsnocturn)\fi
  \fi
}

\newcommand{\writeheading}[1]{
  \begin{center}{
  \addfontfeature{Numbers=Lining}
  \textsc{#1}
  }\end{center}
  \medskip
}
\newcommand{\printseparation}{
  %\hfil\rule{3in}{0.5pt}\hfil
  \bigskip
  \bigskip
}

\def\nogloriapatri{T}

\gresetbarspacing{new}
%\gresetlastline{justified}
\setlength\headheight{0.25in+15pt}
\setlength\headsep{1pc}
\setlength\topskip{0pc}
\setlength\footskip{1pc}
\geometry{outer=0.4in,inner=0.85in,top=0pc+\headheight+\headsep,bottom=0.4in,twoside=true}
\newpagestyle{main}{
%\setheadrule{0pt}
\sethead[\garamond{\thepage}][\garamond{\chaptertitle}][] % even
{}{\garamond{\sectiontitle}}{\garamond{\thepage}} % odd
\setfoot[][][] % even
{}{}{} % odd
}
\pagestyle{main}

%\grechangeglyph{Porrectus*}{*}{.alt}
%\grechangeglyph{TorculusResupinus*}{*}{.alt}

\titleformat
{\section} % command
[block] % shape
{\phantomsection\large\addfontfeature{Numbers=Lining}} % format
{} % label
{} % sep
{
    % \rule{\textwidth}{1pt}
    % \vspace{1ex}
    \centering
} % before-code
%[
% \vspace{-0.5ex}%
% \rule{\textwidth}{0.3pt}
%] % after-code
 
 
\titleformat{\chapter}[block]
{\thispagestyle{empty}\phantomsection\Large\scshape\addfontfeature{Numbers=Lining}}
{}{0.5em}{\centering}
 
\titlespacing{\chapter}{0pt}{6pt-\headheight}{1pc}
\titlespacing{\section}{0pt}{*2.5}{*1}
\titleclass{\chapter}{top}
\newcommand{\chapterbreak}{\clearpage}
%\titleclass{\section}{top}

\contentsmargin{2pc}
%\dottedcontents{chapter}[2.3em]{}{2.3em}{1pc}
\titlecontents{chapter}[2.3em]{}{\contentslabel{2.3em}}{\hspace*{-2.3em}}{}
\dottedcontents{section}[5.5em]{}{3.2em}{1pc}

\newcommand{\printnote}[1]{
	{\normalsize \emph{#1}}%
}
\newcommand{\subtitle}[1]{
{
	\centering
	{\addfontfeature{Numbers=Lining} \normalsize \emph{#1}}\par
}
}
\makeatletter  
\newcounter{score}
\newcounter{tabstop}[score]
\newcommand{\grealign}{%
  \@bsphack%
  \ifgre@boxing\else%
    \kern\gre@dimen@begindifference%
    \stepcounter{tabstop}%
    \expandafter\zsavepos{stop-\thescore-\thetabstop}%
    \kern-\gre@dimen@begindifference%
  \fi%
  \@esphack%
}
\newcommand{\setstops}{%
  \gdef\nstabbing@stops{%
    \checkoddpage
    \hspace*{-\ifoddpage\oddsidemargin\else\evensidemargin\fi}\hspace{-1in}%
    \hspace*{\zposx{stop-\thescore-1} sp}\=%
  }%
  \count@=\@ne
  \loop\ifnum\count@<\value{tabstop}%
    \begingroup\edef\x{\endgroup
      \noexpand\g@addto@macro\noexpand\nstabbing@stops{%
        \noexpand\hspace{-\noexpand\zposx{stop-\thescore-\the\count@} sp}%
        \noexpand\hspace{\noexpand\zposx{stop-\thescore-\the\numexpr\count@+1} sp}\noexpand\=%
      }%
    }\x
    \advance\count@\@ne
  \repeat
  \nstabbing@stops\kill
}
\makeatother

\newenvironment{nstabbing}
  {\setlength{\topsep}{0pt}%
   \setlength{\partopsep}{0pt}%
   \tabbing%
   \setstops}
  {\endtabbing\stepcounter{score}}

\def\beginpsalmcols{\begin{parcolumns}[rulebetween]{2}}
\def\endpsalmcols{\end{parcolumns}}

\newcommand{\deusinadjutorium}{\noindent{\Vbar~Deus in adjutórium, p.~\pageref{deusinadjutorium}}}
%\newcommand{\deusinadjutoriumsolemn}{\noindent\printnote{\Vbar~\emph{Deus in adjutórium}, p.~\pageref{deusinadjutoriumsolemn}.}}
\newcommand{\printcollect}[2]{
	\ifx\undefined\begincollectcols\def\begincollectcols{\begin{parcolumns}[rulebetween]{2}}\fi
	\ifx\printcollectheading\undefined\def\printcollectheading{T}\fi
	\if\printcollectheading T
		\oldneedspace{36pt}
		%\penalty -100% this is to try to replace the oldneedspace
		\medskip
		{\centering\large Collect.\par}
		%\smallskip
	\fi
	\ifx\precollect\undefined\else\precollect\fi
	\begincollectcols
	\sloppy
	\prayer{#1}{#2}
	\end{parcolumns}
	\let\begincollectcols=\undefined
}
\newcommand{\benedicamusdominoreference}[1]{%
	\ifthenelse{\boolean{includebenedicamusdominoreferences}}{
		\Vbar~\emph{Benedicámus Dómino \IfInteger{#1}{#1}{\csname benedicamusdominoname#1\endcsname}}, p.~\pageref{benedicamusdomino-#1}.
	}{}
}
\newcommand{\benedicamusdominoreferencelentoreaster}{%
	\ifthenelse{\boolean{includebenedicamusdominoreferences}}{
		In Lent, \Vbar~\emph{Benedicámus Dómino \benedicamusdominonamelent}, p.~\pageref{benedicamusdomino-lent}, or in Easter, \emph{\benedicamusdominonameeaster}, page \pageref{benedicamusdomino-easter}.
	}{}
}
\newcommand{\benedicamusdominomaster}[1]{
	\ifthenelse{\boolean{includebenedicamusdominoreferences}}{
		\noindent\printnote{\benedicamusdominoreference{#1}}
		\ifx\postbenedicamusdomino\undefined\else\postbenedicamusdomino\fi
	}{}
	\bigskip
	\hrule
}
\newcommand{\benedicamusdominolentoreaster}{
	\ifthenelse{\boolean{includebenedicamusdominoreferences}}{
		\noindent\printnote{\benedicamusdominoreferencelentoreaster}
		\ifx\postbenedicamusdomino\undefined\else\postbenedicamusdomino\fi
	}{}
	\bigskip
	\hrule
}

\ifthenelse{\boolean{lettersize}}{
	\newcommand{\psalmcolsoverride}[1][0]{
	}	
}{
	\newcommand{\psalmcolsoverride}[1][0]{
		\def\beginpsalmcols{\begin{parcolumns}[rulebetween,colwidths={1=0.45\linewidth}]{2}}
		\ifnum#1=110
		\def\beginpsalmcols{\begin{parcolumns}[rulebetween,colwidths={1=0.465\linewidth}]{2}}
		\fi
		\ifnum#1=111
		\def\beginpsalmcols{\begin{parcolumns}[rulebetween,colwidths={1=0.475\linewidth}]{2}}
		\fi
		%\ifnum#1=112
		%\def\beginpsalmcols{\begin{parcolumns}[rulebetween,colwidths={1=0.445\linewidth}]{2}}
		%\fi
		\ifnum#1=116
		\def\beginpsalmcols{\begin{parcolumns}[rulebetween,colwidths={1=0.5655\linewidth}]{2}}
		\fi
		\ifnum#1=121
		\def\beginpsalmcols{\begin{parcolumns}[rulebetween,colwidths={1=0.47\linewidth}]{2}}
		\fi
		\ifnum#1=125
		\def\beginpsalmcols{\begin{parcolumns}[rulebetween,colwidths={1=0.49\linewidth}]{2}}
		\fi
		\ifnum#1=129
		\def\beginpsalmcols{\begin{parcolumns}[rulebetween,colwidths={1=0.475\linewidth}]{2}}
		\fi
		\ifnum#1=131
		\def\beginpsalmcols{\begin{parcolumns}[rulebetween,colwidths={1=0.4875\linewidth},distance=1em]{2}}
		\fi
		\ifnum#1=137
		\def\beginpsalmcols{\begin{parcolumns}[rulebetween]{2}}
		\fi
		\ifnum#1=147
		%\def\beginpsalmcols{\begin{parcolumns}[rulebetween,colwidths={1=0.465\linewidth}]{2}}
		\def\beginpsalmcols{\begin{parcolumns}[rulebetween,colwidths={1=0.475\linewidth}]{2}}
		\fi
	}
}
\DeclareDocumentCommand{\printbothversions}{ O{\undefined} O{\undefined} m m }{
% #1 & #2 are the label
% #3 is what to check for
% #4 is the body of what to print
	\ifx#3\undefined
		\ifx\definevesperspropersalt\undefined\else
	  {
			\ifx\vesperspropersaltnote\undefined\else
		    \oldneedspace{3\baselineskip}
				\printnote{\vesperspropersaltnote}
			\fi
			\definevesperspropersalt
			\ifx#2\undefined\else\oldneedspace{5\baselineskip}\label{#2}\fi
	  	#4
		}
		\medskip
		\fi
		\ifx\definevesperspropers\undefined\else
	  {
			\ifx\vesperspropersnote\undefined\else
	    	\oldneedspace{3\baselineskip}
				\printnote{\vesperspropersnote}
			\fi
			\definevesperspropers
			\ifx#1\undefined\else\oldneedspace{5\baselineskip}\label{#1}\fi
	  	#4
		}
		\fi
	\else
  {
		\ifx#1\undefined\else\label{#1}\fi
  	#4
	}
	\fi
}
\let\printhymn=\undefined
\newcommand{\printpsalmtitle}[1]{%
  {\addfontfeature{Numbers=Lining}\centering Psalm #1\\}%
}
\newcommand{\printtercepsalms}[1]{
  % #1 is psalm tone like 2D
  \printpsalmtitle{118. III.}

  %print first verse in chant
  {
    \gresetinitiallines{0}
    \grechangedim{interwordspacetext}{0.15 cm plus 0.15 cm minus 0.10 cm}{scalable}%
    \gregorioscore{psalms/Psalm118.3-#1}
    % the additional width of the additional lines (compared to the width of the glyph they're associated with)
\grechangedim{additionallineswidth}{0.14584 cm}{scalable}%
% width of the additional lines, used only for the custos (maybe should depend on the width of the custos...)
% the width is the one for the custos at end of lines, the line for custos in the middle of a score is the same
% multiplied by 2.
\grechangedim{additionalcustoslineswidth}{0.09114 cm}{scalable}%
% null space
\grechangedim{zerowidthspace}{0 cm}{scalable}%
% space between glyphs in the same element
\grechangedim{interglyphspace}{0.06927 cm plus 0.00363 cm minus 0.00363 cm}{scalable}%
% space between an alteration (flat or natural) and the next glyph
\grechangedim{alterationspace}{0.07747 cm plus 0.01276 cm minus 0.00455 cm}{scalable}%
% space between a clef and a flat (for clefs with flat)
\grechangedim{clefflatspace}{0.05469 cm plus 0.00638 cm minus 0.00638 cm}{scalable}%
% space before a choral sign
\grechangedim{beforelowchoralsignspace}{0.04556 cm plus 0.00638 cm minus 0.00638 cm}{scalable}%
% when bolshifts are enabled, minimal space between a clef at the beginning of the line and a leading alteration glyph (should be larger than clefflatspace so that a flatted clef can be distinguished from a flat which is part of the first glyph on a line, but also smaller than spaceafterlineclef, the distance from the clef to the first notes)
\grechangedim{beforealterationspace}{0.1 cm}{scalable}%
% space between elements
\grechangedim{interelementspace}{0.06927 cm plus 0.00182 cm minus 0.00363 cm}{scalable}%
% larger space between elements
\grechangedim{largerspace}{0.10938 cm plus 0.01822 cm minus 0.00911 cm}{scalable}%
% space between elements in ancient notation
\grechangedim{nabcinterelementspace}{0.06927 cm plus 0.00182 cm minus 0.00363 cm}{scalable}%
% larger space between elements in ancient notation
\grechangedim{nabclargerspace}{0.10938 cm plus 0.01822 cm minus 0.00911 cm}{scalable}%
% space between elements which has the size of a note
\grechangedim{glyphspace}{0.21877 cm plus 0.01822 cm minus 0.01822 cm}{scalable}%
% space before custos
\grechangedim{spacebeforecustos}{0.1823 cm plus 0.31903 cm minus 0.0638 cm}{scalable}%
% space before punctum mora and augmentum duplex
\grechangedim{spacebeforesigns}{0.05469 cm plus 0.00455 cm minus 0.00455 cm}{scalable}%
% space after punctum mora and augmentum duplex
\grechangedim{spaceaftersigns}{0.08203 cm plus 0.0082 cm minus 0.0082 cm}{scalable}%
% space after a clef at the beginning of a line
\grechangedim{spaceafterlineclef}{0.27345 cm plus 0.14584 cm minus 0.01367 cm}{scalable}%
% minimal space between notes of different words
%\grechangedim{interwordspacenotes}{0.27 cm plus 0.15 cm minus 0.05 cm}{scalable}%
\grechangedim{interwordspacenotes}{0.27 cm plus 0.08 cm minus 0.05 cm}{scalable}%
% minimal space between notes of the same syllable.
% Warning: always keep minus to 0; also keep plus very low, or some words won't be hyphenated
%\grechangedim{intersyllablespacenotes}{0.24 cm plus 0.04cm minus 0cm}{scalable}%
\grechangedim{intersyllablespacenotes}{0.24 cm plus 0.04cm minus 0cm}{scalable}%
% minimal space between letters of different words. Makes sense to have
% the same plus and minus as interwordspacenotes.
%\grechangedim{interwordspacetext}{0.38 cm plus 0.15 cm minus 0.05 cm}{scalable}%
\grechangedim{interwordspacetext}{0.18 cm plus 0.08 cm minus 0.05 cm}{scalable}%
% Versions of interword spaces for euouae blocks
%\grechangedim{interwordspacenotes@euouae}{0.19 cm plus 0.1 cm minus 0.05 cm}{scalable}%
\grechangedim{interwordspacenotes@euouae}{0.13 cm plus 0.1 cm minus 0.05 cm}{1}%
%\grechangedim{interwordspacetext@euouae}{0.27 cm plus 0.1 cm minus 0.05 cm}{scalable}%
\grechangedim{interwordspacetext@euouae}{0.13 cm plus 0.1 cm minus 0.05 cm}{1}%
% space between notes of a bivirga or trivirga
\grechangedim{bitrivirspace}{0.06927 cm plus 0.00182 cm minus 0.00546 cm}{scalable}%
% space between notes of a bistropha or tristrophae
\grechangedim{bitristrospace}{0.06927 cm plus 0.00182 cm minus 0.00546 cm}{scalable}%
% space between two punctum inclinatum
\grechangedim{punctuminclinatumshift}{-0.03918 cm plus 0.0009 cm minus 0.0009 cm}{scalable}%
% space before puncta inclinata
\grechangedim{beforepunctainclinatashift}{0.05286 cm plus 0.00728 cm minus 0.00455 cm}{scalable}%
% space between a punctum inclinatum and a punctum inclinatum deminutus
\grechangedim{punctuminclinatumanddebilisshift}{-0.02278 cm plus 0.0009 cm minus 0.0009 cm}{scalable}%
% space between two punctum inclinatum deminutus
\grechangedim{punctuminclinatumdebilisshift}{-0.00728 cm plus 0.0009 cm minus 0.0009 cm}{scalable}%
% space between puncta inclinata, larger ambitus (range=3rd)
\grechangedim{punctuminclinatumbigshift}{0.07565 cm plus 0.0009 cm minus 0.0009 cm}{scalable}%
% space between puncta inclinata, larger ambitus (range=4th -or more?-)
\grechangedim{punctuminclinatummaxshift}{0.17865 cm plus 0.0009 cm minus 0.0009 cm}{scalable}%
% space for the bars (inside syllables)
%first for virgula and divisio minima
\grechangedim{spacearoundsmallbar}{0.1823 cm plus 0.22787 cm minus 0.00469 cm}{scalable}%
%then divisio minor
\grechangedim{spacearoundminor}{0.1823 cm plus 0.22787 cm minus 0.00469 cm}{scalable}%
%divisio major
\grechangedim{spacearoundmaior}{0.1823 cm plus 0.22787 cm minus 0.00469 cm}{scalable}%
%divisio finalis
\grechangedim{spacearoundfinalis}{0.1823 cm plus 0.22787 cm minus 0.00469 cm}{scalable}%
%a special space for finalis, for when it is the last glyph
\grechangedim{spacebeforefinalfinalis}{0.29169 cm plus 0.07292 cm minus 0.27345 cm}{scalable}%
% additional space that will appear around bars that are preceded by a custos and followed by a key.
\grechangedim{spacearoundclefbars}{0.03645 cm plus 0.00455 cm minus 0.0009 cm}{scalable}%
% space between the text and the text of the bar
\grechangedim{textbartextspace}{0.24611 cm plus 0.13672 cm minus 0.04921 cm}{scalable}%
% minimal space between a note and a bar
\grechangedim{notebarspace}{0.31903 cm plus 0.27345 cm minus 0.02824 cm}{scalable}%
% maximal space between two syllables for which we consider a dash is not needed
\grechangedim{maximumspacewithoutdash}{0.00 cm}{scalable}%
% an extensible space for the beginning of lines
\grechangedim{afterclefnospace}{0 cm plus 0.27345 cm minus 0 cm}{scalable}%
% space between the initial and the beginning of the score
\grechangedim{afterinitialshift}{0.2457 cm}{scalable}%
% space before the initial
\grechangedim{beforeinitialshift}{0.2457 cm}{scalable}%
% when bolshifts are enabled, minimum space between beginning of line and first syllable text
\grechangedim{minimalspaceatlinebeginning}{0.05 cm}{scalable}%
% space to force the initial width to.  Ignored when 0.
\grechangedim{manualinitialwidth}{0 cm}{scalable}%
% distance to move the initial up by
\grechangedim{initialraise}{0 cm}{scalable}%
% Space between lines in the annotation
\grechangedim{annotationseparation}{0.05cm}{scalable}%
% Amount to raise (positive) or lower (negative) the annotations from the default position (base line of top annotation aligned with top line of staff)
\grechangedim{annotationraise}{0cm}{scalable}%
% space at the beginning of the lines if there is no clef
\grechangedim{noclefspace}{0.1 cm}{scalable}%
% space around a clef change
\grechangedim{clefchangespace}{0.01768 cm plus 0.00175 cm minus 0.01768 cm}{scalable}%
%When \gre@clivisalignment is 2, this distance is the maximum length of the consonants after vowels for which the clivis will be aligned on its center.
\grechangedim{clivisalignmentmin}{0.3 cm}{scalable}%



%%%%%%%%%%%%%%%%%%
% vertical spaces
%%%%%%%%%%%%%%%%%%

% first, we have two spaces for the chironomic signs
\grechangedim{abovesignsspace}{0.8 cm}{scalable}%
\grechangedim{belowsignsspace}{0 cm}{scalable}%
% the amount to shift down:
% (a) low choral signs that are not lower than the note, regardless of whether
%     it's on a line or in a space
% (b) high choral signs and low choral signs that are lower than the note which
%     are in a space
\grechangedim{choralsigndownshift}{0.00911 cm}{scalable}%
% the amount to shift up:
% (a) high choral signs and low choral signs that are lower than the note which
%     are on a line
\grechangedim{choralsignupshift}{0.04556 cm}{scalable}%
% the space for the translation
\grechangedim{translationheight}{0.5 cm}{scalable}%
%the space above the lines
\grechangedim{spaceabovelines}{0.45576 cm plus 0.36461 cm minus 0.09114 cm}{scalable}%
%the space between the lines and the bottom of the text
\grechangedim{spacelinestext}{0.60617 cm}{scalable}%
%the space beneath the text
\grechangedim{spacebeneathtext}{0 cm}{scalable}%
% height of the text above the note line
\grechangedim{abovelinestextraise}{-0.1 cm}{scalable}%
% height that is added at the top of the lines if there is text above the lines (it must be bigger than the text for it to be taken into consideration)
\grechangedim{abovelinestextheight}{0.3 cm}{scalable}%
% an additional shift you can give to the brace above the bars if you don't like it
\grechangedim{braceshift}{0 cm}{scalable}%
% a shift you can give to the accentus above the curly brace
\grechangedim{curlybraceaccentusshift}{-0.05 cm}{scalable}%


    \greseteolcustos{auto}
  }
  \vspace{0pt plus 4pt minus 8pt}
  \setlength\parsep{0pt}%
  \setlength\topsep{0pt}%
  \setlength\partopsep{0pt}%
  \setlength\multicolsep{0pt}%
  \setlength{\columnsep}{18pt}
  \setlength{\columnseprule}{.4pt}
  \selectlanguage{latin}
  \begin{multicols}{2}
  \begin{psalmverses}[1]
  \input{psalms/Psalm118.3-#1-verses}
  \end{psalmverses}
  %\medskip
  \printpsalmtitle{118. IV.}
  \begin{psalmverses}
  \input{psalms/Psalm118.4-#1-verses}
  \end{psalmverses}
  %\medskip
  \printpsalmtitle{118. V.}
  \begin{psalmverses}
  \input{psalms/Psalm118.5-#1-verses}
  \end{psalmverses}
  \setlength{\parindent}{6ex}
  \ifx\printrepeatantiphon\undefined
	  \IfStrEq{#1}{no-tone}{}{%if not no-tone

	    \emph{Repeat antiphon.}%
	  }
	\else

		\printrepeatantiphon
	\fi
  \end{multicols}
}
\newcommand{\printtercepsalmstranslation}{
  \setlength\parsep{0pt}%
  \setlength\topsep{0pt}%
  \setlength\partopsep{0pt}%
  \setlength\multicolsep{0pt}%
  \setlength{\columnsep}{18pt}
  \setlength{\columnseprule}{.4pt}
  \selectlanguage{american}
  \begin{multicols}{2}
  \printpsalmtitle{118. III.}
  \begin{psalmverses}
  \emph{\item Set before me for a law the way of thy justifications, O Lord: * and I will always seek after it.
\item Give me understanding, and I will search thy law; * and I will keep it with my whole heart.
\item Lead me into the path of thy commandments; * for this same I have desired.
\item Incline my heart into thy testimonies * and not to covetousness.
\item Turn away my eyes that they may not behold vanity: * quicken me in thy way.
\item Establish thy word to thy servant, * in thy fear.
\item Turn away my reproach, which I have apprehended: * for thy judgments are delightful.
\item Behold I have longed after thy precepts: * quicken me in thy justice.
\item Let thy mercy also come upon me, O Lord: * thy salvation according to thy word.
\item So shall I answer them that reproach me in any thing; * that I have trusted in thy words.
\item And take not thou the word of truth utterly out of my mouth: * for in thy words, I have hoped exceedingly.
\item So shall I always keep thy law, * for ever and ever.
\item And I walked at large: * because I have sought after thy commandments.
\item And I spoke of thy testimonies before kings: * and I was not ashamed.
\item I meditated also on thy commandments, * which I loved.
\item And I lifted up my hands to thy commandments, which I loved: * and I was exercised in thy justifications.}
  \end{psalmverses}
  \medskip
  \printpsalmtitle{118. IV.}
  \begin{psalmverses}
  \emph{\item Be thou mindful of thy word to thy servant, * in which thou hast given me hope.
\item This hath comforted me in my humiliation: * because thy word hath enlivened me.
\item The proud did iniquitously altogether: * but I declined not from thy law.
\item I remembered, O Lord, thy judgments of old: * and I was comforted.
\item A fainting hath taken hold of me, * because of the wicked that forsake thy law.
\item Thy justifications were the subject of my song, * in the place of my pilgrimage.
\item In the night I have remembered thy name, O Lord: * and have kept thy law.
\item This happened to me: * because I sought after thy justifications.
\item O Lord, my portion, * I have said, I would keep thy law.
\item I entreated thy face with all my heart: * have mercy on me according to thy word.
\item I have thought on my ways: * and turned my feet unto thy testimonies.
\item I am ready, and am not troubled: * that I may keep thy commandments.
\item The cords of the wicked have encompassed me: * but I have not forgotten thy law.
\item I rose at midnight to give praise to thee; * for the judgments of thy justification.
\item I am a partaker with all them that fear thee, * and that keep thy commandments.
\item The earth, O Lord, is full of thy mercy: * teach me thy justifications.}
  \end{psalmverses}
  \medskip
  \printpsalmtitle{118. V.}
  \begin{psalmverses}
  \emph{\item Thou hast done well with thy servant, O Lord, * according to thy word.
\item Teach me goodness and discipline and knowledge; * for I have believed thy commandments.
\item Before I was humbled I offended; * therefore have I kept thy word.
\item Thou art good; * and in thy goodness teach me thy justifications.
\item The iniquity of the proud hath been multiplied over me: * but I will seek thy commandments with my whole heart.
\item Their heart is curdled like milk: * but I have meditated on thy law.
\item It is good for me that thou hast humbled me, * that I may learn thy justifications.
\item The law of thy mouth is good to me, * above thousands of gold and silver.
\item Thy hands have made me and formed me: * give me understanding, and I will learn thy commandments.
\item They that fear thee shall see me, and shall be glad: * because I have greatly hoped in thy words.
\item I know, O Lord, that thy judgments are equity: * and in thy truth thou hast humbled me.
\item O! let thy mercy be for my comfort, * according to thy word unto thy servant.
\item Let thy tender mercies come unto me, and I shall live: * for thy law is my meditation.
\item Let the proud be ashamed, because they have done unjustly towards me: * but I will be employed in thy commandments.
\item Let them that fear thee turn to me: * and they that know thy testimonies.
\item Let my heart be undefiled in thy justifications, * that I may not be confounded.}
  \end{psalmverses}
  \end{multicols}
}
\def\printbenedicamusdominoref{%
  \noindent{\Vbar{}~Benedicámus Dómino, p.~\pageref{benedicamus-domino-sunday}.}
}
\def\printhymnsundayrefnospace{%
  \emph{Hymn 1}, p.~\pageref{hymn-sunday}%
}
\def\printhymnsundayref{%
  , \printhymnsundayrefnospace{}.

  \medskip
}
\def\printhymnfeastref{%
  , \emph{Hymn 2}, p.~\pageref{hymn-feast}.

  \medskip
}
\def\printhymnbvmref{%
  , \emph{Hymn 3}, p.~\pageref{hymn-bvm}.

  \medskip
}
\let\printhymn=\printhymnsundayref
\newcommand{\printterce}[3][.]{
	{
		\grechangestaffsize{15}
		\ifx\gabcfolder\undefined
			\def\gabcfolder{#1}
			\input{\gabcfolder/#2}
		\fi

		\deusinadjutorium{}%
		\ifx\printhymn\undefined%
		\else%
		  \printhymn
		\fi

		\ifx\anttwotex\undefined\else%
			\setlabel{#3}{ant}
			\printgabc{Ant.}{\anttwolinetwo}{\anttwoinitial}{\anttwotex .noEuouae}

			\ifx\anttwotranslation\undefined\else%
				\translation[]{\anttwotranslation}
			\fi

			\printtercepsalms{\anttwotoneendexpanded}
		\fi
		\ifx\chaptertext\undefined\else
			\ifx\prechapter\undefined\else%
				\prechapter
			\fi
			\setlabel{#3}{chapter}
			\printchapter{\chaptertext}{\chaptertranslation}

			\medskip
		\fi

		\ifx\printshortresp\undefined%
			\noindent\emph{Short Resp. \emph{Inclina cor meum}}, p. \pageref{shortresp-sunday}, \Vbar{}~Ego dixi, p. \pageref{vr-sunday}.
		\else
			\printshortresp
		\fi

		\ifx\collect\undefined%
			\let\collect=\latincollect
			\let\collecttranslation=\englishcollect
		\fi
		\setlabel{#3}{collect}
		\printcollect{\collect}{\collecttranslation}

		\medskip

		\ifx\printbenedicamusdominoref\undefined\else\printbenedicamusdominoref\fi
	}
}

\sloppy
%\nofiles
\begin{document}
\normalsize
\grechangestaffsize{15}
{
	\thispagestyle{empty}
	% title page
	% general
	\vspace*{5\baselineskip}

	{\centering

	{\Huge
	Terce with Gregorian Chant

	}

	{\Large\medskip
	\emph{for}

	\medskip}

	{\Huge
	Sundays \& Holy Days

	}
	\vfill}
	\pagebreak
	% copyright page
	\thispagestyle{empty}
	\noindent{}Terce with Gregorian Chant for Sundays \& Holy Days: \emph{newly typeset, based on \emph{The Liber Usualis}, edited by the Benedictines of Solesmes (Desclee Company, 1961).}

	\bigskip{}
	\noindent{}%
	Elie Roux's Gregorio (https://gregorio-project.github.io), Olivier Berton's GregoBase (http://gregobase.selapa.net), and my brother Benjamin's chant tools (http://bbloomf.github.io/jgabc) were indispensible in the creation of this book.

	\begin{flushright}
	\emph{Albert Bloomfield}\\
	Cincinnati, Ohio
	\end{flushright}

	\bigskip{}\noindent{}http://asbloomf.github.io/gabc-chants

	\vfill
%editions
	\bigskip{}\noindent{}%
	First edition, 13 July 2021.

	\hangindent=1em % indent all subsequent lines
    \bigskip

\makeatletter
	\noindent{}Typeset using \LuaLaTeX{} and Gregorio version \gre@gregoriotexversion{}
\makeatother

	\bigskip{}\noindent{}This work is free of known copyright restrictions.

	\bigskip{}\noindent{}%CreateSpace
	ISBN: \isbn{}
	%
  %\noindent{}Lulu ISBN: 978-1-329-59990-1
}
%\frontmatter
%\pagenumbering{roman}

\tableofcontents%

%\pagenumbering{arabic}
%\mainmatter

% redefine the  label command
\newcommand{\setlabel}[2]{
% #1 label part 1 (like easter2)
% #2 label part 2 (any of these 3: ant/chapter/collect)
	\label{#1-#2}% now execute the original label command
}

%\include{inc-common-terce}
\ifthenelse{\boolean{testrun}}{
	\newcommand{\printpsalmwithtranslation}[3]{
	% #1 psalm number
	% #2 ending
	% #3 0 to start on 1st verse, 1 for second
	\setlength{\columnsep}{18pt}
	\setlength{\columnseprule}{.4pt}
	\beginpsalmcols
	\colchunk{\vspace{-12pt}%
	\begin{psalmverses}[#3]
	\vspace{-\baselineskip}%
	\input{psalms/Psalm#1-#2-verses}
	\end{psalmverses}
	}

	\selectlanguage{american}
	\colchunk{\vspace{-12pt}%
	\sloppy
	\begin{psalmverses}
	\vspace{-\baselineskip}%
	\input{psalms/Psalm#1-verses-en}
	\end{psalmverses}
	}
	\selectlanguage{latin}
	\endpsalmcols
}

\renewcommand{\printpsalm}[3]{
  % #1 psalm number
  % #2 ending
  % #3 0 to start on 1st verse, 1 for second
  \setlength{\columnsep}{18pt}
  \setlength{\columnseprule}{.4pt}
  \setlength\multicolsep{0pt}%
  \begin{multicols*}{2}
  \begin{psalmverses}[#3]
  \input{psalms/Psalm#1-#2-verses}
  \end{psalmverses}
  \end{multicols*}
}



\newcommand{\printpsalmtranslation}[3]{
  % #1 psalm number
  % #2 ending
  % #3 0 to start on 1st verse, 1 for second
  \setlength{\columnsep}{18pt}
  \setlength{\columnseprule}{.4pt}
  \setlength\multicolsep{0pt}%
  \begin{multicols*}{2}
  \begin{psalmverses}[#3]
  \input{psalms/Psalm#1-verses-en}
  \end{psalmverses}
  \end{multicols*}
}

\def\beginchaptercols{\begin{parcolumns}[rulebetween]{2}}
\newcommand{\printchapternew}[4]{
	% #1 ref
	% #2 latin first words
	% #3 latin remaining
	% #4 english text
	{\noindent\hspace{3em}Chapter.\hfill\emph{#1}\hspace{3em}}
    \beginchaptercols{}
    \colchunk{\sloppy\dropcap{latin}{#2} #3\hfill{}\Rbar{}~Deo~grátias.}
    \colchunk{\sloppy \dropcap{american}{#4}}
    \end{parcolumns}
}
\newcommand{\printvrwithtranslation}{
    {\normalsize
    \ifx\beginvrcols\undefined\def\beginvrcols{\begin{parcolumns}[rulebetween]{2}}\fi
    \beginvrcols
    \colchunk{
      \par\vspace{-\baselineskip}\noindent\selectlanguage{latin}%
      \Vbar{}~\vlatin{}
    }
    \colchunk{%
      \par\vspace{-\baselineskip}\noindent\selectlanguage{american}%
      \Vbar{}~\vtranslation{}
    }%
    \colplacechunks%
    \ifx\rlatin\undefined\else
    \colchunk{
      \par\vspace{-\baselineskip}\noindent\selectlanguage{latin}%
      \Rbar{}~\rlatin{}%
    }
    \colchunk{%
      \par\vspace{-\baselineskip}\noindent\selectlanguage{american}%
      \Rbar{}~\rtranslation{}%
    }%
    \colplacechunks%
    \fi
    \end{parcolumns}
    }
}
{
  \label{sunday}
  \chapter{Sunday at Terce}
  \phantomsection
  \sectionmark{Sunday at Terce throughout the year}
  \addcontentsline{toc}{section}{Sunday at Terce throughout the year}

  {\centering Oratio ante officium.\par}
  {\centering \emph{Aperi. Pater. Ave.}\par}
  \begin{columns}
  \colchunk{\selectlanguage{latin}Aperi, Domine, os meum ad benedicendum nomen
sanctum tuum: munda quoque cor meum ab omnibus
vanis, perversis et alienis cogitationibus; intellectum
illumina, affectum inflamma, ut digne, attente ac
devote hoc Officium recitare valeam, et exaudiri
merear ante conspectum divinae Majestatis tuae. Per
Christum Dóminum nostrum. \Rbar{}~Amen.}
  \colchunk{\selectlanguage{american}Open, O Lord, my mouth to bless Thy holy name;
cleanse my heart from all vain, evil, and
wandering thoughts; enlighten my understanding
and kindle my affections; that I may worthily,
attentively, and devoutly say this Office, and so
deserve to be heard before the presence of Thy
divine Majesty. Through Christ our Lord. \Rbar{}~Amen.}
  \colplacechunks{}
  \colchunk{\selectlanguage{latin}Domine, in unione illius divinae intentionis, qua ipse
in terries laudes Deo persolvisti, hanc tibi
Horam persolvo.}
  \colchunk{\selectlanguage{american}O Lord, in union with that divine intention, with
which Thou didst praise God upon earth, I render
this Hour to Thee.}
  \colplacechunks{}
  \end{columns}
  \medskip

  %\oldneedspace{6\baselineskip}
  \emph{The Ferial Tone is used except before Pontifical Mass.}

  \printgabc{\Vbar}{}{D}{deus-in-adjutorium}\label{deusinadjutorium}
  \medskip

  %\oldneedspace{6\baselineskip}
  \emph{The Festal Tone is used only before Pontifical Mass.}

  {\def\gabcfolder{..}
    \printgabc{\Vbar}{}{D}{DeusInAdjutorium_laustibi}\label{deusinadjutorium-festal}
  }

  \medskip\vfill
  {\centering Hymn.\par}%
  %\vspace{-0.5\baselineskip}
  \def\dotting{\leaders\hbox to 1em{\hfil.\hfil}\hfill}
  %\def\dotting{\hfill}
  \setlength\parsep{0pt}%
  \setlength\topsep{0pt}%
  \setlength\partopsep{0pt}%
  \setlength\multicolsep{0pt}%
  \begin{multicols}{2}%
  \noindent{}\emph{In Advent},\dotting p.~\pageref{hymn-advent}\\
  \emph{Christmas},\dotting p.~\pageref{hymn-christmas}\\
  \emph{Epiphany},\dotting p.~\pageref{hymn-epiphany}\\
  \emph{Holy Family},\dotting p.~\pageref{hymn-holyfamily}\\
  \emph{In Lent},\dotting p.~\pageref{hymn-lent}\\
  \emph{In Passiontide},\dotting p.~\pageref{hymn-passiontide}\\
  \emph{In Paschaltide},\dotting p.~\pageref{hymn-paschaltide}\\
  \emph{Ascension},\dotting p.~\pageref{hymn-ascension}\\
  \emph{Pentecost},\dotting p.~\pageref{hymn-pentecost}\\
  \emph{Christ the King},\dotting p.~\pageref{hymn-oct-last-sunday}%\\
  %\emph{Corpus Christi},\dotting p.~\pageref{hymn-corpus-christi}\\
  %\emph{Sacred Heart},\dotting p.~\pageref{hymn-sacred-heart}
  \end{multicols}%

  \bigskip
  %\oldneedspace{4\baselineskip}
  {\centering 1.~On Ordinary Sundays.\par}\label{hymn-sunday}

  \printgabc{Hymn.}{2.}{N}{hy--nunc_sancte_nobis_(in_dominicis_per_annum)--solesmes_1961}
  \bigskip

  \oldneedspace{4\baselineskip}
  {\centering 2.~On Solemn Feasts.\par}\label{hymn-feast}

  \printgabc{Hymn.}{8.}{N}{hy--nunc_sancte_nobis_(in_festis)--solesmes_1961}
  \bigskip

  \oldneedspace{4\baselineskip}\vfill
  {\centering 3.~On Feasts of the Blessed Virgin Mary.\par}\label{hymn-bvm}

  \printgabc{Hymn.}{2.}{N}{hy--nunc_sancte_nobis_(in_festis_bmv)--solesmes_1961}

  \begin{multicols}{2}
  \begin{psalmverses}
\item Come Holy Ghost, who ever One
Art with the Father and the Son,
It is the hour, our souls possess
With thy full flood of holiness.

\item Let flesh and heart and lips and mind
Sound forth our witness to mankind;
And love light up our mortal frame,
Till others catch the living flame.

\item Almighty Father, hear our cry,
Through Jesus Christ, our Lord most High,
Who, with the Holy Ghost and thee,
Doth live and reign eternally.
Amen.
  \end{psalmverses}
  \end{multicols}

\bigskip

% print antiphon
%    \printgabc[\preant]{\antlineone}{\antlinetwo}{\antinitial}{\anttex}
  \printgabc{Ant.}{2. D}{A}{an--alleluia._(sund._at_terce)--solesmes}
% print translation
  \translation[]{Alleluia, Lead me into the path of thy commandments, alleluia, alleluia.}

  { \def\printrepeatantiphon{\emph{Repeat antiphon, \emph{p.~\pageref{ant2-sunday}.}}}
    \printtercepsalms{2D}%
  }

  \printtercepsalmstranslation{}

% \printpsalm{118.3}{2D}{1}
% \printpsalmtitle{118. IV.}
% \printpsalm{118.4}{2D}{0}
% \printpsalmtitle{118. V.}
% \printpsalm{118.5}{2D}{0}

% print antiphon
%    \printgabc[\preant]{\antlineone}{\antlinetwo}{\antinitial}{\anttex}
  \bigskip\oldneedspace{8\baselineskip}\label{ant2-sunday}
  \printgabc{Ant.}{2. D}{A}{an--alleluia._(sund._at_terce)--solesmes}
% print translation
  %\translation[]{Alleluia, Lead me into the path of thy commandments, alleluia, alleluia.}

\medskip\oldneedspace{6\baselineskip}
\printchapternew{1. John 4.}{Deus cáritas}{est~:~\dag{} et qui manet in caritáte, in Deo manet,~* et Deus in eo.}{God is charity: and he that abideth in charity, abideth in God, and God in him.}

\medskip\label{shortresp-sunday}
\printgabc{Short}{Resp.}{I}{re--inclina_cor_meum--solesmes}

\translation[]{\Vbar{}~Incline my heart into thy testimonies.
\Rbar{}~Incline\dots{}
\Vbar{}~Turn away my eyes that they may not behold vanity: quicken me in thy way.
\Rbar{}~Into thy testimonies.
\Vbar{}~Glory be\dots{}
\Vbar{}~Incline\dots{}}

\bigskip\label{vr-sunday}
\gresetinitiallines{0}
\gregorioscore{vr-ego-dixi}
\newlength{\myhwidth}
\settowidth{\myhwidth}{tib}
\begin{nstabbing}
\>\Rbar{}~Sana ánimam meam, quia peccávi \>\hspace{-\myhwidth}tibi.
\end{nstabbing}

\translation[]{\Vbar{}~I said: O Lord, be thou merciful to me.\\
\Rbar{}~Heal my soul, for I have sinned against thee.}

\bigskip
%\def\beginvrcols{\begin{parcolumns}[rulebetween,colwidths={1=0.48\linewidth}]{2}}
\def\vlatin{Dóminus vobíscum.}
\def\rlatin{Et cum spíritu tuo.}
\def\vtranslation{The Lord be with you.}
\def\rtranslation{And with thy spirit.}
{\centering Collect.\par}
\printvrwithtranslation{}

\bigskip
    %\def\dotting{\leaders\hbox to 1em{\hfil.\hfil}\hfill}
    \def\dotting{\hfill}
    \begin{multicols}{2}%
    %\printnote{Advent,\dotting \emph{p.~\pageref{advent1-collect} to \pageref{advent4-collect}}}

    %\printnote{Christmas,\dotting \emph{p.~\pageref{christmas-eve-collect} to \pageref{holyfamily-collect}}}

    \noindent\emph{Time after Epiphany},\dotting p.~\pageref{epiphany2-collect} to \pageref{epiphany6-collect}\\
    \emph{Septuagesima},\dotting p.~\pageref{septuagesima-collect} to \pageref{quinquagesima-collect}\\
    %\emph{Easter to Pentecost},\dotting p.~\pageref{easter1-collect} to \pageref{pentecost-collect}\\
    \emph{Time after Pentecost},\dotting p.~\pageref{pentecost2-collect} to \pageref{pentecost24-collect}
    \end{multicols}

\bigskip
\printvrwithtranslation{}
\bigskip

\printnote{Except before Pontifical Mass, the Officiant chants the \emph{\Vbar{}~Benedicámus Dómino} to the following tone.}
\medskip
\gresetinitiallines{0}\label{benedicamus-domino-sunday}
\gregorioscore{vr-benedicamus-domino}
\bigskip

%\def\beginvrcols{\begin{parcolumns}[rulebetween,colwidths={1=0.44\linewidth}]{2}}
\def\vlatin{Fidélium ánimæ per misericórdiam Dei requiéscant in pace.\hfill{}Amen.}
\let\rlatin=\undefined
\def\vtranslation{May the souls of the faithful departed through the mercy of God rest in peace.\hfill{}Amen.}
\let\rtranslation=\undefined
\printvrwithtranslation{}

\bigskip
\printnote{If Pontifical Mass is to follow, the \emph{\Vbar{}~Benedicámus Dómino} is sung by the Cantors in one of the following tones; and the \emph{\Vbar{}~Fidelium} is not said.}
\medskip
{
\newcommand{\printbenedicamusdomino}[2]{
	\greseteolcustos{manual}
	\gresetinitiallines{1}
	\def\annot{\small{#1}}
	\alsetinitialspacing{B}
	\gregorioscore{#2}
  \greseteolcustos{auto}
}

\vfil
\grechangestaffsize{15}
\oldneedspace{4\baselineskip}\label{benedicamusdomino-1}
{{\centering \bfseries 1.~On feasts of the I class.\\}
\smallskip
\def\breakbeforeresp{T}
\printbenedicamusdomino{5.}{../BenedicamusDomino/or--benedicamus_(in_festis_i_classis_ad_laudes)--solesmes}
\vfil
}
\oldneedspace{4\baselineskip}\label{benedicamusdomino-2}
{{\centering \bfseries 2.~On feasts of the II class.\\}
\smallskip
{\def\breakbeforeresp{T}
\printbenedicamusdomino{2.}{../BenedicamusDomino/or--benedicamus_domino_laudes_--solesmes_1961}
}
}
\vfil

\oldneedspace{4\baselineskip}\label{benedicamusdomino-mary}
\gdef\benedicamusdominonamemary{3}
{{\centering \bfseries 3.~On feasts of the Blessed Virgin.\\}
\smallskip
%\def\breakbeforeresp{T}
\printbenedicamusdomino{1.}{../BenedicamusDomino/BenedicamusDomino_blessedVirgin}}
\vfil

\needspace{6\baselineskip}
\label{benedicamusdomino-sunday}
\gdef\benedicamusdominonamesunday{4}
{{\centering \bfseries 4.~On Sundays during the Year\\and Septuagesima, Sexagesima, and Quinquagesima.\\}
\smallskip
%\def\breakbeforeresp{T}
\printbenedicamusdomino{1.}{../BenedicamusDomino/BenedicamusDomino_Sundays}}

\vfil
\label{benedicamusdomino-lent}\label{benedicamusdomino-advent}
\gdef\benedicamusdominonamelent{5}
\gdef\benedicamusdominonameadvent{\benedicamusdominonamelent}
{{\centering \bfseries 5.~On Sundays of Advent and Lent.\\}
\smallskip
%\def\breakbeforeresp{T}
\printbenedicamusdomino{6.}{../BenedicamusDomino/BenedicamusDomino_SundaysOfAdventAndLent}
\ifthenelse{\boolean{birmingham}}{
	\medskip
	\emph{\normalsize or from \emph{Mass XVII}:}

	\smallskip
	\printbenedicamusdomino{6.}{../BenedicamusDomino/ky--benedicamus_xviia--solesmes}
}{}
}
\vfil

\label{benedicamusdomino-easter}
\gdef\benedicamusdominonameeaster{6}
{{\centering \bfseries 6.~On Sundays of Paschal Time.\\}
\smallskip
%\def\breakbeforeresp{T}
\printbenedicamusdomino{7.}{../BenedicamusDomino/BenedicamusDomino_SundaysOfPaschalTime}}

}

}

	\chapter{Proper of the Time -- Time After Pentecost}
{
\def\printcommonvespers{
  \vspace{-0.5\baselineskip}
  \subtitle{\nth{2} Class, Green}
  \medskip
  \deusinadjutorium{}
  \hfill
  \emph{\emph{Vespers of Sundays throughout the year,} p.~\pageref{sundayvespers}.\par}
}
\newcommand{\benedicamusdomino}[1][sunday]{
  \benedicamusdominomaster{#1}
}
\def\premagtitle{
  \needspace{8\baselineskip}  
}

\newcommand{\printhymnnote}{}
\newcommand{\printvespersafterpentecost}[1]{
  {
  \section{\nth{#1} \ifnum#1=24{or Last }\fi Sunday after Pentecost}
  \label{pentecost#1}
\printcommonvespers{}
\ifx\printaftercommonvespers\undefined\else\printaftercommonvespers{}\fi
  \ifx\nocommemoration\undefined%
    \def\precollect{\printvrdirigatur\smallskip}%
  \fi
  \ifx\postmagtitle\undefined\def\oldpostmagtitle{}\else\let\oldpostmagtitle=\postmagtitle\fi
  \def\postmagtitle{\oldpostmagtitle\label{pentecost#1-mag}}
  \printvespersmag[../TimeAfterPentecost]{inc-VespersMagnificatPentecost#1}
  \smallskip
  \benedicamusdomino{}
  }
  
}


\ifthenelse{\boolean{includecorpuschristi}}{% true
{
  \section{Corpus Christi}

  \def\definevesperspropers{\newcommand{\maganttex}{an--o_sacrum_convivium--solesmes}
\newcommand{\magantinitial}{O}
\newcommand{\maganttranslation}{O sacred banquet, in which Christ is received; the memory of His Passion is renewed; the mind is filled with grace; and a pledge of future glory is given to us, alleluia.}
\newcommand{\magsolemn}{T}
\definemag{5}{a}
}

  \printvespers[../CorpusChristi]{inc-CorpusChristi}
  \medskip
  \benedicamusdomino{}
}
}{}% false

{\def\begincollectcols{\begin{parcolumns}[rulebetween,colwidths={1=0.455\linewidth}]{2}}
\printvespersafterpentecost{2}}
\ifthenelse{\boolean{includesacredheart}}{% true
{
  \section{Sacred Heart of Jesus}

  \def\definevesperspropers{\newcommand{\antonetex}{an--unus_militum--solesmes}
\newcommand{\antoneinitial}{U}
\newcommand{\antonetranslation}{A soldier with a spear opened his side, and immediately there came forth blood and water.}
\definepsalm{1}{109}{1}{f}

\newcommand{\anttwotex}{an--stans_jesus--solesmes}
\newcommand{\anttwoinitial}{S}
\newcommand{\anttwotranslation}{Jesus stood and cried, saying~: If any man thirst, let him come to me and drink.}
\definepsalm{2}{110}{7}{c}

\newcommand{\antthreetex}{an--in_caritate--solesmes}
\newcommand{\antthreeinitial}{I}
\newcommand{\antthreetranslation}{With an everlasting love has God loved us; lifted up, therefore, from the earth, he has drawn us to his Heart, taking pity on us.}
\definepsalm{3}{115}{3}{a2}

\newcommand{\antfourtex}{an--venite_ad_me--solesmes}
\newcommand{\antfourinitial}{V}
\newcommand{\antfourtranslation}{Come to me, all you that labour and are burdened~: and I will refresh you.}
\definepsalm{4}{127}{4}{E}

\newcommand{\antfivetex}{an--fili_praebe--solesmes}
\newcommand{\antfiveinitial}{F}
\newcommand{\antfivetranslation}{My son, give me thy heart~: and let thine eyes keep my ways.}
\definepsalm{5}{147}{5}{a}

\newcommand{\maganttex}{an--ad_jesum_autem--solesmes}
\newcommand{\magantinitial}{A}
\newcommand{\maganttranslation}{But after they were come to Jesus, when they saw that he was already dead, they did not break his legs.  But one of the soldiers with a spear opened his side~: and immediately there came out blood and water.}
\newcommand{\magsolemn}{T}
\definemag{1}{f}

\newcommand{\vrtex}{vrHaurietis}
\newcommand{\vtranslation}{Ye shall draw waters with joy.}
\newcommand{\rtranslation}{Out of the Saviour's fountains.}

}

  \printvespers[../SacredHeart]{inc-SacredHeart}
  \medskip
  \benedicamusdomino{}
}
}{}% false
\ifthenelse{\boolean{testrun}}{}{
\printvespersafterpentecost{3}
\printvespersafterpentecost{4}
{\def\begincollectcols{\begin{parcolumns}[rulebetween,colwidths={1=0.44\linewidth}]{2}}
\printvespersafterpentecost{5}}
\printvespersafterpentecost{6}
\printvespersafterpentecost{7}
{
  \def\premagtitle{}
  \def\postmagtitle{
    \pagebreak[2]
  }
\printvespersafterpentecost{8}}
\printvespersafterpentecost{9}
\printvespersafterpentecost{10}
\printvespersafterpentecost{11}
{\def\begincollectcols{\begin{parcolumns}[rulebetween,colwidths={1=0.455\linewidth}]{2}}
%\def\printaftercommonvespers{\needspace{6\baselineskip}}
\printvespersafterpentecost{12}}
{
\def\premagnificat{\bigskip}
\printvespersafterpentecost{13}}
\printvespersafterpentecost{14}
\printvespersafterpentecost{15}
\printvespersafterpentecost{16}
{
  \def\premagtitle{}
  \def\postmagtitle{
    \pagebreak[2]
  }
\printvespersafterpentecost{17}}
\printvespersafterpentecost{18}
{\def\begincollectcols{\begin{parcolumns}[rulebetween,colwidths={1=0.47\linewidth}]{2}}
\def\postbenedicamusdomino{\bigskip}
%\def\premagverses{\vspace{-0.7\baselineskip}}
\printvespersafterpentecost{19}}
\bigskip\bigskip
\printvespersafterpentecost{20}
\printvespersafterpentecost{21}
{
\def\postmagtitle{\medskip}
\printvespersafterpentecost{22}
}
{
\def\postbenedicamusdomino{\bigskip}
\printvespersafterpentecost{23}
}
\bigskip
{%\def\prevespers{
 % \let\oldthing=\englishmagantiphon
 % \def\englishmagantiphon{\oldthing\pagebreak}  
%}
\def\begincollectcols{\begin{parcolumns}[rulebetween,colwidths={1=0.455\linewidth}]{2}}
\printvespersafterpentecost{24}}
}
}
	%\clearpage
\chapter{Proper of the Saints}
{
\newcommand{\benedicamusdomino}[1][1]{
  \noindent\printnote{\Vbar~\emph{Benedicámus Dómino \IfInteger{#1}{#1}{\csname benedicamusdominoname#1\endcsname}}, p.~\pageref{benedicamusdomino-#1}.}
  \bigskip
  \hrule
}

%December 8: Immaculate Conception
%Second Vespers of Immaculate Conception
{
\label{immaculateconception}
\section{December 8: Immaculate Conception}
\subtitle{\nth{1} Class}
\subtitle{I \& II Vespers}

\def\premagverses{\greseteolcustos{manual}}
\def\definevesperspropersalt{\def\noeuouae{T}\newcommand{\maganttex}{MagnificatAntiphon1}
\newcommand{\magantinitial}{B}
\newcommand{\maganttranslation}{All generations shall call me blessed, because he that is mighty, hath done great things for me, alleluia.}
\def\magpsalmclef{c3}
\definemag{8}{G}
}
\def\definevesperspropers{\def\noeuouae{T}\newcommand{\maganttex}{MagnificatAntiphon2}
\newcommand{\magantinitial}{H}
\newcommand{\maganttranslation}{This day a rod came forth from the root of Jesse: this day Mary was conceived without any stain of sin: this day the head of the old serpent was crushed by her.  Alleluia.}
\definemag{1}{f}
}
\def\vesperspropersaltnote{At I Vespers:}
\def\vesperspropersnote{At II Vespers:}
\def\prehymn{\printnote{All kneel for the first stanza of the following hymn.}}
\def\hymnlabel{hymn-avemarisstella}
\def\vrlinebreak{F}
\printvespers[../December8-ImmaculateConception]{inc-ImmaculateConceptionVespers}
}

{
\bigskip

\bigskip
\noindent
\printnote{Then follows a Commemoration of the Advent Sunday or Feria according to the day of the week on which the Feast of the Immaculate Conception falls.}
\bigskip
}

{
  \oldneedspace{5\baselineskip}
  \subtitle{First Week of Advent.}
  \vspace{-\baselineskip}
  \subtitle{\small{Thursday.}}
  {
  \def\noeuouae{T}
  \printgabc{At Magn.}{\oldstylenums{Ant.~4.}}{E}{../Advent1/MagAntThursday-Exspectabo}
  }
  \translation[]{I will look for the Lord my Saviour, and await Him, while He is near, alleluia.}
  \medskip

  \oldneedspace{5\baselineskip}
  \vspace{-\baselineskip}
  \subtitle{\small{Friday.}}
  {
  \def\noeuouae{T}
  \printgabc{At Magn.}{\oldstylenums{Ant.~4.}}{E}{../Advent1/MagAntFriday-ExAegypto}
  }
  \translation[]{Out of Egypt I have called my Son; he shall come to save His people.}
  \medskip

  \oldneedspace{5\baselineskip}
  \subtitle{Second Week of Advent.}
  \vspace{-\baselineskip}
  \subtitle{\small{Saturday.}}
  {
  \def\noeuouae{T}
  \printgabc{At Magn.}{\oldstylenums{Ant.~7.}}{V}{../Advent1/MagAntSaturday-VeniDomine}
  }
  \translation[]{Come, Lord, to visit us in peace, that we may rejoice before Thee with a perfect heart.}
  \medskip

  \oldneedspace{5\baselineskip}
  \vspace{-\baselineskip}
  \subtitle{\small{Sunday.}}
  {
  \def\noeuouae{T}
  \printgabc{At Magn.}{\oldstylenums{Ant.~8.~G}}{T}{../Advent2/MagnificatAntiphon-noEuouae}
  }
  \translation[]{Art Thou He that art to come, or look we for another? Relate to John what you have seen: The blind recover their sight, the dead rise again, the poor have the Gospel preached to them, alleluia.}
  \medskip

  \oldneedspace{5\baselineskip}
  \vspace{-\baselineskip}
  \subtitle{\small{Monday.}}
  {
  \def\noeuouae{T}
  \printgabc{At Magn.}{\oldstylenums{Ant.~4.}}{E}{../Advent2/MagAntMonday-EcceRexVeniet}
  }
  \translation[]{Behold, the King shall come, the Lord of the land; and He shall take away the yoke of our captivity.}
  \medskip

  \oldneedspace{5\baselineskip}
  \vspace{-\baselineskip}
  \subtitle{\small{Tuesday.}}
  {
  \def\noeuouae{T}
  \printgabc{At Magn.}{\oldstylenums{Ant.~5.}}{V}{../Advent2/MagAntTuesday-VoxClamantis}
  }
  \translation[]{A voice of one crying in the desert, Prepare ye the way of the Lord, make straight His paths}
  \medskip

  \oldneedspace{5\baselineskip}
  \vspace{-\baselineskip}
  \subtitle{\small{Wednesday.}}
  {
  \def\noeuouae{T}
  \printgabc{At Magn.}{\oldstylenums{Ant.~4.}}{S}{../Advent2/MagAntWednesday-Sion}
  }
  \translation[]{Sion, thou shalt be restored, and shalt see the Just One who shall appear in thee.}
  \medskip

  \oldneedspace{5\baselineskip}
  \vspace{-\baselineskip}
  \subtitle{\small{Thursday.}}
  {
  \def\noeuouae{T}
  \printgabc{At Magn.}{\oldstylenums{Ant.~4.}}{Q}{../Advent2/MagAntThursday-QuiPostMeVenit}
  }
  \translation[]{He that shall come after me is preferred before me; whose shoes I am not worthy to loose.}
  \medskip
  \hrule
  \medskip
  {
      \newcommand{\commvlatin}{Roráte cæli désuper, et nubes pluant \textbf{ju}stum.}
      \newcommand{\commrlatin}{Aperiátur terra, et gérminet Salva\textbf{tó}rem.}
      \newcommand{\commvtranslation}{Ye heavens, drop down dew from above, and let the clouds rain down the Just One.}
      \newcommand{\commrtranslation}{Let the earth open and bud forth the Saviour.}
  \printvrcommem{}
  }

  \oldneedspace{3\baselineskip}
  \begin{center}{\large Collect.}\end{center}
  \vspace{-\baselineskip}
  \def\printcollectheading{F}
  {
  \begin{center}{First Week of Advent.}\end{center}
  \def\gabcfolder{../Advent1}
  \newcommand{\antonetex}{Ant1-InIllaDie}
\newcommand{\antoneinitial}{I}
\newcommand{\antonetranslation}{In that day the mountains shall drop down sweetness, and the hills shall flow with milk and honey, alleluia.}
\definepsalm{1}{109}{8}{G}

\newcommand{\anttwotex}{Ant2-Jucundare}
\newcommand{\anttwoinitial}{J}
\newcommand{\anttwotranslation}{Shout for joy, O daughter of Sion, rejoice greatly, O daughter of Jerusalem, alleluia.}
\definepsalm{2}{110}{8}{G*}

\newcommand{\antthreetex}{Ant3-EcceDominusVeniet}
\newcommand{\antthreeinitial}{E}
\newcommand{\antthreetranslation}{Behold, the Lord shall come, and all His Saints with Him: and there shall be in that day a great light, alleluia.}
\definepsalm{3}{111}{5}{a}

\newcommand{\antfourtex}{Ant4-Omnes}
\newcommand{\antfourinitial}{O}
\newcommand{\antfourtranslation}{All ye that thirst come to the waters: seek the Lord while He can be found, alleluia.}
\definepsalm{4}{112}{7}{c}

\newcommand{\antfivetex}{Ant5-EcceVeniet}
\newcommand{\antfiveinitial}{E}
\newcommand{\antfivetranslation}{Behold there shall come a great Prophet, and He shall renew Jerusalem, alleluia.}
\definepsalm{5}{113}{4}{A*}

\newcommand{\chaptertext}{\dropcap{latin}{Fratres~: Hora est jam nos de somno} \textbf{súr}\-ge\-re~:~\gredagger{} nunc enim própior est \emph{no\-stra} \textbf{sá}\-lus,~* quam cum cre\-\textbf{dí}\-dimus.}
\newcommand{\chaptertranslation}{Brethren: it is now the hour for us to rise from sleep.  For now our salvation is nearer than when we believed.}

\newcommand{\magantinitial}{N}
\newcommand{\maganttex}{MagnificatAntiphon}
\newcommand{\maganttranslation}{Fear not, Mary, for thou hast found grace with the Lord: behold thou shalt conceive and bring forth a son, alleluia.}
\def\magsolemn{F}
\definemag{8}{G}

\newcommand{\collect}{Excita, quǽsumus Dómine, poténtiam tuam, et veni~:~† ut ab imminéntibus peccatórum nostrórum perículis, te mereámur protegénte éripi,~* te liberánte salvári.  Qui vivis et regnas cum Deo Patre in unitáte Spíritus Sancti Deus~:~* per ómnia sǽcula sæculórum.}
\newcommand{\collecttranslation}{Stir up Thy power, we beseech Thee, O Lord, and come: that from the threatening dangers of our sins we may deserve to be rescued by Thy protection, and to be saved by Thy deliverance: Who livest and reignest with God the Father in the unity of the Holy Ghost, world without end.}

  \printcollect{\collect}{\collecttranslation}
  }
  {
  \medskip
  \begin{center}{Second Week of Advent.}\end{center}
  \def\gabcfolder{../Advent2}
  % !TEX TS-program = lualatex
% !TEX encoding = UTF-8

% This is a simple template for a LuaLaTeX document using gregorio scores.

\newcommand{\comheadingtext}{Commemoration of 2nd Sunday of Advent}

\newcommand{\latincomcollect}{Excita Dómine corda nostra ad præparándas Unigéniti tui vias~:~† ut per ejus advéntum,~* purificátis tibi méntibus servíre mereámur. Qui tecum vivit et regnat.}
\newcommand{\englishcomcollect}{Stir up our hearts, O Lord, to prepare the ways of Thine only-begotten Son: that through His coming we may deserve to serve Thee with purified minds: Who with Thee liveth and reigneth.}

\newcommand{\englishcommagantiphon}{Art Thou He that art to come, or look we for another? Relate to John what you have seen: The blind recover their sight, the dead rise again, the poor have the Gospel preached to them, alleluia.}

\newcommand{\commagantlinetwo}{Ant. 8. G}
\newcommand{\commaganttex}{MagnificatAntiphon-noEuouae}
\newcommand{\commagantinitial}{T}
\newcommand{\commagantinitialsize}{35}

\newcommand{\commvrtex}{../Advent/vr-commemoration}
\newcommand{\commvtranslation}{Ye heavens, drop down dew from above, and let the clouds rain down the Just One.}
\newcommand{\commrtranslation}{Let the earth open and bud forth the Saviour.}

\setgrefactor{17}
  \printcollect{\latincomcollect}{\englishcomcollect}
  }

  \bigskip
  \benedicamusdomino{}
}


%Purification & Presentation (2nd class no commem of Sunday because it is a feast of the Lord)
{
\section{February 2: Purification \& Presentation}
\subtitle{\nth{2} Class}
\subtitle{I \& II Vespers}
\printnote{At I Vespers, Psalms and Antiphons of the Circumcision, p.~\pageref{circumcision}, continuing with the chapter on p.~\pageref{purification-chapter}.}

\def\definevesperspropers{\newcommand{\maganttex}{an--hodie_beata_virgo--solesmes}
\newcommand{\magantinitial}{H}
\newcommand{\maganttranslation}{Today did the Blessed Virgin Mary present the Child Jesus in the temple; and Simeon, filled with the Holy Ghost, took Him up in his arms, and blessed God for ever.}
\def\magpsalmclef{c3}
\definemag{8}{G*}

  \def\prepsalmfive{\greseteolcustos{manual}}
}
\def\definevesperspropersalt{\newcommand{\maganttex}{an--senex_puerum--solesmes}
\newcommand{\magantinitial}{S}
\newcommand{\maganttranslation}{The old man carried the Child, but the Child led the old man. The Virgin bore the Child, and after child-bearing was virgin still: Whom she bore, Him she adored.}
\definemag{1}{D}
}
\def\vesperspropersnote{At II Vespers:}
\def\vesperspropersaltnote{At I Vespers:}
%\def\premag{\def\noeuouae{T}}
\def\premagverses{\greseteolcustos{manual}}
\def\prechapter{\label{purification-chapter}}
\def\printhymnnote{
  {
    \oldneedspace{3\baselineskip}
    \printnote{Hymn.~\emph{Ave Maris Stella}, p.~\pageref{hymn-avemarisstella}.\\}
    %
    % \def\vrlinebreak{T}
    % \oldneedspace{3\baselineskip}
    % \printvr[\greseteolcustos{manual}]{\vrtex}{\vtranslation}{\rtranslation}
  }
}
\printvespers[../February2-PurificationOfBlessedVirginMary]{inc-purification}
\bigskip
\benedicamusdomino[2]{}
}

%May 1: St Joseph the Worker (1st class)
{
\section{May 1: St Joseph the Worker}
\subtitle{\nth{1} Class}
\subtitle{I \& II Vespers}

\def\definevesperspropers{\newcommand{\vrtex}{vrOraProNobis}
\newcommand{\vtranslation}{Pray for us, St Joseph, alleluia.}
\newcommand{\rtranslation}{Faithful protector of our labors, alleluia.}

\newcommand{\maganttex}{an--et_ipse_jesus--solesmes}
\newcommand{\magantinitial}{E}
\newcommand{\maganttranslation}{}
\definemag{7}{d}

  \def\prepsalmfive{\greseteolcustos{manual}}
}
\def\definevesperspropersalt{\newcommand{\vrtex}{vrSolemnitasEstHodie}
\newcommand{\vtranslation}{Today is the solemnity of St Joseph, alleluia.}
\newcommand{\rtranslation}{Who ministered with his hands to the Son of God, alleluia.}

\newcommand{\maganttex}{an--christus_dominus--solesmes}
\newcommand{\magantinitial}{C}
\newcommand{\maganttranslation}{}
\definemag{7}{c2}

  \def\vrlinebreak{F}
}
\def\vesperspropersnote{At II Vespers:}
\def\vesperspropersaltnote{At I Vespers:}
%\def\premag{\def\noeuouae{T}}
\def\premagverses{\greseteolcustos{manual}}

\printvespers[../May1-StJosephWorker]{inc-StJosephWorker}
%if feast of St Joseph the worker falls from 2nd through 5th Sunday after Easter, it outranks the Sunday and the Sunday is commemorated
\medskip
\printnote{If today is Sunday, the Vespers of the Sunday is commemorated with \emph{Magnificat antiphon}, \emph{\Vbar{} Mane nobíscum.} in simple commemoration tone, p.~\pageref{vr-manenobiscum} and \emph{Collect}.  Otherwise \Vbar~\emph{Benedicámus Dómino 1}, p.~\pageref{benedicamusdomino-1}.

\begin{multicols}{2}
\noindent\emph{\nth{2} Sunday after Easter}, p.~\pageref{easter2}.\\
\emph{\nth{3} Sunday after Easter}, p.~\pageref{easter3}.\\
\emph{\nth{4} Sunday after Easter}, p.~\pageref{easter4}.\\
\emph{\nth{5} Sunday after Easter}, p.~\pageref{easter5}.
\end{multicols}
}
% \bigskip
% \benedicamusdomino{}
}

%June 24: Nativity of St John the Baptist (1st class)
{
\section{June 24: Nativity of St John the Baptist}
\subtitle{\nth{1} Class}
\subtitle{I \& II Vespers}

\def\definevesperspropers{\newcommand{\antonetex}{Ant1-ElisabethZachariae}
\newcommand{\antoneinitial}{E}
\newcommand{\antonetranslation}{Elizabeth, the wife of Zacharias, gave birth to a great man, John the Baptist, the forerunner of the Lord.}
\definepsalm{1}{109}{3}{a}

\newcommand{\anttwotex}{Ant2-Innuebant}
\newcommand{\anttwoinitial}{I}
\newcommand{\anttwotranslation}{They made signs unto his father, by what name he should be called: and he wrote, saying: His name is John.}
\definepsalm{2}{110}{4}{E}

\newcommand{\antthreetex}{Ant3-JoannesVocabitur}
\newcommand{\antthreeinitial}{J}
\newcommand{\antthreetranslation}{His name shall be called John, and many shall rejoice in his birth.}
\definepsalm{3}{111}{1}{f}

\newcommand{\antfourtex}{Ant4-InterNatos}
\newcommand{\antfourinitial}{I}
\newcommand{\antfourtranslation}{Among those born of women, there hath not risen a greater than John the Baptist.}
\definepsalm{4}{112}{3}{b}

\newcommand{\antfivetex}{Ant5-TuPuer}
\newcommand{\antfiveinitial}{T}
\newcommand{\antfivetranslation}{Thou, child, shalt be called the Prophet of the Most High: thou shalt go before the Lord to prepare His ways.}
\definepsalm{5}{116}{3}{b}

\newcommand{\vrtex}{vrIstePuerMagnus}
\newcommand{\vtranslation}{This child is great before the Lord.}
\newcommand{\rtranslation}{For in truth His hand is with him.}

\newcommand{\maganttex}{MagnificatAntiphon2}
\newcommand{\magantinitial}{P}
\newcommand{\maganttranslation}{The child that is born to us is more than a prophet; for this is he of whom the Saviour said: Among those born of women there hath not risen a greater than John the Baptist.}
\def\magsolemn{T}
\definemag{7}{d}

  \def\prepsalmfive{\greseteolcustos{manual}}
}
\def\definevesperspropersalt{\newcommand{\antonetex}{Ant1-IpsePraeibit}
\newcommand{\antoneinitial}{I}
\newcommand{\antonetranslation}{He shall go before Him in the spirit and power of Elias, to prepare unto the Lord a perfect people.\vspace{0ex plus 0ex minus 3ex}}
\definepsalm{1}{109}{7}{a}

\newcommand{\anttwotex}{Ant2-Joannes}
\newcommand{\anttwoinitial}{J}
\newcommand{\anttwotranslation}{John is his name.  Wine and strong drink he shall not drink, and many shall rejoice in his birth.}
\definepsalm{2}{110}{8}{G}

\newcommand{\antthreetex}{Ant3-ExUteroSenectutis}
\newcommand{\antthreeinitial}{E}
\newcommand{\antthreetranslation}{From the barren womb of age was born John, the forerunner of the Lord.}
\definepsalm{3}{111}{1}{f}

\newcommand{\antfourtex}{Ant4-IstePuer}
\newcommand{\antfourinitial}{I}
\newcommand{\antfourtranslation}{This child is great before the Lord, for the hand of God is with him.}
\definepsalm{4}{112}{4}{A*}

\newcommand{\antfivetex}{Ant5-Nazaraeus}
\newcommand{\antfiveinitial}{N}
\newcommand{\antfivetranslation}{This child shall be called a Nazarite; wine and strong drink he shall not drink, and from his mother's womb he shall eat nothing unclean.}
\definepsalm{5}{116}{5}{a}

\newcommand{\vrtex}{vrFuitHomo}
\newcommand{\vtranslation}{There was a man sent from God.}
\newcommand{\rtranslation}{Whose name was John.}

\newcommand{\maganttex}{MagnificatAntiphon1}
\newcommand{\magantinitial}{I}
\newcommand{\maganttranslation}{When Zacharias had entered the temple of the Lord, there appeared to him the Angel Gabriel, standing at the right hand of the altar of incense.\vspace{-4pt plus 4pt}}
\def\magpsalmclef{c3}
\def\magsolemn{T}
\definemag{8}{G}
}
\def\vesperspropersnote{At II Vespers:}
\def\vesperspropersaltnote{At I Vespers:}
\def\prevesperspsalms{\noindent\printnote{Chapter and following, page \pageref{june24-chapter}.\\}}
\def\vesperspsalmslabel{\label{june24-2vespers}}
\def\prevesperspsalmsalt{\noindent\printnote{II Vespers psalms and antiphons, page \pageref{june24-2vespers}.}\medskip}
\def\prechapter{\label{june24-chapter}}
%\def\premag{\def\noeuouae{T}}
\def\premagverses{\greseteolcustos{manual}}

\printvespers[../June24-BirthOfJohnTheBaptist]{inc-BirthOfJohnTheBaptist}

\medskip
\printnote{If today is Sunday, the Vespers of the Sunday is commemorated with \emph{Magnificat antiphon}, \emph{\Vbar{} Dirigátur.} in simple commemoration tone, p.~\pageref{vr-dirigatur} and \emph{Collect}.  Otherwise \Vbar~\emph{Benedicámus Dómino 1}, p.~\pageref{benedicamusdomino-1}.

\begin{multicols}{2}
\noindent\emph{\nth{2} Sunday after Pentecost}, p.~\pageref{pentecost2}.\\
\emph{\nth{3} Sunday after Pentecost}, p.~\pageref{pentecost3}.\\
\emph{\nth{4} Sunday after Pentecost}, p.~\pageref{pentecost4}.\\
\emph{\nth{5} Sunday after Pentecost}, p.~\pageref{pentecost5}.\\
\emph{\nth{6} Sunday after Pentecost}, p.~\pageref{pentecost6}.
\end{multicols}
}
% \bigskip
% \benedicamusdomino{}
}

%June 29: Sts Peter \& Paul (1st class)
{
\section{June 29: Sts Peter \& Paul}
\subtitle{\nth{1} Class}
\subtitle{I \& II Vespers}

\def\definevesperspropers{\import{../CommonOfApostles/}{inc-CommonOfApostles-2Vespers-psalms}
\edef\antonetex{../CommonOfApostles/\antonetex}
\edef\anttwotex{../CommonOfApostles/\anttwotex}
\edef\antthreetex{../CommonOfApostles/\antthreetex}
\edef\antfourtex{../CommonOfApostles/\antfourtex}
\edef\antfivetex{../CommonOfApostles/\antfivetex}

\newcommand{\vrtex}{vrAnnuntiaverunt}
\newcommand{\vtranslation}{They declared the works of God.}
\newcommand{\rtranslation}{And understood His doings.}

\newcommand{\maganttex}{MagnificatAntiphon2-Hodie}
\newcommand{\magantinitial}{H}
\newcommand{\maganttranslation}{Today, Simon Peter went up upon the gibbet of the cross, alleluia; today, he that holdeth the keys of the kingdom, departed with joy to be with Christ; today, the Apostle Paul, the light of the world, bowing his head, for Christ's sake was crowned with martyrdom, alleluia.}
\def\magsolemn{T}
\definemag{1}{D}

  \def\prepsalmfive{\greseteolcustos{manual}}
}
\def\definevesperspropersalt{\newcommand{\antonetex}{Ant1-PetrusEtJoannes}
\newcommand{\antoneinitial}{P}
\newcommand{\antonetranslation}{Peter and John went up together into the Temple at the hour of prayer, being the ninth hour.}
\definepsalm{1}{109}{8}{G}

\newcommand{\anttwotex}{Ant2-Argentum}
\newcommand{\anttwoinitial}{A}
\newcommand{\anttwotranslation}{Silver and gold have I none, but such as I have, give I thee.}
\definepsalm{2}{110}{7}{b}

\newcommand{\antthreetex}{Ant3-DixitAngelusAdPetrum}
\newcommand{\antthreeinitial}{D}
\newcommand{\antthreetranslation}{The Angel said unto Peter: Cast thy garment about thee, and follow me.}
\definepsalm{3}{111}{8}{c}

\newcommand{\antfourtex}{Ant4-MisitDominus}
\newcommand{\antfourinitial}{M}
\newcommand{\antfourtranslation}{The Lord hath sent His Angel, and hath delivered me out of the hand of Herod.  Alleluia.}
\definepsalm{4}{112}{7}{c2}

\newcommand{\antfivetex}{Ant5-TuEsPetrus}
\newcommand{\antfiveinitial}{T}
\newcommand{\antfivetranslation}{Thou art Peter and upon this Rock I will build My Church.}
\definepsalm{5}{116}{7}{c}

\newcommand{\vrtex}{vrInOmnemTerram}
\newcommand{\vtranslation}{Their sound hath gone forth into all the earth.}
\newcommand{\rtranslation}{And their words unto the ends of the world.}

\newcommand{\maganttex}{MagnificatAntiphon1}
\newcommand{\magantinitial}{T}
\newcommand{\maganttranslation}{Thou art the Shepherd of the sheep and the Prince of the Apostles, and unto thee are given the keys of the kingdom of heaven.}
\def\magsolemn{T}
\definemag{1}{f}
}
\def\vesperspropersnote{At II Vespers:}
\def\vesperspropersaltnote{At I Vespers:}
\def\prevesperspsalms{\noindent\printnote{Chapter and following, page \pageref{june29-chapter}.\\}}
\def\vesperspsalmslabel{\label{june29-2vespers}}
\def\prevesperspsalmsalt{\noindent\printnote{II Vespers psalms and antiphons, page \pageref{june29-2vespers}.}\medskip}
\def\prechapter{\label{june29-chapter}}
%\def\premag{\def\noeuouae{T}}
\def\premagverses{\greseteolcustos{manual}}

\def\begincollectcols{\begin{parcolumns}[rulebetween,colwidths={1=0.45\linewidth}]{2}}
\printvespers[../June29-StsPeterAndPaul]{inc-StsPeterAndPaul}

\medskip
\printnote{If today is Sunday, the Vespers of the Sunday is commemorated with \emph{Magnificat antiphon}, \emph{\Vbar{} Dirigátur.} in simple commemoration tone, p.~\pageref{vr-dirigatur} and \emph{Collect}.  Otherwise \Vbar~\emph{Benedicámus Dómino 1}, p.~\pageref{benedicamusdomino-1}.

\begin{multicols}{2}
\noindent\emph{\nth{3} Sunday after Pentecost}, p.~\pageref{pentecost3}.\\
\emph{\nth{4} Sunday after Pentecost}, p.~\pageref{pentecost4}.\\
\emph{\nth{5} Sunday after Pentecost}, p.~\pageref{pentecost5}.\\
\emph{\nth{6} Sunday after Pentecost}, p.~\pageref{pentecost6}.\\
\emph{\nth{7} Sunday after Pentecost}, p.~\pageref{pentecost7}.
\end{multicols}
}
% \bigskip
% \benedicamusdomino{}
}

%July 1: Most Precious Blood (1st class)
{
\global\let\psalmclefthree=\undefined
\section{July 1: The Precious Blood of Our Lord Jesus Christ}
\subtitle{\nth{1} Class}
\subtitle{I \& II Vespers}

\def\definevesperspropers{\definepsalm{5}{147}{2}{D}

\newcommand{\vrtex}{vrTeErgo}
\newcommand{\vtranslation}{We therefore pray Thee help Thy servants.}
\newcommand{\rtranslation}{Whom Thou hast redeemed with Thy precious Blood.}

\newcommand{\maganttex}{MagnificatAntiphon2}
\newcommand{\magantinitial}{H}
\newcommand{\maganttranslation}{\vspace{-16pt plus 16pt}Ye shall observe this day for a memorial: and ye shall keep it holy unto the Lord, in your generations with an everlasting worship.\vspace{-4pt plus 4pt}}
\newcommand{\magsolemn}{F}
\definemag{1}{D2}

  \def\prepsalmfive{\greseteolcustos{manual}}
}
\def\definevesperspropersalt{\definepsalm{5}{116}{2}{D}

\newcommand{\vrtex}{vrRedemistiNos}
\newcommand{\vtranslation}{Thou hast redeemed us, O Lord, in Thy Blood.}
\newcommand{\rtranslation}{And hast made of us a kingdom unto our God.}

\newcommand{\maganttex}{MagnificatAntiphon1}
\newcommand{\magantinitial}{A}
\newcommand{\maganttranslation}{Ye are come to Mount Sion, to the city of the living God, the heavenly Jerusalem, and to Jesus the Mediator of the new Testament, and to the sprinkling of blood which speaketh better than that of Abel.}
\newcommand{\magsolemn}{F}
\definemag{3}{a}
}
\def\vesperspropersnote{At II Vespers:}
\def\vesperspropersaltnote{At I Vespers:}
%\def\premag{\def\noeuouae{T}}
\def\premagverses{\greseteolcustos{manual}}

\def\begincollectcols{\begin{parcolumns}[rulebetween,colwidths={1=0.45\linewidth}]{2}}
\printvespers[../July1-MostPreciousBloodOfChrist]{inc-MostPreciousBloodOfChrist}
\bigskip
\benedicamusdomino{}
}

%Aug 6: Transfiguration (2nd class)
{
\section{August 6: Transfiguration of Our Lord Jesus Christ}
\subtitle{\nth{2} Class}
\subtitle{I \& II Vespers}

\def\definevesperspropers{\newcommand{\maganttex}{MagnificatAntiphon2}
\newcommand{\magantinitial}{E}
\newcommand{\maganttranslation}{And the disciples hearing, fell on their faces, and were sore afraid; and Jesus came, and touched them, and said to them, Arise, and fear not, alleluia.}
\newcommand{\magsolemn}{F}
\definemag{8}{G}

  \def\prepsalmfive{\greseteolcustos{manual}}
}
\def\definevesperspropersalt{\newcommand{\maganttex}{an--christus_jesus--solesmes}
\newcommand{\magantinitial}{C}
\newcommand{\maganttranslation}{Christ Jesus, radiance of the Father and image of His Being, upholding all things by the word of His power; making atonement for sins, has deigned to appear today in glory on the high mountain.}
\newcommand{\magsolemn}{F}
\definemag{4}{E}
}
\def\vesperspropersnote{At II Vespers:}
\def\vesperspropersaltnote{At I Vespers:}
%\def\premag{\def\noeuouae{T}}
\def\premagverses{\greseteolcustos{manual}}

\def\begincollectcols{\begin{parcolumns}[rulebetween,colwidths={1=0.44\linewidth}]{2}}
\printvespers[../August6-TransfigurationOfOurLord]{inc-Transfiguration}
\bigskip
\benedicamusdomino[2]{}
}

%Aug 15: Assumption (1st class)
{
\section{August 15: Assumption of the B.~V.~M.}
\subtitle{\nth{1} Class}
\subtitle{I \& II Vespers}

\def\definevesperspropers{% hymn is ave maris stella
%\input{inc-hymn-avemarisstella}

\newcommand{\maganttex}{an--hodie_maria_virgo--solesmes}
\newcommand{\magantinitial}{H}
\newcommand{\maganttranslation}{Today the Virgin Mary has gone up to heaven: rejoice, for with Christ she reigns forever.}
\newcommand{\magsolemn}{T}
\definemag{8}{G*}

  \def\prepsalmfive{\greseteolcustos{manual}}
}
\def\definevesperspropersalt{\newcommand{\hymnlinetwo}{2.}
\newcommand{\hymntex}{Hymn-OPrimaVirgoProdita}
\newcommand{\hymninitial}{O}
\newcommand{\hymntranslation}{
\item O Virgin who was first to receive
The Creator’s grace by the spirit,
Who was predestined by the Most High
To bear in her womb the Son.

\item O woman, who was foretold to be
The perpetual enemy of the demon;
Who alone was filled with grace,
Undefiled from conception.

\item Thou who conceives Life itself in thy womb,
Life that was lost by Adam;
Furnishing the divine Victim,
A body for his sacrifice.

\item Death, the recompense for sin,
Had no victory over thee, and now departs;
And then thou hastened bodily to heaven
To be thy loving Son’s companion.

\item Illuminated by so great a Glory,
All nature is raised up;
And in thee is called to reach
The pinnacle of all glory and splendour.

\item In thy triumph O our Queen,
Turn thine eyes to us exiles;
That through thy patronage,
We may come to heaven, our blessed homeland.

\item Praise to the Father! praise to Him,
The Virgin’s holy Son!
Praise to the Spirit Paraclete,
While endless ages run! 
Amen.
}

\newcommand{\maganttex}{MagAntiphon-VirgoPrudentissima}
\newcommand{\magantinitial}{V}
\newcommand{\maganttranslation}{O Virgin most prudent, whither goest thou, like the golden dawn?  Daughter of Sion, thou art all beautiful and sweet; fair as the moon, bright as the sun.}
\newcommand{\magsolemn}{T}
\definemag{1}{f}
}
\def\vesperspropersnote{At II Vespers:}
\def\vesperspropersaltnote{At I Vespers:}
%\def\premag{\def\noeuouae{T}}
\def\premagverses{\greseteolcustos{manual}}
\def\printfullhymn{
  {
    \oldneedspace{3\baselineskip}
    \printnote{At II Vespers: Hymn.~\emph{Ave Maris Stella}, p.~\pageref{hymn-avemarisstella}. \Vbar{} \emph{Exaltata.} p.~\pageref{vr-assumption}.\\}

    \printnote{\vesperspropersaltnote}
    \definevesperspropersalt
    \printhymn{\oldstylenums{\hymnlinetwo}}{\hymninitial}{\hymntex}{\hymntranslation}
  }
  {
    \def\vrlinebreak{T}
    \oldneedspace{3\baselineskip}
    \label{vr-assumption}
    \printvr[\greseteolcustos{manual}]{\vrtex}{\vtranslation}{\rtranslation}
  }
}

\printvespers[../August15-AssumptionOfTheBlessedVirginMary]{inc-Assumption}

%     TODO Feast of Assumption (8/15) could have commem of Saturday before 3rd Sunday of August or 9th to 13th Sunday after Pentecost
\medskip
\printnote{If today is Sunday, the Vespers of the Sunday is commemorated with \emph{Magnificat antiphon}, \emph{\Vbar{} Dirigátur.} in simple commemoration tone, p.~\pageref{vr-dirigatur} and \emph{Collect}.  Otherwise \Vbar~\emph{Benedicámus Dómino 1}, p.~\pageref{benedicamusdomino-1}.

\begin{multicols}{2}
\noindent\emph{\nth{9} Sunday after Pentecost}, p.~\pageref{pentecost9}.\\
\emph{\nth{10} Sunday after Pentecost}, p.~\pageref{pentecost10}.\\
\emph{\nth{11} Sunday after Pentecost}, p.~\pageref{pentecost11}.\\
\emph{\nth{12} Sunday after Pentecost}, p.~\pageref{pentecost12}.\\
\emph{\nth{13} Sunday after Pentecost}, p.~\pageref{pentecost13}.
\end{multicols}
}
\medskip
\hrule
% \bigskip
% \benedicamusdomino{}
}

%Sep 14: Exaltation of Holy Cross (2nd class)
{
\section{September 14: Exaltation of the Holy Cross}
\subtitle{\nth{2} Class}
\subtitle{I \& II Vespers}

\def\definevesperspropers{\newcommand{\maganttex}{MagnificatAntiphon2-OCrux}
\newcommand{\magantinitial}{O}
\newcommand{\maganttranslation}{O blessed art thou, O Cross which wast counted the only tree worthy to bear the Lord and King of heaven. Alleluia.}
\def\magsolemn{F}
\definemag{1}{D2}

  \def\prepsalmfive{\greseteolcustos{manual}}
}
\def\definevesperspropersalt{\newcommand{\maganttex}{MagnificatAntiphon1-OCrux}
\newcommand{\magantinitial}{T}
\newcommand{\maganttranslation}{Hail, O Cross Brighter than all the stars thy name is honourable upon earth; To the eyes of men thou art exceeding lovely; holy art thou among all things that are earthly; thy transom made the one worthy balance whereon the price of the world was weighed; sweetest wood and sweetest iron, sweetest weight is hung on thee; O that every one that is here gathered this day to praise thee may find that thou art indeed salvation for him.}
\def\magsolemn{F}
\definemag{1}{D}
}
\def\vesperspropersnote{At II Vespers:}
\def\vesperspropersaltnote{At I Vespers:}
%\def\premag{\def\noeuouae{T}}
\def\premagverses{\greseteolcustos{manual}}

\def\begincollectcols{\begin{parcolumns}[rulebetween,colwidths={1=0.42\linewidth}]{2}}
\printvespers[../September14-ExaltationOfTheHolyCross]{inc-ExaltationOfTheHolyCross}
\bigskip
\benedicamusdomino[2]{}
}

%Sep 29: Dedication of St Michael (1st class)
{
\section{September 29: Dedication of St Michael the Archangel}
\subtitle{\nth{1} Class}
\subtitle{I \& II Vespers}

\def\definevesperspropers{\definepsalm{5}{137}{7}{c}

\newcommand{\vrtex}{vrInConspectuAngelorum}
\newcommand{\vtranslation}{In the sight of the Angels, I will sing praise to Thee, O my God.}
\newcommand{\rtranslation}{I will worship towards Thy holy temple, and give glory to Thy name.}

\newcommand{\maganttex}{MagnificatAntiphon2}
\definemag{1}{D2}
\newcommand{\magantinitial}{P}
\newcommand{\maganttranslation}{O most glorious prince, Michael the Archangel, be mindful of us, and here and everywhere entreat the Son of God for us, alleluia, alleluia.}

  \def\prepsalmfive{\greseteolcustos{manual}}
}
\def\definevesperspropersalt{\definepsalm{5}{116}{7}{c}

\newcommand{\vrtex}{vrStetitAngelus}
\newcommand{\vtranslation}{The Angel stood by the altar of the temple.}
\newcommand{\rtranslation}{Holding in his hand a censer of gold.}

\newcommand{\maganttex}{MagnificatAntiphon1}
\newcommand{\magantinitial}{D}
\definemag{8}{G}
\newcommand{\maganttranslation}{While John was beholding the sacred Mystery, the Archangel Michael sounded a trumpet.  Forgive us, O Lord our God, Thou who openest the book, and loosest the seals thereof.  Alleluia.\vspace{-1ex}}
}
\def\vesperspropersnote{At II Vespers:}
\def\vesperspropersaltnote{At I Vespers:}
%\def\premag{\def\noeuouae{T}}
\def\premagverses{\greseteolcustos{manual}}

\printvespers[../September29-DedicationOfChurchOfStMichaelArchangel]{inc-DedicationStMichael}

\medskip
\printnote{If today is Sunday, the Vespers of the Sunday is commemorated with \emph{Magnificat antiphon}, \emph{\Vbar{} Dirigátur.} in simple commemoration tone, p.~\pageref{vr-dirigatur} and \emph{Collect}.  Otherwise \Vbar~\emph{Benedicámus Dómino 1}, p.~\pageref{benedicamusdomino-1}.

\begin{multicols}{2}
\noindent\emph{\nth{16} Sunday after Pentecost}, p.~\pageref{pentecost16}.\\
\emph{\nth{17} Sunday after Pentecost}, p.~\pageref{pentecost17}.\\
\emph{\nth{18} Sunday after Pentecost}, p.~\pageref{pentecost18}.\\
\emph{\nth{19} Sunday after Pentecost}, p.~\pageref{pentecost19}.\\
\emph{\nth{20} Sunday after Pentecost}, p.~\pageref{pentecost20}.
\end{multicols}
}
\medskip
\hrule
% \bigskip
% \benedicamusdomino{}
}

%Last Sunday in October: Christ the King (1st class)
{
\section{Last Sunday in October: Jesus Christ, King}
\subtitle{\nth{1} Class}
\subtitle{II Vespers}

\def\definevesperspropers{\newcommand{\vrtex}{vr}
\newcommand{\vtranslation}{His dominion shall be increased.}
\newcommand{\rtranslation}{And of peace there shall be no end.}

\newcommand{\magantinitial}{H}
\newcommand{\maganttex}{MagnificatAntiphon}
\newcommand{\maganttranslation}{He hath on His garment and on His thigh written: King of kings and Lord of lords.  To Him be glory and empire for ever and ever.}
\def\magsolemn{T}
\definemag{7}{a}

  \def\prepsalmfive{\greseteolcustos{manual}}
}
%\def\premag{\def\noeuouae{T}}
\def\premagverses{\greseteolcustos{manual}}
\def\beginchaptercols{\begin{parcolumns}[rulebetween,colwidths={1=0.46\linewidth}]{2}}
%\def\begincollectcols{\begin{parcolumns}[rulebetween,colwidths={1=0.45\linewidth}]{2}}

\printvespers[../OctoberLastSunday-ChristTheKing]{inc-ChristTheKing}
\noindent
\printnote{If today is October 31, the First Vespers of All Saints is commemorated with \emph{Magnificat}, p.~\pageref{allsaints1-magnificat}; \emph{\Vbar{}~Lætámini}.~in simple commemoration tone, p.~\pageref{allsaints1-vr}; and \emph{Collect}, p.~\pageref{allsaints-collect}.  Otherwise \Vbar~\emph{Benedicámus Dómino 1}, p.~\pageref{benedicamusdomino-1}.}
\bigskip
\hrule
%\bigskip
%\benedicamusdomino{}
}

%Nov 1: All Saints (1st class)
{
\section{November 1: All Saints}
\subtitle{\nth{1} Class}
\subtitle{I \& II Vespers}

\def\definevesperspropers{\definepsalm{5}{115}{8}{G}

\newcommand{\vrtex}{vrExsultabunt}
\newcommand{\vtranslation}{The Saints will rejoice in glory.}
\newcommand{\rtranslation}{They will be joyful upon their beds.}

\newcommand{\maganttex}{an--o_quam_gloriosum--solesmes}
\newcommand{\magantinitial}{O}
\newcommand{\maganttranslation}{Oh! how glorious is the kingdom where all the Saints rejoice with Christ; clothed in white robes, they follow the Lamb whithersoever he goeth!}
\def\magsolemn{T}
\definemag{6}{F}

  \def\prepsalmfive{\greseteolcustos{manual}}
}
\def\definevesperspropersalt{\definepsalm{5}{116}{8}{G}

\newcommand{\vrtex}{vrLaetamini}
\newcommand{\vtranslation}{Be glad in the Lord, and rejoice ye righteous.}
\newcommand{\rtranslation}{And shout for joy, all ye that are upright in heart.}

\newcommand{\maganttex}{an--angeli_archangeli_all_saints--solesmes}
\newcommand{\magantinitial}{A}
\newcommand{\maganttranslation}{O ye Angels, Archangels, Thrones and Dominions, Principalities and Powers, Virtues, Cherubim and Seraphim, Patriarchs and Prophets, holy Teachers of the Law, all Apostles, Martyrs of Christ, holy Confessors, Virgins of the Lord, Hermits, and all Saints, intercede for us.}
\def\magsolemn{T}
\definemag{1}{D}
\def\premag{\label{allsaints1-magnificat}}}
\def\vraltlabel{allsaints1-vr}
\def\vesperspropersnote{At II Vespers:}
\def\vesperspropersaltnote{At I Vespers:}
\def\begincollectcols{\label{allsaints-collect}\begin{parcolumns}[rulebetween]{2}}
%\def\premag{\def\noeuouae{T}}
\def\premagverses{\greseteolcustos{manual}}

\printvespers[../November1-AllSaints]{inc-AllSaints}

\medskip
\printnote{If today is Sunday, the Vespers of the Sunday is commemorated with \emph{Magnificat antiphon}, \emph{\Vbar{} Dirigátur.} in simple commemoration tone, p.~\pageref{vr-dirigatur} and \emph{Collect}.  Otherwise \Vbar~\emph{Benedicámus Dómino 1}, p.~\pageref{benedicamusdomino-1}.

\begin{multicols}{2}
\noindent\emph{\nth{21} Sunday after Pentecost}, p.~\pageref{pentecost21}.\\
\emph{\nth{22} Sunday after Pentecost}, p.~\pageref{pentecost22}.\\
\emph{\nth{23} Sunday after Pentecost}, p.~\pageref{pentecost23}.\\
\emph{\nth{4} Sunday after Epiphany}, p.~\pageref{epiphany4}.
\end{multicols}
}
\medskip
\hrule

%     TODO Feast of All Saints (11/1) commem of Sunday, Saturday before 1st Sunday of November, 21st - 23rd Sunday after Pentecost or 4th after Epiphany
% \bigskip
% \benedicamusdomino{}
}

%Nov 9: Dedication of Archbasilica of Holy Savior (2nd class)
{
\section{November 9: Dedication of Archbasilica of Holy Savior}
\subtitle{\nth{2} Class}
\subtitle{I \& II Vespers}
\printnote{All from the Common of the Dedication of a Church.}

\def\definevesperspropers{\newcommand{\vrtex}{vrDomumTuam}
\newcommand{\vtranslation}{Holiness becometh thy house, O Lord.}
\newcommand{\rtranslation}{Forever.}

\newcommand{\maganttex}{MagnificatAntiphon2-OQuamMetuendusEst}
\newcommand{\magantinitial}{O}
\newcommand{\maganttranslation}{How dreadful is this place. Surely this is none other but the house of God, and the gate of heaven.}
\def\magsolemn{T}
\def\magoneline{T}
\definemag{6}{F}

  \def\prepsalmfive{\greseteolcustos{manual}}
}
\def\definevesperspropersalt{\newcommand{\vrtex}{vrHaecEstDomusDomini}
\newcommand{\vtranslation}{This is the house of the Lord, strongly built.}
\newcommand{\rtranslation}{It is well founded upon strong rock.}

\newcommand{\maganttex}{MagnificatAntiphon1-Sanctificavit}
\newcommand{\magantinitial}{S}
\newcommand{\maganttranslation}{.}
\def\magsolemn{T}
\definemag{1}{g}
}
\def\vesperspropersnote{At II Vespers:}
\def\vesperspropersaltnote{At I Vespers:}
%\def\premag{\def\noeuouae{T}}
\def\premagverses{\greseteolcustos{manual}}

\printvespers[../CommonOfDedicationOfChurch]{inc-DedicationOfChurch}
\bigskip
\benedicamusdomino[2]{}
}

}

}{

	\clearpage

	\newcommand{\printpsalmwithtranslation}[3]{
	% #1 psalm number
	% #2 ending
	% #3 0 to start on 1st verse, 1 for second
	\setlength{\columnsep}{18pt}
	\setlength{\columnseprule}{.4pt}
	\beginpsalmcols
	\colchunk{\vspace{-12pt}%
	\begin{psalmverses}[#3]
	\vspace{-\baselineskip}%
	\input{psalms/Psalm#1-#2-verses}
	\end{psalmverses}
	}

	\selectlanguage{american}
	\colchunk{\vspace{-12pt}%
	\sloppy
	\begin{psalmverses}
	\vspace{-\baselineskip}%
	\input{psalms/Psalm#1-verses-en}
	\end{psalmverses}
	}
	\selectlanguage{latin}
	\endpsalmcols
}

\renewcommand{\printpsalm}[3]{
  % #1 psalm number
  % #2 ending
  % #3 0 to start on 1st verse, 1 for second
  \setlength{\columnsep}{18pt}
  \setlength{\columnseprule}{.4pt}
  \setlength\multicolsep{0pt}%
  \begin{multicols*}{2}
  \begin{psalmverses}[#3]
  \input{psalms/Psalm#1-#2-verses}
  \end{psalmverses}
  \end{multicols*}
}



\newcommand{\printpsalmtranslation}[3]{
  % #1 psalm number
  % #2 ending
  % #3 0 to start on 1st verse, 1 for second
  \setlength{\columnsep}{18pt}
  \setlength{\columnseprule}{.4pt}
  \setlength\multicolsep{0pt}%
  \begin{multicols*}{2}
  \begin{psalmverses}[#3]
  \input{psalms/Psalm#1-verses-en}
  \end{psalmverses}
  \end{multicols*}
}

\def\beginchaptercols{\begin{parcolumns}[rulebetween]{2}}
\newcommand{\printchapternew}[4]{
	% #1 ref
	% #2 latin first words
	% #3 latin remaining
	% #4 english text
	{\noindent\hspace{3em}Chapter.\hfill\emph{#1}\hspace{3em}}
    \beginchaptercols{}
    \colchunk{\sloppy\dropcap{latin}{#2} #3\hfill{}\Rbar{}~Deo~grátias.}
    \colchunk{\sloppy \dropcap{american}{#4}}
    \end{parcolumns}
}
\newcommand{\printvrwithtranslation}{
    {\normalsize
    \ifx\beginvrcols\undefined\def\beginvrcols{\begin{parcolumns}[rulebetween]{2}}\fi
    \beginvrcols
    \colchunk{
      \par\vspace{-\baselineskip}\noindent\selectlanguage{latin}%
      \Vbar{}~\vlatin{}
    }
    \colchunk{%
      \par\vspace{-\baselineskip}\noindent\selectlanguage{american}%
      \Vbar{}~\vtranslation{}
    }%
    \colplacechunks%
    \ifx\rlatin\undefined\else
    \colchunk{
      \par\vspace{-\baselineskip}\noindent\selectlanguage{latin}%
      \Rbar{}~\rlatin{}%
    }
    \colchunk{%
      \par\vspace{-\baselineskip}\noindent\selectlanguage{american}%
      \Rbar{}~\rtranslation{}%
    }%
    \colplacechunks%
    \fi
    \end{parcolumns}
    }
}
{
  \label{sunday}
  \chapter{Sunday at Terce}
  \phantomsection
  \sectionmark{Sunday at Terce throughout the year}
  \addcontentsline{toc}{section}{Sunday at Terce throughout the year}

  {\centering Oratio ante officium.\par}
  {\centering \emph{Aperi. Pater. Ave.}\par}
  \begin{columns}
  \colchunk{\selectlanguage{latin}Aperi, Domine, os meum ad benedicendum nomen
sanctum tuum: munda quoque cor meum ab omnibus
vanis, perversis et alienis cogitationibus; intellectum
illumina, affectum inflamma, ut digne, attente ac
devote hoc Officium recitare valeam, et exaudiri
merear ante conspectum divinae Majestatis tuae. Per
Christum Dóminum nostrum. \Rbar{}~Amen.}
  \colchunk{\selectlanguage{american}Open, O Lord, my mouth to bless Thy holy name;
cleanse my heart from all vain, evil, and
wandering thoughts; enlighten my understanding
and kindle my affections; that I may worthily,
attentively, and devoutly say this Office, and so
deserve to be heard before the presence of Thy
divine Majesty. Through Christ our Lord. \Rbar{}~Amen.}
  \colplacechunks{}
  \colchunk{\selectlanguage{latin}Domine, in unione illius divinae intentionis, qua ipse
in terries laudes Deo persolvisti, hanc tibi
Horam persolvo.}
  \colchunk{\selectlanguage{american}O Lord, in union with that divine intention, with
which Thou didst praise God upon earth, I render
this Hour to Thee.}
  \colplacechunks{}
  \end{columns}
  \medskip

  %\oldneedspace{6\baselineskip}
  \emph{The Ferial Tone is used except before Pontifical Mass.}

  \printgabc{\Vbar}{}{D}{deus-in-adjutorium}\label{deusinadjutorium}
  \medskip

  %\oldneedspace{6\baselineskip}
  \emph{The Festal Tone is used only before Pontifical Mass.}

  {\def\gabcfolder{..}
    \printgabc{\Vbar}{}{D}{DeusInAdjutorium_laustibi}\label{deusinadjutorium-festal}
  }

  \medskip\vfill
  {\centering Hymn.\par}%
  %\vspace{-0.5\baselineskip}
  \def\dotting{\leaders\hbox to 1em{\hfil.\hfil}\hfill}
  %\def\dotting{\hfill}
  \setlength\parsep{0pt}%
  \setlength\topsep{0pt}%
  \setlength\partopsep{0pt}%
  \setlength\multicolsep{0pt}%
  \begin{multicols}{2}%
  \noindent{}\emph{In Advent},\dotting p.~\pageref{hymn-advent}\\
  \emph{Christmas},\dotting p.~\pageref{hymn-christmas}\\
  \emph{Epiphany},\dotting p.~\pageref{hymn-epiphany}\\
  \emph{Holy Family},\dotting p.~\pageref{hymn-holyfamily}\\
  \emph{In Lent},\dotting p.~\pageref{hymn-lent}\\
  \emph{In Passiontide},\dotting p.~\pageref{hymn-passiontide}\\
  \emph{In Paschaltide},\dotting p.~\pageref{hymn-paschaltide}\\
  \emph{Ascension},\dotting p.~\pageref{hymn-ascension}\\
  \emph{Pentecost},\dotting p.~\pageref{hymn-pentecost}\\
  \emph{Christ the King},\dotting p.~\pageref{hymn-oct-last-sunday}%\\
  %\emph{Corpus Christi},\dotting p.~\pageref{hymn-corpus-christi}\\
  %\emph{Sacred Heart},\dotting p.~\pageref{hymn-sacred-heart}
  \end{multicols}%

  \bigskip
  %\oldneedspace{4\baselineskip}
  {\centering 1.~On Ordinary Sundays.\par}\label{hymn-sunday}

  \printgabc{Hymn.}{2.}{N}{hy--nunc_sancte_nobis_(in_dominicis_per_annum)--solesmes_1961}
  \bigskip

  \oldneedspace{4\baselineskip}
  {\centering 2.~On Solemn Feasts.\par}\label{hymn-feast}

  \printgabc{Hymn.}{8.}{N}{hy--nunc_sancte_nobis_(in_festis)--solesmes_1961}
  \bigskip

  \oldneedspace{4\baselineskip}\vfill
  {\centering 3.~On Feasts of the Blessed Virgin Mary.\par}\label{hymn-bvm}

  \printgabc{Hymn.}{2.}{N}{hy--nunc_sancte_nobis_(in_festis_bmv)--solesmes_1961}

  \begin{multicols}{2}
  \begin{psalmverses}
\item Come Holy Ghost, who ever One
Art with the Father and the Son,
It is the hour, our souls possess
With thy full flood of holiness.

\item Let flesh and heart and lips and mind
Sound forth our witness to mankind;
And love light up our mortal frame,
Till others catch the living flame.

\item Almighty Father, hear our cry,
Through Jesus Christ, our Lord most High,
Who, with the Holy Ghost and thee,
Doth live and reign eternally.
Amen.
  \end{psalmverses}
  \end{multicols}

\bigskip

% print antiphon
%    \printgabc[\preant]{\antlineone}{\antlinetwo}{\antinitial}{\anttex}
  \printgabc{Ant.}{2. D}{A}{an--alleluia._(sund._at_terce)--solesmes}
% print translation
  \translation[]{Alleluia, Lead me into the path of thy commandments, alleluia, alleluia.}

  { \def\printrepeatantiphon{\emph{Repeat antiphon, \emph{p.~\pageref{ant2-sunday}.}}}
    \printtercepsalms{2D}%
  }

  \printtercepsalmstranslation{}

% \printpsalm{118.3}{2D}{1}
% \printpsalmtitle{118. IV.}
% \printpsalm{118.4}{2D}{0}
% \printpsalmtitle{118. V.}
% \printpsalm{118.5}{2D}{0}

% print antiphon
%    \printgabc[\preant]{\antlineone}{\antlinetwo}{\antinitial}{\anttex}
  \bigskip\oldneedspace{8\baselineskip}\label{ant2-sunday}
  \printgabc{Ant.}{2. D}{A}{an--alleluia._(sund._at_terce)--solesmes}
% print translation
  %\translation[]{Alleluia, Lead me into the path of thy commandments, alleluia, alleluia.}

\medskip\oldneedspace{6\baselineskip}
\printchapternew{1. John 4.}{Deus cáritas}{est~:~\dag{} et qui manet in caritáte, in Deo manet,~* et Deus in eo.}{God is charity: and he that abideth in charity, abideth in God, and God in him.}

\medskip\label{shortresp-sunday}
\printgabc{Short}{Resp.}{I}{re--inclina_cor_meum--solesmes}

\translation[]{\Vbar{}~Incline my heart into thy testimonies.
\Rbar{}~Incline\dots{}
\Vbar{}~Turn away my eyes that they may not behold vanity: quicken me in thy way.
\Rbar{}~Into thy testimonies.
\Vbar{}~Glory be\dots{}
\Vbar{}~Incline\dots{}}

\bigskip\label{vr-sunday}
\gresetinitiallines{0}
\gregorioscore{vr-ego-dixi}
\newlength{\myhwidth}
\settowidth{\myhwidth}{tib}
\begin{nstabbing}
\>\Rbar{}~Sana ánimam meam, quia peccávi \>\hspace{-\myhwidth}tibi.
\end{nstabbing}

\translation[]{\Vbar{}~I said: O Lord, be thou merciful to me.\\
\Rbar{}~Heal my soul, for I have sinned against thee.}

\bigskip
%\def\beginvrcols{\begin{parcolumns}[rulebetween,colwidths={1=0.48\linewidth}]{2}}
\def\vlatin{Dóminus vobíscum.}
\def\rlatin{Et cum spíritu tuo.}
\def\vtranslation{The Lord be with you.}
\def\rtranslation{And with thy spirit.}
{\centering Collect.\par}
\printvrwithtranslation{}

\bigskip
    %\def\dotting{\leaders\hbox to 1em{\hfil.\hfil}\hfill}
    \def\dotting{\hfill}
    \begin{multicols}{2}%
    %\printnote{Advent,\dotting \emph{p.~\pageref{advent1-collect} to \pageref{advent4-collect}}}

    %\printnote{Christmas,\dotting \emph{p.~\pageref{christmas-eve-collect} to \pageref{holyfamily-collect}}}

    \noindent\emph{Time after Epiphany},\dotting p.~\pageref{epiphany2-collect} to \pageref{epiphany6-collect}\\
    \emph{Septuagesima},\dotting p.~\pageref{septuagesima-collect} to \pageref{quinquagesima-collect}\\
    %\emph{Easter to Pentecost},\dotting p.~\pageref{easter1-collect} to \pageref{pentecost-collect}\\
    \emph{Time after Pentecost},\dotting p.~\pageref{pentecost2-collect} to \pageref{pentecost24-collect}
    \end{multicols}

\bigskip
\printvrwithtranslation{}
\bigskip

\printnote{Except before Pontifical Mass, the Officiant chants the \emph{\Vbar{}~Benedicámus Dómino} to the following tone.}
\medskip
\gresetinitiallines{0}\label{benedicamus-domino-sunday}
\gregorioscore{vr-benedicamus-domino}
\bigskip

%\def\beginvrcols{\begin{parcolumns}[rulebetween,colwidths={1=0.44\linewidth}]{2}}
\def\vlatin{Fidélium ánimæ per misericórdiam Dei requiéscant in pace.\hfill{}Amen.}
\let\rlatin=\undefined
\def\vtranslation{May the souls of the faithful departed through the mercy of God rest in peace.\hfill{}Amen.}
\let\rtranslation=\undefined
\printvrwithtranslation{}

\bigskip
\printnote{If Pontifical Mass is to follow, the \emph{\Vbar{}~Benedicámus Dómino} is sung by the Cantors in one of the following tones; and the \emph{\Vbar{}~Fidelium} is not said.}
\medskip
{
\newcommand{\printbenedicamusdomino}[2]{
	\greseteolcustos{manual}
	\gresetinitiallines{1}
	\def\annot{\small{#1}}
	\alsetinitialspacing{B}
	\gregorioscore{#2}
  \greseteolcustos{auto}
}

\vfil
\grechangestaffsize{15}
\oldneedspace{4\baselineskip}\label{benedicamusdomino-1}
{{\centering \bfseries 1.~On feasts of the I class.\\}
\smallskip
\def\breakbeforeresp{T}
\printbenedicamusdomino{5.}{../BenedicamusDomino/or--benedicamus_(in_festis_i_classis_ad_laudes)--solesmes}
\vfil
}
\oldneedspace{4\baselineskip}\label{benedicamusdomino-2}
{{\centering \bfseries 2.~On feasts of the II class.\\}
\smallskip
{\def\breakbeforeresp{T}
\printbenedicamusdomino{2.}{../BenedicamusDomino/or--benedicamus_domino_laudes_--solesmes_1961}
}
}
\vfil

\oldneedspace{4\baselineskip}\label{benedicamusdomino-mary}
\gdef\benedicamusdominonamemary{3}
{{\centering \bfseries 3.~On feasts of the Blessed Virgin.\\}
\smallskip
%\def\breakbeforeresp{T}
\printbenedicamusdomino{1.}{../BenedicamusDomino/BenedicamusDomino_blessedVirgin}}
\vfil

\needspace{6\baselineskip}
\label{benedicamusdomino-sunday}
\gdef\benedicamusdominonamesunday{4}
{{\centering \bfseries 4.~On Sundays during the Year\\and Septuagesima, Sexagesima, and Quinquagesima.\\}
\smallskip
%\def\breakbeforeresp{T}
\printbenedicamusdomino{1.}{../BenedicamusDomino/BenedicamusDomino_Sundays}}

\vfil
\label{benedicamusdomino-lent}\label{benedicamusdomino-advent}
\gdef\benedicamusdominonamelent{5}
\gdef\benedicamusdominonameadvent{\benedicamusdominonamelent}
{{\centering \bfseries 5.~On Sundays of Advent and Lent.\\}
\smallskip
%\def\breakbeforeresp{T}
\printbenedicamusdomino{6.}{../BenedicamusDomino/BenedicamusDomino_SundaysOfAdventAndLent}
\ifthenelse{\boolean{birmingham}}{
	\medskip
	\emph{\normalsize or from \emph{Mass XVII}:}

	\smallskip
	\printbenedicamusdomino{6.}{../BenedicamusDomino/ky--benedicamus_xviia--solesmes}
}{}
}
\vfil

\label{benedicamusdomino-easter}
\gdef\benedicamusdominonameeaster{6}
{{\centering \bfseries 6.~On Sundays of Paschal Time.\\}
\smallskip
%\def\breakbeforeresp{T}
\printbenedicamusdomino{7.}{../BenedicamusDomino/BenedicamusDomino_SundaysOfPaschalTime}}

}

}

	{
	\label{advent}
	\chapter{Advent}
	\section{First Sunday of Advent}

  {
	  \def\printhymn{%
		{\centering Hymn.\par}\label{hymn-advent}

	  	{\def\gabcfolder{.}
	  	\printgabc{1.}{}{N}{hy--nunc_sancte_nobis_(in_adventu)--solesmes_1961}}

		\bigskip
	  }
    \def\printshortresp{%
    \label{shortresp-advent}%
    {\def\gabcfolder{.}
    \printgabc{Short}{Resp.}{V}{re--veni_ad_liberandum--solesmes}}

    \translation[]{\Vbar{}~Come to my rescue O God, Lord of hosts.
    \Rbar{}~Come\dots{}
    \Vbar{}~Show us thy face, and we shall be saved.
    \Rbar{}~O God, Lord of hosts.
    \Vbar{}~Glory be to the Father, and to the Son, and to the Holy Ghost.
    \Rbar{}~Come\dots{}
    }

    \bigskip
    \gresetinitiallines{0}
    \gregorioscore{vr-timebunt-gentes}
    \let\myhwidth\relax
    \let\myhhwidth\relax
    \newlength{\myhwidth}
    \settowidth{\myhwidth}{tu}
    \newlength{\myhhwidth}
    \settowidth{\myhhwidth}{n}
    \addtolength{\myhhwidth}{-\myhwidth}
    \def\myhspace{\hspace{1.4ex}}
    \begin{nstabbing}
    \>\Rbar{}~Et \myhspace{} omnes \myhspace{} reges \myhspace{} terræ \myhspace{} glóriam \>\hspace{\myhhwidth}tuam.
    \end{nstabbing}

    \translation[]{\Vbar{}~And the Gentiles shall fear thy name, O Lord.\\
    \Rbar{}~And all the kings of the earth thy glory.}

    \bigskip
    }

	  \printterce[../Advent1]{inc-Advent1}
  }

  \def\printhymn{
  	\noindent\emph{Hymn}, page \pageref{hymn-advent}

  	\medskip
  }
  \def\printshortresp{
    \noindent\emph{Short Resp. \emph{Veni ad liberándum nos}, p.} \pageref{shortresp-advent}, \Vbar{}~Timébunt gentes. 
  }
  \section{Second Sunday of Advent}
  \printterce[../Advent2]{inc-Advent2-Vespers2}

  \section{Third Sunday of Advent}
  \printterce[../Advent3]{inc-Advent3}

  \section{Fourth Sunday of Advent}
  \printterce[../Advent4]{inc-Advent4}

}

	{
	\label{christmas}
	\chapter{Christmas}
	\section{Christmas Eve}
  \def\printbenedicamusdomino{
    \noindent\emph{Benedicamus Domino}, p. \pageref{benedicamus-domino-sunday}.
  }
  {
    \def\printhymn{%
    	, \emph{Hymn}, page \pageref{hymn-advent}

    	\medskip
    }
    \def\printshortresp{%
      \label{shortresp-christmas-eve}%
      {\def\gabcfolder{.}
      \printgabc{Short}{Resp.}{H}{re--hodie_scietis--solesmes_1961}}

      \translation[]{\Vbar{}~This day ye shall know that the Lord cometh.
      \Rbar{}~This\dots{}
      \Vbar{}~And in the morning, then ye shall see His glory.
      \Rbar{}~That the Lord cometh.
      \Vbar{}~Glory be to the Father, and to the Son, and to the Holy Ghost.
      \Rbar{}~This\dots{}
      }

      \bigskip
      \gresetinitiallines{0}\label{vr-christmas-eve}
      \gregorioscore{vr-constantes-estote}
      \let\myhwidth\relax
      \let\myhhwidth\relax
      \newlength{\myhwidth}
      \settowidth{\myhwidth}{v} %this is what precedes the last vowel in the response
      \newlength{\myhhwidth}
      \settowidth{\myhhwidth}{t} %this is what preceded the last vowel in the verse
      \addtolength{\myhhwidth}{-\myhwidth}
      \def\myhspace{\hspace{1.4ex}}
      \begin{nstabbing}
      \>\Rbar{}~Vidébitis auxílium Dómini super \>\hspace{\myhhwidth}vos.
      \end{nstabbing}

      \translation[]{\Vbar{}~Stand ye still.\\
      \Rbar{}~And ye shall see the salvation of the Lord with you.}

      \bigskip
    }
    \printterce{inc-christmas-eve}{christmas-eve}
  }
  {
    \newcommand{\printrefshymn}[1]{%
      \def\dotting{\hfill%\leaders\hbox to 1em{\hfil.\hfil}\hfill
        }%
      \begin{multicols}{2}%
      \noindent{}Christmas \& Sunday,\dotting \emph{below}\\
      Circumcision,\dotting \emph{p.~\pageref{circumcision-#1}}\\
      Holy Name,\dotting \emph{p.~\pageref{holyname-#1}}
      \end{multicols}%
      \smallskip
    }
    \def\printhymn{

      {\centering Hymn.\par}\label{hymn-christmas}

      {\def\gabcfolder{.}
       \printgabc{8.}{}{N}{hy--nunc_sancte_nobis_(in_nativitate_domini)--solesmes_1961}
      }

      \printrefshymn{ant}
      \bigskip
    }
    \newcommand{\printrefs}[1]{%
      \def\dotting{\hfill%\leaders\hbox to 1em{\hfil.\hfil}\hfill
        }%
      \begin{multicols}{2}%
      \noindent{}Christmas,\dotting \emph{below}\\
      Sunday within the octave,\dotting \emph{p.~\pageref{christmas-sunday-#1}}
      \end{multicols}%
      \smallskip
    }
    \newcommand{\anttwotex}{an--genuit_puerpera_regem--solesmes_1961}
    \newcommand{\anttwoinitial}{G}
    \newcommand{\anttwotranslation}{The Mother brought forth the King, Whose name is called The Eternal; the joy of a Mother was hers, remaining a Virgin unsullied; neither before nor henceforth hath there been or shall be such another, alleluia.}
    \definepsalm{2}{109}{2}{D}

    \def\printshortresp{%
      \label{shortresp-christmas}%
      {\def\gabcfolder{.}
      \printgabc{Short}{Resp.}{V}{rb--verbum_caro_factum_est--solesmes_1961}}

      \translation[]{\Vbar{}~The Word was made flesh. Alleluia, Alleluia.
      \Rbar{}~The Word\dots{}
      \Vbar{}~And dwelt among us.
      \Rbar{}~Alleluia, Alleluia.
      \Vbar{}~Glory be to the Father, and to the Son, and to the Holy Ghost.
      \Rbar{}~The Word\dots{}
      }

      \bigskip
      \gresetinitiallines{0}\label{vr-christmas}
      \gregorioscore{vr-ipse-invocabit}
      \let\myhwidth\relax
      \let\myhhwidth\relax
      \newlength{\myhwidth}
      \settowidth{\myhwidth}{allelui} %this is what precedes the last vowel in the response
      \newlength{\myhhwidth}
      \settowidth{\myhhwidth}{i} %this is what preceded the last vowel in the verse
      \addtolength{\myhhwidth}{-\myhwidth}
      \def\myhspace{\hspace{0.3ex}}
      \begin{nstabbing}
      \>\Rbar{}~Pater \myhspace{} meus \myhspace{} es \myhspace{} tu, \>\hspace{\myhhwidth}allelúia.
      \end{nstabbing}

      \translation[]{\Vbar{}~He shall cry unto Me, Alleluia.\\
      \Rbar{}~Thou art My Father, Alleluia.}

      \bigskip
    }
    \def\prechapter{\printrefs{chapter}}
    \section{Christmas}
    \printterce{../Christmas/inc-Christmas-Vespers2}{christmas}
  }

  \def\printshortresp{
    \noindent\emph{Short Resp. \emph{Verbum caro}}, p. \pageref{shortresp-christmas}, \Vbar{}~Ipse invocabit, p. \pageref{vr-christmas}.
  }
  {
    \def\printhymn{%
      , \emph{Hymn}, page \pageref{hymn-sunday}%
      , Ant. \emph{Genuit puerpera Regem}, \pageref{christmas-ant}

      \medskip
    }

    \section{Sunday within the octave}
    \printterce{../Christmas/inc-SundayWithinOctaveOfChristmas-Vespers2}{christmas-sunday}
  }
  \def\printhymn{%
    , \emph{Hymn}, page \pageref{hymn-christmas}%

    \medskip
  }
  {
    \section{Octave of the Nativity}
    \printterce[../ChristmasOctave-Circumcision]{inc-Circumcision-Vespers-common}{circumcision}
  }
  {
    % holy name
    % proper ant, chapter from vespers, short resp
    \newcommand{\antonetex}{an--omnis_qui_invocaverit--solesmes}
\newcommand{\antoneinitial}{O}
\newcommand{\antonetranslation}{All who shall call on the name of the Lord shall be saved.}
\definepsalm{1}{109}{8}{G}

\newcommand{\anttwotex}{an--sanctum_et_terribile--solesmes}
\newcommand{\anttwoinitial}{S}
\newcommand{\anttwotranslation}{Holy and terrible is His name; the fear of the Lord is the beginning of wisdom.}
\definepsalm{2}{110}{5}{a}

\newcommand{\antthreetex}{an--ego_autem_in_domino--solesmes}
\newcommand{\antthreeinitial}{E}
\newcommand{\antthreetranslation}{Yet I will rejoice in the Lord and exult in the God of my salvation.}
\definepsalm{3}{111}{3}{a2}

\newcommand{\antfourtex}{an--a_solis_ortu--solesmes}
\newcommand{\antfourinitial}{A}
\newcommand{\antfourtranslation}{From the rising of the sun to its setting, the Lord's Name is to be praised.}
\definepsalm{4}{112}{4}{E}

\newcommand{\antfivetex}{an--sacrificabo_hostiam--solesmes}
\newcommand{\antfiveinitial}{S}
\newcommand{\antfivetranslation}{I will offer the sacrifice of praise, and will call upon the Name of the Lord.}
\def\psalmclef{2} %if I didn't use definepsalm, I would define this as psalmcleffive
\definepsalm{5}{115}{8}{c}
\let\psalmclef=\undefined % but then I have to undefine it!
%\renewcommand{\psalmcleffive}{2}

\newcommand{\chaptertext}{\dropcap{latin}{Fratres~: Christus humiliávit semetípsum, factus obédiens usque ad mortem, mortem autem} \textbf{crú}\-cis.~\dag{} Propter quod et Deus exaltávit illum, et donávit illi nomen quod est super \emph{om\-ne} \textbf{nó}\-men~:~* ut in nómine Jesu omne genu flec\textbf{tá}tur.}
\newcommand{\chaptertranslation}{Christ humbled Himself, becoming obedient unto death, even the death of the cross. For which cause God also hath exalted Him, and given Him a name which is above all names: That in the name of Jesus every knee should bow.}

\newcommand{\hymnlinetwo}{1.}
\newcommand{\hymntex}{hy--jesu_dulcis_memoria--solesmes}
\newcommand{\hymninitial}{J}
\newcommand{\hymntranslation}{
\item Jesu, the very thought of Thee
With sweetness fills my breast;
But sweeter far Thy face to see,
And in Thy presence rest.

\item Nor voice can sing, nor heart can frame,
Nor can the memory find,
A sweeter sound than Thy blest Name,
O Saviour of mankind!

\item O Hope of every contrite heart,
O Joy of all the meek,
To those who fall, how kind Thou art!
How good to those who seek!

\item But what to those who find? Ah! this
Nor tongue nor pen can show:
The love of Jesus, what it is
None but His loved ones know.

\item Jesu, our only joy be Thou,
As Thou our prize wilt be;
Jesu, be Thou our glory now,
And through eternity.
Amen.%\grechangestaffsize{15}
}

\newcommand{\vrtex}{vrSitNomenDominiBenedictum}
\newcommand{\vtranslation}{Blessed be the Name of the Lord, alleluia.}
\newcommand{\rtranslation}{From this time forth, and for evermore, alleluia.}

\newcommand{\collect}{Deus, qui unigénitum Fílium tuum constituísti humáni géneris Salvatórem, et Jesum vocári jussísti~:~\dag{} concéde propítius; ut cujus sanctum nomen venerámur in terris,~* ejus quoque aspéctu perfruámur in cælis. Per eúmdem Dóminum.}
\newcommand{\collecttranslation}{O God, Who hast appointed Thine Only-begotten Son to be the Saviour of mankind, and commanded that He should be called Jesus, mercifully grant that we, who here on earth do worship His Holy Name, may be made glad in heaven by His Presence. Through the same our Lord.}

    \renewcommand{\anttwotex}{an--scitote_quia_dominus--solesmes_1961}
    \renewcommand{\anttwoinitial}{S}
    \renewcommand{\anttwotranslation}{}
    \definepsalm{2}{109}{3}{a}

    \def\printshortresp{%
      \label{shortresp-holyname}%
      {\def\gabcfolder{.}
      \printgabc{Short}{Resp.}{S}{rb--sit_nomen--solesmes}}

      \translation[]{\Vbar{}~Blessed be the Name of the Lord, Alleluia, alleluia.
        \Rbar{}~Blessed\dots{}
        \Vbar{}~From this time forth, and for evermore.
        \Rbar{}~Alleluia, alleluia.
        \Vbar{}~Glory be to the Father, and to the Son, and to the Holy Ghost.
        \Rbar{}~Blessed\dots{}
      }

      \bigskip
      \gresetinitiallines{0}\label{vr-holyname}
      \gregorioscore{vr-afferte-domino}
      \let\myhwidth\relax
      \let\myhhwidth\relax
      \newlength{\myhwidth}
      \settowidth{\myhwidth}{allelui} %this is what precedes the last vowel in the response
      \newlength{\myhhwidth}
      \settowidth{\myhhwidth}{i} %this is what preceded the last vowel in the verse
      \addtolength{\myhhwidth}{-\myhwidth}
      \def\myhspace{\hspace{0.5ex}}
      \begin{nstabbing}
      %\>\Rbar{}~Pater \myhspace{} meus \myhspace{} es \myhspace{} tu, \>\hspace{\myhhwidth}allelúia.
      \>\Rbar{}~Afférte \myhspace{}Dómino\myhspace{} glóriam\myhspace{} nómini\myhspace{} ejus, \>\hspace{\myhhwidth}allelúia.
      \end{nstabbing}

      \translation[]{\Vbar{}~Give unto the Lord glory and honour, alleluia.\\
        \Rbar{}~Give unto the Lord the glory due unto His Name, alleluia.}

      \bigskip
    }
    \section{Holy Name}
    \def\gabcfolder{.}
    \printterce{}{holyname}
  }
  {
    % epiphany
    % proper hymn, ant & chapter from vespers, short resp
    \def\printhymn{

      {\centering Hymn.\par}\label{hymn-epiphany}

      {\def\gabcfolder{.}
       \printgabc{8.}{}{N}{hy--nunc_sancte_nobis_(in_epiphania_domini)--solesmes_1961}
      }

      %\printrefshymn{ant}
      \bigskip
    }
    \def\printshortresp{%
      \label{shortresp-epiphany}%
      {\def\gabcfolder{.}
      \printgabc{Short}{Resp.}{R}{re--reges_tharsis--solesmes_1961}}

      \translation[]{\Vbar{}~The kings of Tarshish and of the isles shall bring presents. Alleluia, Alleluia.
        \Rbar{}~The kings\dots{}
        \Vbar{}~The kings of Arabia and Saba shall offer gifts.
        \Rbar{}~Alleluia, Alleluia.
        \Vbar{}~Glory be to the Father, and to the Son, and to the Holy Ghost.
        \Rbar{}~The kings\dots{}
      }

      \bigskip
      \gresetinitiallines{0}\label{vr-epiphany}
      \gregorioscore{vr-omnes-de-saba}
      \let\myhwidth\relax
      \let\myhhwidth\relax
      \newlength{\myhwidth}
      \settowidth{\myhwidth}{allelui} %this is what precedes the last vowel in the response
      \newlength{\myhhwidth}
      \settowidth{\myhhwidth}{i} %this is what preceded the last vowel in the verse
      \addtolength{\myhhwidth}{-\myhwidth}
      \def\myhspace{\hspace{0.1ex}}
      \begin{nstabbing}
      %\>\Rbar{}~Pater \myhspace{} meus \myhspace{} es \myhspace{} tu, \>\hspace{\myhhwidth}allelúia.
      \>\Rbar{}~Aurum et thus deferéntes, \>\hspace{\myhhwidth}allelúia.
      \end{nstabbing}

      \translation[]{\Vbar{}~All they from Saba shall come. Alleluia.\\
        \Rbar{}~They shall bring gold and incense. Alleluia.}

      \bigskip
    }
    \section{Epiphany}
    \printterce[../Epiphany]{inc-Epiphany-vespers}{epiphany}
  }
  {
    % holy family
    % ant & chapter from vespers, proper short resp
    % slightly proper hymn (rest from epiphany)
    \def\printhymn{

      {\centering Hymn.\par}\label{hymn-holyfamily}

      {\def\gabcfolder{.} %TODO: this has the correct verse 3 but first 2 verses are wrong!
       \printgabc{8.}{}{N}{hy--te_lucis_ante_terminum_(holy_family)--solesmes_1961}
      }

      %\printrefshymn{ant}
      \bigskip
    }
    \def\printshortresp{%
      \label{shortresp-holyfamily}%
      {\def\gabcfolder{.}
      \printgabc{Short}{Resp.}{P}{re--propter_nos--solesmes_1961}}

      \translation[]{\Vbar{}~For our sake he became poor, Being rich.
        \Rbar{}~For our sake\dots{}
        \Vbar{}~That through his poverty we might be rich.
        \Rbar{}~Being rich.
        \Vbar{}~Glory be to the Father, and to the Son, and to the Holy Ghost.
        \Rbar{}~For our sake\dots{}
      }

      \bigskip
      \gresetinitiallines{0}\label{vr-holyfamily}
      \gregorioscore{vr-dominus-vias-suas}
      \let\myhwidth\relax
      \let\myhhwidth\relax
      \newlength{\myhwidth}
      \settowidth{\myhwidth}{ej} %this is what precedes the last vowel in the response
      \newlength{\myhhwidth}
      \settowidth{\myhhwidth}{n} %this is what preceded the last vowel in the verse
      \addtolength{\myhhwidth}{-\myhwidth}
      \def\myhspace{\hspace{0.1ex}}
      \begin{nstabbing}
      %\>\Rbar{}~Pater \myhspace{} meus \myhspace{} es \myhspace{} tu, \>\hspace{\myhhwidth}allelúia.
      \>\Rbar{}~Et \myhspace{} ambulábimus \myhspace{} in \myhspace{} sémitis \>\hspace{\myhhwidth}ejus.
      \end{nstabbing}

      \translation[]{\Vbar{}~The Lord will teach us his ways.\\
        \Rbar{}~And we will walk in his paths.}

      \bigskip
    }
    \section{Holy Family}
    \printterce[../HolyFamily]{inc-HolyFamily-Vespers2}{holyfamily}
  }
}

	\chapter{Proper of the Time -- After Epiphany}
{
\newcommand{\benedicamusdomino}[1][sunday]{
  \benedicamusdominomaster{#1}
}
\def\printhymnnote{}
\def\printcommonvespers{
	\subtitle{\nth{2} Class, Green}
	\printnote{From \emph{Vespers of Sundays throughout the year}, page \pageref{sundayvespers}.\\}
}

{
\section{\nth{2} Sunday after Epiphany}
\label{epiphany2}
\printcommonvespers{}
\printvespersmag[../TimeAfterEpiphany]{inc-VespersMagnificatEpiphany2}

\bigskip
\benedicamusdomino{}
}

{
\section{\nth{3} Sunday after Epiphany}
\label{epiphany3}
\printcommonvespers{}
\printvespersmag[../TimeAfterEpiphany]{inc-VespersMagnificatEpiphany3}

\bigskip
\benedicamusdomino{}
}

{
\section{\nth{4} Sunday after Epiphany}
\label{epiphany4}
\def\precollect{\printvrdirigatur}
\printcommonvespers{}
\printvespersmag[../TimeAfterEpiphany]{inc-VespersMagnificatEpiphany4}

\bigskip
\benedicamusdomino{}
}

{
\section{\nth{5} Sunday after Epiphany}
\label{epiphany5}
\printcommonvespers{}
\printvespersmag[../TimeAfterEpiphany]{inc-VespersMagnificatEpiphany5}

\bigskip
\benedicamusdomino{}
}

{
\section{\nth{6} Sunday after Epiphany}
\label{epiphany6}
\printcommonvespers{}
\def\prevespers{%
  \let\oldthing=\englishmagantiphon
  \def\englishmagantiphon{\oldthing\pagebreak}
}
\printvespersmag[../TimeAfterEpiphany]{inc-VespersMagnificatEpiphany6}

\bigskip
\benedicamusdomino{}
}
}

	{
	\label{septuagesima}
	\chapter{Septuagesima \& Lent}
  \def\printbenedicamusdomino{
    \noindent\emph{Benedicamus Domino}, p. \pageref{benedicamus-domino-sunday}.
  }
  \section{Septuagesima}
  \printterce{inc-septuagesima}{septuagesima}

  \section{Sexagesima}
  \printterce{inc-sexagesima}{sexagesima}

  \section{Quinquagesima}
  \printterce{inc-quinquagesima}{quinquagesima}

  {
    \newcommand{\printrefs}[1]{%
      \def\dotting{\hfill%\leaders\hbox to 1em{\hfil.\hfil}\hfill
        }%
      \begin{multicols}{2}%
      \noindent{}1st Sunday of Lent,\dotting \emph{below}\\
      2nd Sunday of Lent,\dotting \emph{p.~\pageref{lent2-#1}}\\
      3rd Sunday of Lent,\dotting \emph{p.~\pageref{lent3-#1}}\\
      4th Sunday of Lent,\dotting \emph{p.~\pageref{lent4-#1}}
      \end{multicols}%
      \smallskip
    }

	  \def\printhymn{

		  {\centering Hymn.\par}\label{hymn-lent}

	  	{\def\gabcfolder{.}
	  	\printgabc{1.}{}{N}{hy--nunc_sancte_nobis_(in_quadragesima)--solesmes_1961}}

      \printrefs{ant}
		  \bigskip
	  }

    \def\printshortresp{%
    \label{shortresp-lent}%
    {\def\gabcfolder{.}
    \printgabc{Short}{Resp.}{I}{re--ipse_liberavit--solesmes}}

    \translation[]{\Vbar{}~For he hath delivered me from the snare of the hunters.
    \Rbar{}~For he hath\dots{}
    \Vbar{}~And from the sharp word.
    \Rbar{}~From the snare of the hunters.
    \Vbar{}~Glory be to the Father, and to the Son, and to the Holy Ghost.
    \Rbar{}~For he hath\dots{}
    }

    \bigskip
    \gresetinitiallines{0}\label{vr-lent}
    \gregorioscore{vr-scapulis-suis}
    \let\myhwidth\relax
    \let\myhhwidth\relax
    \newlength{\myhwidth}
    \settowidth{\myhwidth}{speráb} % text in last word before last vowel of response
    \newlength{\myhhwidth}
    \settowidth{\myhhwidth}{b} % text in last syllable before vowel of versicle
    \addtolength{\myhhwidth}{-\myhwidth}
    \def\myhspace{\hspace{1ex}}
    \begin{nstabbing}
    %\>\Rbar{}~Et \myhspace{} omnes \myhspace{} reges \myhspace{} terræ \myhspace{} glóriam \>\hspace{\myhhwidth}tuam.
    \>\Rbar{}~Et \myhspace{} sub \myhspace{} pennis \myhspace{} ejus \>\hspace{\myhhwidth}sperábis.
    \end{nstabbing}

    \translation[]{\Vbar{}~He will overshadow thee with his shoulders.\\
    \Rbar{}~And under his wings thou shalt trust.}

    \printrefs{collect}
    \bigskip
    }

    \section{First Sunday of Lent}
	  \printterce{inc-Lent1}{lent1}
  }

  \def\printhymn{%
  	, \emph{Hymn}, page \pageref{hymn-lent}

  	\medskip
  }
  \def\printshortresp{
    \noindent\emph{Short Resp. \emph{Ipse liberavit me}}, p. \pageref{shortresp-lent}, \Vbar{}~Scapulis suis, p. \pageref{vr-lent}.
  }
  \section{Second Sunday of Lent}
  \printterce{inc-Lent2}{lent2}

  \section{Third Sunday of Lent}
  \printterce{inc-Lent3}{lent3}

  \section{Fourth Sunday of Lent}
  \printterce{inc-Lent4}{lent4}

  {
    \newcommand{\printrefs}[1]{%
      \def\dotting{\hfill%\leaders\hbox to 1em{\hfil.\hfil}\hfill
        }%
      \begin{multicols}{2}%
      \noindent{}1st Sunday of the Passion,\dotting \emph{below}\\
      2nd Sunday of the Passion,\dotting \emph{p.~\pageref{passion2-#1}}
      \end{multicols}%
      \smallskip
    }

    \def\printhymn{

      {\centering Hymn.\par}\label{hymn-passiontide}

      {\def\gabcfolder{.}
      \printgabc{2.}{}{N}{hy--nunc_sancte_nobis_(in_tempore_passionis)--solesmes_1961}}

      \printrefs{ant}
      \bigskip
    }

    \def\printshortresp{%
    \label{shortresp-passiontide}%
    {\def\gabcfolder{.}
    \printgabc{Short}{Resp.}{E}{re--erue_a_frama--solesmes}}

    \translation[]{\Vbar{}~Deliver from the sword, O God, my soul.
    \Rbar{}~Deliver\dots{}
    \Vbar{}~My only one from the hand of the dog.
    \Rbar{}~O God, my soul.
    \Rbar{}~Deliver, * O God, my soul from the sword.For he hath delivered me from the snare of the hunters.
    }

    \bigskip
    \gresetinitiallines{0}\label{vr-passiontide}
    \gregorioscore{vr-de-ore-leonis}
    \let\myhwidth\relax
    \let\myhhwidth\relax
    \newlength{\myhwidth}
    \settowidth{\myhwidth}{me} % text in last word before last vowel of response
    \newlength{\myhhwidth}
    \settowidth{\myhhwidth}{n} % text in last syllable before vowel of versicle
    \addtolength{\myhhwidth}{-\myhwidth}
    \def\myhspace{\hspace{1.4ex}}
    \begin{nstabbing}
    %\>\Rbar{}~Et \myhspace{} omnes \myhspace{} reges \myhspace{} terræ \myhspace{} glóriam \>\hspace{\myhhwidth}tuam.
    \>\Rbar{}~Et a córnibus unicórnium humilitátem \>\hspace{\myhhwidth}meam.
    \end{nstabbing}

    \translation[]{\Vbar{}~From the lion's mouth, O Lord, save me.
    \Rbar{}~And my lowness from the horns of the unicorns.}

    \printrefs{collect}
    \bigskip
    }

    \section{First Sunday of the Passion}
    \printterce{inc-Passion1}{passion1}
  }
 
  \def\printhymn{%
    , \emph{Hymn}, page \pageref{hymn-passiontide}

    \medskip
  }
  \def\printshortresp{
    \noindent\emph{Short Resp. \emph{Erue a framea}}, p. \pageref{shortresp-passiontide}, \Vbar{}~De ore leonis, p. \pageref{vr-passiontide}.
  }
  \section{Second Sunday of the Passion}
  \printterce{inc-Passion2}{passion2}
}

	{
\newcommand{\printcommemnote}[1][easter]{\smallskip
\noindent
\printnote{\commemorations{}  Otherwise \benedicamusdominoreference{#1}}
}
{
\chapter{Proper of the Time -- Easter}
\section{Easter Sunday}
\subtitle{\nth{1} Class, White or Gold}
\def\printfullhymn{
    \vspace{-0.5\baselineskip}
    \emph{Chapter, Hymn, and Versicle are all omitted, but the following Antiphon is said :}

    \bigskip
    \def\annot{\small{Ant.}}
    \def\annottwo{\small{\chapterhymnversicleantiphonmode.}}
    \alsetinitialspacing{\chapterhymnversicleantiphoninitial}
    \gregorioscore{\gabcfolder/\chapterhymnversicleantiphontex}
    \translation[]{\chapterhymnversicleantiphontranslation}
    \vspace{-0.5\baselineskip}
    \bigskip
}
\def\chapterreplacement{\bigskip}
\def\begincollectcols{\begin{parcolumns}[rulebetween,colwidths={1=0.45\linewidth}]{2}}
\def\postmag{\vspace{-0.05\baselineskip}}
\printvespers[../Easter]{inc-EasterVespers}
\newcommand{\printbenedicamusdomino}[2]{
	\def\annot{\small{#1}}
	\def\annottwo{}
	\alsetinitialspacing{B}
    \greseteolcustos{manual}
	\gregorioscore{#2}
    \greseteolcustos{auto}
    \bigskip
    \hrule
}
\def\breakbeforeresp{T}
\printbenedicamusdomino{\Vbar}{../BenedicamusDomino/BenedicamusDomino_Easter}
}

\newcommand{\printcommonvespers}[1][2]{
    \subtitle{\nth{#1} Class, \liturgicalcolor{}}
    \deusinadjutorium{}
    \printnote{\emph{Vespers of Sundays in Paschaltide}, p.~\pageref{sundayvespers-easter}.\\}
}
\def\liturgicalcolor{White}
{
\newcommand{\benedicamusdomino}[1][easter]{
  \benedicamusdominomaster{#1}
}

\newcommand{\printhymnnote}{
    \noindent\printnote{Hymn.~\emph{Ad Régias Agni Dapes}, page \pageref{hymn-adregiasagnidapes}.
    \Vbar~\emph{Mane nobíscum}, page \pageref{vr-manenobiscum}.}
}

{
\section{Low Sunday}
\label{easter1}
\printcommonvespers[1]
\def\begincollectcols{\begin{parcolumns}[rulebetween,colwidths={1=0.42\linewidth}]{2}}
\printvespersmag[../TimeAfterEaster]{inc-VespersMagnificatEaster1}

\def\commemorations{If the Feast of the Annunciation has been transferred to the Monday following Low Sunday, First Vespers is commemorated as on page \pageref{annunciation-commem}.  If today is April 30, May 1, or May 2, First Vespers of St Joseph the Worker is commemorated as follows.}
\printcommemnote{}
}

\medskip
\hrule
{
\label{stjoseph-worker-commem}
\def\begincollectcols{\begin{parcolumns}[rulebetween,colwidths={1=0.43\linewidth}]{2}}
\def\vrlinebreak{T}
\printcommemoration[../May1-StJosephWorker]{commemorationStJosephWorker-Vespers1}

\bigskip
\benedicamusdomino{}
}

{
\section{\nth{2} Sunday after Easter}
\label{easter2}
\printcommonvespers{}
\def\precollect{\printvrmanenobiscum}
\printvespersmag[../TimeAfterEaster]{inc-VespersMagnificatEaster2}
\benedicamusdomino{}
}
{
\section{\nth{3} Sunday after Easter}
\label{easter3}
\printcommonvespers{}
\def\begincollectcols{\begin{parcolumns}[rulebetween,colwidths={1=0.44\linewidth}]{2}}
\def\precollect{\printvrmanenobiscum}
\printvespersmag[../TimeAfterEaster]{inc-VespersMagnificatEaster3}
\benedicamusdomino{}
}

{
\section{\nth{4} Sunday after Easter}
\label{easter4}
\printcommonvespers{}
\def\precollect{\printvrmanenobiscum}
\printvespersmag[../TimeAfterEaster]{inc-VespersMagnificatEaster4}
\benedicamusdomino{}
}

{
\section{\nth{5} Sunday after Easter}
\label{easter5}
\printcommonvespers{}
\def\precollect{\printvrmanenobiscum}
\printvespersmag[../TimeAfterEaster]{inc-VespersMagnificatEaster5}
\benedicamusdomino{}
}

{
\section{Ascension of Our Lord}
\label{ascension}
\subtitle{\nth{1} Class, White or Gold}
\vspace{-0.5\baselineskip}
\subtitle{I \& II Vespers}
\vspace{-0.5\baselineskip}

\def\premagverses{\greseteolcustos{manual}}
\def\begincollectcols{%\vspace{-0.5\baselineskip}
\begin{parcolumns}[rulebetween,colwidths={1=0.46\linewidth}]{2}}
\def\definevesperspropers{\newcommand{\maganttex}{an--o_rex_gloriae--solesmes}
\newcommand{\magantinitial}{O}
\newcommand{\maganttranslation}{O King of glory, Lord of hosts, who today didst ascend in triumph above all heavens, leave us not orphans, but send us the promise of the Father, the Spirit of truth, alleluia.}
\def\magsolemn{T}
\definemag{2}{D}

    \def\prepsalmfive{\greseteolcustos{manual}}
}
\def\definevesperspropersalt{\newcommand{\maganttex}{MagnificatAntiphon1}
\newcommand{\magantinitial}{P}
\newcommand{\maganttranslation}{Father, I have manifested Thy Name unto the men whom Thou gavest Me; and now I pray for them, not for the world, because I come to Thee, alleluia.}
\def\magsolemn{T}
\definemag{6}{F}

    %\def\premagverses{\pagebreak}
}
\def\vesperspropersnote{At II Vespers:}
\def\vesperspropersaltnote{At I Vespers:}
\let\printhymnnote=\undefined
\def\hymnlabel{hymn-salutishumanaesator}
\printvespers[../Ascension]{inc-Ascension}

\def\commemorations{If today is April 30 or May 1, First Vespers of St Joseph the Worker is commemorated as on page \pageref{stjoseph-worker-commem}.}
\printcommemnote[1]{}
\medskip
\hrule
}

{
\section{Sunday after the Ascension}
\label{easter6}\label{sundayafterascension}
\printcommonvespers{}
%\let\printhymnnote=\undefined
\renewcommand{\printhymnnote}{
    \noindent\printnote{Hymn.~\emph{Salútis humánæ Sator}, page \pageref{hymn-salutishumanaesator}.}
    \def\vrlinebreak{T}
    \printvr[\greseteolcustos{manual}]{\vrtex}{\vtranslation}{\rtranslation}
}
\def\precollect{\printvrmanenobiscum}
%\def\premagverses{\pagebreak}
\printvespersmag[../TimeAfterEaster]{inc-VespersMagnificatSundayAfterAscension}
\benedicamusdomino{}
}
}
}

	\chapter{Proper of the Time -- Pentecost Octave}
{
\newcommand{\benedicamusdomino}[1][1]{
  \benedicamusdominomaster{#1}
}
{
\section{Pentecost Sunday}
\subtitle{\nth{1} Class}
\subtitle{I \& II Vespers}

\def\deusinadjutoriumsolemn{T}
\def\definevesperspropers{\definepsalm{5}{113}{7}{c2}

\newcommand{\vrtex}{vrLoquebantur}
\newcommand{\vtranslation}{The apostles spoke in divers tongues.}
\newcommand{\rtranslation}{The wonderful works of God.}

\newcommand{\maganttex}{MagnificatAntiphon2}
\newcommand{\magantinitial}{H}
\newcommand{\maganttranslation}{Today the days of Pentecost are complete, alleluia; today the Holy Ghost appeared in fire to the disciples, gave them gifts and graces, sent them into all the world to preach and to bear witness; whoever believes and is baptised shall be saved, alleluia.}

	%\def\prepsalmfivetitle{\medskip}
	\def\prepsalmfive{\greseteolcustos{manual}}
}
\def\definevesperspropersalt{\definepsalm{5}{116}{7}{c2}

\newcommand{\vrtex}{vrRepletiSunt}
\newcommand{\vtranslation}{They were all filled with the Holy Ghost.}
\newcommand{\rtranslation}{And they began to speak.}

\newcommand{\maganttex}{MagnificatAntiphon1}
\newcommand{\magantinitial}{N}
\newcommand{\maganttranslation}{I will not leave you orphans, alleluia; I go away, and I come unto you, alleluia; and your heart shall rejoice, alleluia.}
%
%\let\oldthing=\maganttranslation
%\def\maganttranslation{\oldthing\vspace{-0.5\baselineskip}}
%\def\postmagtitle{\vspace{-\baselineskip}}%
}%
%
%\def\prepsalmtitleone{\vspace{-0.5\baselineskip}}
%\def\prepsalmtitlethree{\vspace{-0.5\baselineskip}}
\ifthenelse{\boolean{birmingham}}{}{
	\def\prepsalmtitleone{\bigskip{}}
	\def\postpsalmtitleone{\bigskip{}}
}
\def\preantthree{\bigskip}
\def\prepsalmtitlefour{\needspace{4\baselineskip}}
\def\vesperspropersnote{At II Vespers:}
\def\vesperspropersaltnote{At I Vespers:}

%\def\premag{\def\noeuouae{T}}
\def\premagverses{\greseteolcustos{manual}}
\def\prehymn{\printnote{All kneel for the first stanza of the following hymn.}}
\def\prehymntranslation{\vspace{-0.3\baselineskip}}
\def\prevespers{%
%	\let\oldthing\antthreetranslation
%	\def\antthreetranslation{\oldthing\vspace{-\baselineskip}}%
	%\let\oldthingb\antfivetranslation
	%\def\antfivetranslation{\oldthingb\vspace{-\baselineskip}}%
}
%\def\prevr{\vspace{-0.7\baselineskip}}
\printvespers[../Pentecost]{inc-PentecostVespers}
\bigskip
\benedicamusdomino{}
}

{
\section{Trinity Sunday}
\subtitle{\nth{1} Class}

\def\deusinadjutoriumsolemn{T}
%\def\premag{\def\noeuouae{T}}
%\def\prepsalmtwoverses{\vspace{-0.05\baselineskip}}
%\def\prepsalmtitleone{\needspace{10\baselineskip}}
\def\prepsalmone{\needspace{10\baselineskip}}

\def\prerepeatantiphontwo{}
\def\preantfive{\bigskip}
\newcommand{\psalmcolsoverrideoverride}[1][0]{
	\psalmcolsoverride[#1]
	\ifnum#1=110
		\def\beginpsalmcols{\begin{parcolumns}[rulebetween,colwidths={1=0.49\linewidth},distance=1.3em]{2}}
	\fi
}
%\def\prepsalmtitlethree{\vspace{-0.5\baselineskip}}
%\def\prerepeatantiphonthree{\vspace{-0.25\baselineskip}}
\def\premagnificat{\needspace{12\baselineskip}}
\def\premagverses{\greseteolcustos{manual}}
\def\prehymn{\oldneedspace{16\baselineskip}}
\def\begincollectcols{\begin{parcolumns}[rulebetween,colwidths={1=0.44\linewidth}]{2}}
\printvespers[../TrinitySunday]{inc-TrinitySunday-Vespers}
\bigskip
\noindent
\printnote{If the feast of St Philip Neri is celebrated tomorrow, then \emph{First Vespers} is commemorated as follows.
\ifthenelse{\boolean{includebenedicamusdominoreferences}}{
	Otherwise \benedicamusdomino{}%
}{}
}
\medskip
%\hrule
%\medskip
}

% commemoration of First Vespers of Immaculate Conception
{
\def\beginvrcols{\begin{parcolumns}[rulebetween,colwidths={1=0.45\linewidth}]{2}}
\def\vrlinebreak{T}
\printcommemoration[../StPhilipNeri]{Commemoration-StPhilipNeri-Vespers1}

%\bigskip
\benedicamusdomino{}
}

}


	\chapter{Proper of the Time -- Time After Pentecost}
{
\def\printcommonvespers{
  \vspace{-0.5\baselineskip}
  \subtitle{\nth{2} Class, Green}
  \medskip
  \deusinadjutorium{}
  \hfill
  \emph{\emph{Vespers of Sundays throughout the year,} p.~\pageref{sundayvespers}.\par}
}
\newcommand{\benedicamusdomino}[1][sunday]{
  \benedicamusdominomaster{#1}
}
\def\premagtitle{
  \needspace{8\baselineskip}  
}

\newcommand{\printhymnnote}{}
\newcommand{\printvespersafterpentecost}[1]{
  {
  \section{\nth{#1} \ifnum#1=24{or Last }\fi Sunday after Pentecost}
  \label{pentecost#1}
\printcommonvespers{}
\ifx\printaftercommonvespers\undefined\else\printaftercommonvespers{}\fi
  \ifx\nocommemoration\undefined%
    \def\precollect{\printvrdirigatur\smallskip}%
  \fi
  \ifx\postmagtitle\undefined\def\oldpostmagtitle{}\else\let\oldpostmagtitle=\postmagtitle\fi
  \def\postmagtitle{\oldpostmagtitle\label{pentecost#1-mag}}
  \printvespersmag[../TimeAfterPentecost]{inc-VespersMagnificatPentecost#1}
  \smallskip
  \benedicamusdomino{}
  }
  
}


\ifthenelse{\boolean{includecorpuschristi}}{% true
{
  \section{Corpus Christi}

  \def\definevesperspropers{\newcommand{\maganttex}{an--o_sacrum_convivium--solesmes}
\newcommand{\magantinitial}{O}
\newcommand{\maganttranslation}{O sacred banquet, in which Christ is received; the memory of His Passion is renewed; the mind is filled with grace; and a pledge of future glory is given to us, alleluia.}
\newcommand{\magsolemn}{T}
\definemag{5}{a}
}

  \printvespers[../CorpusChristi]{inc-CorpusChristi}
  \medskip
  \benedicamusdomino{}
}
}{}% false

{\def\begincollectcols{\begin{parcolumns}[rulebetween,colwidths={1=0.455\linewidth}]{2}}
\printvespersafterpentecost{2}}
\ifthenelse{\boolean{includesacredheart}}{% true
{
  \section{Sacred Heart of Jesus}

  \def\definevesperspropers{\newcommand{\antonetex}{an--unus_militum--solesmes}
\newcommand{\antoneinitial}{U}
\newcommand{\antonetranslation}{A soldier with a spear opened his side, and immediately there came forth blood and water.}
\definepsalm{1}{109}{1}{f}

\newcommand{\anttwotex}{an--stans_jesus--solesmes}
\newcommand{\anttwoinitial}{S}
\newcommand{\anttwotranslation}{Jesus stood and cried, saying~: If any man thirst, let him come to me and drink.}
\definepsalm{2}{110}{7}{c}

\newcommand{\antthreetex}{an--in_caritate--solesmes}
\newcommand{\antthreeinitial}{I}
\newcommand{\antthreetranslation}{With an everlasting love has God loved us; lifted up, therefore, from the earth, he has drawn us to his Heart, taking pity on us.}
\definepsalm{3}{115}{3}{a2}

\newcommand{\antfourtex}{an--venite_ad_me--solesmes}
\newcommand{\antfourinitial}{V}
\newcommand{\antfourtranslation}{Come to me, all you that labour and are burdened~: and I will refresh you.}
\definepsalm{4}{127}{4}{E}

\newcommand{\antfivetex}{an--fili_praebe--solesmes}
\newcommand{\antfiveinitial}{F}
\newcommand{\antfivetranslation}{My son, give me thy heart~: and let thine eyes keep my ways.}
\definepsalm{5}{147}{5}{a}

\newcommand{\maganttex}{an--ad_jesum_autem--solesmes}
\newcommand{\magantinitial}{A}
\newcommand{\maganttranslation}{But after they were come to Jesus, when they saw that he was already dead, they did not break his legs.  But one of the soldiers with a spear opened his side~: and immediately there came out blood and water.}
\newcommand{\magsolemn}{T}
\definemag{1}{f}

\newcommand{\vrtex}{vrHaurietis}
\newcommand{\vtranslation}{Ye shall draw waters with joy.}
\newcommand{\rtranslation}{Out of the Saviour's fountains.}

}

  \printvespers[../SacredHeart]{inc-SacredHeart}
  \medskip
  \benedicamusdomino{}
}
}{}% false
\ifthenelse{\boolean{testrun}}{}{
\printvespersafterpentecost{3}
\printvespersafterpentecost{4}
{\def\begincollectcols{\begin{parcolumns}[rulebetween,colwidths={1=0.44\linewidth}]{2}}
\printvespersafterpentecost{5}}
\printvespersafterpentecost{6}
\printvespersafterpentecost{7}
{
  \def\premagtitle{}
  \def\postmagtitle{
    \pagebreak[2]
  }
\printvespersafterpentecost{8}}
\printvespersafterpentecost{9}
\printvespersafterpentecost{10}
\printvespersafterpentecost{11}
{\def\begincollectcols{\begin{parcolumns}[rulebetween,colwidths={1=0.455\linewidth}]{2}}
%\def\printaftercommonvespers{\needspace{6\baselineskip}}
\printvespersafterpentecost{12}}
{
\def\premagnificat{\bigskip}
\printvespersafterpentecost{13}}
\printvespersafterpentecost{14}
\printvespersafterpentecost{15}
\printvespersafterpentecost{16}
{
  \def\premagtitle{}
  \def\postmagtitle{
    \pagebreak[2]
  }
\printvespersafterpentecost{17}}
\printvespersafterpentecost{18}
{\def\begincollectcols{\begin{parcolumns}[rulebetween,colwidths={1=0.47\linewidth}]{2}}
\def\postbenedicamusdomino{\bigskip}
%\def\premagverses{\vspace{-0.7\baselineskip}}
\printvespersafterpentecost{19}}
\bigskip\bigskip
\printvespersafterpentecost{20}
\printvespersafterpentecost{21}
{
\def\postmagtitle{\medskip}
\printvespersafterpentecost{22}
}
{
\def\postbenedicamusdomino{\bigskip}
\printvespersafterpentecost{23}
}
\bigskip
{%\def\prevespers{
 % \let\oldthing=\englishmagantiphon
 % \def\englishmagantiphon{\oldthing\pagebreak}  
%}
\def\begincollectcols{\begin{parcolumns}[rulebetween,colwidths={1=0.455\linewidth}]{2}}
\printvespersafterpentecost{24}}
}
}

	\clearpage
\chapter{Proper of the Saints}
{
\newcommand{\benedicamusdomino}[1][1]{
  \noindent\printnote{\Vbar~\emph{Benedicámus Dómino \IfInteger{#1}{#1}{\csname benedicamusdominoname#1\endcsname}}, p.~\pageref{benedicamusdomino-#1}.}
  \bigskip
  \hrule
}

%December 8: Immaculate Conception
%Second Vespers of Immaculate Conception
{
\label{immaculateconception}
\section{December 8: Immaculate Conception}
\subtitle{\nth{1} Class}
\subtitle{I \& II Vespers}

\def\premagverses{\greseteolcustos{manual}}
\def\definevesperspropersalt{\def\noeuouae{T}\newcommand{\maganttex}{MagnificatAntiphon1}
\newcommand{\magantinitial}{B}
\newcommand{\maganttranslation}{All generations shall call me blessed, because he that is mighty, hath done great things for me, alleluia.}
\def\magpsalmclef{c3}
\definemag{8}{G}
}
\def\definevesperspropers{\def\noeuouae{T}\newcommand{\maganttex}{MagnificatAntiphon2}
\newcommand{\magantinitial}{H}
\newcommand{\maganttranslation}{This day a rod came forth from the root of Jesse: this day Mary was conceived without any stain of sin: this day the head of the old serpent was crushed by her.  Alleluia.}
\definemag{1}{f}
}
\def\vesperspropersaltnote{At I Vespers:}
\def\vesperspropersnote{At II Vespers:}
\def\prehymn{\printnote{All kneel for the first stanza of the following hymn.}}
\def\hymnlabel{hymn-avemarisstella}
\def\vrlinebreak{F}
\printvespers[../December8-ImmaculateConception]{inc-ImmaculateConceptionVespers}
}

{
\bigskip

\bigskip
\noindent
\printnote{Then follows a Commemoration of the Advent Sunday or Feria according to the day of the week on which the Feast of the Immaculate Conception falls.}
\bigskip
}

{
  \oldneedspace{5\baselineskip}
  \subtitle{First Week of Advent.}
  \vspace{-\baselineskip}
  \subtitle{\small{Thursday.}}
  {
  \def\noeuouae{T}
  \printgabc{At Magn.}{\oldstylenums{Ant.~4.}}{E}{../Advent1/MagAntThursday-Exspectabo}
  }
  \translation[]{I will look for the Lord my Saviour, and await Him, while He is near, alleluia.}
  \medskip

  \oldneedspace{5\baselineskip}
  \vspace{-\baselineskip}
  \subtitle{\small{Friday.}}
  {
  \def\noeuouae{T}
  \printgabc{At Magn.}{\oldstylenums{Ant.~4.}}{E}{../Advent1/MagAntFriday-ExAegypto}
  }
  \translation[]{Out of Egypt I have called my Son; he shall come to save His people.}
  \medskip

  \oldneedspace{5\baselineskip}
  \subtitle{Second Week of Advent.}
  \vspace{-\baselineskip}
  \subtitle{\small{Saturday.}}
  {
  \def\noeuouae{T}
  \printgabc{At Magn.}{\oldstylenums{Ant.~7.}}{V}{../Advent1/MagAntSaturday-VeniDomine}
  }
  \translation[]{Come, Lord, to visit us in peace, that we may rejoice before Thee with a perfect heart.}
  \medskip

  \oldneedspace{5\baselineskip}
  \vspace{-\baselineskip}
  \subtitle{\small{Sunday.}}
  {
  \def\noeuouae{T}
  \printgabc{At Magn.}{\oldstylenums{Ant.~8.~G}}{T}{../Advent2/MagnificatAntiphon-noEuouae}
  }
  \translation[]{Art Thou He that art to come, or look we for another? Relate to John what you have seen: The blind recover their sight, the dead rise again, the poor have the Gospel preached to them, alleluia.}
  \medskip

  \oldneedspace{5\baselineskip}
  \vspace{-\baselineskip}
  \subtitle{\small{Monday.}}
  {
  \def\noeuouae{T}
  \printgabc{At Magn.}{\oldstylenums{Ant.~4.}}{E}{../Advent2/MagAntMonday-EcceRexVeniet}
  }
  \translation[]{Behold, the King shall come, the Lord of the land; and He shall take away the yoke of our captivity.}
  \medskip

  \oldneedspace{5\baselineskip}
  \vspace{-\baselineskip}
  \subtitle{\small{Tuesday.}}
  {
  \def\noeuouae{T}
  \printgabc{At Magn.}{\oldstylenums{Ant.~5.}}{V}{../Advent2/MagAntTuesday-VoxClamantis}
  }
  \translation[]{A voice of one crying in the desert, Prepare ye the way of the Lord, make straight His paths}
  \medskip

  \oldneedspace{5\baselineskip}
  \vspace{-\baselineskip}
  \subtitle{\small{Wednesday.}}
  {
  \def\noeuouae{T}
  \printgabc{At Magn.}{\oldstylenums{Ant.~4.}}{S}{../Advent2/MagAntWednesday-Sion}
  }
  \translation[]{Sion, thou shalt be restored, and shalt see the Just One who shall appear in thee.}
  \medskip

  \oldneedspace{5\baselineskip}
  \vspace{-\baselineskip}
  \subtitle{\small{Thursday.}}
  {
  \def\noeuouae{T}
  \printgabc{At Magn.}{\oldstylenums{Ant.~4.}}{Q}{../Advent2/MagAntThursday-QuiPostMeVenit}
  }
  \translation[]{He that shall come after me is preferred before me; whose shoes I am not worthy to loose.}
  \medskip
  \hrule
  \medskip
  {
      \newcommand{\commvlatin}{Roráte cæli désuper, et nubes pluant \textbf{ju}stum.}
      \newcommand{\commrlatin}{Aperiátur terra, et gérminet Salva\textbf{tó}rem.}
      \newcommand{\commvtranslation}{Ye heavens, drop down dew from above, and let the clouds rain down the Just One.}
      \newcommand{\commrtranslation}{Let the earth open and bud forth the Saviour.}
  \printvrcommem{}
  }

  \oldneedspace{3\baselineskip}
  \begin{center}{\large Collect.}\end{center}
  \vspace{-\baselineskip}
  \def\printcollectheading{F}
  {
  \begin{center}{First Week of Advent.}\end{center}
  \def\gabcfolder{../Advent1}
  \newcommand{\antonetex}{Ant1-InIllaDie}
\newcommand{\antoneinitial}{I}
\newcommand{\antonetranslation}{In that day the mountains shall drop down sweetness, and the hills shall flow with milk and honey, alleluia.}
\definepsalm{1}{109}{8}{G}

\newcommand{\anttwotex}{Ant2-Jucundare}
\newcommand{\anttwoinitial}{J}
\newcommand{\anttwotranslation}{Shout for joy, O daughter of Sion, rejoice greatly, O daughter of Jerusalem, alleluia.}
\definepsalm{2}{110}{8}{G*}

\newcommand{\antthreetex}{Ant3-EcceDominusVeniet}
\newcommand{\antthreeinitial}{E}
\newcommand{\antthreetranslation}{Behold, the Lord shall come, and all His Saints with Him: and there shall be in that day a great light, alleluia.}
\definepsalm{3}{111}{5}{a}

\newcommand{\antfourtex}{Ant4-Omnes}
\newcommand{\antfourinitial}{O}
\newcommand{\antfourtranslation}{All ye that thirst come to the waters: seek the Lord while He can be found, alleluia.}
\definepsalm{4}{112}{7}{c}

\newcommand{\antfivetex}{Ant5-EcceVeniet}
\newcommand{\antfiveinitial}{E}
\newcommand{\antfivetranslation}{Behold there shall come a great Prophet, and He shall renew Jerusalem, alleluia.}
\definepsalm{5}{113}{4}{A*}

\newcommand{\chaptertext}{\dropcap{latin}{Fratres~: Hora est jam nos de somno} \textbf{súr}\-ge\-re~:~\gredagger{} nunc enim própior est \emph{no\-stra} \textbf{sá}\-lus,~* quam cum cre\-\textbf{dí}\-dimus.}
\newcommand{\chaptertranslation}{Brethren: it is now the hour for us to rise from sleep.  For now our salvation is nearer than when we believed.}

\newcommand{\magantinitial}{N}
\newcommand{\maganttex}{MagnificatAntiphon}
\newcommand{\maganttranslation}{Fear not, Mary, for thou hast found grace with the Lord: behold thou shalt conceive and bring forth a son, alleluia.}
\def\magsolemn{F}
\definemag{8}{G}

\newcommand{\collect}{Excita, quǽsumus Dómine, poténtiam tuam, et veni~:~† ut ab imminéntibus peccatórum nostrórum perículis, te mereámur protegénte éripi,~* te liberánte salvári.  Qui vivis et regnas cum Deo Patre in unitáte Spíritus Sancti Deus~:~* per ómnia sǽcula sæculórum.}
\newcommand{\collecttranslation}{Stir up Thy power, we beseech Thee, O Lord, and come: that from the threatening dangers of our sins we may deserve to be rescued by Thy protection, and to be saved by Thy deliverance: Who livest and reignest with God the Father in the unity of the Holy Ghost, world without end.}

  \printcollect{\collect}{\collecttranslation}
  }
  {
  \medskip
  \begin{center}{Second Week of Advent.}\end{center}
  \def\gabcfolder{../Advent2}
  % !TEX TS-program = lualatex
% !TEX encoding = UTF-8

% This is a simple template for a LuaLaTeX document using gregorio scores.

\newcommand{\comheadingtext}{Commemoration of 2nd Sunday of Advent}

\newcommand{\latincomcollect}{Excita Dómine corda nostra ad præparándas Unigéniti tui vias~:~† ut per ejus advéntum,~* purificátis tibi méntibus servíre mereámur. Qui tecum vivit et regnat.}
\newcommand{\englishcomcollect}{Stir up our hearts, O Lord, to prepare the ways of Thine only-begotten Son: that through His coming we may deserve to serve Thee with purified minds: Who with Thee liveth and reigneth.}

\newcommand{\englishcommagantiphon}{Art Thou He that art to come, or look we for another? Relate to John what you have seen: The blind recover their sight, the dead rise again, the poor have the Gospel preached to them, alleluia.}

\newcommand{\commagantlinetwo}{Ant. 8. G}
\newcommand{\commaganttex}{MagnificatAntiphon-noEuouae}
\newcommand{\commagantinitial}{T}
\newcommand{\commagantinitialsize}{35}

\newcommand{\commvrtex}{../Advent/vr-commemoration}
\newcommand{\commvtranslation}{Ye heavens, drop down dew from above, and let the clouds rain down the Just One.}
\newcommand{\commrtranslation}{Let the earth open and bud forth the Saviour.}

\setgrefactor{17}
  \printcollect{\latincomcollect}{\englishcomcollect}
  }

  \bigskip
  \benedicamusdomino{}
}


%Purification & Presentation (2nd class no commem of Sunday because it is a feast of the Lord)
{
\section{February 2: Purification \& Presentation}
\subtitle{\nth{2} Class}
\subtitle{I \& II Vespers}
\printnote{At I Vespers, Psalms and Antiphons of the Circumcision, p.~\pageref{circumcision}, continuing with the chapter on p.~\pageref{purification-chapter}.}

\def\definevesperspropers{\newcommand{\maganttex}{an--hodie_beata_virgo--solesmes}
\newcommand{\magantinitial}{H}
\newcommand{\maganttranslation}{Today did the Blessed Virgin Mary present the Child Jesus in the temple; and Simeon, filled with the Holy Ghost, took Him up in his arms, and blessed God for ever.}
\def\magpsalmclef{c3}
\definemag{8}{G*}

  \def\prepsalmfive{\greseteolcustos{manual}}
}
\def\definevesperspropersalt{\newcommand{\maganttex}{an--senex_puerum--solesmes}
\newcommand{\magantinitial}{S}
\newcommand{\maganttranslation}{The old man carried the Child, but the Child led the old man. The Virgin bore the Child, and after child-bearing was virgin still: Whom she bore, Him she adored.}
\definemag{1}{D}
}
\def\vesperspropersnote{At II Vespers:}
\def\vesperspropersaltnote{At I Vespers:}
%\def\premag{\def\noeuouae{T}}
\def\premagverses{\greseteolcustos{manual}}
\def\prechapter{\label{purification-chapter}}
\def\printhymnnote{
  {
    \oldneedspace{3\baselineskip}
    \printnote{Hymn.~\emph{Ave Maris Stella}, p.~\pageref{hymn-avemarisstella}.\\}
    %
    % \def\vrlinebreak{T}
    % \oldneedspace{3\baselineskip}
    % \printvr[\greseteolcustos{manual}]{\vrtex}{\vtranslation}{\rtranslation}
  }
}
\printvespers[../February2-PurificationOfBlessedVirginMary]{inc-purification}
\bigskip
\benedicamusdomino[2]{}
}

%May 1: St Joseph the Worker (1st class)
{
\section{May 1: St Joseph the Worker}
\subtitle{\nth{1} Class}
\subtitle{I \& II Vespers}

\def\definevesperspropers{\newcommand{\vrtex}{vrOraProNobis}
\newcommand{\vtranslation}{Pray for us, St Joseph, alleluia.}
\newcommand{\rtranslation}{Faithful protector of our labors, alleluia.}

\newcommand{\maganttex}{an--et_ipse_jesus--solesmes}
\newcommand{\magantinitial}{E}
\newcommand{\maganttranslation}{}
\definemag{7}{d}

  \def\prepsalmfive{\greseteolcustos{manual}}
}
\def\definevesperspropersalt{\newcommand{\vrtex}{vrSolemnitasEstHodie}
\newcommand{\vtranslation}{Today is the solemnity of St Joseph, alleluia.}
\newcommand{\rtranslation}{Who ministered with his hands to the Son of God, alleluia.}

\newcommand{\maganttex}{an--christus_dominus--solesmes}
\newcommand{\magantinitial}{C}
\newcommand{\maganttranslation}{}
\definemag{7}{c2}

  \def\vrlinebreak{F}
}
\def\vesperspropersnote{At II Vespers:}
\def\vesperspropersaltnote{At I Vespers:}
%\def\premag{\def\noeuouae{T}}
\def\premagverses{\greseteolcustos{manual}}

\printvespers[../May1-StJosephWorker]{inc-StJosephWorker}
%if feast of St Joseph the worker falls from 2nd through 5th Sunday after Easter, it outranks the Sunday and the Sunday is commemorated
\medskip
\printnote{If today is Sunday, the Vespers of the Sunday is commemorated with \emph{Magnificat antiphon}, \emph{\Vbar{} Mane nobíscum.} in simple commemoration tone, p.~\pageref{vr-manenobiscum} and \emph{Collect}.  Otherwise \Vbar~\emph{Benedicámus Dómino 1}, p.~\pageref{benedicamusdomino-1}.

\begin{multicols}{2}
\noindent\emph{\nth{2} Sunday after Easter}, p.~\pageref{easter2}.\\
\emph{\nth{3} Sunday after Easter}, p.~\pageref{easter3}.\\
\emph{\nth{4} Sunday after Easter}, p.~\pageref{easter4}.\\
\emph{\nth{5} Sunday after Easter}, p.~\pageref{easter5}.
\end{multicols}
}
% \bigskip
% \benedicamusdomino{}
}

%June 24: Nativity of St John the Baptist (1st class)
{
\section{June 24: Nativity of St John the Baptist}
\subtitle{\nth{1} Class}
\subtitle{I \& II Vespers}

\def\definevesperspropers{\newcommand{\antonetex}{Ant1-ElisabethZachariae}
\newcommand{\antoneinitial}{E}
\newcommand{\antonetranslation}{Elizabeth, the wife of Zacharias, gave birth to a great man, John the Baptist, the forerunner of the Lord.}
\definepsalm{1}{109}{3}{a}

\newcommand{\anttwotex}{Ant2-Innuebant}
\newcommand{\anttwoinitial}{I}
\newcommand{\anttwotranslation}{They made signs unto his father, by what name he should be called: and he wrote, saying: His name is John.}
\definepsalm{2}{110}{4}{E}

\newcommand{\antthreetex}{Ant3-JoannesVocabitur}
\newcommand{\antthreeinitial}{J}
\newcommand{\antthreetranslation}{His name shall be called John, and many shall rejoice in his birth.}
\definepsalm{3}{111}{1}{f}

\newcommand{\antfourtex}{Ant4-InterNatos}
\newcommand{\antfourinitial}{I}
\newcommand{\antfourtranslation}{Among those born of women, there hath not risen a greater than John the Baptist.}
\definepsalm{4}{112}{3}{b}

\newcommand{\antfivetex}{Ant5-TuPuer}
\newcommand{\antfiveinitial}{T}
\newcommand{\antfivetranslation}{Thou, child, shalt be called the Prophet of the Most High: thou shalt go before the Lord to prepare His ways.}
\definepsalm{5}{116}{3}{b}

\newcommand{\vrtex}{vrIstePuerMagnus}
\newcommand{\vtranslation}{This child is great before the Lord.}
\newcommand{\rtranslation}{For in truth His hand is with him.}

\newcommand{\maganttex}{MagnificatAntiphon2}
\newcommand{\magantinitial}{P}
\newcommand{\maganttranslation}{The child that is born to us is more than a prophet; for this is he of whom the Saviour said: Among those born of women there hath not risen a greater than John the Baptist.}
\def\magsolemn{T}
\definemag{7}{d}

  \def\prepsalmfive{\greseteolcustos{manual}}
}
\def\definevesperspropersalt{\newcommand{\antonetex}{Ant1-IpsePraeibit}
\newcommand{\antoneinitial}{I}
\newcommand{\antonetranslation}{He shall go before Him in the spirit and power of Elias, to prepare unto the Lord a perfect people.\vspace{0ex plus 0ex minus 3ex}}
\definepsalm{1}{109}{7}{a}

\newcommand{\anttwotex}{Ant2-Joannes}
\newcommand{\anttwoinitial}{J}
\newcommand{\anttwotranslation}{John is his name.  Wine and strong drink he shall not drink, and many shall rejoice in his birth.}
\definepsalm{2}{110}{8}{G}

\newcommand{\antthreetex}{Ant3-ExUteroSenectutis}
\newcommand{\antthreeinitial}{E}
\newcommand{\antthreetranslation}{From the barren womb of age was born John, the forerunner of the Lord.}
\definepsalm{3}{111}{1}{f}

\newcommand{\antfourtex}{Ant4-IstePuer}
\newcommand{\antfourinitial}{I}
\newcommand{\antfourtranslation}{This child is great before the Lord, for the hand of God is with him.}
\definepsalm{4}{112}{4}{A*}

\newcommand{\antfivetex}{Ant5-Nazaraeus}
\newcommand{\antfiveinitial}{N}
\newcommand{\antfivetranslation}{This child shall be called a Nazarite; wine and strong drink he shall not drink, and from his mother's womb he shall eat nothing unclean.}
\definepsalm{5}{116}{5}{a}

\newcommand{\vrtex}{vrFuitHomo}
\newcommand{\vtranslation}{There was a man sent from God.}
\newcommand{\rtranslation}{Whose name was John.}

\newcommand{\maganttex}{MagnificatAntiphon1}
\newcommand{\magantinitial}{I}
\newcommand{\maganttranslation}{When Zacharias had entered the temple of the Lord, there appeared to him the Angel Gabriel, standing at the right hand of the altar of incense.\vspace{-4pt plus 4pt}}
\def\magpsalmclef{c3}
\def\magsolemn{T}
\definemag{8}{G}
}
\def\vesperspropersnote{At II Vespers:}
\def\vesperspropersaltnote{At I Vespers:}
\def\prevesperspsalms{\noindent\printnote{Chapter and following, page \pageref{june24-chapter}.\\}}
\def\vesperspsalmslabel{\label{june24-2vespers}}
\def\prevesperspsalmsalt{\noindent\printnote{II Vespers psalms and antiphons, page \pageref{june24-2vespers}.}\medskip}
\def\prechapter{\label{june24-chapter}}
%\def\premag{\def\noeuouae{T}}
\def\premagverses{\greseteolcustos{manual}}

\printvespers[../June24-BirthOfJohnTheBaptist]{inc-BirthOfJohnTheBaptist}

\medskip
\printnote{If today is Sunday, the Vespers of the Sunday is commemorated with \emph{Magnificat antiphon}, \emph{\Vbar{} Dirigátur.} in simple commemoration tone, p.~\pageref{vr-dirigatur} and \emph{Collect}.  Otherwise \Vbar~\emph{Benedicámus Dómino 1}, p.~\pageref{benedicamusdomino-1}.

\begin{multicols}{2}
\noindent\emph{\nth{2} Sunday after Pentecost}, p.~\pageref{pentecost2}.\\
\emph{\nth{3} Sunday after Pentecost}, p.~\pageref{pentecost3}.\\
\emph{\nth{4} Sunday after Pentecost}, p.~\pageref{pentecost4}.\\
\emph{\nth{5} Sunday after Pentecost}, p.~\pageref{pentecost5}.\\
\emph{\nth{6} Sunday after Pentecost}, p.~\pageref{pentecost6}.
\end{multicols}
}
% \bigskip
% \benedicamusdomino{}
}

%June 29: Sts Peter \& Paul (1st class)
{
\section{June 29: Sts Peter \& Paul}
\subtitle{\nth{1} Class}
\subtitle{I \& II Vespers}

\def\definevesperspropers{\import{../CommonOfApostles/}{inc-CommonOfApostles-2Vespers-psalms}
\edef\antonetex{../CommonOfApostles/\antonetex}
\edef\anttwotex{../CommonOfApostles/\anttwotex}
\edef\antthreetex{../CommonOfApostles/\antthreetex}
\edef\antfourtex{../CommonOfApostles/\antfourtex}
\edef\antfivetex{../CommonOfApostles/\antfivetex}

\newcommand{\vrtex}{vrAnnuntiaverunt}
\newcommand{\vtranslation}{They declared the works of God.}
\newcommand{\rtranslation}{And understood His doings.}

\newcommand{\maganttex}{MagnificatAntiphon2-Hodie}
\newcommand{\magantinitial}{H}
\newcommand{\maganttranslation}{Today, Simon Peter went up upon the gibbet of the cross, alleluia; today, he that holdeth the keys of the kingdom, departed with joy to be with Christ; today, the Apostle Paul, the light of the world, bowing his head, for Christ's sake was crowned with martyrdom, alleluia.}
\def\magsolemn{T}
\definemag{1}{D}

  \def\prepsalmfive{\greseteolcustos{manual}}
}
\def\definevesperspropersalt{\newcommand{\antonetex}{Ant1-PetrusEtJoannes}
\newcommand{\antoneinitial}{P}
\newcommand{\antonetranslation}{Peter and John went up together into the Temple at the hour of prayer, being the ninth hour.}
\definepsalm{1}{109}{8}{G}

\newcommand{\anttwotex}{Ant2-Argentum}
\newcommand{\anttwoinitial}{A}
\newcommand{\anttwotranslation}{Silver and gold have I none, but such as I have, give I thee.}
\definepsalm{2}{110}{7}{b}

\newcommand{\antthreetex}{Ant3-DixitAngelusAdPetrum}
\newcommand{\antthreeinitial}{D}
\newcommand{\antthreetranslation}{The Angel said unto Peter: Cast thy garment about thee, and follow me.}
\definepsalm{3}{111}{8}{c}

\newcommand{\antfourtex}{Ant4-MisitDominus}
\newcommand{\antfourinitial}{M}
\newcommand{\antfourtranslation}{The Lord hath sent His Angel, and hath delivered me out of the hand of Herod.  Alleluia.}
\definepsalm{4}{112}{7}{c2}

\newcommand{\antfivetex}{Ant5-TuEsPetrus}
\newcommand{\antfiveinitial}{T}
\newcommand{\antfivetranslation}{Thou art Peter and upon this Rock I will build My Church.}
\definepsalm{5}{116}{7}{c}

\newcommand{\vrtex}{vrInOmnemTerram}
\newcommand{\vtranslation}{Their sound hath gone forth into all the earth.}
\newcommand{\rtranslation}{And their words unto the ends of the world.}

\newcommand{\maganttex}{MagnificatAntiphon1}
\newcommand{\magantinitial}{T}
\newcommand{\maganttranslation}{Thou art the Shepherd of the sheep and the Prince of the Apostles, and unto thee are given the keys of the kingdom of heaven.}
\def\magsolemn{T}
\definemag{1}{f}
}
\def\vesperspropersnote{At II Vespers:}
\def\vesperspropersaltnote{At I Vespers:}
\def\prevesperspsalms{\noindent\printnote{Chapter and following, page \pageref{june29-chapter}.\\}}
\def\vesperspsalmslabel{\label{june29-2vespers}}
\def\prevesperspsalmsalt{\noindent\printnote{II Vespers psalms and antiphons, page \pageref{june29-2vespers}.}\medskip}
\def\prechapter{\label{june29-chapter}}
%\def\premag{\def\noeuouae{T}}
\def\premagverses{\greseteolcustos{manual}}

\def\begincollectcols{\begin{parcolumns}[rulebetween,colwidths={1=0.45\linewidth}]{2}}
\printvespers[../June29-StsPeterAndPaul]{inc-StsPeterAndPaul}

\medskip
\printnote{If today is Sunday, the Vespers of the Sunday is commemorated with \emph{Magnificat antiphon}, \emph{\Vbar{} Dirigátur.} in simple commemoration tone, p.~\pageref{vr-dirigatur} and \emph{Collect}.  Otherwise \Vbar~\emph{Benedicámus Dómino 1}, p.~\pageref{benedicamusdomino-1}.

\begin{multicols}{2}
\noindent\emph{\nth{3} Sunday after Pentecost}, p.~\pageref{pentecost3}.\\
\emph{\nth{4} Sunday after Pentecost}, p.~\pageref{pentecost4}.\\
\emph{\nth{5} Sunday after Pentecost}, p.~\pageref{pentecost5}.\\
\emph{\nth{6} Sunday after Pentecost}, p.~\pageref{pentecost6}.\\
\emph{\nth{7} Sunday after Pentecost}, p.~\pageref{pentecost7}.
\end{multicols}
}
% \bigskip
% \benedicamusdomino{}
}

%July 1: Most Precious Blood (1st class)
{
\global\let\psalmclefthree=\undefined
\section{July 1: The Precious Blood of Our Lord Jesus Christ}
\subtitle{\nth{1} Class}
\subtitle{I \& II Vespers}

\def\definevesperspropers{\definepsalm{5}{147}{2}{D}

\newcommand{\vrtex}{vrTeErgo}
\newcommand{\vtranslation}{We therefore pray Thee help Thy servants.}
\newcommand{\rtranslation}{Whom Thou hast redeemed with Thy precious Blood.}

\newcommand{\maganttex}{MagnificatAntiphon2}
\newcommand{\magantinitial}{H}
\newcommand{\maganttranslation}{\vspace{-16pt plus 16pt}Ye shall observe this day for a memorial: and ye shall keep it holy unto the Lord, in your generations with an everlasting worship.\vspace{-4pt plus 4pt}}
\newcommand{\magsolemn}{F}
\definemag{1}{D2}

  \def\prepsalmfive{\greseteolcustos{manual}}
}
\def\definevesperspropersalt{\definepsalm{5}{116}{2}{D}

\newcommand{\vrtex}{vrRedemistiNos}
\newcommand{\vtranslation}{Thou hast redeemed us, O Lord, in Thy Blood.}
\newcommand{\rtranslation}{And hast made of us a kingdom unto our God.}

\newcommand{\maganttex}{MagnificatAntiphon1}
\newcommand{\magantinitial}{A}
\newcommand{\maganttranslation}{Ye are come to Mount Sion, to the city of the living God, the heavenly Jerusalem, and to Jesus the Mediator of the new Testament, and to the sprinkling of blood which speaketh better than that of Abel.}
\newcommand{\magsolemn}{F}
\definemag{3}{a}
}
\def\vesperspropersnote{At II Vespers:}
\def\vesperspropersaltnote{At I Vespers:}
%\def\premag{\def\noeuouae{T}}
\def\premagverses{\greseteolcustos{manual}}

\def\begincollectcols{\begin{parcolumns}[rulebetween,colwidths={1=0.45\linewidth}]{2}}
\printvespers[../July1-MostPreciousBloodOfChrist]{inc-MostPreciousBloodOfChrist}
\bigskip
\benedicamusdomino{}
}

%Aug 6: Transfiguration (2nd class)
{
\section{August 6: Transfiguration of Our Lord Jesus Christ}
\subtitle{\nth{2} Class}
\subtitle{I \& II Vespers}

\def\definevesperspropers{\newcommand{\maganttex}{MagnificatAntiphon2}
\newcommand{\magantinitial}{E}
\newcommand{\maganttranslation}{And the disciples hearing, fell on their faces, and were sore afraid; and Jesus came, and touched them, and said to them, Arise, and fear not, alleluia.}
\newcommand{\magsolemn}{F}
\definemag{8}{G}

  \def\prepsalmfive{\greseteolcustos{manual}}
}
\def\definevesperspropersalt{\newcommand{\maganttex}{an--christus_jesus--solesmes}
\newcommand{\magantinitial}{C}
\newcommand{\maganttranslation}{Christ Jesus, radiance of the Father and image of His Being, upholding all things by the word of His power; making atonement for sins, has deigned to appear today in glory on the high mountain.}
\newcommand{\magsolemn}{F}
\definemag{4}{E}
}
\def\vesperspropersnote{At II Vespers:}
\def\vesperspropersaltnote{At I Vespers:}
%\def\premag{\def\noeuouae{T}}
\def\premagverses{\greseteolcustos{manual}}

\def\begincollectcols{\begin{parcolumns}[rulebetween,colwidths={1=0.44\linewidth}]{2}}
\printvespers[../August6-TransfigurationOfOurLord]{inc-Transfiguration}
\bigskip
\benedicamusdomino[2]{}
}

%Aug 15: Assumption (1st class)
{
\section{August 15: Assumption of the B.~V.~M.}
\subtitle{\nth{1} Class}
\subtitle{I \& II Vespers}

\def\definevesperspropers{% hymn is ave maris stella
%\input{inc-hymn-avemarisstella}

\newcommand{\maganttex}{an--hodie_maria_virgo--solesmes}
\newcommand{\magantinitial}{H}
\newcommand{\maganttranslation}{Today the Virgin Mary has gone up to heaven: rejoice, for with Christ she reigns forever.}
\newcommand{\magsolemn}{T}
\definemag{8}{G*}

  \def\prepsalmfive{\greseteolcustos{manual}}
}
\def\definevesperspropersalt{\newcommand{\hymnlinetwo}{2.}
\newcommand{\hymntex}{Hymn-OPrimaVirgoProdita}
\newcommand{\hymninitial}{O}
\newcommand{\hymntranslation}{
\item O Virgin who was first to receive
The Creator’s grace by the spirit,
Who was predestined by the Most High
To bear in her womb the Son.

\item O woman, who was foretold to be
The perpetual enemy of the demon;
Who alone was filled with grace,
Undefiled from conception.

\item Thou who conceives Life itself in thy womb,
Life that was lost by Adam;
Furnishing the divine Victim,
A body for his sacrifice.

\item Death, the recompense for sin,
Had no victory over thee, and now departs;
And then thou hastened bodily to heaven
To be thy loving Son’s companion.

\item Illuminated by so great a Glory,
All nature is raised up;
And in thee is called to reach
The pinnacle of all glory and splendour.

\item In thy triumph O our Queen,
Turn thine eyes to us exiles;
That through thy patronage,
We may come to heaven, our blessed homeland.

\item Praise to the Father! praise to Him,
The Virgin’s holy Son!
Praise to the Spirit Paraclete,
While endless ages run! 
Amen.
}

\newcommand{\maganttex}{MagAntiphon-VirgoPrudentissima}
\newcommand{\magantinitial}{V}
\newcommand{\maganttranslation}{O Virgin most prudent, whither goest thou, like the golden dawn?  Daughter of Sion, thou art all beautiful and sweet; fair as the moon, bright as the sun.}
\newcommand{\magsolemn}{T}
\definemag{1}{f}
}
\def\vesperspropersnote{At II Vespers:}
\def\vesperspropersaltnote{At I Vespers:}
%\def\premag{\def\noeuouae{T}}
\def\premagverses{\greseteolcustos{manual}}
\def\printfullhymn{
  {
    \oldneedspace{3\baselineskip}
    \printnote{At II Vespers: Hymn.~\emph{Ave Maris Stella}, p.~\pageref{hymn-avemarisstella}. \Vbar{} \emph{Exaltata.} p.~\pageref{vr-assumption}.\\}

    \printnote{\vesperspropersaltnote}
    \definevesperspropersalt
    \printhymn{\oldstylenums{\hymnlinetwo}}{\hymninitial}{\hymntex}{\hymntranslation}
  }
  {
    \def\vrlinebreak{T}
    \oldneedspace{3\baselineskip}
    \label{vr-assumption}
    \printvr[\greseteolcustos{manual}]{\vrtex}{\vtranslation}{\rtranslation}
  }
}

\printvespers[../August15-AssumptionOfTheBlessedVirginMary]{inc-Assumption}

%     TODO Feast of Assumption (8/15) could have commem of Saturday before 3rd Sunday of August or 9th to 13th Sunday after Pentecost
\medskip
\printnote{If today is Sunday, the Vespers of the Sunday is commemorated with \emph{Magnificat antiphon}, \emph{\Vbar{} Dirigátur.} in simple commemoration tone, p.~\pageref{vr-dirigatur} and \emph{Collect}.  Otherwise \Vbar~\emph{Benedicámus Dómino 1}, p.~\pageref{benedicamusdomino-1}.

\begin{multicols}{2}
\noindent\emph{\nth{9} Sunday after Pentecost}, p.~\pageref{pentecost9}.\\
\emph{\nth{10} Sunday after Pentecost}, p.~\pageref{pentecost10}.\\
\emph{\nth{11} Sunday after Pentecost}, p.~\pageref{pentecost11}.\\
\emph{\nth{12} Sunday after Pentecost}, p.~\pageref{pentecost12}.\\
\emph{\nth{13} Sunday after Pentecost}, p.~\pageref{pentecost13}.
\end{multicols}
}
\medskip
\hrule
% \bigskip
% \benedicamusdomino{}
}

%Sep 14: Exaltation of Holy Cross (2nd class)
{
\section{September 14: Exaltation of the Holy Cross}
\subtitle{\nth{2} Class}
\subtitle{I \& II Vespers}

\def\definevesperspropers{\newcommand{\maganttex}{MagnificatAntiphon2-OCrux}
\newcommand{\magantinitial}{O}
\newcommand{\maganttranslation}{O blessed art thou, O Cross which wast counted the only tree worthy to bear the Lord and King of heaven. Alleluia.}
\def\magsolemn{F}
\definemag{1}{D2}

  \def\prepsalmfive{\greseteolcustos{manual}}
}
\def\definevesperspropersalt{\newcommand{\maganttex}{MagnificatAntiphon1-OCrux}
\newcommand{\magantinitial}{T}
\newcommand{\maganttranslation}{Hail, O Cross Brighter than all the stars thy name is honourable upon earth; To the eyes of men thou art exceeding lovely; holy art thou among all things that are earthly; thy transom made the one worthy balance whereon the price of the world was weighed; sweetest wood and sweetest iron, sweetest weight is hung on thee; O that every one that is here gathered this day to praise thee may find that thou art indeed salvation for him.}
\def\magsolemn{F}
\definemag{1}{D}
}
\def\vesperspropersnote{At II Vespers:}
\def\vesperspropersaltnote{At I Vespers:}
%\def\premag{\def\noeuouae{T}}
\def\premagverses{\greseteolcustos{manual}}

\def\begincollectcols{\begin{parcolumns}[rulebetween,colwidths={1=0.42\linewidth}]{2}}
\printvespers[../September14-ExaltationOfTheHolyCross]{inc-ExaltationOfTheHolyCross}
\bigskip
\benedicamusdomino[2]{}
}

%Sep 29: Dedication of St Michael (1st class)
{
\section{September 29: Dedication of St Michael the Archangel}
\subtitle{\nth{1} Class}
\subtitle{I \& II Vespers}

\def\definevesperspropers{\definepsalm{5}{137}{7}{c}

\newcommand{\vrtex}{vrInConspectuAngelorum}
\newcommand{\vtranslation}{In the sight of the Angels, I will sing praise to Thee, O my God.}
\newcommand{\rtranslation}{I will worship towards Thy holy temple, and give glory to Thy name.}

\newcommand{\maganttex}{MagnificatAntiphon2}
\definemag{1}{D2}
\newcommand{\magantinitial}{P}
\newcommand{\maganttranslation}{O most glorious prince, Michael the Archangel, be mindful of us, and here and everywhere entreat the Son of God for us, alleluia, alleluia.}

  \def\prepsalmfive{\greseteolcustos{manual}}
}
\def\definevesperspropersalt{\definepsalm{5}{116}{7}{c}

\newcommand{\vrtex}{vrStetitAngelus}
\newcommand{\vtranslation}{The Angel stood by the altar of the temple.}
\newcommand{\rtranslation}{Holding in his hand a censer of gold.}

\newcommand{\maganttex}{MagnificatAntiphon1}
\newcommand{\magantinitial}{D}
\definemag{8}{G}
\newcommand{\maganttranslation}{While John was beholding the sacred Mystery, the Archangel Michael sounded a trumpet.  Forgive us, O Lord our God, Thou who openest the book, and loosest the seals thereof.  Alleluia.\vspace{-1ex}}
}
\def\vesperspropersnote{At II Vespers:}
\def\vesperspropersaltnote{At I Vespers:}
%\def\premag{\def\noeuouae{T}}
\def\premagverses{\greseteolcustos{manual}}

\printvespers[../September29-DedicationOfChurchOfStMichaelArchangel]{inc-DedicationStMichael}

\medskip
\printnote{If today is Sunday, the Vespers of the Sunday is commemorated with \emph{Magnificat antiphon}, \emph{\Vbar{} Dirigátur.} in simple commemoration tone, p.~\pageref{vr-dirigatur} and \emph{Collect}.  Otherwise \Vbar~\emph{Benedicámus Dómino 1}, p.~\pageref{benedicamusdomino-1}.

\begin{multicols}{2}
\noindent\emph{\nth{16} Sunday after Pentecost}, p.~\pageref{pentecost16}.\\
\emph{\nth{17} Sunday after Pentecost}, p.~\pageref{pentecost17}.\\
\emph{\nth{18} Sunday after Pentecost}, p.~\pageref{pentecost18}.\\
\emph{\nth{19} Sunday after Pentecost}, p.~\pageref{pentecost19}.\\
\emph{\nth{20} Sunday after Pentecost}, p.~\pageref{pentecost20}.
\end{multicols}
}
\medskip
\hrule
% \bigskip
% \benedicamusdomino{}
}

%Last Sunday in October: Christ the King (1st class)
{
\section{Last Sunday in October: Jesus Christ, King}
\subtitle{\nth{1} Class}
\subtitle{II Vespers}

\def\definevesperspropers{\newcommand{\vrtex}{vr}
\newcommand{\vtranslation}{His dominion shall be increased.}
\newcommand{\rtranslation}{And of peace there shall be no end.}

\newcommand{\magantinitial}{H}
\newcommand{\maganttex}{MagnificatAntiphon}
\newcommand{\maganttranslation}{He hath on His garment and on His thigh written: King of kings and Lord of lords.  To Him be glory and empire for ever and ever.}
\def\magsolemn{T}
\definemag{7}{a}

  \def\prepsalmfive{\greseteolcustos{manual}}
}
%\def\premag{\def\noeuouae{T}}
\def\premagverses{\greseteolcustos{manual}}
\def\beginchaptercols{\begin{parcolumns}[rulebetween,colwidths={1=0.46\linewidth}]{2}}
%\def\begincollectcols{\begin{parcolumns}[rulebetween,colwidths={1=0.45\linewidth}]{2}}

\printvespers[../OctoberLastSunday-ChristTheKing]{inc-ChristTheKing}
\noindent
\printnote{If today is October 31, the First Vespers of All Saints is commemorated with \emph{Magnificat}, p.~\pageref{allsaints1-magnificat}; \emph{\Vbar{}~Lætámini}.~in simple commemoration tone, p.~\pageref{allsaints1-vr}; and \emph{Collect}, p.~\pageref{allsaints-collect}.  Otherwise \Vbar~\emph{Benedicámus Dómino 1}, p.~\pageref{benedicamusdomino-1}.}
\bigskip
\hrule
%\bigskip
%\benedicamusdomino{}
}

%Nov 1: All Saints (1st class)
{
\section{November 1: All Saints}
\subtitle{\nth{1} Class}
\subtitle{I \& II Vespers}

\def\definevesperspropers{\definepsalm{5}{115}{8}{G}

\newcommand{\vrtex}{vrExsultabunt}
\newcommand{\vtranslation}{The Saints will rejoice in glory.}
\newcommand{\rtranslation}{They will be joyful upon their beds.}

\newcommand{\maganttex}{an--o_quam_gloriosum--solesmes}
\newcommand{\magantinitial}{O}
\newcommand{\maganttranslation}{Oh! how glorious is the kingdom where all the Saints rejoice with Christ; clothed in white robes, they follow the Lamb whithersoever he goeth!}
\def\magsolemn{T}
\definemag{6}{F}

  \def\prepsalmfive{\greseteolcustos{manual}}
}
\def\definevesperspropersalt{\definepsalm{5}{116}{8}{G}

\newcommand{\vrtex}{vrLaetamini}
\newcommand{\vtranslation}{Be glad in the Lord, and rejoice ye righteous.}
\newcommand{\rtranslation}{And shout for joy, all ye that are upright in heart.}

\newcommand{\maganttex}{an--angeli_archangeli_all_saints--solesmes}
\newcommand{\magantinitial}{A}
\newcommand{\maganttranslation}{O ye Angels, Archangels, Thrones and Dominions, Principalities and Powers, Virtues, Cherubim and Seraphim, Patriarchs and Prophets, holy Teachers of the Law, all Apostles, Martyrs of Christ, holy Confessors, Virgins of the Lord, Hermits, and all Saints, intercede for us.}
\def\magsolemn{T}
\definemag{1}{D}
\def\premag{\label{allsaints1-magnificat}}}
\def\vraltlabel{allsaints1-vr}
\def\vesperspropersnote{At II Vespers:}
\def\vesperspropersaltnote{At I Vespers:}
\def\begincollectcols{\label{allsaints-collect}\begin{parcolumns}[rulebetween]{2}}
%\def\premag{\def\noeuouae{T}}
\def\premagverses{\greseteolcustos{manual}}

\printvespers[../November1-AllSaints]{inc-AllSaints}

\medskip
\printnote{If today is Sunday, the Vespers of the Sunday is commemorated with \emph{Magnificat antiphon}, \emph{\Vbar{} Dirigátur.} in simple commemoration tone, p.~\pageref{vr-dirigatur} and \emph{Collect}.  Otherwise \Vbar~\emph{Benedicámus Dómino 1}, p.~\pageref{benedicamusdomino-1}.

\begin{multicols}{2}
\noindent\emph{\nth{21} Sunday after Pentecost}, p.~\pageref{pentecost21}.\\
\emph{\nth{22} Sunday after Pentecost}, p.~\pageref{pentecost22}.\\
\emph{\nth{23} Sunday after Pentecost}, p.~\pageref{pentecost23}.\\
\emph{\nth{4} Sunday after Epiphany}, p.~\pageref{epiphany4}.
\end{multicols}
}
\medskip
\hrule

%     TODO Feast of All Saints (11/1) commem of Sunday, Saturday before 1st Sunday of November, 21st - 23rd Sunday after Pentecost or 4th after Epiphany
% \bigskip
% \benedicamusdomino{}
}

%Nov 9: Dedication of Archbasilica of Holy Savior (2nd class)
{
\section{November 9: Dedication of Archbasilica of Holy Savior}
\subtitle{\nth{2} Class}
\subtitle{I \& II Vespers}
\printnote{All from the Common of the Dedication of a Church.}

\def\definevesperspropers{\newcommand{\vrtex}{vrDomumTuam}
\newcommand{\vtranslation}{Holiness becometh thy house, O Lord.}
\newcommand{\rtranslation}{Forever.}

\newcommand{\maganttex}{MagnificatAntiphon2-OQuamMetuendusEst}
\newcommand{\magantinitial}{O}
\newcommand{\maganttranslation}{How dreadful is this place. Surely this is none other but the house of God, and the gate of heaven.}
\def\magsolemn{T}
\def\magoneline{T}
\definemag{6}{F}

  \def\prepsalmfive{\greseteolcustos{manual}}
}
\def\definevesperspropersalt{\newcommand{\vrtex}{vrHaecEstDomusDomini}
\newcommand{\vtranslation}{This is the house of the Lord, strongly built.}
\newcommand{\rtranslation}{It is well founded upon strong rock.}

\newcommand{\maganttex}{MagnificatAntiphon1-Sanctificavit}
\newcommand{\magantinitial}{S}
\newcommand{\maganttranslation}{.}
\def\magsolemn{T}
\definemag{1}{g}
}
\def\vesperspropersnote{At II Vespers:}
\def\vesperspropersaltnote{At I Vespers:}
%\def\premag{\def\noeuouae{T}}
\def\premagverses{\greseteolcustos{manual}}

\printvespers[../CommonOfDedicationOfChurch]{inc-DedicationOfChurch}
\bigskip
\benedicamusdomino[2]{}
}

}


	{
	\label{appendix}
  \cleardoublepage  
  \chapter{Appendix}
  \section{Common of the Dedication of a Church}
  \label{commondedicationofchurch}
  {
    \def\printhymn{%
      , \emph{Hymn 2}, p.~\pageref{hymn-feast}.

      \medskip
    }
    \def\printshortresp{%
      \label{shortresp-dedication}%
      {\def\gabcfolder{.}
      \printgabc{Short}{Resp.}{D}{rb--domum_tuam--solesmes_1961}
      \translation[]{\Vbar{}~Holiness becometh thine house, O Lord.
        \Rbar{}~Holiness\dots{}
        \Vbar{}~For ever.
        \Rbar{}~Holiness becometh.
        \Vbar{}~Glory be\dots{}
        \Rbar{}~Holiness\dots{}
      }
      \medskip
      {\centering In Paschal Time.\par}
      \printgabc{Short}{Resp.}{D}{rb--domum_tuam_(paschal_time)--solesmes_1961}
      }

      \bigskip
      \gresetinitiallines{0}\label{vr-dedication}
      \gregorioscore{vr-locus-iste}
      \let\myhwidth\relax
      \let\myhhwidth\relax
      \let\myhwidthb\relax
      \let\myhhwidthb\relax
      \newlength{\myhwidth}
      \settowidth{\myhwidth}{pópul}
      \newlength{\myhhwidth}
      \settowidth{\myhhwidth}{do}
      \addtolength{\myhhwidth}{-\myhwidth}
      \newlength{\myhwidthb}
      \settowidth{\myhwidthb}{\emph{T. P.} Allelui}
      \newlength{\myhhwidthb}
      \settowidth{\myhhwidthb}{i}
      \addtolength{\myhhwidthb}{-\myhwidthb}
      \def\myhspace{\hspace{3.5ex}}
      \begin{nstabbing}
      %\>\Rbar{}~Et \myhspace{} omnes \myhspace{} reges \myhspace{} terræ \myhspace{} glóriam \>\hspace{\myhhwidth}tuam.
      \>\Rbar{}~Pro\myhspace{} delíctis\myhspace{} et\myhspace{} peccátis \>\hspace{\myhhwidth}pópuli. \>\hspace{\myhhwidthb}\emph{T. P.} Alleluia.
      \end{nstabbing}

      \translation[]{\Vbar{}~This place is holy, wherein the Priest prayeth.\\
      \Rbar{}~For the pardon of the transgressions and offences of the people.}

      \bigskip
    }

    \printterce[../CommonOfDedicationOfChurch]{inc-DedicationOfChurch}{dedication}
  }

}


}
\end{document}

