% !TEX TS-program = lualatex
% !TEX encoding = UTF-8

% This is a simple template for a LuaLaTeX document using gregorio scores.

\documentclass[letterpaper,12pt]{book} % use larger type; default would be 10pt
\usepackage{../definepsalms}
\usepackage{titlesec}
\usepackage{titletoc}
\usepackage{titleps}
\usepackage{letltxmacro}
\usepackage{changepage} % gives us \ifoddpage use [strict]
\usepackage{adjustbox}
\usepackage{amsfonts}

\LetLtxMacro{\oldneedspace}{\needspace}
\renewcommand{\needspace}[1]{
	\checkoddpage\ifoddpage\oldneedspace{#1}\else\fi
}
\newcommand{\raisedot}[1]{\adjustbox{raise=1pt}{#1}}
\newcommand{\pushhepisemusright}[1]{\adjustbox{lap=1pt}{#1}}

%\usepackage{hyperref}
\newcommand{\phantomsection}{}

\setcounter{secnumdepth}{-1}

%\def\mywidth{6in}
%\def\myheight{9in}

\def\mywidth{8.5in}
\def\myheight{11in}

% !TEX TS-program = lualatex
% !TEX encoding = UTF-8
% usual packages loading:
%\usepackage{luatextra}
%\usepackage{graphicx} % support the \includegraphics command and options
\usepackage{geometry} % See geometry.pdf to learn the layout options. There are lots.
\ifx\undefined\mywidth
    \geometry{letterpaper} % or letterpaper (US) or a5paper or....
\else
    \geometry{papersize={\mywidth,\myheight}}
\fi
\usepackage{expl3}
\let\luatexlocalrightbox\localrightbox
\let\luatexlocalleftbox\localleftbox
\usepackage{gregoriotex} % for gregorio score inclusion
\usepackage{import}

% If you use usual TeX fonts, here is a starting point:
%\usepackage{palatino}
%\input{glyphtounicode} \pdfglyphtounicode{f_f}{FB00} \pdfglyphtounicode{f_f_i}{FB03} \pdfglyphtounicode{f_f_l}{FB04}
%\pdfglyphtounicode{Q_u}{E048} \pdfglyphtounicode{O_e}{0152} \pdfglyphtounicode{o_e}{0153}
%\pdfgentounicode=1
% to change the font to something better, you can install the kpfonts package (if not already installed). To do so
% go open the "TeX Live Manager" in the Menu Start->All Programs->TeX Live 2010
% the additional width of the additional lines (compared to the width of the glyph they're associated with)
\grechangedim{additionallineswidth}{0.14584 cm}{scalable}%
% width of the additional lines, used only for the custos (maybe should depend on the width of the custos...)
% the width is the one for the custos at end of lines, the line for custos in the middle of a score is the same
% multiplied by 2.
\grechangedim{additionalcustoslineswidth}{0.09114 cm}{scalable}%
% null space
\grechangedim{zerowidthspace}{0 cm}{scalable}%
% space between glyphs in the same element
\grechangedim{interglyphspace}{0.06927 cm plus 0.00363 cm minus 0.00363 cm}{scalable}%
% space between an alteration (flat or natural) and the next glyph
\grechangedim{alterationspace}{0.07747 cm plus 0.01276 cm minus 0.00455 cm}{scalable}%
% space between a clef and a flat (for clefs with flat)
\grechangedim{clefflatspace}{0.05469 cm plus 0.00638 cm minus 0.00638 cm}{scalable}%
% space before a choral sign
\grechangedim{beforelowchoralsignspace}{0.04556 cm plus 0.00638 cm minus 0.00638 cm}{scalable}%
% when bolshifts are enabled, minimal space between a clef at the beginning of the line and a leading alteration glyph (should be larger than clefflatspace so that a flatted clef can be distinguished from a flat which is part of the first glyph on a line, but also smaller than spaceafterlineclef, the distance from the clef to the first notes)
\grechangedim{beforealterationspace}{0.1 cm}{scalable}%
% space between elements
\grechangedim{interelementspace}{0.06927 cm plus 0.00182 cm minus 0.00363 cm}{scalable}%
% larger space between elements
\grechangedim{largerspace}{0.10938 cm plus 0.01822 cm minus 0.00911 cm}{scalable}%
% space between elements in ancient notation
\grechangedim{nabcinterelementspace}{0.06927 cm plus 0.00182 cm minus 0.00363 cm}{scalable}%
% larger space between elements in ancient notation
\grechangedim{nabclargerspace}{0.10938 cm plus 0.01822 cm minus 0.00911 cm}{scalable}%
% space between elements which has the size of a note
\grechangedim{glyphspace}{0.21877 cm plus 0.01822 cm minus 0.01822 cm}{scalable}%
% space before custos
\grechangedim{spacebeforecustos}{0.1823 cm plus 0.31903 cm minus 0.0638 cm}{scalable}%
% space before punctum mora and augmentum duplex
\grechangedim{spacebeforesigns}{0.05469 cm plus 0.00455 cm minus 0.00455 cm}{scalable}%
% space after punctum mora and augmentum duplex
\grechangedim{spaceaftersigns}{0.08203 cm plus 0.0082 cm minus 0.0082 cm}{scalable}%
% space after a clef at the beginning of a line
\grechangedim{spaceafterlineclef}{0.27345 cm plus 0.14584 cm minus 0.01367 cm}{scalable}%
% minimal space between notes of different words
%\grechangedim{interwordspacenotes}{0.27 cm plus 0.15 cm minus 0.05 cm}{scalable}%
\grechangedim{interwordspacenotes}{0.27 cm plus 0.08 cm minus 0.05 cm}{scalable}%
% minimal space between notes of the same syllable.
% Warning: always keep minus to 0; also keep plus very low, or some words won't be hyphenated
%\grechangedim{intersyllablespacenotes}{0.24 cm plus 0.04cm minus 0cm}{scalable}%
\grechangedim{intersyllablespacenotes}{0.24 cm plus 0.04cm minus 0cm}{scalable}%
% minimal space between letters of different words. Makes sense to have
% the same plus and minus as interwordspacenotes.
%\grechangedim{interwordspacetext}{0.38 cm plus 0.15 cm minus 0.05 cm}{scalable}%
\grechangedim{interwordspacetext}{0.18 cm plus 0.08 cm minus 0.05 cm}{scalable}%
% Versions of interword spaces for euouae blocks
%\grechangedim{interwordspacenotes@euouae}{0.19 cm plus 0.1 cm minus 0.05 cm}{scalable}%
\grechangedim{interwordspacenotes@euouae}{0.13 cm plus 0.1 cm minus 0.05 cm}{1}%
%\grechangedim{interwordspacetext@euouae}{0.27 cm plus 0.1 cm minus 0.05 cm}{scalable}%
\grechangedim{interwordspacetext@euouae}{0.13 cm plus 0.1 cm minus 0.05 cm}{1}%
% space between notes of a bivirga or trivirga
\grechangedim{bitrivirspace}{0.06927 cm plus 0.00182 cm minus 0.00546 cm}{scalable}%
% space between notes of a bistropha or tristrophae
\grechangedim{bitristrospace}{0.06927 cm plus 0.00182 cm minus 0.00546 cm}{scalable}%
% space between two punctum inclinatum
\grechangedim{punctuminclinatumshift}{-0.03918 cm plus 0.0009 cm minus 0.0009 cm}{scalable}%
% space before puncta inclinata
\grechangedim{beforepunctainclinatashift}{0.05286 cm plus 0.00728 cm minus 0.00455 cm}{scalable}%
% space between a punctum inclinatum and a punctum inclinatum deminutus
\grechangedim{punctuminclinatumanddebilisshift}{-0.02278 cm plus 0.0009 cm minus 0.0009 cm}{scalable}%
% space between two punctum inclinatum deminutus
\grechangedim{punctuminclinatumdebilisshift}{-0.00728 cm plus 0.0009 cm minus 0.0009 cm}{scalable}%
% space between puncta inclinata, larger ambitus (range=3rd)
\grechangedim{punctuminclinatumbigshift}{0.07565 cm plus 0.0009 cm minus 0.0009 cm}{scalable}%
% space between puncta inclinata, larger ambitus (range=4th -or more?-)
\grechangedim{punctuminclinatummaxshift}{0.17865 cm plus 0.0009 cm minus 0.0009 cm}{scalable}%
% space for the bars (inside syllables)
%first for virgula and divisio minima
\grechangedim{spacearoundsmallbar}{0.1823 cm plus 0.22787 cm minus 0.00469 cm}{scalable}%
%then divisio minor
\grechangedim{spacearoundminor}{0.1823 cm plus 0.22787 cm minus 0.00469 cm}{scalable}%
%divisio major
\grechangedim{spacearoundmaior}{0.1823 cm plus 0.22787 cm minus 0.00469 cm}{scalable}%
%divisio finalis
\grechangedim{spacearoundfinalis}{0.1823 cm plus 0.22787 cm minus 0.00469 cm}{scalable}%
%a special space for finalis, for when it is the last glyph
\grechangedim{spacebeforefinalfinalis}{0.29169 cm plus 0.07292 cm minus 0.27345 cm}{scalable}%
% additional space that will appear around bars that are preceded by a custos and followed by a key.
\grechangedim{spacearoundclefbars}{0.03645 cm plus 0.00455 cm minus 0.0009 cm}{scalable}%
% space between the text and the text of the bar
\grechangedim{textbartextspace}{0.24611 cm plus 0.13672 cm minus 0.04921 cm}{scalable}%
% minimal space between a note and a bar
\grechangedim{notebarspace}{0.31903 cm plus 0.27345 cm minus 0.02824 cm}{scalable}%
% maximal space between two syllables for which we consider a dash is not needed
\grechangedim{maximumspacewithoutdash}{0.00 cm}{scalable}%
% an extensible space for the beginning of lines
\grechangedim{afterclefnospace}{0 cm plus 0.27345 cm minus 0 cm}{scalable}%
% space between the initial and the beginning of the score
\grechangedim{afterinitialshift}{0.2457 cm}{scalable}%
% space before the initial
\grechangedim{beforeinitialshift}{0.2457 cm}{scalable}%
% when bolshifts are enabled, minimum space between beginning of line and first syllable text
\grechangedim{minimalspaceatlinebeginning}{0.05 cm}{scalable}%
% space to force the initial width to.  Ignored when 0.
\grechangedim{manualinitialwidth}{0 cm}{scalable}%
% distance to move the initial up by
\grechangedim{initialraise}{0 cm}{scalable}%
% Space between lines in the annotation
\grechangedim{annotationseparation}{0.05cm}{scalable}%
% Amount to raise (positive) or lower (negative) the annotations from the default position (base line of top annotation aligned with top line of staff)
\grechangedim{annotationraise}{0cm}{scalable}%
% space at the beginning of the lines if there is no clef
\grechangedim{noclefspace}{0.1 cm}{scalable}%
% space around a clef change
\grechangedim{clefchangespace}{0.01768 cm plus 0.00175 cm minus 0.01768 cm}{scalable}%
%When \gre@clivisalignment is 2, this distance is the maximum length of the consonants after vowels for which the clivis will be aligned on its center.
\grechangedim{clivisalignmentmin}{0.3 cm}{scalable}%



%%%%%%%%%%%%%%%%%%
% vertical spaces
%%%%%%%%%%%%%%%%%%

% first, we have two spaces for the chironomic signs
\grechangedim{abovesignsspace}{0.8 cm}{scalable}%
\grechangedim{belowsignsspace}{0 cm}{scalable}%
% the amount to shift down:
% (a) low choral signs that are not lower than the note, regardless of whether
%     it's on a line or in a space
% (b) high choral signs and low choral signs that are lower than the note which
%     are in a space
\grechangedim{choralsigndownshift}{0.00911 cm}{scalable}%
% the amount to shift up:
% (a) high choral signs and low choral signs that are lower than the note which
%     are on a line
\grechangedim{choralsignupshift}{0.04556 cm}{scalable}%
% the space for the translation
\grechangedim{translationheight}{0.5 cm}{scalable}%
%the space above the lines
\grechangedim{spaceabovelines}{0.45576 cm plus 0.36461 cm minus 0.09114 cm}{scalable}%
%the space between the lines and the bottom of the text
\grechangedim{spacelinestext}{0.60617 cm}{scalable}%
%the space beneath the text
\grechangedim{spacebeneathtext}{0 cm}{scalable}%
% height of the text above the note line
\grechangedim{abovelinestextraise}{-0.1 cm}{scalable}%
% height that is added at the top of the lines if there is text above the lines (it must be bigger than the text for it to be taken into consideration)
\grechangedim{abovelinestextheight}{0.3 cm}{scalable}%
% an additional shift you can give to the brace above the bars if you don't like it
\grechangedim{braceshift}{0 cm}{scalable}%
% a shift you can give to the accentus above the curly brace
\grechangedim{curlybraceaccentusshift}{-0.05 cm}{scalable}%


%\def\greinitialformat#1{{\fontsize{37}{37}\selectfont #1}}
%small > footnotesize > scriptsize > tiny


% my stuff
\usepackage[garamond]{../mypackage}
% end my stuff

\setgrefactor{17}

%\marginsize{25pt}{25pt}{25pt}{30pt}
\usepackage{calc}
%\setlength\headsep{20pt}
%\setlength\footskip{15pt}
\setlength\headheight{15pt}
\setlength\headsep{22pt}
\ifx\undefined\tenebrae
    \geometry{outer=25pt,inner=25pt,top=22pt+\headsep+\headheight,bottom=25pt+\footskip}
\else
    \ifx\undefined\mywidth
        %had been .3 outer, .4 inner
        %let's try .75 for inner and .5 for outer
        %now let's go to .35 outer, .9 inner
        %this time let's try .4 and .85
        \ifbook{\geometry{outer=0.4in,inner=0.85in,top=25pt+\headsep+\headheight,bottom=25pt+\footskip,twoside=true}}
        \ifnotbook{\geometry{outer=0.625in,inner=0.625in,top=25pt+\headsep+\headheight,bottom=25pt+\footskip,twoside=true}}
    \else
        \setlength\headsep{0.25in}
        \setlength\footskip{0.3in}
        \geometry{outer=0.5in,inner=0.5in,top=0.25in+\headsep+\headheight,bottom=0.25in+\footskip,twoside=true}
    \fi
\fi

\pagestyle{fancy} % no header or footers
\let\oldheadrulewidth\headrulewidth
\renewcommand\headrulewidth{\ifnum\thepage=1
0pt
\else
\oldheadrulewidth
\fi}

\ifx\undefined\ifbook
    \newcommand{\ifbook}[1]{}
    \newcommand{\ifnotbook}[1]{#1}
\fi
\ifx\undefined\ifsmallbook
    \newcommand{\ifsmallbook}[1]{}
    \newcommand{\ifnotsmallbook}[1]{#1}
\fi
%\cfoot{\thepage}

\setlength\headheight{0.25in+15pt}
\setlength\headsep{1pc}
\setlength\topskip{0pc}
\setlength\footskip{1pc}
\geometry{outer=0.4in,inner=0.85in,top=0pc+\headheight+\headsep,bottom=0.4in,twoside=true}
\newpagestyle{main}{
\sethead[\garamond{\thepage}][\garamond{\chaptertitle}][] % even
{}{\garamond{\sectiontitle}}{\garamond{\thepage}} % odd
\setfoot[][][] % even
{}{}{} % odd
}
\pagestyle{main}


\titleformat
{\section} % command
[block] % shape
{\thispagestyle{empty}\phantomsection
\large\addfontfeature{Numbers=Lining}\scshape} % format
{} % label
{} % sep
{
    % \rule{\textwidth}{1pt}
    % \vspace{1ex}
    \centering
} % before-code
%[
% \vspace{-0.5ex}%
% \rule{\textwidth}{0.3pt}
%] % after-code
 
 
\titleformat{\chapter}[block]
{\thispagestyle{empty}\phantomsection\Large\scshape\addfontfeature{Numbers=Lining}}
{}{0.5em}{\centering}
 
\titlespacing{\chapter}{0pt}{-\headheight}{1pc}
\titlespacing{\section}{0pt}{-\headheight}{1pc}
\titleclass{\chapter}{top}
\titleclass{\section}{top}
\newcommand{\chapterbreak}{\clearpage}
%\titleclass{\section}{top}

\contentsmargin{1pc}
\dottedcontents{chapter}[0pc]{}{2pc}{1pc}
\dottedcontents{section}[0pc]{}{0pc}{1pc}

\newcommand{\printnote}[1]{
	{\normalsize \emph{#1}}
}
\newcommand{\subtitle}[1]{
\begin{center}{
	{\addfontfeature{Numbers=Lining} \normalsize \emph{#1}}
}\end{center}
}

\newcommand{\printcollect}[2]{
	\ifx\undefined\begincollectcols\def\begincollectcols{\begin{parcolumns}[rulebetween]{2}}\fi
	\ifx\printcollectheading\undefined\def\printcollectheading{T}\fi
	\if\printcollectheading T
	\needspace{3\baselineskip}
	\begin{center}{\large Collect.}\end{center}
	\vspace{-0.4\baselineskip}
	\fi
	\begincollectcols
	\sloppy
	\prayer{#1}{#2}
	\end{parcolumns}
	\let\begincollectcols=\undefined
}
\newcommand{\psalmcolsoverride}[1][0]{
\def\beginpsalmcols{\begin{parcolumns}[rulebetween,colwidths={1=0.45\linewidth}]{2}}
\ifnum#1=2
\def\beginpsalmcols{\begin{parcolumns}[rulebetween,colwidths={1=0.46\linewidth}]{2}}
\fi
\ifnum#1=18
\def\beginpsalmcols{\begin{parcolumns}[rulebetween,colwidths={1=0.4425\linewidth}]{2}}
\fi
\ifnum#1=44
\def\beginpsalmcols{\begin{parcolumns}[rulebetween,colwidths={1=0.46\linewidth}]{2}}
\fi
\ifnum#1=71
\def\beginpsalmcols{\begin{parcolumns}[rulebetween,colwidths={1=0.4325\linewidth}]{2}}
\fi
\ifnum#1=84
\def\beginpsalmcols{\begin{parcolumns}[rulebetween,colwidths={1=0.5\linewidth}]{2}}
\fi
\ifnum#1=88
\def\beginpsalmcols{\begin{parcolumns}[rulebetween,colwidths={1=0.475\linewidth}]{2}}
\fi
}
\newcommand{\writeheading}[1]{
	\section{#1}
}

\sloppy
\begin{document}
\normalsize
\setgrefactor{15}
%\pagenumbering{roman}
%\frontmatter

%\pagenumbering{arabic}
%\mainmatter

\chapter{The Office of Matins}

{
	\def\matins{\emph{Matins}}
	\def\lauds{\emph{Lauds}}
	\def\vespers{\emph{Vespers}}
	\def\compline{\emph{Compline}}
	\def\prime{\emph{Prime}}
	\def\terce{\emph{Terce}}
	\def\sext{\emph{Sext}}
	\def\none{\emph{None}}
	\def\th{\textsuperscript{th}}
	\def\rd{\textsuperscript{rd}}
	\setlength{\parindent}{0cm}
	\setlength{\parskip}{\baselineskip}
	The Liturgy of the Roman Rite is comprised of the rites and ceremonies as described in certain Liturgical books. This
	includes the Sacraments, various rites of blessings, but in particular of the Holy Sacrifice of the Mass and the Divine Office.

	The Divine Office consists of major and minor ``hours'' which are spread throughout the day. Minor hours such as \prime,
	\terce{}, \sext{} and \none{} take their name from the hour of the day. The day was considered to start at around 6:00~AM, so
	\prime{} would correspond to the first hour of the day – or 6:00~AM. \terce{}, the third hour, would be around 9:00~AM; \sext{}
	(6\th{} hour) at noon; and \none{} (9\th{} hour) at 3:00~PM. There is an additional minor hour – that of \compline{} – which is not
	specific to an hour in the day, but which is the final night prayer.

	Major hours are those of \matins{}, \lauds{}, and \vespers{}. \matins{} and \lauds{} are the hours typically associated with the early
	hours of the morning to sunrise, while \vespers{} is an evening prayer. Thus the complete cycle of hours of the Divine office would
	consist of \matins{}, \lauds{}, \prime{}, \terce{}, \sext{}, \none{}, \vespers{}, and \compline{}. In religious communities, \matins{} and \lauds{}
	may actually start around 2:30~AM; \prime{} at 6:00~AM; \terce{} at 9:00~AM; \sext{} at 12:00~noon; \none{} at 3:00~PM; \vespers{}
	around 5:30~PM; and finally \compline{} around 8:00~PM.

	The Divine Office constitutes a continuous outpouring of prayer from Church Militant to her Creator, and ideally spans the
	entirety of the Psalter within the course of a single week. The widespread incorporation of the psalms into the Liturgy of
	the Church demonstrates the connective thread from thousands of years of history – from the Old Testament before Christ
	all the way to the present – something that is also consistent in the Sacrifice of the Mass.

	The authorship of the Psalms is attributed to King David roughly 1000 years before Christ, although modern theories
	ascribe multiple authors over a period of time. Regardless of authorship, the psalms are wonderful songs of praise that
	span all the purposes of prayer – Adoration, Petition, Contrition, and Thanksgiving. Thus their prominence throughout all
	of the Liturgy – including as part of the Sacrifice of the Mass.

	The minor hours of the Office essentially consist of 3 antiphons and 3 psalms, with the antiphons sung before and after
	each psalm. The major hours of \lauds{} and \vespers{} essentially consist of 5 antiphons and 5 psalms (or sections of psalms),
	again with the antiphons sung before and after each psalm. Additionally, \lauds{} and \vespers{} have a hymn and a canticle
	(the \emph{Benedictus} in \lauds{}, the \emph{Magnificat} in \vespers{}). \matins{}, however, is very unique in terms of structure. It is subdivided
	into three nocturns, each consisting not only of 3 antiphons and 3 psalms, but also having 3 lessons and 3 responsories.
	There are additional elements – the hymn as with \lauds{} and \vespers{}, but also an \emph{Invitatory} at the very opening and the
	use of \emph{Te Deum} in lieu of the final responsory on feast days. \matins{} is by far the longest of all of the hours of the Office.

	As mentioned above, the day was divided according to hours from 6:00~AM to 6:00~PM. Thus the reference in the Gospel
	to the owner of the vineyard seeking laborers at the 3\rd{}, 6\th{}, 9\th{}, and 11\th{} hours. The Romans divided the hours of the night
	in similar fashion. The hours from 6:00~PM to 6:00~AM were divided into 4 watches of three hours each – and these were
	called ``vigils'', i.e. when the night watch was kept. This type of division is mimicked in dividing the office into the three
	nocturns.

	The term ``\matins{}'' corresponds to morning, specifically dawn, but the origin of the office had its usage during the night
	as the last part of the ``vigils''. \lauds{} (which take name from the use of the Laudate psalms) was the office actually
	associated with dawn. Somewhat surprisingly, there were periods of time in which \lauds{} preceded \matins{}, though the
	order today is \matins{} followed by \lauds{}, often with \lauds{} immediately following the office of \matins{}. For Christmas,
	however, the offices of \matins{} and \lauds{} are separated from one another. \matins{} is followed by the Midnight Mass after
	which \lauds{} would be celebrated\ldots{} followed by the Mass at Dawn, \prime{}, \terce{}, and then the Mass of the Day\ldots{} followed
	then in due course by the remainder of the office.
}

\chapter{Christmas at Matins.}

{

\centering{Pater noster.\hspace{1em}Ave María.\hspace{1em}Credo.}

}
\bigskip{}
\noindent\emph{Celebrant intones:}

\smallskip{}
{% Domine, labia mea aperies...
	\def\annot{\small \Vbar{}}
	\alsetinitialspacing{D}
	\includescore{DomineLabiaMea}

	\emph{\maltese{} on the lips}

	\translation[]{\Vbar{}~Lord, Thou shalt open my lips. \Rbar{}~And my mouth will announce Thy praise.}
	\bigskip
}

%\subtitle{Festal Tone}
\def\includelaustibi{F}
\noindent\emph{Celebrant intones:}

\smallskip{}
\printdeusinadjutorium{}

{
	
	\centering{Invitatory.}

}

\newcommand{\printinvitatoryant}{
	\def\annot{\small 4. g}
	\alsetinitialspacing{C}
	\includescore{in-ChristusNatusEst}	
}
\newcommand{\printinvitatoryantparttwo}{
	\includescore{in-VeniteAdoremus}
}

\noindent\emph{Cantors intone:}

{
	\printinvitatoryant{}

	\smallskip{}
	\emph{The Choir repeats~:} Christus natus est nobis~:~* Veníte, adorémus.
}

\bigskip{}
%\emph{The Cantors sing the verses together from the center; the Choir sings the antiphon after each verse as indicated.}
\noindent\emph{The Cantors:}

\smallskip{}
{
	\def\annot{\small Ps. 94.}
	\alsetinitialspacing{V}
	\includescore{in-ps94-1-Venite}

%	\bigskip
%	\printinvitatoryant{}
}

\bigskip
\noindent\emph{The Cantors:}

{
	\smallskip
	\def\annot{\small \Vbar{} 2.}
	\alsetinitialspacing{Q}
	\includescore{in-ps94-2-QuoniamDeus}

%	\bigskip
%	\printinvitatoryantparttwo{}
}

{
	\bigskip
	\emph{\small In the following Verse, at the words \emph{Veníte, adorémus et procidámus ante Déum.} the Choir kneels.}

	\noindent\emph{The Cantors:}
	\smallskip

	\def\annot{\small \Vbar{} 3.}
	\alsetinitialspacing{Q}
	\includescore{in-ps94-3-QuoniamIpsius}

%	\bigskip
%	\printinvitatoryant{}
}

\bigskip
\noindent\emph{The Cantors:}

{
	\smallskip
	\def\annot{\small \Vbar{} 4.}
	\alsetinitialspacing{Q}
	\includescore{in-ps94-4-Hodie}

%	\bigskip
%	\printinvitatoryantparttwo{}
}

\bigskip
\noindent\emph{The Cantors:}

{
	\smallskip
	\def\annot{\small \Vbar{} 5.}
	\alsetinitialspacing{Q}
	\includescore{in-ps94-5-Quadraginta}

%	\bigskip
%	\printinvitatoryant{}
}

\bigskip
\needspace{6\baselineskip}
\noindent\emph{The Cantors:}

{
	\def\annot{\small \Vbar{} 6.}
	\alsetinitialspacing{Q}
	\includescore{in-ps94-6-GloriaPatri}

%	\bigskip
%	\printinvitatoryantparttwo{}

	\bigskip
	\noindent\emph{The Cantors:}

	\printinvitatoryant{}
}

% hymn

{
	\bigskip
	\noindent\emph{Celebrant intones, alternate verses between Choir 1 (Epistle Side) \& Choir 2 (Gospel Side).}

	\printgabc{Hymn.}{1.}{J}{../Christmas/Hymn-JesuRedemptorOmnium}
}

\pagebreak
\def\alwaysrepeatantiphon{T}
{
	{
  \writeheading{In the 1st Nocturn}
  %\section{1st Nocturn}

  {
    \def\preant{\setgrefactor{17}\ifprintwhointones{\normalsize\vspace{-\baselineskip}\noindent\emph{Celebrant intones:}

    }{}
    \large}
    \def\preantafter{\setgrefactor{17}\large}
    \def\prepsalm{\normalsize}
    \def\prepsalmtext{\ifprintwhointones{\vspace{-\baselineskip}\noindent\emph{First Cantor intones:}

    }{}}
    \def\anttranslation{The Lord hath said unto me: Thou art My son, this day have I begotten thee.}
    %\def\psalmtranslationsmall{T}
    %\def\prerepeatantiphon{\vspace{-20pt}}
    \def\psalmclef{2}
    \printpsalm{1}{2}{8c}{an--dominus_dixit_ad_me--solesmes}{D}
  }

  {
    \bigskip
    \def\preant{\setgrefactor{17}\ifprintwhointones{\noindent\normalsize\emph{Second Cantor intones:}

    }{}
    \large}
    \def\preantafter{\setgrefactor{17}\large}
    \def\prepsalm{\normalsize}
    \def\anttranslation{The Lord is as a bridegroom coming out of his chamber.}
    %\def\psalmtranslationsmall{T}
    %\def\prerepeatantiphon{\vspace{-20pt}}
    \printpsalm{2}{18}{8G}{an--tamquam_sponsus--solesmes}{T}
  }

  {
    %\smallskip
    \def\preant{\setgrefactor{17}\ifprintwhointones{\noindent\normalsize\emph{First Cantor intones:}

    }{}
    \large}
    \def\preantafter{\setgrefactor{17}\large}
    \def\prepsalm{\normalsize}
    \def\anttranslation{Grace is poured into thy lips, therefore God hath blessed thee for ever.}
    %\def\psalmtranslationsmall{T}
    %\def\prerepeatantiphon{\vspace{-20pt}}
    \printpsalm{3}{44}{1a2}{an--diffusa_est--solesmes}{T}
  }

  \bigskip
  {
    %\def\vrnospace{T}
    \printvr[\normalsize]{vr-TamquamSponsus}
    {The Lord is as a bridegroom.}
    {Coming out of his chamber.}
  }

  {
    \bigskip
    \includescore[a]{paterNoster}

    \medskip
    \emph{Absolution.} Exáudi, Dómine Jesu Christe, preces servórum tuórum~\dag{} et miserére nobis,~* qui cum Patre et Spíritu Sancto vivis et regnas in sǽcula sæculórum. \Rbar{}~Amen.

    \emph{The Lector~:} Jube, domne, benedícere.

    \emph{Blessing.} Benedictióne perpétua~* benedícat nos Pater ætérnus. \Rbar{}~Amen.

    \smallskip
    \emph{The following three Lessons of Isaias are read without a Title.}
  }

  \bigskip
  {
    \lessontitle{I}{Chap. 9}

    \begin{parcolumns}[rulebetween,colwidths={1=.44\linewidth}]{2}
    \lesson[\normalsize]{Primo témpore alleviáta est terra Zábulon, et terra Néphthali~: et novíssimo aggraváta est via maris trans Jordánem Galilǽæ géntium.
      Pópulus, qui ambulábat in ténebris, vidit lucem magnam~: habitántibus in regióne umbræ mortis, lux orta est eis.
      Multiplicásti gentem, et non magnificásti lætítiam. Lætabúntur coram te, sicut qui lætántur in messe, sicut exsúltant victóres, capta præda, quando dívidunt spólia.
      Jugum enim óneris ejus, et virgam húmeri ejus, et sceptrum exactóris ejus superásti sicut in die Mádian.
      Quia omnis violénta prædátio cum tumúltu, et vestiméntum mistum sánguine, erit in combustiónem, et cibus ignis.
      Párvulus enim natus est nobis, et fílius datus est nobis, et factus est principátus super húmerum ejus~: et vocábitur nomen ejus, Admirábilis, Consiliárius, Deus, Fortis, Pater futúri sǽculi, Princeps pacis.}{%
      Tu autem, Dómine, miserére nobis.
      \Rbar{}~Deo~grátias.}
    {At the first time the land of Zabulon, and the land of Nephtali was lightly touched: and at the last the way of the sea beyond the Jordan of the Galilee of the Gentiles was heavily loaded.
      The people that walked in darkness, have seen a great light: to them that dwelt in the region of the shadow of death, light is risen.
      Thou hast multiplied the nation, and hast not increased the joy. They shall rejoice before thee, as they that rejoice in the harvest, as conquerors rejoice after taking a prey, when they divide the spoils.
      For the yoke of their burden, and the rod of their shoulder, and the sceptre of their oppressor thou hast overcome, as in the day of Median.
      For every violent taking of spoils, with tumult, and garment mingled with blood, shall be burnt, and be fuel for the fire.
      For a child is born to us, and a son is given to us, and the government is upon his shoulder: and his name shall be called, Wonderful, Counsellor, God the Mighty, the Father of the world to come, the Prince of Peace.
      But thou, O Lord, have mercy upon us.}
    \end{parcolumns}

  }

  {
    \bigskip
    \setgrefactor{17}
    \printgabc{1. Resp.}{5.}{H}{re--hodie_nobis_caelorum--solesmes}
    \vspace{-\baselineskip}
    {
    \translationcolumns[\normalsize]{\Rbar{}~Today the King of heaven was pleased to be born of a Virgin, that He might bring back to heaven man who was lost.
* There is joy among the hosts of Angels, because eternal salvation hath appeared unto men.
\Vbar{}~Glory to God in the highest, and on earth peace to men of goodwill.
\Rbar{}~There is joy~\dots{}
\Vbar{}~Glory be~\dots{}
\Rbar{}~Today~\ldots{}}
    }

    \emph{The Lector~:} Jube, domne, benedícere.

    \emph{Blessing.} Unigénitus Dei Fílius~* nos benedícere et adjuváre dignétur. \Rbar{}~Amen.
  }

  \bigskip
  {
    \lessontitle{II}{Chap. 40}

    \begin{parcolumns}[rulebetween,colwidths={1=.45\linewidth}]{2}
    \lesson[\normalsize]{Consolámini, consolámini, pópule meus, dicit Deus vester.}{%
      Loquímini ad cor Jerúsalem, et advocáte eam~: quóniam compléta est malítia ejus, dimíssa est iníquitas illíus~: suscépit de manu Dómini duplícia pro ómnibus peccátis suis.
      Vox clamántis in desérto~: Paráte viam Dómini, rectas fácite in solitúdine sémitas Dei nostri.
      Omnis vallis exaltábitur, et omnis mons et collis humiliábitur~: et erunt prava in dirécta, et áspera in vias planas.
      Et revelábitur glória Dómini~: et vidébit omnis caro páriter quod os Dómini locútum est.
      Vox dicéntis~: Clama. Et dixi~: ¿Quid \emph{clamá}bo? Omnis caro fœnum, et omnis glória ejus quasi flos agri.
      Exsiccátum est fœnum, et cécidit flos~: quia spíritus Dómini sufflávit in eo. Vere fœnum est pópulus~:
      exsiccátum est fœnum, et cécidit flos~: Verbum autem Dómini nostri manet in ætérnum.
      Tu autem, Dómine, miserére nobis.
      \Rbar{}~Deo~grátias.}
    {Be comforted, be comforted, my people, saith your God.
      Speak ye to the heart of Jerusalem, and call to her: for her evil is come to an end, her iniquity is forgiven: she hath received of the hand of the Lord double for all her sins.
      The voice of one crying in the desert: Prepare ye the way of the Lord, make straight in the wilderness the paths of our God.
      Every valley shall be exalted, and every mountain and hill shall be made low, and the crooked shall become straight, and the rough ways plain.
      And the glory of the Lord shall be revealed, and all flesh together shall see, that the mouth of the Lord hath spoken.
      The voice of one saying: Cry. And I said: What shall I cry? All flesh is grass, and all the glory thereof as the flower of the field.
      The grass is withered, and the flower is fallen, because the spirit of the Lord hath blown upon it. Indeed the people is grass:
      The grass is withered, and the flower is fallen: but the word of our Lord endureth for ever.
      But thou, O Lord, have mercy upon us.}
    \end{parcolumns}

  }

  {
    \bigskip
    \setgrefactor{17}
    \printgabc{2. Resp.}{8.}{H}{re--hodie_nobis_de_caelo--solesmes}
    {
    \translationcolumns[\normalsize]{\Rbar{}~Today true peace comes down unto us from heaven.
* Today throughout the whole world the skies drop down sweetness.
\Vbar{}~Today is the daybreak of our new redemption, of the restoring of the old, of everlasting joy.
\Rbar{}~Today throughout the whole world the skies drop down sweetness.}
    }
    \medskip

    \emph{The Lector~:} Jube, domne, benedícere.

    \emph{Blessing.} Spíritus Sancti grátia~* illúminet sensus et corda nostra. \Rbar{}~Amen.
  }

  \bigskip
  {
    \lessontitle{III}{Chap. 52}

    \begin{parcolumns}[rulebetween]{2}
    \lesson[\normalsize]{Consúrge, consúrge, indúere fortitúdine tua, Sion, indúere vestiméntis glóriæ tuæ, Jerúsalem,}{cívitas sancti~: quia non adjíciet ultra ut pertránseat per te incircumcísus et immúndus.
      Excútere de púlvere, consúrge, sede, Jerúsalem~: solve víncula colli tui, captíva fília Sion.
      Quia hæc dicit Dóminus~: Gratis venúmdati estis, et sine argénto re\-di\-mé\-mi\-ni.
      Quia hæc dicit Dóminus Deus~: In Ægýptum descéndit pópulus meus in princípio, ut colónus esset ibi~: et Assur absque ulla causa calumniátus est eum.
      Et nunc quid mihi est hic, dicit Dóminus, ¿quóniam ablátus est pópulus me\emph{us gra}tis? Dominatóres ejus iníque agunt, dicit Dóminus~: et júgiter tota die nomen meum blasphemátur.
      Propter hoc sciet pópulus meus nomen meum, in die illa~: quia ego ipse qui loquébar, ecce adsum.
      Tu autem, Dómine, miserére nobis.
      \Rbar{}~Deo~grátias.}
    {Arise, arise, put on thy strength, O Sion, put on the garments of thy glory, O Jerusalem, the city of the Holy One: for henceforth the uncircumcised, and unclean shall no more pass through thee.
      Shake thyself from the dust, arise, sit up, O Jerusalem: loose the bonds from off thy neck, O captive daughter of Sion.
      For thus saith the Lord: You were sold gratis, and you shall be redeemed without money.
      For thus saith the Lord God: My people went down into Egypt at the beginning to sojourn there: and the Assyrian hath oppressed them without any cause at all.
      And now what have I here, saith the Lord: for my people is taken away gratis. They that rule over them treat them unjustly, saith the Lord, and my name is continually blasphemed all the day long.
      Therefore my people shall know my name in that day: for I myself that spoke, behold I am here.
      But thou, O Lord, have mercy upon us.}
    \end{parcolumns}

  }

  {
    \bigskip
    \setgrefactor{17}
    \printgabc{3. Resp.}{4.}{Q}{re--quem_vidistis--solesmes}
  }
  {
    \translationcolumns[\normalsize]{\Rbar{}~O ye shepherds, speak, and tell us what ye have seen; who is appeared in the earth?
* We saw the new-born Child, and Angels singing praise to the Lord.
\Vbar{}~Speak; what have ye seen? And tell us of the Birth of Christ.
\Rbar{}~We saw the new-born Child, and Angels singing praise to the Lord.
\Vbar{}~Glory be~\dots{}
\Rbar{}~We saw the new-born~\dots{}}
  }
  \medskip
}
}

{
	{
  \writeheading{In the 2nd Nocturn}

  {
    \def\preant{\setgrefactor{17}\large}
    \def\prepsalm{\normalsize}
    \def\anttranslation{We have drunk in thy loving-kindness, O God, in the midst of thy temple.}
    %\def\psalmtranslationsmall{T}
    %\def\prerepeatantiphon{\vspace{-20pt}}
    %\def\psalmclef{2}
    \printpsalm{1}{47}{8G}{an--suscepimus_deus--solesmes}{S}
  }

  {
    \def\preant{\setgrefactor{17}\large}
    \def\prepsalm{\normalsize}
    \def\anttranslation{In the Lord's days shall abundance of peace arise and flourish.}
    %\def\psalmtranslationsmall{T}
    %\def\prerepeatantiphon{\vspace{-20pt}}
    \printpsalm{2}{71}{3b}{an--orietur--solesmes}{O}
  }

  {
    \def\preant{\setgrefactor{17}\large}
    \def\prepsalm{\normalsize}
    \def\anttranslation{Truth is sprung out of the earth, and righteousness hath looked down from heaven.}
    %\def\psalmtranslationsmall{T}
    %\def\prerepeatantiphon{\vspace{-20pt}}
    \printpsalm{3}{84}{8c}{an--veritas--solesmes}{V}
  }

  \bigskip
  {
    %\def\vrnospace{T}
    \printvr[\normalsize]{vr-Speciosus}
    {Thou art fairer than the children of men.}
    {Grace is poured into thy lips.}
  }

  {
    {Pater noster.}

    \emph{Absolution.} Ipsíus píetas et misericórdia nos ádjuvet,~* qui cum Patre et Spíritu Sancto vivit et regnat in sǽcula sæculórum. \Rbar{}~Amen.

    \emph{The Lector~:} Jube, domne, benedícere.

    \emph{Blessing.} Deus Pater omnípotens,~* sit nobis propítius et clemens. \Rbar{}~Amen.
  }

  \bigskip{}
  {

    \centering{Sermo sancti Leónis Papæ.}

  }

  {
    \hfil{Lesson IV.}\hfil

    \begin{parcolumns}[rulebetween,colwidths={1=.51\linewidth}]{2}
    \lesson[\normalsize]{Salvátor noster, dilectíssimi, hódie natus est~: gaudeámus. Neque enim fas est locum esse tristítiæ, ubi natális est vitæ~: quæ, consúmpto mortalitátis timóre, nobis íngerit de promíssa æternitáte lætítiam. Nemo ab hujus alacritátis participatióne secérnitur. Una cunctis lætítiæ commúnis est ratio~: quia Dóminus noster, peccáti mortísque destrúctor, sicut nullum a reátu líberum réperit, ita liberándis ómnibus venit. Exsúltet sanctus, quia appropínquat ad palmam~: gáudeat peccátor, quia invitátur ad véniam~: animétur Gentílis, quia vocátur ad vitam. Dei namque Fílius secúndum plenitúdinem témporis, quam divíni consílii inscrutábilis altitúdo dispósuit, reconciliándam auctóri suo natúram géneris assúmpsit humáni, ut invéntor mortis diábolus, per ipsam, quam vícerat, vincerétur.}{
      Tu autem, Dómine, mise\textbf{ré}re \textbf{no}bis.
      \Rbar{}~Deo~grátias.}
    {Dearly beloved brethren, Unto us is born this day a Saviour. Let us rejoice. It would be unlawful to be sad to-day, for today is Life's Birthday; the Birthday of that Life, Which, for us dying creatures, taketh away the sting of death, and bringeth the bright promise of the eternal gladness hereafter. It would be unlawful for any man to refuse to partake in our rejoicing. All men have an equal share in the great cause of our joy, for, since our Lord, Who is the destroyer of sin and of death, findeth that all are bound under the condemnation, He is come to make all free. Rejoice, O thou that art holy, thou drawest nearer to thy crown! Rejoice, O thou that art sinful, thy Saviour offereth thee pardon! Rejoice also, O thou Gentile, God calleth thee to life! For the Son of God, when the fulness of the time was come, which had been fixed by the unsearchable counsel of God, took upon Him the nature of man, that He might reconcile that nature to Him Who made it, and so the devil, the inventor of death, is met and beaten in that very flesh which hath been the field of his victory.
      But thou, O Lord, have mercy upon us.}
    \end{parcolumns}

  }

  {
    \setgrefactor{17}
    \printgabc{4. Resp.}{3.}{O}{re--o_magnum_mysterium--solesmes}
    \medskip

    \emph{The Lector~:} Jube, domne, benedícere.

    \emph{Blessing.} Christus perpétuæ~* det nobis gáudia vitæ. \Rbar{}~Amen.
  }

  \bigskip{}
  {
    \hfil{Lesson V.}\hfil

    \begin{parcolumns}[rulebetween,colwidths={1=.51\linewidth}]{2}
    \lesson[\normalsize]{In quo conflíctu pro nobis ínito, magno et mirábili æquitátis iure certátum est, dum omnípotens Dóminus cum sævíssimo hoste non in sua majestáte, sed in nostra congréditur humilitáte~: objíciens ei eámdem formam, eamdémque natúram, mortalitátis quidem nostræ partícipem, sed peccáti totíus expértem. Aliénum quippe ab hac nativitáte est, quod de ómnibus légitur~: Nemo mundus a sorde, nec infans, cujus est uníus diéi vita super terram. Nihil ergo in istam singulárem nativitátem de carnis concupiscéntia transívit, nihil de peccáti lege manávit. Virgo régia Davídicæ stirpis elígitur, quæ sacro gravidánda fœtu, divínam humanámque prolem prius concíperet mente, quam córpore. Et ne supérni ignára consílii ad inusitátos pavéret affátus, quod in ea operándum erat a Spíritu Sancto, collóquio discit angélico~: nec damnum credit pudóris, Dei Génitrix mox futúra.}{
      Tu autem, Dómine, mise\textbf{ré}re \textbf{no}bis.
      \Rbar{}~Deo~grátias.}
    {When our Lord entered the field of battle against the devil, He did so with a great and wonderful fairness. Being Himself the Almighty, He laid aside His uncreated Majesty to fight with our cruel enemy in our weak flesh. He brought against him the very shape, the very nature of our mortality, yet without sin. Heb. iv. 15. His birth however was not a birth like other births for no other is born pure, nay, not the little child whose life endureth but a day on the earth. To His birth alone the throes of human passion had not contributed, in His alone no consequence of sin had had -part. For His Mother was chosen a Virgin of the kingly lineage of David, and when she was to grow heavy with the sacred Child, her soul had already conceived Him before her body. She knew the counsel of God announced to her by the Angel, lest the unwonted events should alarm her. The future Mother of God knew what was to be wrought in her by the Holy Ghost, and that her modesty was absolutely safe.
      But thou, O Lord, have mercy upon us.}
    \end{parcolumns}

  }

  {
    \setgrefactor{17}
    \printgabc{5. Resp.}{7.}{B}{re--beata_dei_genitrix--solesmes}
    \medskip

    \emph{The Lector~:} Jube, domne, benedícere.

    \emph{Blessing.} Ignem sui amóris~* accéndat Deus in córdibus nostris. \Rbar{}~Amen.
  }

  \bigskip{}
  {
    \hfil{Lesson VI.}\hfil

    \begin{parcolumns}[rulebetween,colwidths={1=.51\linewidth}]{2}
    \lesson[\normalsize]{Agámus ergo, dilectíssimi, grátias Deo Patri, per Fílium ejus in Spíritu Sancto~: qui propter multam caritátem suam, qua diléxit nos, misértus est nostri~: et cum essémus mórtui peccátis, convivificávit nos Christo, ut essémus in ipso nova creatúra, novúmque figméntum. Deponámus ergo véterem hóminem cum áctibus suis~: et adépti participatiónem generatiónis Christi, carnis renuntiémus opéribus. Agnósce, o Christiáne, dignitátem tuam~: et divínæ consors factus natúræ, noli in véterem vilitátem degéneri conversatióne redíre. Meménto, cujus cápitis et cujus córporis sis membrum. Reminíscere, quia érutus de potestáte tenebrárum, translátus es in Dei lumen et regnum.}{
      Tu autem, Dómine, mise\textbf{ré}re \textbf{no}bis.
      \Rbar{}~Deo~grátias.}
    {Therefore, dearly beloved brethren, let us give thanks to God the Father, through His Son, in the Holy Ghost: Who, for His great love wherewith He loved us, hath had mercy on us and, even when we were dead in sins, hath quickened us together with Christ, Eph. ii. 4, 5, that in Him we might be a new creature, and a new workmanship. Let us then put off the old man with his deeds; and, having obtained a share in the Sonship of Christ, let us renounce the deeds of the flesh. Learn, O Christian, how great thou art, who hast been made partaker of the Divine nature, and fall not again by corrupt conversation into the beggarly elements above which thou art lifted. Remember Whose Body it is Whereof thou art made a member, and Who is its Head, Remember that it is He That hath delivered thee from the power of darkness and hath translated thee into God's light, and God's kingdom.
      But thou, O Lord, have mercy upon us.}
    \end{parcolumns}

  }

  {
    \setgrefactor{17}
    \printgabc{6. Resp.}{2.}{S}{re--sancta_et_immaculata--solesmes}
    \medskip
  }
}
}

{
	{
  \writeheading{In the 3rd Nocturn}

  {
    \def\preant{\setgrefactor{17}\large}
    \def\prepsalm{\normalsize}
    \def\anttranslation{He shall cry unto Me, alleluia, Thou art My Father, alleluia.}
    %\def\psalmtranslationsmall{T}
    %\def\prerepeatantiphon{\vspace{-20pt}}
    %\def\psalmclef{2}
    \printpsalm{1}{88}{6F}{an--ipse_invocabit_me--solesmes}{I}
  }

  {
    \def\preant{\setgrefactor{17}\large}
    \def\prepsalm{\normalsize}
    \def\anttranslation{Let the heavens rejoice, and let the earth be glad before the Lord, for He cometh.}
    %\def\psalmtranslationsmall{T}
    %\def\prerepeatantiphon{\vspace{-20pt}}
    \printpsalm{2}{95}{4A}{an--laetentur_caeli--solesmes}{L}
  }

  {
    \def\preant{\setgrefactor{17}\large}
    \def\prepsalm{\normalsize}
    \def\anttranslation{The Lord hath made known, alleluia, His salvation, alleluia.}
    %\def\psalmtranslationsmall{T}
    %\def\prerepeatantiphon{\vspace{-20pt}}
    \printpsalm{3}{97}{6F}{an--notum_fecit_dominus--solesmes}{N}
  }

  \bigskip
  {
    %\def\vrnospace{T}
    \printvr[\normalsize]{vr-IpseInvocabit}
    {He shall cry unto Me, alleluia.}
    {Thou art My Father, alleluia.}
  }

  {
    {Pater noster.}

    \emph{Absolution.} A vínculis peccatórum nostrórum~* absólvat nos omnípotens et miséricors Dóminus. \Rbar{}~Amen.

    \emph{The Lector~:} Jube, domne, benedícere.

    \emph{Blessing.} Evangélica léctio~* sit nobis salus et protéctio. \Rbar{}~Amen.
  }

  \bigskip{}
  {

    \centering{Léctio sancti Evangélii secúndum Lucam.}

  }

  {
    \hspace{10ex}{Lesson VII.}\hfill\emph{Chap. 9}\hspace{10ex}

    \begin{parcolumns}[rulebetween,colwidths={1=.51\linewidth}]{2}
    \lesson[\normalsize]{In illo témpore~: Exiit edíctum a Cǽsare Augústo, ut describerétur univérsus orbis. Et réliqua.}{

Homilía sancti Gregórii Papæ.

Quia, largiénte Dómino, Missárum solémnia ter hódie celebratúri sumus, loqui diu de evangélica lectióne non póssumus; sed nos áliquid vel bréviter dícere, Redemptóris nostri Natívitas ipsa compéllit. Quid est enim, quod nascitúro Dómino mundus descríbitur, nisi hoc, quod apérte monstrátur, quia ille apparébat in carne, qui eléctos suos adscríberet in æternitáte? Quo contra de réprobis per prophétam dícitur~: Deleántur de libro vivéntium, et cum justis non scribántur. Qui bene étiam in Béthlehem náscitur~: Béthlehem quippe domus panis interpretátur. Ipse namque est, qui ait~: Ego sum panis vivus, qui de cælo descéndi. Locus ergo, in quo Dóminus náscitur, domus panis ántea vocátus est; quia futúrum profécto erat, ut ille ibi per matériam carnis apparéret, qui electórum mentes intérna satietáte refíceret. Qui non in paréntum domo, sed in via náscitur; ut profécto osténderet, quia per humanitátem suam, quam assúmpserat, quasi in aliéno nascebátur.
      Tu autem, Dómine, mise\textbf{ré}re \textbf{no}bis.
      \Rbar{}~Deo~grátias.}
    {At that time, it came to pass that there went out a decree from Caesar Augustus, that the whole world should be enrolled. And so on.
Homily by Pope St Gregory the Great.
By God's mercy we are to say three Masses today, so that there is not much time left for preaching; but at the same time the occasion of the Lord's Birth-day itself obliges me to speak a few words. I will first ask why, when the Lord was to be born, the world was enrolled? Was it not to herald the appearing of Him by Whom the elect are enrolled in the book of life? Whereas the Prophet saith of the reprobate Let them be blotted out of the book of the living, and not be written with the righteous. Then, the Lord is born in Bethlehem. Now the name Bethlehem signifieth the House of Bread, and thus it is the birth-place of Him Who hath said, I am the Living Bread, Which came down from heaven. We see then that this name of Bethlehem was prophetically given to the place where Christ was born, because it was there that He was to appear in the flesh by Whom the souls of the faithful are fed unto life eternal. He was born, not in His Mother's house, but away from home. And this is a mystery, showing that this our mortality into which He was born was not the home of Him Who is begotten of the Father before the worlds.
      But thou, O Lord, have mercy upon us.}
    \end{parcolumns}

  }

  {
    \setgrefactor{17}
    \printgabc{7. Resp.}{7.}{B}{re--beata_viscera--solesmes}
    \medskip

    \emph{The Lector~:} Jube, domne, benedícere.

    \emph{Blessing.} Per evangélica dicta~* deleántur nostra delícta. \Rbar{}~Amen.
  }

  \bigskip{}
  {

    \centering{Léctio sancti Evangélii secúndum Lucam.}

  }

  {
    \hspace{10ex}{Lesson VI.}\hfill\emph{Chap. 2. b}\hspace{10ex}

    \begin{parcolumns}[rulebetween,colwidths={1=.51\linewidth}]{2}
    \lesson[\normalsize]{In illo témpore~: Pastóres loquebántur ad ínvicem~: Transeámus usque Béthlehem, et videámus hoc verbum, quod factum est, quod Dóminus osténdit nobis. Et réliqua.}{

Homilía sancti Ambrósii Epíscopi.

Vidéte Ecclésiae surgéntis exórdium~: Christus náscitur, et pastóres vigiláre cœpérunt; qui Géntium greges, pécudum modo ante vivéntes, in caulam Dómini congregárent, ne quos spiritálium bestiárum, per offúsas nóctium ténebras pateréntur incúrsus. Et bene pastóres vígilant, quos bonus pastor infórmat. Grex ígitur pópulus, nox sǽculum, pastóres sunt sacerdótes. Aut fortásse étiam ille sit pastor, cui dícitur~: Esto vígilans, et confírma. Quia non solum epíscopos ad tuéndum gregem Dóminus ordinávit, sed étiam Angelos destinávit.
      Tu autem, Dómine, mise\textbf{ré}re \textbf{no}bis.
      \Rbar{}~Deo~grátias.}
    {At that time, the shepherds said one to another: Let us go over to Bethlehem, and let us see this word that is come to pass, which the Lord hath showed to us. And so on.
Homily by St Ambrose, Bishop of Milan.
Behold the beginning of the Church. Christ is born, and the shepherds watch; shepherds, to gather together the scattered sheep of the Gentiles, and to lead them into the fold of Christ, that they might no longer be a prey to the ravages of spiritual wolves in the night of this world's darkness. And that shepherd is wide awake, whom the Good Shepherd stirreth up. The flock then is the people, the night is the world, and the shepherds are the Priests. And perhaps he is a shepherd to whom it is said, Be watchful and strengthen, Apoc. iii. 2, for God hath ordained as the shepherds of His flock not Bishops only, but also Angels.
      But thou, O Lord, have mercy upon us.}
    \end{parcolumns}

  }

  {
    \setgrefactor{17}
    \printgabc{8. Resp.}{8.}{V}{re--verbum_caro--solesmes}
    \medskip

    \emph{The Lector~:} Jube, domne, benedícere.

    \emph{Blessing.} Verba sancti Evangélii~* dóceat nos Christus Fílius Dei. \Rbar{}~Amen.
  }

  \bigskip{}
  {

    \centering{Léctio sancti Evangélii secúndum Joánnem.}

  }

  {
    \hspace{10ex}{Lesson IX.}\hfill\emph{Chap. 1}\hspace{10ex}

    \begin{parcolumns}[rulebetween,colwidths={1=.51\linewidth}]{2}
    \lesson[\normalsize]{In princípio erat Verbum, et Verbum erat apud Deum, et Deus erat Verbum. Et réliqua.}{

Homilía sancti Augustíni Epíscopi.

Ne vile áliquid putáres quale consuevísti cogitáre, cum verba humána soléres audíre, audi quid cógites~: Deus erat Verbum. Exeat nunc néscio quis infidélis Ariánus et dicat quia Verbum Dei factum est. Quómodo potest fíeri, ut Verbum Dei factum sit, quando Deus per Verbum fecit ómnia? Si et Verbum Dei ipsum factum est; per quod áliud verbum factum est? Si hoc dicis, quia hoc est verbum Verbi, per quod factum est illud; ipsum dico ego únicum Fílium Dei. Si autem non dicis verbum Verbi, concéde non factum, per quod facta sunt ómnia. Non enim per seípsum fíeri pótuit, per quod facta sunt ómnia. Crede ergo Evangelístæ.
      Tu autem, Dómine, mise\textbf{ré}re \textbf{no}bis.
      \Rbar{}~Deo~grátias.}
    {In the beginning was the Word, and the Word was with God, and the Word was God. The same was in the beginning with God. And so on.
Homily by St Augustine, Bishop of Hippo.
Lest thou shouldest think all things mean, as thou art accustomed to think of things human, hear and digest this The Word was God. Now perhaps there will come forward some Arian unbeliever, and say that the Word of God was a creature. How can the Word of God be a creature, when it was by the Word that all creatures were made? If He be a creature, then there must have been some other Word, not a creature, by which He was made. And what Word is that? If thou sayest that it was by the word of the Word Himself that He was made, I tell thee that God had no other, but One Only-begotten Son. But if thou say not that it was by the word of the Word Himself that He was made, thou art forced to confess that. He by Whom all things were made was not Himself made at all. Believe the Gospel.
      But thou, O Lord, have mercy upon us.}
    \end{parcolumns}

  }

  {
    \setgrefactor{17}
    \printgabc{3.}{}{T}{../misc/te-deum-solemn}
    \medskip
  }

  \Vbar{}~Dóminus vobíscum. \Rbar{}~Et cum spíritu túo.
  \bigskip

  {

    \centering{Prayer.}

  }
  \smallskip

  {
    \begin{columns}
    \prayer{Concéde, quǽsumus, omnípotens Deus~:~† ut nos Unigéniti tui nova per carnem Natívitatis líberet;~* quos sub peccáti jugo vetústa sérvitus tenet. Per eúmdem Dóminum.}
           {Grant, we beseech thee, almighty God, that the new birth of thy Only-begotten Son as man may set us free, who are held by the old bondage under the yoke of sin. Through the same Lord.}
    \end{columns}
  }
  \bigskip{}

  \Vbar{}~Dóminus vobíscum. \Rbar{}~Et cum spíritu túo.

  \bigskip{}
  {
    \gresetfirstlineaboveinitial{\small \textbf{5.}}{\small \textbf{5.}}
    \large
    \includescore{../BenedicamusDomino/BenedicamusDomino_1classLauds}
  }
}
}

\pagebreak
{
	\def\gabcfolder{../Tenebrae}
	%\vspace*{-50pt plus 20pt}
\selectlanguage{american}
\begin{center}{%\addfontfeature{Numbers=Lining}%
%\textsc
\huge{Tone for the Prophecy.}
}\end{center}
\large\setgrefactor{17}
{
\def\greabovelinestextstyle#1{%
  \adjustbox{padding=0pt 0pt 0pt 3pt,raise=-1pt}{\small\emph{#1}}%
  \relax %
}
\printgabc{}{}{L}{prophecyTone}
}

\bigskip\bigskip
\emph{Examples of the flex in the case of a monosyllable or of a Hebrew word.}
\medskip
%\let\oldgreabovelinestextstyle=\greabovelinestextstyle
{
\def\greabovelinestextstyle#1{%
  \adjustbox{padding=0pt 0pt 0pt 3pt,raise=-1pt}{\small{#1}}%
  \relax %
}
\greblockcustos
\printgabc{}{}{}{prophecyToneFlex}

\bigskip\bigskip
\emph{Examples of the full stop in the case of a monosyllable or of a Hebrew word.}
\medskip
\greblockcustos
\printgabc{}{}{}{prophecyToneFullStop}
}
\let\greabovelinestextstle=\oldgreabovelinestextstyle
\bigskip\bigskip
\begin{multicols}{2}%
\emph{The flex is made towards the middle of each sentence; it is omitted only if the sentence is very short.  In the longer sentences, the flex may be repeated several times if the sense allows this being done.  The metrum does not occur in this tone.

The flex is made by lowering the voice on the last syllable only, even if the penultimate syllable be not accented.  At the full stop, the voice is lowered on the penultimate syllable, if this be short.

The interrogation has only one modulation, that which has already been indicated for the Prophecy.  It is the same for the Epistle, the Gospel, and the Lessons.  Very short interrogative sentences are sung as follows:}
\end{multicols}

\large
\medskip
\printgabc{}{}{}{ShortInterrogatives}
\bigskip
\begin{multicols}{2}%
\emph{However, the interrogatory formula must never be used at the end, unless the \emph{Tu autem}. is to follow.  If an interrogation occurs at the end of the text to which the \emph{Tu autem}. is not to be added, the ordinary cadential formula must be used.}
\end{multicols}
}
\end{document}

