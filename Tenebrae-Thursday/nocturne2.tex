%\pagebreak
\def\matinsnocturn{2nd Nocturn}
\writeheading{In the 2nd Nocturn}
%
\large
%\printgabc{1. Ant.}{8. c.}{Z}{an--zelus_domus_tuae--solesmes}
{
\def\nogloriapatri{T}
\def\preant{\setgrefactor{17}\large}
\def\preantafter{\setgrefactor{15}\normalsize}
\def\psalmtranslationsmall{T}
%\def\lalinebreakaftersixteen{T}
\def\lalinebreakafterverse{14}
\def\prepsalm{\setgrefactor{15}\normalsize}
\def\anttranslation{The Lord delivered the poor from the mighty: and the needy that had no helper.}
\printpsalm{1}{71}{7c}{an--liberavit_dominus--solesmes}{L}
}
%
\printseparation
\bigskip
{
\def\preant{\setgrefactor{17}\large}
\def\preantafter{\setgrefactor{15}\normalsize}
\def\psalmtranslationsmall{T}
\def\prepsalm{\setgrefactor{17}\normalsize}
\def\anttranslation{The impious have thought and spoken wickedness: they have spoken iniquity on high.}
\def\prepsalmverses{\vspace{-9pt}}
%\def\dontrepeatantiphon{T}
\printpsalm{2}{72}{8c}{an--cogitaverunt_impii--solesmes}{C}
%
}
\printseparation
\bigskip
{
\def\preant{\setgrefactor{17}\large}
\let\preantafter=\undefined
\def\psalmtranslationsmall{T}
\def\prepsalm{\setgrefactor{17}\normalsize\vspace{-5.5pt}}
\def\prepsalmverses{\vspace{-1pt}}
\def\anttranslation{Arise, O Lord, and judge my cause.}
%\def\lalinebreakaftertwentyone{T}
\printpsalm{3}{73}{1g}{an--exsurge_domine_et_judica--solesmes}{E}
}
%psalm 70 - In te, Domine, speravi

\bigskip
\printvr[\setgrefactor{14}\normalsize]{vr-DeusMeus}{O my God, deliver me from the hand of the sinner:}{And out of the hand of the law-breaker and of the unjust man.}

\bigskip
Pater noster. \emph{in silence.}

\begin{center}
Ex Tractátu sancti Augustíni Epíscopi super Psalmos
%\translation{From the treatise of St Augustine, Bishop, upon the psalms}
\end{center}
{
\hspace{10ex}\textsc{lesson iv}\hfill\emph{On Ps. 54, at verse 1}\hspace{10ex}

\begin{parcolumns}[rulebetween,colwidths={1=0.5\linewidth}]{2}
\lesson{Exaudi, Deus, oratiónem meam, et ne despéxeris deprecatiónem meam~: inténde mihi, et exáudi me. Satagéntis, sollíciti, in tribulatióne pósiti, verba sunt ista. Orat multa pátiens, de malo liberári desíderans. Súperest ut videámus in quo malo sit~: et cum dícere cœperit: agnoscámus ibi nos esse~: ut communicáta tribulatióne, conjungámus oratiónem. Contristátus sum, inquit, in exercitatióne mea, et conturbátus sum. Ubi contristátus? ubi conturbátus? In exercitatióne mea, inquit. Hómines malos, quos pátitur, commemorátus est~: eamdémque passiónem malórum hóminum, exercitatiónem suam dixit. Ne putétis gratis esse malos in hoc mundo, et nihil boni de illis ágere Deum. Omnis malus aut ídeo vivit, ut corrigátur; aut ídeo vivit, ut per illum bonus exerceátur.}
{Hear my prayer, O God, and despise not my petition: attend to me and hear me. These are the words of a man travailing, anxious, and troubled. He prayeth in the midst of much suffering, longing to be rid of his affliction. Our part is to see what his affliction was, and when he hath told us, to acknowledge that we also suffer therefrom; so that, sharing in his trouble, we may also join in his prayer. I mourn in my exercise, he says, and am troubled. Wherein mourned he? Wherein was he troubled? He saith: In my exercise. He hath in mind the wicked that cause him affliction, and this suffering which came upon him at the hands of wicked men, he hath called his exercise. Think not that wicked men are in this world for nothing, and that God doth no good with them. Every wicked man liveth, either to repent, or to exercise the righteous.}
\end{parcolumns}
\medskip

\vfil

\setgrefactor{17}
\printgabc{Resp. 4}{8.}{A}{re--amicus_meus--solesmes}

\translationcolumns[\normalsize]{\Rbar{}. My friend betrayed Me by the sign of a kiss: Whom I shall kiss, That is He, hold Him fast. This was the traitorous sign which he gave, who murdered with a kiss.
* Unhappy man, he relinquished the price of blood, and in the end hanged himself.
\Vbar{}. It had been good for that man, if he had not been born.
\Rbar{}. Woe unto that man \dots}
}
\pagebreak
{
\begin{center}{\textsc{lesson v}}\end{center}

\begin{parcolumns}[rulebetween,colwidths={1=250pt}]{2}
\lesson{Utinam ergo qui nos modo exércent, convertántur, et nobíscum exerceántur~: tamen quámdiu ita sunt ut exérceant, non eos odérimus~: quia in eo quod malus est quis eórum, utrum usque in finem perseveratúrus sit, ignorámus. Et plerúmque cum tibi vidéris odísse inimícum, fratrem odísti, et nescis. Diábolus, et ángeli ejus in Scriptúris sanctis manifestáti sunt nobis, quod ad ignem ætérnum sint destináti. Ipsórum tantum desperánda est corréctio, contra quos habémus occúltam luctam~: ad quam luctam nos armat Apóstolus, dicens~: Non est nobis colluctátio advérsus carnem et sánguinem~: id est, non advérsus hómines, quos vidétis, sed advérsus príncipes, et potestátes, et rectóres mundi, tenebrárum harum. Ne forte cum dixísset, mundi, intellígeres dǽmones esse rectóres cæli et terræ, mundi dixit, tenebrárum harum~: mundi dixit amatórum mundi~: mundi dixit, impiórum et iniquórum~: mundi dixit, de quo dicit Evangélium~: Et mundus eum non cognóvit.}
{Would, therefore, that they who now exercise us were converted and exercised with us! Yet, while they are as they are, and exercise us, we will not hate them: for we know not of any one of them whether he will endure to the end in his sin. Yea, oftentimes, when thou deemest that thou hatest thine enemy, thou hatest thy brother, and knowest it not. The Holy Scriptures show us that the devil and his angels are already damned unto everlasting fire: their repentance alone is hopeless, against whom we wage a hidden strife. For which strife the Apostle would arm us, saying: We wrestle not against flesh and blood, (that is, not against men whom we see,) but against principalities, against powers, against the rulers of the darkness of this world. He saith not the rulers of this world, lest perchance thou shouldest deem that devils are the lords of heaven and earth; what he doth say is, rulers of the darkness of this world, of that world which they love who love the world, of that world wherein the ungodly and unrighteous do prosper, of that world, of which the Gospel saith: And the world knew Him not.}
\end{parcolumns}
}

\bigskip%\bigskip

{
\printgabc{Resp. 5}{2.}{J}{re--judas_mercator--solesmes}

\translationcolumns[\normalsize]{\Rbar{}. Judas, worst of traffickers, came to the Lord to kiss Him: He, as an innocent Lamb, refused not the kiss of Judas,
* Who, for a few pence, betrayed Christ to the Jews.
\Vbar{}. It had been better for him if he had not been born.
\Rbar{}. Who, for a few pence, betrayed Christ to the Jews.}
}
{
\needspace{10\baselineskip}
\begin{center}{\textsc{lesson vi}}\end{center}

\begin{parcolumns}[rulebetween,colwidths={1=250pt}]{2}
\lesson{Quóniam vidi iniquitátem, et contradictiónem in civitáte. Atténde glóriam crucis ipsíus. Jam in fronte regum crux illa fixa est, cui inimíci insultavérunt. Efféctus probávit virtútem~: dómuit orbem non ferro, sed ligno. Lignum crucis contuméliis dignum visum est inimícis, et ante ipsum lignum stantes caput agitábant, et dicébant~: Si Fílius Dei est, descéndat de cruce. Extendébat ille manus suas ad pópulum non credéntem et contradicéntem. Si enim justus est, qui ex fide vivit; iníquus est, qui non habet fidem. Quod ergo hic ait, iniquitátem~: perfídiam intéllige. Vidébat ergo Dóminus in civitáte iniquitátem et contradictiónem, et extendébat manus suas ad pópulum non credéntem, et contradicéntem~: et tamen et ipsos exspéctans dicébat~: Pater, ignósce illis, quia nésciunt quid fáciunt.}
{I have seen iniquity and strife in the city. Behold, the glory of the Cross. That Cross is established now above the brows of kings, which once enemies did deride. The end hath shown the measure of its power: He hath conquered the world, not with a sword, but with wood. The enemies of God thought the Cross a meet object of insult and ridicule, yea, they stood before it, wagging their heads and saying: If He be the Son of God, let Him come down from the Cross! And He stretched forth His Hands unto a disobedient and gainsaying people. If he is just which liveth by faith, he is unjust that hath not faith. Therefore where is written iniquity we may understand unbelief. The Lord therefore saith that He saw iniquity and strife in the city, and that He stretched forth His Hands unto that disobedient and gainsaying people, and yet, looking upon the very same, He said: Father, forgive them, for they know not what they do.}
\end{parcolumns}

\bigskip\bigskip

\printgabc{Resp. 6}{8.}{U}{re--unus_ex_discipulis--solesmes}

\translationcolumns[\normalsize]{\Rbar{}. One of My disciples will betray Me today. Woe unto him by whom I am betrayed!
* It had been better for him if he had not been born.
\Vbar{}. He that dippeth his hand with Me in the dish, the same will deliver Me into the hands of sinners.
\Rbar{}. It had been better for him if he had not been born.
\Rbar{}. One of My disciples will betray \dots}
}