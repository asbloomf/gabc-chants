% !TEX TS-program = lualatex
% !TEX encoding = UTF-8

% This is a simple template for a LuaLaTeX document using gregorio scores.

\documentclass[letterpaper,12pt]{article} % use larger type; default would be 10pt
\usepackage{../definepsalms}

\newcommand{\chapter}{\dropcap{latin}{Fratres~: Nescítis quod ii qui in stádio} \textbf{cúr}\-runt,~\dag{} omnes quidem currunt, sed unus áccipit bravíum?  Sic cúrrite ut comprehen\textbf{dá}tis.}
\newcommand{\chaptertranslation}{Brethren: Know you not that they that run in the race, all run indeed, but one receiveth the prize?  So run, that you may obtain.}

\newcommand{\hymn}{Lúcis Creátor}
\newcommand{\vr}{Dirigátur}

\newcommand{\headingtext}{Septuagesima.}
\newcommand{\maganttex}{MagnificatAntiphonSeptuagesima.tex}

\newcommand{\latincollect}{Preces pópuli tui, quaésumus Dómine, cleménter exáudi~:~\dag{} ut qui juste pro peccátis nostris afflígimur,~* pro tui nóminis glória misericórditer liberémur. Per Dóminum nostrum.}
\newcommand{\englishcollect}{Graciously hear, we beseech Thee, O Lord, the prayers of Thy people, that we, who are justly afflicted for our sins, may for the glory of Thy name be mercifully delivered. Through our Lord.}

\newcommand{\englishmagantiphon}{The householder said unto his labourers: Why stand ye here all the day idle? And they answering said, Because no man hath hired us. Go ye into my vineyard, and I will give you what shall be just.}

\def\magsolemn{F}
\def\magoneline{T}
\newcommand{\magantinitial}{D}
\newcommand{\psalmclef}{c2}
\definemag{7}{a}

\newcommand{\magmedianttermination}{\medianttermination{Mediant of 2 accents.}{Termination of 2 accents.}}

\newcommand{\premagant}{%
%\normalsize
\large%\greblockcustos
}

\newcommand{\magantinitialsize}{35}

\newcommand{\pretranslation}{%
%\bigskip
%\vspace{-4ex}
}

\newcommand{\premagverses}{\normalsize
%\greblockcustos
\setgrefactor{14}}

\input{../vespersmagnificat}
