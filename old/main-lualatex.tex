% !TEX TS-program = lualatex
% !TEX encoding = UTF-8

% This is a simple template for a LuaLaTeX document using gregorio scores.

\documentclass[11pt]{extarticle} % use larger type; default would be 10pt

% usual packages loading:
%\usepackage{luatextra}
%\usepackage{graphicx} % support the \includegraphics command and options
\usepackage{geometry} % See geometry.pdf to learn the layout options. There are lots.
\geometry{letterpaper} % or letterpaper (US) or a5paper or....
\usepackage{gregoriotex} % for gregorio score inclusion
%\usepackage{fullpage} % to reduce the margins
\usepackage{anysize} % for marginsize
\usepackage{multicol} % for columns
\usepackage{lettrine} % for drop caps
\usepackage{wrapfig}

% choose the language of the document here
\usepackage[latin,american]{babel}

% use the two following package for using normal TeX fonts
\usepackage[T1]{fontenc}
\usepackage[utf8]{luainputenc}

% If you use usual TeX fonts, here is a starting point:
%\usepackage{palatino}
\usepackage{libertine}
\renewcommand*\oldstylenums[1]{{\fontfamily{fxlj}\selectfont #1}}
%\usepackage{fontspec}
% to change the font to something better, you can install the kpfonts package (if not already installed). To do so
% go open the "TeX Live Manager" in the Menu Start->All Programs->TeX Live 2010
%\fontspec[Numbers=OldStyle]{Linux Libertine O}
%\setgrefactor{17} \def\greinitialformat#1{{\fontsize{37}{37}\selectfont #1}}
%small > footnotesize > scriptsize > tiny

\pagestyle{empty}
\marginsize{40pt}{40pt}{40pt}{40pt}

% here we begin the document
\begin{document}

% The title:
\begin{center}\begin{huge}Ave Regina cælorum.\end{huge}\end{center}

\selectlanguage{american}
\textit{From Compline of Feb 2nd (even if the Feast of the Purification be transferred) until Compline of Wednesday in Holy Week.}
\begin{center}\textbf{Solemn Tone.}\end{center}
\selectlanguage{latin}

% Here we set the space around the initial.
% Please report to http://home.gna.org/gregorio/gregoriotex/details for more details and options
\setspaceafterinitial{2.2mm plus 0em minus 0em}
\setspacebeforeinitial{2.2mm plus 0em minus 0em}

% Here we set the initial font. Change 43 if you want a bigger initial.
\def\greinitialformat#1{%
{\fontsize{43}{43}\selectfont #1}%
}

% We set red lines here, comment it if you want black ones.
%\redlines

% We set VII above the initial.
\gresetfirstlineaboveinitial{\small \textbf{Ant.}}{\small \textbf{Ant.}}
\setsecondannotation{\small \oldstylenums{ \textbf{6.}}}

% We type a text in the top right corner of the score:
%\commentary{{\small \emph{Cf. Is. 30, 19 . 30 ; Ps. 79}}}

% and finally we include the score. The file must be in the same directory as this one.
\includescore{AveReginaSolemn.tex}

\selectlanguage{american}
\begin{quote}
\textit{Hail, Queen of Heaven! Hail, Queen of Angels! Hail, blest Root and Gate, from which came light upon the world! Rejoice, O glorious Virgin, that surpassest all in beauty! Hail, O most lovely, and pray for us to Christ.}
\end{quote}
\begin{center}\textbf{Simple Tone.}\end{center}
\selectlanguage{latin}

\gresetfirstlineaboveinitial{\small \textbf{6.}}{\small \textbf{6.}}

\includescore{AveReginaSimple.tex}
\setlength{\columnsep}{18pt}
\setlength{\columnseprule}{1pt}
\begin{multicols}{2}
\noindent\Vbar{}~Dignáre me laudáre te Vírgo sacráta.
\\
\Rbar Da míhi virtútem cóntra hóstes túos.\\

\noindent\hspace*{3em}Orémus.
\lettrine{C}Oncéde, miséricors Deus, fragilitáti nostræ præsídium : \gredagger{} ut qui sanctæ Dei Genitrícis memóriam ágimus, * intercessiónis ejus auxílio a nostris iniquitátibus resurgámus. Per eúmdem Christum Dóminum nostrum. \Rbar Amen. %
\columnbreak

\selectlanguage{american}
\noindent\Vbar{}~Vouchsafe, O holy Virgin, that I may praise thee.\\
\Rbar Give me power against thine enemies.\\

\noindent\hspace*{3em}Let us pray,
\noindent{}\lettrine{G}Rant, O merciful God, Thy protection to us in our weakness; that we, who celebrate the memory of the holy Mother of God, may, through the aid of her intercession, rise again from our sins. Through the same Christ our Lord. \Rbar Amen. %
\selectlanguage{latin}
\end{multicols}



\end{document}
