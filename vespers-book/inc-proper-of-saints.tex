\clearpage
\chapter{Proper of the Saints}
{
\newcommand{\benedicamusdomino}[1][1]{
  \benedicamusdominomaster{#1}
}
\newcommand{\sundaycommemnote}{%
  If today is Sunday, Vespers of the Sunday is commemorated with \emph{Magnificat antiphon}, \emph{\Vbar{}~Dirigátur},
  %in simple commemoration tone, p.~\pageref{vr-dirigatur},
  and \emph{Collect}.
}
\newcommand{\sundaycommemnoteeaster}{%
  If today is Sunday, Vespers of the Sunday is commemorated with \emph{Magnificat antiphon}, \emph{\Vbar{}~Mane nobíscum}, %in simple commemoration tone, p.~\pageref{vr-manenobiscum},
  and \emph{Collect}.
}
\newcommand{\sundaycommemnoteeasterpentecost}{%
  If today is Sunday, Vespers of the Sunday is commemorated with \emph{Magnificat antiphon}, \emph{\Vbar{}~Mane nobíscum.} in Paschaltide, %p.~\pageref{vr-manenobiscum},
  or \emph{\Vbar{}~Dirigátur.} after Pentecost, %p.~\pageref{vr-dirigatur}, in simple commemoration tone,
  and \emph{Collect}.
}

%December 8: Immaculate Conception
%Second Vespers of Immaculate Conception
{
\label{immaculateconception}
\section{December 8: Immaculate Conception}
\subtitle{\nth{1} Class, White or Gold}
\subtitle{I \& II Vespers}
\smallskip

\def\preanttranslationone{\vspace{-0.35\baselineskip}}
\def\prepsalmtitleone{\vspace{-0.15\baselineskip}}
\def\prepsalmone{\vspace{-0.1\baselineskip}}
\def\prepsalmoneverses{\vspace{-0.05\baselineskip}}
\def\prerepeatantiphonone{}

\def\prepsalmtitlethree{\vspace{-0.75\baselineskip}}

\def\preanttranslationfive{\vspace{-0.35\baselineskip}}
\def\prepsalmtitlefive{\vspace{-\baselineskip}}
\def\prepsalmfive{\vspace{-0.2\baselineskip}}
\def\prepsalmfiveverses{\vspace{-0.025\baselineskip}}
\def\prerepeatantiphonfive{}
\def\premagverses{\greseteolcustos{manual}}
\def\definevesperspropersalt{\def\noeuouae{T}\newcommand{\maganttex}{MagnificatAntiphon1}
\newcommand{\magantinitial}{B}
\newcommand{\maganttranslation}{All generations shall call me blessed, because he that is mighty, hath done great things for me, alleluia.}
\def\magpsalmclef{c3}
\definemag{8}{G}
}
\def\definevesperspropers{\def\noeuouae{T}\newcommand{\maganttex}{MagnificatAntiphon2}
\newcommand{\magantinitial}{H}
\newcommand{\maganttranslation}{This day a rod came forth from the root of Jesse: this day Mary was conceived without any stain of sin: this day the head of the old serpent was crushed by her.  Alleluia.}
\definemag{1}{f}
}
\def\vesperspropersaltnote{At I Vespers:}
\def\vesperspropersnote{At II Vespers:}
\def\prehymn{\printnote{All kneel for the first stanza of the following hymn.}}
\def\hymnlabel{hymn-avemarisstella}
\def\vrlinebreak{F}
%\def\prechapter{\vspace{-\baselineskip}}
\printvespers[../December8-ImmaculateConception]{inc-ImmaculateConceptionVespers}
}

{
\bigskip

\bigskip
\noindent
\printnote{Then follows a Commemoration of the Advent Sunday or Feria according to the day of the week.  The antiphon is followed by \emph{\Vbar{} Roráte cæli}, page \pageref{vr-rorate-dec8}, and then the appropriate Collect, page \pageref{collect-dec8}.}
\bigskip
}

{
  \oldneedspace{5\baselineskip}
  \subtitle{First Week of Advent.}
  \subtitle{\small{Thursday.}}
  {
  \def\noeuouae{T}
  \printgabc{At Magn.}{\oldstylenums{Ant.~4.}}{E}{../Advent1/MagAntThursday-Exspectabo}
  }
  \translation[]{I will look for the Lord my Saviour, and await Him, while He is near, alleluia.}
  \medskip

  \oldneedspace{5\baselineskip}
  \subtitle{\small{Friday.}}
  {
  \def\noeuouae{T}
  \printgabc{At Magn.}{\oldstylenums{Ant.~4.}}{E}{../Advent1/MagAntFriday-ExAegypto}
  }
  \translation[]{Out of Egypt I have called my Son; he shall come to save His people.}
  \medskip

  \oldneedspace{5\baselineskip}
  \subtitle{Second Week of Advent.}
  \subtitle{\small{Saturday.}}
  {
  \def\noeuouae{T}
  \printgabc{At Magn.}{\oldstylenums{Ant.~7.}}{V}{../Advent1/MagAntSaturday-VeniDomine}
  }
  \translation[]{Come, Lord, to visit us in peace, that we may rejoice before Thee with a perfect heart.}
  \medskip

  \oldneedspace{5\baselineskip}
  \subtitle{\small{Sunday.}}
  {
  \def\noeuouae{T}
  \printgabc{At Magn.}{\oldstylenums{Ant.~8.~G}}{T}{../Advent2/MagnificatAntiphon-noEuouae}
  }
  \translation[]{Art Thou He that art to come, or look we for another? Relate to John what you have seen: The blind recover their sight, the dead rise again, the poor have the Gospel preached to them, alleluia.}
  \medskip

  \oldneedspace{5\baselineskip}
  \subtitle{\small{Monday.}}
  {
  \def\noeuouae{T}
  \printgabc{At Magn.}{\oldstylenums{Ant.~4.}}{E}{../Advent2/MagAntMonday-EcceRexVeniet}
  }
  \translation[]{Behold, the King shall come, the Lord of the land; and He shall take away the yoke of our captivity.}
  \medskip

  \oldneedspace{5\baselineskip}
  \subtitle{\small{Tuesday.}}
  {
  \def\noeuouae{T}
  \printgabc{At Magn.}{\oldstylenums{Ant.~5.}}{V}{../Advent2/MagAntTuesday-VoxClamantis}
  }
  \translation[]{A voice of one crying in the desert, Prepare ye the way of the Lord, make straight His paths}
  \medskip

  \oldneedspace{5\baselineskip}
  \subtitle{\small{Wednesday.}}
  {
  \def\noeuouae{T}
  \printgabc{At Magn.}{\oldstylenums{Ant.~4.}}{S}{../Advent2/MagAntWednesday-Sion}
  }
  \translation[]{Sion, thou shalt be restored, and shalt see the Just One who shall appear in thee.}
  \medskip

  \oldneedspace{5\baselineskip}
  \subtitle{\small{Thursday.}}
  {
  \def\noeuouae{T}
  \printgabc{At Magn.}{\oldstylenums{Ant.~4.}}{Q}{../Advent2/MagAntThursday-QuiPostMeVenit}
  }
  \translation[]{He that shall come after me is preferred before me; whose shoes I am not worthy to loose.}
  \medskip
  \hrule
  \medskip
  {
      \label{vr-rorate-dec8}
      \newcommand{\commvlatin}{Roráte cæli désuper, et nubes pluant \textbf{ju}stum.}
      \newcommand{\commrlatin}{Aperiátur terra, et gérminet Salva\textbf{tó}rem.}
      \newcommand{\commvtranslation}{Ye heavens, drop down dew from above, and let the clouds rain down the Just One.}
      \newcommand{\commrtranslation}{Let the earth open and bud forth the Saviour.}
  \printvrcommem{}
  }

  \oldneedspace{3\baselineskip}
  \label{collect-dec8}
  \begin{center}{\large Collect.}\end{center}
  \vspace{-1.5\baselineskip}
  \def\printcollectheading{F}
  {
  \begin{center}{First Week of Advent.}\end{center}
  \def\gabcfolder{../Advent1}
  \newcommand{\antonetex}{Ant1-InIllaDie}
\newcommand{\antoneinitial}{I}
\newcommand{\antonetranslation}{In that day the mountains shall drop down sweetness, and the hills shall flow with milk and honey, alleluia.}
\definepsalm{1}{109}{8}{G}

\newcommand{\anttwotex}{Ant2-Jucundare}
\newcommand{\anttwoinitial}{J}
\newcommand{\anttwotranslation}{Shout for joy, O daughter of Sion, rejoice greatly, O daughter of Jerusalem, alleluia.}
\definepsalm{2}{110}{8}{G*}

\newcommand{\antthreetex}{Ant3-EcceDominusVeniet}
\newcommand{\antthreeinitial}{E}
\newcommand{\antthreetranslation}{Behold, the Lord shall come, and all His Saints with Him: and there shall be in that day a great light, alleluia.}
\definepsalm{3}{111}{5}{a}

\newcommand{\antfourtex}{Ant4-Omnes}
\newcommand{\antfourinitial}{O}
\newcommand{\antfourtranslation}{All ye that thirst come to the waters: seek the Lord while He can be found, alleluia.}
\definepsalm{4}{112}{7}{c}

\newcommand{\antfivetex}{Ant5-EcceVeniet}
\newcommand{\antfiveinitial}{E}
\newcommand{\antfivetranslation}{Behold there shall come a great Prophet, and He shall renew Jerusalem, alleluia.\vspace{1ex}}
\definepsalm{5}{113}{4}{A*}

\newcommand{\chaptertext}{\dropcap{latin}{Fratres~: Hora est jam nos de somno} \textbf{súr}\-ge\-re~:~\gredagger{} nunc enim própior est \emph{no\-stra} \textbf{sá}\-lus,~* quam cum cre\-\textbf{dí}\-dimus.}
\newcommand{\chaptertranslation}{Brethren: it is now the hour for us to rise from sleep.  For now our salvation is nearer than when we believed.}

\newcommand{\magantinitial}{N}
\newcommand{\maganttex}{MagnificatAntiphon}
\newcommand{\maganttranslation}{Fear not, Mary, for thou hast found grace with the Lord: behold thou shalt conceive and bring forth a son, alleluia.}
\def\magsolemn{F}
\definemag{8}{G}

\newcommand{\collect}{Excita, quaésumus Dómine, poténtiam tuam, et veni~:~† ut ab imminéntibus peccatórum nostrórum perículis, te mereámur protegénte éripi,~* te liberánte salvári.  Qui vivis et regnas cum Deo Patre in unitáte Spíritus Sancti Deus~:~* per ómnia saécula sæculórum.}
\newcommand{\collecttranslation}{Stir up Thy power, we beseech Thee, O Lord, and come: that from the threatening dangers of our sins we may deserve to be rescued by Thy protection, and to be saved by Thy deliverance: Who livest and reignest with God the Father in the unity of the Holy Ghost, world without end.}

  \printcollect{\collect}{\collecttranslation}
  }
  {
  \begin{center}{Second Week of Advent.}\end{center}
  \def\gabcfolder{../Advent2}
  % !TEX TS-program = lualatex
% !TEX encoding = UTF-8

% This is a simple template for a LuaLaTeX document using gregorio scores.

\newcommand{\comheadingtext}{Commemoration of 2nd Sunday of Advent}

\newcommand{\latincomcollect}{Excita Dómine corda nostra ad præparándas Unigéniti tui vias~:~† ut per ejus advéntum,~* purificátis tibi méntibus servíre mereámur. Qui tecum.}
\newcommand{\englishcomcollect}{Stir up our hearts, O Lord, to prepare the ways of Thine only-begotten Son: that through His coming we may deserve to serve Thee with purified minds: Who with Thee.}

\newcommand{\englishcommagantiphon}{Art Thou He that art to come, or look we for another? Relate to John what you have seen: The blind recover their sight, the dead rise again, the poor have the Gospel preached to them, alleluia.}

\newcommand{\commagantlinetwo}{Ant. 8. G}
\newcommand{\commaganttex}{MagnificatAntiphon-noEuouae}
\newcommand{\commagantinitial}{T}
\newcommand{\commagantinitialsize}{35}

\newcommand{\commvrtex}{../Advent/vr-commemoration}
\newcommand{\commvtranslation}{Ye heavens, drop down dew from above, and let the clouds rain down the Just One.}
\newcommand{\commrtranslation}{Let the earth open and bud forth the Saviour.}

\grechangestaffsize{17}
  \printcollect{\latincomcollect}{\englishcomcollect}
  }

  \bigskip
  \benedicamusdomino{}
}


%January 29, St Francis de Sales
{
\section{January 29: St Francis de Sales}
\subtitle{\nth{1} Class (patron of the Archdiocese of Cincinnati), White or Gold}
\vspace{-\baselineskip}
\subtitle{I \& II Vespers}

\def\definevesperspropers{\definepsalm{5}{131}{3}{g}

\newcommand{\vrtex}{vrJustumDeduxit}
\newcommand{\vtranslation}{The Lord led the just by right ways.}
\newcommand{\rtranslation}{And showed him the kingdom of God.}

}
\def\definevesperspropersalt{\definepsalm{5}{116}{3}{g}

\newcommand{\vrtex}{vrAmavitEum}
\newcommand{\vtranslation}{The Lord lloved him and adorned him.}
\newcommand{\rtranslation}{He clothed him with a robe of glory.}

}
\def\vesperspropersnote{At II Vespers:}
\def\vesperspropersaltnote{At I Vespers:}
\def\premagverses{\vspace{-0.1\baselineskip}}
\def\prevespers{%
  %\let\oldthing=\maganttranslation
  %\def\maganttranslation{\vspace{-0.25\baselineskip}\oldthing\vspace{-0.25\baselineskip}}
  \def\premagtexverses{\smallskip}
}
\printvespers[../CommonOfConfessorBishop]{inc-StFrancisDeSales}
\medskip
\printnote{\sundaycommemnote{}

\bigskip{}

\emph{\nth{4} Sunday after Epiphany}, p.~\pageref{epiphany4}.

\emph{Septuagesima}, p.~\pageref{septuagesima}.

\emph{Sexagesima}, p.~\pageref{sexagesima}.
}

\bigskip{}
\benedicamusdomino{}
}


%Purification & Presentation (2nd class no commem of Sunday because it is a feast of the Lord)
{
\section{February 2: Purification \& Presentation}
\subtitle{\nth{2} Class, White}
\subtitle{I \& II Vespers}
\printnote{At I Vespers, Psalms and Antiphons of the Circumcision, p.~\pageref{circumcision}, continuing with the chapter on p.~\pageref{purification-chapter}.}

\def\definevesperspropers{\newcommand{\maganttex}{an--hodie_beata_virgo--solesmes}
\newcommand{\magantinitial}{H}
\newcommand{\maganttranslation}{Today did the Blessed Virgin Mary present the Child Jesus in the temple; and Simeon, filled with the Holy Ghost, took Him up in his arms, and blessed God for ever and ever.\vspace{-2ex}}
\definemag{8}{G*}
}
\def\definevesperspropersalt{\newcommand{\maganttex}{an--senex_puerum--solesmes}
\newcommand{\magantinitial}{S}
\newcommand{\maganttranslation}{This day did the Blessed Virgin Mary present the Child Jesus in the temple; and Simeon, filled with the Holy Ghost, took Him up in his arms, and blessed God for ever and ever.}
\definemag{1}{D}
}
\def\vesperspropersnote{At II Vespers:}
\def\vesperspropersaltnote{At I Vespers:}
\def\prechapter{\label{purification-chapter}}
\def\printhymnnote{
  {
    \oldneedspace{3\baselineskip}
    \printnote{Hymn.~\emph{Ave Maris Stella}, p.~\pageref{hymn-avemarisstella}.\\}
    %
    % \def\vrlinebreak{T}
    % \oldneedspace{3\baselineskip}
    % \printvr[\greseteolcustos{manual}]{\vrtex}{\vtranslation}{\rtranslation}
  }
}
\printvespers[../February2-PurificationOfBlessedVirginMary]{inc-purification}
\bigskip{}
\benedicamusdomino[2]{}
}

%March 25: Annunciation 
{
\def\vrlinebreak{F}
\section{March 25: Annunciation of the Blessed Virgin Mary}
\subtitle{\nth{1} Class}
\subtitle{I \& II Vespers}
\medskip

\def\deusinadjutoriumsolemn{T}
\def\definevesperspropers{\newcommand{\maganttex}{an--gabriel_angelus_ave--solesmes}
\newcommand{\magantinitial}{G}
\newcommand{\maganttranslation}{The Angel Gabriel spoke to Mary, saying: Hail, full of grace, the Lord is with thee; blessed art thou amongst women.}
\def\magsolemn{T}
\def\psalmclef{c3}
\definemag{7}{d}

  \def\prepsalmfive{\greseteolcustos{manual}}
}
\def\definevesperspropersalt{\newcommand{\maganttex}{an--spiritus_sanctus--solesmes}
\newcommand{\magantinitial}{S}
\newcommand{\maganttranslation}{The Holy Ghost shall come down upon thee, Mary, and the power of the Most High shall overshadow thee.}
\def\magsolemn{T}
\definemag{8}{G}
}
\def\vesperspropersnote{At II Vespers:}
\def\vesperspropersaltnote{At I Vespers:}
%\def\premag{\def\noeuouae{T}}
\ifthenelse{\boolean{birmingham}}{
  \def\preantthree{\vspace{-0.75\baselineskip}}
  \def\prepsalmtitlethree{\vspace{-0.25\baselineskip}}
  \def\preantfive{\vspace{-0.75\baselineskip}}
  \def\prepsalmtitlefive{\vspace{-0.75\baselineskip}}
  \def\premagtitle{\oldneedspace{2\baselineskip}}
  \def\precollect{\vspace{-0.5\baselineskip}}
}{
  \def\prepsalmtitlefive{\vspace{-0.5\baselineskip}}
}
\def\prepsalmtitletwo{\vspace{-0.5\baselineskip}}
\def\premagverses{\greseteolcustos{manual}}
\def\printhymnnote{
  {
    \oldneedspace{3\baselineskip}
    \printnote{Hymn.~\emph{Ave Maris Stella}, p.~\pageref{hymn-avemarisstella}.\\}

    \printnote{At II Vespers in Paschal time, add Alleluia to the versicle and response.\\}
    %
    % \def\vrlinebreak{T}
    % \oldneedspace{3\baselineskip}
    % \printvr[\greseteolcustos{manual}]{\vrtex}{\vtranslation}{\rtranslation}
  }
}
\def\beginchaptercols{\begin{parcolumns}[rulebetween,colwidths={1=0.44\linewidth}]{2}}
\def\begincollectcols{\begin{parcolumns}[rulebetween,colwidths={1=0.43\linewidth}]{2}}
%\def\prepsalmthreeverses{\pagebreak}
\def\prevespers{
  %\let\oldthing=\anttwotranslation  
  %\def\anttwotranslation{\oldthing\vspace{-\baselineskip}}
  %\let\oldthingb=\antfourtranslation  
  %\def\antfourtranslation{\oldthing\pagebreak}
}
\def\vrlabel{vr-march25}

\printvespers[../March25-Annunciation]{inc-Annunciation}
}

{
\bigskip
\noindent
\printnote{Then follows a Commemoration of the Lenten Feria.  Finally, \benedicamusdominoreference{1}}
\bigskip
}


% Easter can be as early as March 22 or as late as April 25, so we basically need to include all the lenten ferias
% I think if it falls on or after Palm Sunday, it is translated to Monday after Low Sunday
% so I should only need to go from Wednesday of the second week unless I want to add St Joseph
{
\newcommand{\printcommem}[8][../Lent]{
% #1 Folder (../Lent)
% #2 name of the day of the week (Thursday)
% #3 Tone (2)
% #4 First letter of antiphon (Q)
% #5 gabc (an--qui_me_sanum_fecit--solesmes)
% #6 ant translation (He who healed me commanded me: Take up thy mat and walk in peace.)
% #7 collect
% #8 collect translation
  \oldneedspace{4\baselineskip}
  \sectionmark{#2 in \beforecommemweek{}\commemweek{}.}
  \subtitle{\small{#2.}}
  \smallskip
  {
  \def\noeuouae{T}
  \printgabc{At Magn.}{\oldstylenums{Ant.~#3.}}{#4}{#1/#5}
  }
  \translation[]{#6}
  \smallskip
  \ifx\vrtitle\undefined%
  {
    \label{\vrlabel}
    \printvrcommem{}
  }\else%
  \printnote{\emph{\Vbar{}~\vrtitle},
  \ifnum\getpagerefnumber{\vrlabel}=\thepage
  below%
  \else
  page \pageref{\vrlabel}%
  \fi.}
  \fi
  \ifx\precollect\undefined\else\precollect\fi
  \printcollect{#7}{#8}
  \bigskip
  \hrule
  \bigskip
}
\newcommand{\printweektitle}[2][the ]{
  \edef\commemweek{#2}%
  \edef\beforecommemweek{#1}%
  \subtitle{#2.}%
}
{
  \newcommand{\vrtitle}{Roráte cæli}
  \newcommand{\vrlabel}{commvrrorate}
  \newcommand{\commvlatin}{Roráte cæli désuper, et nubes pluant \textbf{ju}stum.}
  \newcommand{\commrlatin}{Aperiátur terra, et gérminet Salva\textbf{tó}rem.}
  \newcommand{\commvtranslation}{Ye heavens, drop down dew from above, and let the clouds rain down the Just One.}
  \newcommand{\commrtranslation}{Let the earth open and bud forth the Saviour.}
  % \oldneedspace{5\baselineskip}
  % \subtitle{First Week of Lent.}\vspace{-0.5\baselineskip}
  % \printcommem{Thursday}{4}{O}{an--o_mulier--solesmes}{O woman, great is thy faith: let it be unto thee as thou hast asked.}
  % {Da, quǽsumus Dómine, pópulis christiánis, et quæ profiténtur agnóscere~:~* et cæléste munus dilígere, quod frequéntant. Per Dóminum.}{Grant, Lord, we beseech Thee, unto all Christian people, that what they now believe they may one day know and see in love unchecked, that heavenly gift whereof now they are the worshippers and the partakers.}

  % \printcommem{Friday}{1}{Q}{an--qui_me_sanum_fecit--solesmes}{He who healed me commanded me: Take up thy bed and walk in peace.}
  % {Exáudi nos miséricors Deus~:~* et méntibus nostris grátiæ tuæ lumen osténde. Per Dóminum nostrum.}
  % {Hear us, O merciful God, and cause the bright beams of thy grace to shine upon our souls.}
  
  \oldneedspace{5\baselineskip}
  \printweektitle{Second Week of Lent}
  % \printcommem{Monday}{1}{Q}{an--qui_me_misit--solesmes}
  % {He who sent Me is with Me and hath not left Me alone: for I always do those things that please Him.}
  % {Adésto supplicatiónibus nostris omnípotesn Deus~:~* et quibus fidúciam sperándæ pietátis indúlges, consuétæ misericórdiæ tríbue benígnus efféctum. Per Dóminum.}
  % {Graciously hear our prayers, O Almighty God, and as Thou dost give us to look with confidence for Thy favour for which we hope, so grant us, in Thy goodness, the manifestation of Thine accustomed mercy.}

  % \printcommem{Tuesday}{4}{O}{an--omnes_autem_vos--solesmes}
  % {And all ye are brethren, and call no man your father upon earth for One is your Father, Which is in heaven neither be ye called masters, for One is your Master, even Christ.}
  % {Propitiáre Dómine supplicatiónibus nostris, et animárum nostrárum medére languóribus~:~* ut remissióne percépta, in tua semper benedictióne lætémur. Per Dóminum.}
  % {Tend thy merciful ears, O Lord, we beseech thee, unto our supplications, and heal the sickness of our souls, that we, receiving thy pardon, may rejoice forever in thy blessing.}
  
  \printcommem{Wednesday}{1}{T}{an--tradetur_enim_gentibus--solesmes}
  {He will be handed over to the gentiles to be mocked and scourged and crucified.}
  {Deus innocéntiæ restitútor et amátor, dírige ad te tuórum corda servórum~:~* ut spíritus tui fervóre concépto, et in fide inveniántur stábiles, et in ópere efficáces. Per Dóminum.}
  {God, the Renewer and Lover of innocence, turn the hearts of Thy servants to Thyself, so that they, enkindled with the fire of Thy Spirit, may be found ever rooted in faith, and fruitful in works.}

  %\newcommand{\vrtitle}{Roráte cæli}
  \printcommem{Thursday}{7}{D}{an--dives_ille_guttam--solesmes}
  {That rich man who had refused Lazarus bread-crumbs, cried for a drop of water.}
  {Adésto Dómine fámulis tuis, et perpétuam benignitátem largíre poscéntibus~:~* ut iis, qui te auctóre et gubernatóre gloriántur, et congregáta restáures, et restauráta consérves. Per Dóminum.}
  {Lord, be present to Thy servants, and grant unto those asking an abiding mercy; that as they boast in Thee, their creator and governer, so Thou wilt renew in them the gifts bestowed, and preserve what Thou hast renewed.}

  \printcommem{Friday}{3}{Q}{an--quaerentes_eum_tenere--solesmes}
  {Seeking to lay hands on Him, they feared the multitude, because they took Him for a Prophet.}
  {Da, quǽsumus Dómine, pópulo tuo salútem mentis et córporis~:~* ut bonis opéribus inhæréndo, tuæ semper virtútis mereátur protectióne deféndi. Per Dóminum.}
  {Grant unto Thy people, O Lord, we beseech Thee, soundness of mind and body, that they, cleaving unto good works, may evermore worthily be defended by the protection of Thy might.}

  \oldneedspace{5\baselineskip}
  \printweektitle{Third Week of Lent}
  \printcommem{Monday}{1}{J}{an--jesus_autem_transiens--solesmes}
  {But Jesus passing through their midst, went His way.}
  {Subvéniat nobis Dómine misericórdia tua~:~* ut ab imminéntibus peccatórum nostrórum perículis te mereámur protegénte éripi, te liberánte salvári. Per Dóminum.}
  {Let our help, O Lord, be in Thy mercy, that we over whom Thy wrath doth most justly hang because of our sins, may in all dangers worthily be shielded by Thy protection and delivered by Thy salvation.}

  %\let\vrtitle=\undefined
  \printcommem{Tuesday}{4}{U}{an--ubi_duo_vel_tres--solesmes}
  {Where two or three are gathered in my name, I am in their midst, saith the Lord.}
  {Tua nos Dómine protectióne defénde~:~* et ab omni semper iniquitáte costódi. Per Dóminum.}
  {O Lord, shield us by Thy protection, and keep us ever from all iniquity.}

  %\newcommand{\vrtitle}{Roráte cæli}
  \printcommem{Wednesday}{7}{N}{an--non_lotis_manibus--solesmes}
  {To eat with unwashen hands, defileth not a man.}
  {Concéde, quǽsumus omnípotens Deus~: ut qui protectiónis tuæ grátiam quǽrimus,~* liberáti a malis ómnibus, secúra tibi mente serviámus. Per Dóminum.}
  {Grant, we beseech Thee, Almighty God, that we who seek the proctection of Thy grace, freed from all evils, may serve Thee in peace and quietness of spirit.}

  %\let\vrtitle=\undefined
  \printcommem{Thursday}{1}{O}{an--omnes_qui_habebant--solesmes}
  {All who had any sick brought them to Jesus, and they were healed.}
  {Subjéctum tibi pópulum, quǽsumus Dómine, propitiátio cæléstis amplíficet~:~* et tuis semper fáciat servíre mandátis. Per Dóminum.}
  {Lord, we beseech Thee that Thine heavenly Peace-Offering may so effectually work for all Thy people, bowing down before Thee, that they may ever continue to keep Thy commandments.}

  %\newcommand{\vrtitle}{Roráte cæli}
  \printcommem{Friday}{3}{D}{an--domine_ut_video--solesmes}
  {Lord, I see that Thou art a prophet: our fathers worshipped on this mountain.}
  {Præsta, quǽsumus omnípotens Deus~:~\dag{} ut qui in tua protectióne confídimus,~* cuncta nobis adversántia te adjuvánte vincámus. Per Dóminum.}
  {Grant, we beseech Thee, Almighty God, that we who trust is Thy protection, may, with Thy help, overcome all evils that rise up against us.}

  \oldneedspace{5\baselineskip}
  \printweektitle{Fourth Week of Lent}
  \printcommem{Monday}{5}{S}{an--solvite_templum_hoc--solesmes}
  {Thus saith the Lord: Destroy this temple, and in three days I will rebuild it. But He was speaking of the temple of His Body.}
  {Deprecatiónem nostram, quǽ\-su\-mus Dómine, benígnus exáudi~:~* et quibus supplicándi præstas afféctum, tríbue defensiónis auxílium. Per Dóminum.}
  {Lord, we beseech Thee, graciously hear our supplication, and evermore help and defend all those to whom Thou hast given the mind to pray.}

  \printcommem{Tuesday}{1}{N}{an--nemo_in_eum_misit--solesmes}
  {No man laid hands on Him; because His hour was not yet come.}
  {Miserére Dómine pópulo tuo~:~* et contínuis tribulatiónibus laborántem, propítius respiráre concéde. Per Dóminum.}
  {Lord, have mercy upon Thy people, and be graciously pleased to grant relief unto the same, who are ever toiling amid the storms of diverse tribulations.}

  %\newcommand{\vrtitle}{Roráte cæli}
  \printcommem{Wednesday}{1}{I}{an--ille_homo--solesmes}
  {The man that is called Jesus, made clay of spittle, and anointed mine eyes, and now I see.}
  {Páteant aures misericórdiæ tuæ Dómine précibus supplicántium~:~* et ut peténtibus desideráta concédas, fac eos quæ tibi sunt plácita postuláre. Per Dóminum.}
  {Let Thy merciful ears, Lord, be open unto the prayers of those entreating Thee, and that Thou mayest grant what we ask, teach us ever to ask what is pleasing to Thee.}

  \printcommem{Thursday}{4}{P}{an--propheta_magnus--solesmes}
  {A great prophet is risen up among us, and God hath visited His people.}
  {Pópuli tui Deus institútor et rector, peccáta quibus impugnátur, expélle~:~* ut semper tibi plácitus, et tuo munímine sit secúrus. Per Dóminum.}
  {O God, Teacher and Shepherd of Thy people, free the same from all sins assailing them, that they may be ever pleasing in Thy sight and safe under Thy shelter.}

  \let\vrtitle=\undefined
  \printcommem{Friday}{1}{D}{an--domine_si_hic_fuisses--solesmes}
  {Lord, if Thou hadst been here, Lazarus had not died; behold already he stinketh, for he hath lain in the grave four days.}
  {Da nobis, quǽsumus omnípotens Deus~:~* ut qui infirmitátis nostræ cónscii, de tua virtúte confídimus, sub tua semper pietáte gaudeámus. Per Dóminum.}
  {Grant, we beseech Thee, Almighty God, unto us who know our weakness and who trust in Thy strength, to ever rejoice in Thy loving kindness.}
}
  
{
  \newcommand{\commvlatin}{Eripe me, Dómine, ab hómine \textbf{ma}lo.}
  \newcommand{\commrlatin}{A viro iníquo éri\textbf{pe} me.}
  \newcommand{\commvtranslation}{Deliver me, O Lord, from the wicked man.}
  \newcommand{\commrtranslation}{And save me from the evil-doer.}
  \newcommand{\vrtitle}{Eripe me}
  \newcommand{\vrlabel}{commvreripeme}
  \oldneedspace{7\baselineskip}
  \printweektitle[]{Passiontide}
  {
  \ifthenelse{\boolean{birmingham}}{
    \def\precollect{\vspace{-0.75\baselineskip}}
  }{
    \def\precollect{\vspace{-0.5\baselineskip}}
  }
  \printcommem{Monday}{4}{S}{an--si_quis_sitit--solesmes}
    {If anyone thirst, let him come and drink; and from his belly will flow living water.}
    {Da, quǽsumus Dómine, pópulo tuo salútem mentis et córporis~:~* ut bonis opéribus inhæréndo, tua semper mereátur protectióne deféndi. Per Dóminum.}
    {Grant unto Thy people, O Lord, we beseech Thee, soundness of mind and body, that they, cleaving unto good works, may evermore worthily be defended by Thy protection.}
  }
  {%\newcommand{\vrtitle}{Eripe me}
  \ifthenelse{\boolean{birmingham}}{
    \def\precollect{\vspace{-0.75\baselineskip}}
  }{}
  \printcommem{Tuesday}{1}{V}{an--vos_ascendite--solesmes}
  {Go ye up unto this Feast; I go not, for My time is not yet full come.}
  {Da nobis, quǽsumus Dómine, perseverántem in tua voluntáte famulátum~:~* ut in diébus nostris, et mérito et número pópulus tibi sérviens augeátur. Per Dóminum.}
  {Grant, we beseech Thee, Lord, to give us grace to endure to the end in doing Thy will, that in our days Thy people which serve Thee may have increase, both in merit and number.}
  }

  {
  \def\precollect{\vspace{-0.5\baselineskip}}
  \printcommem{Wednesday}{4}{M}{an--multa_bona_opera--solesmes}
    {I have wrought many good works among you: on account of which work do you want to kill me?}
    {Adésto supplicatiónibus nostris omnípotens Deus~:~* et quibus fidúciam sperándæ pietátis indúlges, consuétæ misericórdiæ tríbue benígnus efféctum. Per Dóminum.}
    {Gratiously hear our prayers, Almighty God, and as Thou dost give us to look with confidence for Thy favour for which we hope, so grant us, in Thy goodness, Thine accustomed mercy.}
  }
  %\newcommand{\vrtitle}{Eripe me}
  \printcommem{Thursday}{4}{D}{an--desiderio_desideravi--solesmes}
  {With desire I have desired to eat this Pasch with you before I suffer.}
  {Esto, quǽsumus Dómine, propítius plebi tuæ~:~* ut quæ tibi non placent respuéntes, tuórum pótius repleántur delectatiónibus mandatórum. Per Dóminum.}
  {Lord, we beseech Thee, deal mercifully with Thy people, and fill plentifully with the rich things of Thy commandments those who shun that which displeaseth Thee.}

  \let\vrtitle=\undefined
  \printcommem{Friday}{1}{P}{an--principes_consilium--solesmes}
  {The chief Priests consulted that they might kill Jesus, but they said: Not on the Feast-day, lest there be an uproar among the people.}
  {Concéde, quǽsumus omnípotens Deus~:~* ut qui protectiónis tuæ grátiam quǽrimus, liberáti a malis ómnibus, secúra tibi mente serviámus. Per Dóminum.}
  {Grant, we beseech Thee, Almighty God, that we who seek the proctection of Thy grace, freed from all evils, may serve Thee in peace and quietness of spirit.}
}
\vspace{-\baselineskip}
  % \bigskip
  % \benedicamusdomino{}
}


%May 1: St Joseph the Worker (1st class)
{
\section{May 1: St Joseph the Worker}
\subtitle{\nth{1} Class, White or Gold}
\subtitle{I \& II Vespers}

\def\definevesperspropers{\newcommand{\vrtex}{vrOraProNobis}
\newcommand{\vtranslation}{Pray for us, St Joseph, alleluia.}
\newcommand{\rtranslation}{Faithful protector of our labors, alleluia.}

\newcommand{\maganttex}{an--et_ipse_jesus--solesmes}
\newcommand{\magantinitial}{E}
\newcommand{\maganttranslation}{}
\definemag{7}{d}

  \let\oldthing=\maganttranslation
  \def\maganttranslation{\oldthing\needspace{10\baselineskip}}
}
\def\definevesperspropersalt{\newcommand{\vrtex}{vrSolemnitasEstHodie}
\newcommand{\vtranslation}{Today is the solemnity of St Joseph, alleluia.}
\newcommand{\rtranslation}{Who ministered with his hands to the Son of God, alleluia.}

\newcommand{\maganttex}{an--christus_dominus--solesmes}
\newcommand{\magantinitial}{C}
\newcommand{\maganttranslation}{Christ the Lord deigned to be thought the son of a carpenter, alleluia.}
\definemag{7}{c2}

  \def\vrlinebreak{T}
  \let\oldthing=\maganttranslation
  \def\maganttranslation{\oldthing\needspace{10\baselineskip}}
}
\def\vesperspropersnote{At II Vespers:}
\def\vesperspropersaltnote{At I Vespers:}
\def\prepsalmthreeverses{\vspace{-0.1\baselineskip}}
\def\prerepeatontiphonthree{}

\printvespers[../May1-StJosephWorker]{inc-StJosephWorker}
%if feast of St Joseph the worker falls from 2nd through 5th Sunday after Easter, it outranks the Sunday and the Sunday is commemorated
\medskip
\printnote{\sundaycommemnoteeaster{}

\begin{multicols}{2}
\noindent\emph{\nth{2} Sunday after Easter}, p.~\pageref{easter2}.\\
\emph{\nth{3} Sunday after Easter}, p.~\pageref{easter3}.\\
\emph{\nth{4} Sunday after Easter}, p.~\pageref{easter4}.\\
\emph{\nth{5} Sunday after Easter}, p.~\pageref{easter5}.
\end{multicols}
}
\bigskip
\benedicamusdomino{}
}

%May 22: St Philip Neri (1)
{
\section{May 22: St Philip Neri}
\subtitle{\nth{1} Class (proper to the Oratory), White or Gold}
\subtitle{I \& II Vespers}

\def\definevesperspropers{\newcommand{\maganttex}{MagnificatAntiphon2}
\newcommand{\magantinitial}{H}
\newcommand{\maganttranslation}{Come, children, hearken to me: I will teach you the fear of the Lord.}
\definemag{4}{A}

}
\def\definevesperspropersalt{\newcommand{\maganttex}{MagnificatAntiphon1}
\newcommand{\magantinitial}{D}
\newcommand{\maganttranslation}{My house shall be called a house of prayer, says the Lord.}

}
\def\vesperspropersnote{\bigskip{}

At II Vespers:}
\def\vesperspropersaltnote{\vspace{-1\baselineskip}At I Vespers:}
%\def\prehymn{\pagebreak}

\printvespers[../StPhilipNeri]{inc-StPhilipNeri}
%when can feast fall?
%if feast falls from 2nd through 5th Sunday after Easter, it outranks the Sunday and the Sunday is commemorated
\medskip
\printnote{\sundaycommemnoteeasterpentecost{}

\bigskip{}
\emph{\nth{5} Sunday after Easter}, p.~\pageref{easter5}.

\emph{Sunday after the Ascension}, p.~\pageref{easter6}.

% pentecost and trinity sunday would outrank this feast
\emph{\nth{2} Sunday after Pentecost}, p.~\pageref{pentecost2}.
}

\bigskip{}
\benedicamusdomino{}
%\vfil
%\pagebreak
}

%June 24: Nativity of St John the Baptist (1st class)
{
\section{June 24: Nativity of St John the Baptist}
\subtitle{\nth{1} Class, White or Gold}
\subtitle{I \& II Vespers}

\def\definevesperspropers{\newcommand{\antonetex}{Ant1-ElisabethZachariae}
\newcommand{\antoneinitial}{E}
\newcommand{\antonetranslation}{Elizabeth, the wife of Zacharias, gave birth to a great man, John the Baptist, the forerunner of the Lord.}
\definepsalm{1}{109}{3}{a}

\newcommand{\anttwotex}{Ant2-Innuebant}
\newcommand{\anttwoinitial}{I}
\newcommand{\anttwotranslation}{They made signs unto his father, by what name he should be called: and he wrote, saying: His name is John.}
\definepsalm{2}{110}{4}{E}

\newcommand{\antthreetex}{Ant3-JoannesVocabitur}
\newcommand{\antthreeinitial}{J}
\newcommand{\antthreetranslation}{His name shall be called John, and many shall rejoice in his birth.}
\definepsalm{3}{111}{1}{f}

\newcommand{\antfourtex}{Ant4-InterNatos}
\newcommand{\antfourinitial}{I}
\newcommand{\antfourtranslation}{Among those born of women, there hath not risen a greater than John the Baptist.}
\definepsalm{4}{112}{3}{b}

\newcommand{\antfivetex}{Ant5-TuPuer}
\newcommand{\antfiveinitial}{T}
\newcommand{\antfivetranslation}{Thou, child, shalt be called the Prophet of the Most High: thou shalt go before the Lord to prepare His ways.}
\definepsalm{5}{116}{3}{b}

\newcommand{\vrtex}{vrIstePuerMagnus}
\newcommand{\vtranslation}{This child is great before the Lord.}
\newcommand{\rtranslation}{For in truth His hand is with him.}

\newcommand{\maganttex}{MagnificatAntiphon2}
\newcommand{\magantinitial}{P}
\newcommand{\maganttranslation}{The child that is born to us is more than a prophet; for this is he of whom the Saviour said: Among those born of women there hath not risen a greater than John the Baptist.}
\def\magsolemn{T}
\definemag{7}{d}

  \let\oldthing=\antonetranslation
  \def\antonetranslation{\oldthing\needspace{10\baselineskip}}
}
\def\definevesperspropersalt{\newcommand{\antonetex}{Ant1-IpsePraeibit}
\newcommand{\antoneinitial}{I}
\newcommand{\antonetranslation}{He shall go before Him in the spirit and power of Elias, to prepare unto the Lord a perfect people.\vspace{0ex plus 0ex minus 3ex}}
\definepsalm{1}{109}{7}{a}

\newcommand{\anttwotex}{Ant2-Joannes}
\newcommand{\anttwoinitial}{J}
\newcommand{\anttwotranslation}{John is his name.  Wine and strong drink he shall not drink, and many shall rejoice in his birth.}
\definepsalm{2}{110}{8}{G}

\newcommand{\antthreetex}{Ant3-ExUteroSenectutis}
\newcommand{\antthreeinitial}{E}
\newcommand{\antthreetranslation}{From the barren womb of age was born John, the forerunner of the Lord.}
\definepsalm{3}{111}{1}{f}

\newcommand{\antfourtex}{Ant4-IstePuer}
\newcommand{\antfourinitial}{I}
\newcommand{\antfourtranslation}{This child is great before the Lord, for the hand of God is with him.}
\definepsalm{4}{112}{4}{A*}

\newcommand{\antfivetex}{Ant5-Nazaraeus}
\newcommand{\antfiveinitial}{N}
\newcommand{\antfivetranslation}{This child shall be called a Nazarite; wine and strong drink he shall not drink, and from his mother's womb he shall eat nothing unclean.}
\definepsalm{5}{116}{5}{a}

\newcommand{\vrtex}{vrFuitHomo}
\newcommand{\vtranslation}{There was a man sent from God.}
\newcommand{\rtranslation}{Whose name was John.}

\newcommand{\maganttex}{MagnificatAntiphon1}
\newcommand{\magantinitial}{I}
\newcommand{\maganttranslation}{When Zacharias had entered the temple of the Lord, there appeared to him the Angel Gabriel, standing at the right hand of the altar of incense.\vspace{-4pt plus 4pt}}
\def\magpsalmclef{c3}
\def\magsolemn{T}
\definemag{8}{G}
}
\def\vesperspropersnote{At II Vespers:}
\def\vesperspropersaltnote{At I Vespers:}
\def\prevesperspsalms{\noindent\printnote{Chapter and following, page \pageref{june24-chapter}.\\}}
\def\vesperspsalmslabel{\label{june24-2vespers}}
\def\prevesperspsalmsalt{\noindent\printnote{II Vespers psalms and antiphons, page \pageref{june24-2vespers}.}\medskip}
\def\prechapter{\label{june24-chapter}}
%\def\premagnificat{\pagebreak}

\printvespers[../June24-BirthOfJohnTheBaptist]{inc-BirthOfJohnTheBaptist}

\printnote{\sundaycommemnote{}
%\vspace{-\baselineskip}
\begin{multicols}{2}
\noindent\emph{\nth{2} Sunday after Pentecost}, p.~\pageref{pentecost2}.\\
\emph{\nth{3} Sunday after Pentecost}, p.~\pageref{pentecost3}.\\
\emph{\nth{4} Sunday after Pentecost}, p.~\pageref{pentecost4}.\\
\emph{\nth{5} Sunday after Pentecost}, p.~\pageref{pentecost5}.\\
\emph{\nth{6} Sunday after Pentecost}, p.~\pageref{pentecost6}.
\end{multicols}
}
\benedicamusdomino{}
}

%June 29: Sts Peter \& Paul (1st class)
{
\section{June 29: Sts Peter \& Paul}
\subtitle{\nth{1} Class, Red}
%\vspace{-\baselineskip}
\subtitle{I \& II Vespers}
%\vspace{-\baselineskip}

\def\definevesperspropers{\import{../CommonOfApostles/}{inc-CommonOfApostles-2Vespers-psalms}
\edef\antonetex{../CommonOfApostles/\antonetex}
\edef\anttwotex{../CommonOfApostles/\anttwotex}
\edef\antthreetex{../CommonOfApostles/\antthreetex}
\edef\antfourtex{../CommonOfApostles/\antfourtex}
\edef\antfivetex{../CommonOfApostles/\antfivetex}

\newcommand{\vrtex}{vrAnnuntiaverunt}
\newcommand{\vtranslation}{They declared the works of God.}
\newcommand{\rtranslation}{And understood His doings.}

\newcommand{\maganttex}{MagnificatAntiphon2-Hodie}
\newcommand{\magantinitial}{H}
\newcommand{\maganttranslation}{Today, Simon Peter went up upon the gibbet of the cross, alleluia; today, he that holdeth the keys of the kingdom, departed with joy to be with Christ; today, the Apostle Paul, the light of the world, bowing his head, for Christ's sake was crowned with martyrdom, alleluia.}
\def\magsolemn{T}
\definemag{1}{D}

  \let\oldthing=\antonetranslation
  \def\antonetranslation{\vspace{-0.5\baselineskip}\oldthing\vspace{-0.5\baselineskip}}
  \let\oldthingb=\antthreetranslation
  \def\antthreetranslation{\vspace{-0.5\baselineskip}\oldthingb\vspace{-1.4\baselineskip}}
  \let\oldthingc=\antfourtranslation
  \def\antfourtranslation{\vspace{-0.5\baselineskip}\oldthingc\vspace{-0.6\baselineskip}}
  \def\prepsalmoneverses{\vspace{-0.1\baselineskip}}
}
\def\definevesperspropersalt{\newcommand{\antonetex}{Ant1-PetrusEtJoannes}
\newcommand{\antoneinitial}{P}
\newcommand{\antonetranslation}{Peter and John went up together into the Temple at the hour of prayer, being the ninth hour.}
\definepsalm{1}{109}{8}{G}

\newcommand{\anttwotex}{Ant2-Argentum}
\newcommand{\anttwoinitial}{A}
\newcommand{\anttwotranslation}{Silver and gold have I none, but such as I have, give I thee.}
\definepsalm{2}{110}{7}{b}

\newcommand{\antthreetex}{Ant3-DixitAngelusAdPetrum}
\newcommand{\antthreeinitial}{D}
\newcommand{\antthreetranslation}{The Angel said unto Peter: Cast thy garment about thee, and follow me.}
\definepsalm{3}{111}{8}{c}

\newcommand{\antfourtex}{Ant4-MisitDominus}
\newcommand{\antfourinitial}{M}
\newcommand{\antfourtranslation}{The Lord hath sent His Angel, and hath delivered me out of the hand of Herod.  Alleluia.}
\definepsalm{4}{112}{7}{c2}

\newcommand{\antfivetex}{Ant5-TuEsPetrus}
\newcommand{\antfiveinitial}{T}
\newcommand{\antfivetranslation}{Thou art Peter and upon this Rock I will build My Church.}
\definepsalm{5}{116}{7}{c}

\newcommand{\vrtex}{vrInOmnemTerram}
\newcommand{\vtranslation}{Their sound hath gone forth into all the earth.}
\newcommand{\rtranslation}{And their words unto the ends of the world.}

\newcommand{\maganttex}{MagnificatAntiphon1}
\newcommand{\magantinitial}{T}
\newcommand{\maganttranslation}{Thou art the Shepherd of the sheep and the Prince of the Apostles, and unto thee are given the keys of the kingdom of heaven.}
\def\magsolemn{T}
\definemag{1}{f}

  \let\oldthing=\antonetranslation
  \def\antonetranslation{\vspace{-0.5\baselineskip}\oldthing\vspace{-0.5\baselineskip}}
  \def\postpsalmtitleone{\oldneedspace{10\baselineskip}}
  \def\postpsalmtitletwo{\oldneedspace{10\baselineskip}}
  \def\preanttwo{\vspace{-0.4\baselineskip}}
  \def\prepsalmtwoverses{\vspace{-0.1\baselineskip}}
  %\let\oldthingb=\antfourtranslation
  %\def\antfourtranslation{\oldthingb\pagebreak}
  \let\oldthingc=\maganttranslation
  \def\maganttranslation{\oldthingc\needspace{6\baselineskip}}
}
\def\vesperspropersnote{At II Vespers:}
\def\vesperspropersaltnote{At I Vespers:}
\def\prevesperspsalms{\noindent\printnote{Chapter and following, page \pageref{june29-chapter}.\\}}
\def\vesperspsalmslabel{\label{june29-2vespers}}
\def\prevesperspsalmsalt{\noindent\printnote{II Vespers psalms and antiphons, page \pageref{june29-2vespers}.}\medskip}
\def\prechapter{\label{june29-chapter}}
%\def\precollect{\vspace{-0.5\baselineskip}}

\def\begincollectcols{\begin{parcolumns}[rulebetween,colwidths={1=0.45\linewidth}]{2}}
\printvespers[../June29-StsPeterAndPaul]{inc-StsPeterAndPaul}

\smallskip
\printnote{\sundaycommemnote{}
\vspace{-0.5\baselineskip}
\begin{multicols}{2}
\noindent\emph{\nth{3} Sunday after Pentecost}, p.~\pageref{pentecost3}.\\
\emph{\nth{4} Sunday after Pentecost}, p.~\pageref{pentecost4}.\\
\emph{\nth{5} Sunday after Pentecost}, p.~\pageref{pentecost5}.\\
\emph{\nth{6} Sunday after Pentecost}, p.~\pageref{pentecost6}.\\
\emph{\nth{7} Sunday after Pentecost}, p.~\pageref{pentecost7}.
\end{multicols}
}
\benedicamusdomino{}
}

%July 1: Most Precious Blood (1st class)
{
\global\let\psalmclefthree=\undefined
\section{July 1: The Precious Blood of Our Lord Jesus Christ}
\subtitle{\nth{1} Class, Red}
\subtitle{I \& II Vespers}

\def\prepsalmtitlefour{\vspace{-0.5\baselineskip}}
\def\definevesperspropers{\definepsalm{5}{147}{2}{D}

\newcommand{\vrtex}{vrTeErgo}
\newcommand{\vtranslation}{We therefore pray Thee, help Thy servants.}
\newcommand{\rtranslation}{Whom Thou hast redeemed with Thy precious Blood.}

\newcommand{\maganttex}{MagnificatAntiphon2}
\newcommand{\magantinitial}{H}
\newcommand{\maganttranslation}{Ye shall observe this day for a memorial: and ye shall keep it holy unto the Lord, in your generations with an everlasting worship.}
\newcommand{\magsolemn}{F}
\definemag{1}{D2}

  \def\prepsalmtitlefive{\bigskip{}\medskip{}}
}
\def\definevesperspropersalt{\definepsalm{5}{116}{2}{D}

\newcommand{\vrtex}{vrRedemistiNos}
\newcommand{\vtranslation}{Thou hast redeemed us, O Lord, in Thy Blood.}
\newcommand{\rtranslation}{And hast made of us a kingdom unto our God.}

\newcommand{\maganttex}{MagnificatAntiphon1}
\newcommand{\magantinitial}{A}
\newcommand{\maganttranslation}{Ye are come to Mount Sion, to the city of the living God, the heavenly Jerusalem, and to Jesus the Mediator of the new Testament, and to the sprinkling of blood which speaketh better than that of Abel.}
\newcommand{\magsolemn}{F}
\definemag{3}{a}
%
  \def\postmagtitle{\vspace{-0.5\baselineskip}}%
}
\def\vesperspropersnote{At II Vespers:}
\def\vesperspropersaltnote{At I Vespers:}

\def\begincollectcols{\begin{parcolumns}[rulebetween,colwidths={1=0.45\linewidth}]{2}}
\printvespers[../July1-MostPreciousBloodOfChrist]{inc-MostPreciousBloodOfChrist}
\medskip
\benedicamusdomino{}
}

%Aug 6: Transfiguration (2nd class)
{
\section{August 6: Transfiguration of Our Lord Jesus Christ}
\vspace{-0.5\baselineskip}
\subtitle{\nth{2} Class, White}
\vspace{-0.8\baselineskip}
\subtitle{I \& II Vespers}
\vspace{-0.5\baselineskip}

\def\prepsalmtitleone{\oldneedspace{4\baselineskip}}
\def\definevesperspropers{\newcommand{\maganttex}{MagnificatAntiphon2}
\newcommand{\magantinitial}{E}
\newcommand{\maganttranslation}{And the disciples hearing, fell on their faces, and were sore afraid; and Jesus came, and touched them, and said to them, Arise, and fear not, alleluia.}
\newcommand{\magsolemn}{F}
\definemag{1}{f}
}
\def\definevesperspropersalt{\newcommand{\maganttex}{an--christus_jesus--solesmes}
\newcommand{\magantinitial}{C}
\newcommand{\maganttranslation}{Christ Jesus, radiance of the Father and image of His Being, upholding all things by the word of His power; making atonement for sins, has deigned to appear today in glory on the high mountain.}
\newcommand{\magsolemn}{F}
\definemag{4}{E}
}
\def\vesperspropersnote{At II Vespers:}
\def\vesperspropersaltnote{At I Vespers:}

\def\begincollectcols{\begin{parcolumns}[rulebetween,colwidths={1=0.44\linewidth}]{2}}
\printvespers[../August6-TransfigurationOfOurLord]{inc-Transfiguration}
\bigskip
\benedicamusdomino[2]{}
}

%Aug 15: Assumption (1st class)
{
\section{August 15: Assumption of the B.~V.~M.}
\subtitle{\nth{1} Class, White or Gold}
\subtitle{I \& II Vespers}

\def\definevesperspropers{% hymn is ave maris stella
%\input{inc-hymn-avemarisstella}

\newcommand{\maganttex}{an--hodie_maria_virgo--solesmes}
\newcommand{\magantinitial}{H}
\newcommand{\maganttranslation}{Today the Virgin Mary has gone up to heaven: rejoice, for with Christ she reigns forever.}
\newcommand{\magsolemn}{T}
\definemag{8}{G*}
}
\def\definevesperspropersalt{\newcommand{\hymnlinetwo}{2.}
\newcommand{\hymntex}{Hymn-OPrimaVirgoProdita}
\newcommand{\hymninitial}{O}
\newcommand{\hymntranslation}{
\item O Virgin who was first to receive
The Creator’s grace by the spirit,
Who was predestined by the Most High
To bear in her womb the Son.

\item O woman, who was foretold to be
The perpetual enemy of the demon;
Who alone was filled with grace,
Undefiled from conception.

\item Thou who conceives Life itself in thy womb,
Life that was lost by Adam;
Furnishing the divine Victim,
A body for his sacrifice.

\item Death, the recompense for sin,
Had no victory over thee, and now departs;
And then thou hastened bodily to heaven
To be thy loving Son’s companion.

\item Illuminated by so great a Glory,
All nature is raised up;
And in thee is called to reach
The pinnacle of all glory and splendour.

\item In thy triumph O our Queen,
Turn thine eyes to us exiles;
That through thy patronage,
We may come to heaven, our blessed homeland.

\item Praise to the Father! praise to Him,
The Virgin’s holy Son!
Praise to the Spirit Paraclete,
While endless ages run! 
Amen.
}

\newcommand{\maganttex}{MagAntiphon-VirgoPrudentissima}
\newcommand{\magantinitial}{V}
\newcommand{\maganttranslation}{O Virgin most prudent, whither goest thou, like the golden dawn?  Daughter of Sion, thou art all beautiful and sweet; fair as the moon, bright as the sun.}
\newcommand{\magsolemn}{T}
\definemag{1}{f}

\let\oldthing=\maganttranslation
\def\maganttranslation{\oldthing\needspace{10\baselineskip}}
}
\def\vesperspropersnote{At II Vespers:}
\def\vesperspropersaltnote{At I Vespers:}
\def\printfullhymn{
  {
    \oldneedspace{3\baselineskip}
    \printnote{At II Vespers: Hymn.~\emph{Ave Maris Stella}, p.~\pageref{hymn-avemarisstella}. \Vbar{} \emph{Exaltata.} p.~\pageref{vr-assumption}.\\}

    \printnote{\vesperspropersaltnote}
    \definevesperspropersalt
    \printhymn{\oldstylenums{\hymnlinetwo}}{\hymninitial}{\hymntex}{\hymntranslation}
  }
  {
    \def\vrlinebreak{T}
    \oldneedspace{3\baselineskip}
    \label{vr-assumption}
    \printvr[\greseteolcustos{manual}]{\vrtex}{\vtranslation}{\rtranslation}
  }
}

\printvespers[../August15-AssumptionOfTheBlessedVirginMary]{inc-Assumption}

%     TODO Feast of Assumption (8/15) could have commem of Saturday before 3rd Sunday of August or 9th to 13th Sunday after Pentecost
\medskip
\printnote{\sundaycommemnote{}

\begin{multicols}{2}
\noindent\emph{\nth{9} Sunday after Pentecost}, p.~\pageref{pentecost9}.\\
\emph{\nth{10} Sunday after Pentecost}, p.~\pageref{pentecost10}.\\
\emph{\nth{11} Sunday after Pentecost}, p.~\pageref{pentecost11}.\\
\emph{\nth{12} Sunday after Pentecost}, p.~\pageref{pentecost12}.\\
\emph{\nth{13} Sunday after Pentecost}, p.~\pageref{pentecost13}.
\end{multicols}
}
\bigskip
\benedicamusdomino{}
}

%Sep 14: Exaltation of Holy Cross (2nd class)
{
\section{September 14: Exaltation of the Holy Cross}
\subtitle{\nth{2} Class, Red}
\subtitle{I \& II Vespers}

\def\definevesperspropers{\newcommand{\maganttex}{MagnificatAntiphon2-OCrux}
\newcommand{\magantinitial}{O}
\newcommand{\maganttranslation}{O blessed art thou, O Cross which wast counted the only tree worthy to bear the Lord and King of heaven. Alleluia.}
\def\magsolemn{F}
\definemag{1}{D2}
}
\def\definevesperspropersalt{\newcommand{\maganttex}{MagnificatAntiphon1-OCrux}
\newcommand{\magantinitial}{T}
\newcommand{\maganttranslation}{O Cross, brighter than all the stars thy name is honourable upon earth; exceeding lovely to mankind; holier than all things; thou alone wast worthy to carry the ransom of the world; sweet wood, sweet nails, bearing a burden sweeter still; save this people gathered here to praise thee.}
\def\magsolemn{F}
\definemag{1}{D}
%
%\def\premagverses{\oldneedspace{4\baselineskip}}%
}
\def\vesperspropersnote{At II Vespers:}
\def\vesperspropersaltnote{At I Vespers:}
\def\postpsalmtitlethree{\oldneedspace{9\baselineskip}}
\def\prerepeatantiphonthree{}
\def\prepsalmthreeverses{\vspace{-0.1\baselineskip}}
\def\prepsalmtitlefour{\oldneedspace{8\baselineskip}}
\def\prepsalmtitlefive{\needspace{8\baselineskip}}
\def\prepsalmfourverses{\oldneedspace{2\baselineskip}}
\def\prehymn{\printnote{All kneel for the sixth verse of the following hymn.}}
\def\prehymntranslation{\oldneedspace{3\baselineskip}}
%\def\precollect{\vspace{-0.5\baselineskip}}

\def\begincollectcols{\begin{parcolumns}[rulebetween,colwidths={1=0.42\linewidth}]{2}}
\printvespers[../September14-ExaltationOfTheHolyCross]{inc-ExaltationOfTheHolyCross}
\bigskip
\benedicamusdomino[2]{}
}

%Sep 29: Dedication of St Michael (1st class)
{
\section{September 29: Dedication of St Michael the Archangel}
\subtitle{\nth{1} Class, White or Gold}
\vspace{-0.5\baselineskip}
\subtitle{I \& II Vespers}
\def\postdeusinadjutorium{\needspace{10\baselineskip}}

\def\definevesperspropers{\definepsalm{5}{137}{7}{c}

\newcommand{\vrtex}{vrInConspectuAngelorum}
\newcommand{\vtranslation}{In the sight of the Angels, I will sing praise to Thee, O my God.}
\newcommand{\rtranslation}{I will worship towards Thy holy temple, and I will give glory to Thy name.}

\newcommand{\maganttex}{MagnificatAntiphon2}
\definemag{1}{D2}
\newcommand{\magantinitial}{P}
\newcommand{\maganttranslation}{O most glorious prince, Michael Archangel, be mindful of us, and here and everywhere entreat the Son of God for us, alleluia, alleluia.}

  \def\prepsalmtitlefive{\bigskip}
  \def\vrlinebreak{F}
}
\def\definevesperspropersalt{\definepsalm{5}{116}{7}{c}

\newcommand{\vrtex}{vrStetitAngelus}
\newcommand{\vtranslation}{The Angel stood by the altar of the temple.}
\newcommand{\rtranslation}{Holding in his hand a censer of gold.}

\newcommand{\maganttex}{MagnificatAntiphon1}
\newcommand{\magantinitial}{D}
\definemag{8}{G}
\newcommand{\maganttranslation}{While John was beholding the sacred Mystery, the Archangel Michael sounded a trumpet.  Forgive us, O Lord our God, Thou who openest the book, and loosest the seals thereof.  Alleluia.\vspace{-1ex}}
}
\def\vesperspropersnote{At II Vespers:\medskip}
\def\vesperspropersaltnote{At I Vespers:\medskip}
\def\prepsalmtitleone{\vspace{-0.5\baselineskip}}
\def\prechapter{\vspace{-0.5\baselineskip}}

\printvespers[../September29-DedicationOfChurchOfStMichaelArchangel]{inc-DedicationStMichael}

\medskip
\printnote{\sundaycommemnote{}

\begin{multicols}{2}
\noindent\emph{\nth{16} Sunday after Pentecost}, p.~\pageref{pentecost16}.\\
\emph{\nth{17} Sunday after Pentecost}, p.~\pageref{pentecost17}.\\
\emph{\nth{18} Sunday after Pentecost}, p.~\pageref{pentecost18}.\\
\emph{\nth{19} Sunday after Pentecost}, p.~\pageref{pentecost19}.\\
\emph{\nth{20} Sunday after Pentecost}, p.~\pageref{pentecost20}.
\end{multicols}
}
\medskip
\benedicamusdomino{}
}

%Oct 11: Maternity of the BVM (2nd class)
{
\section{October 11: Maternity of the B.~V.~M.}
\subtitle{\nth{2} Class, White}
\vspace{-1.5\baselineskip}
\subtitle{I \& II Vespers}
\printnote{This feast is outranked by any Sunday and no commemoration is made.}

\def\definevesperspropers{\newcommand{\maganttex}{MagnificatAntiphon2}
\newcommand{\magantinitial}{M}
\newcommand{\maganttranslation}{Thy Motherhood, O Virgin Mother of God, heralded joy to the whole world: for out of thee has arisen the sun of justice, Christ our God.}
\definemag{4}{E}

}
\def\definevesperspropersalt{\newcommand{\maganttex}{MagnificatAntiphon1}
\newcommand{\magantinitial}{C}
\newcommand{\maganttranslation}{Let us celebrate with joy the Motherhood of blessed Mary ever Virgin.}
\definemag{7}{a}
}
\def\vesperspropersnote{At II Vespers:}
\def\vesperspropersaltnote{At I Vespers:}
\def\printhymnnote{
  {
    \oldneedspace{3\baselineskip}
    \printnote{Hymn.~\emph{Ave Maris Stella}, p.~\pageref{hymn-avemarisstella}.\\}
  }
}
\def\prepsalmtitleone{\vspace{-0.4\baselineskip}}
\def\prepsalmtitletwo{}
\def\prevespers{
  \let\oldthing=\anttwotranslation
  \def\anttwotranslation{\vspace{-0.3\baselineskip}\oldthing\vspace{-0.7\baselineskip}}
}
\printvespers[../October11-MaternityOfBlessedVirginMary]{inc-MaternityOfBVM}

\medskip
\benedicamusdomino[mary]{}
}

%Last Sunday in October: Christ the King (1st class)
{
\section{Last Sunday in October: Jesus Christ, King}
\subtitle{\nth{1} Class, White or Gold}
\subtitle{II Vespers}

\def\postpsalmtitletwo{\needspace{12\baselineskip}}
\def\prerepeatantiphontwo{}
\def\definevesperspropers{\newcommand{\vrtex}{vr}
\newcommand{\vtranslation}{His dominion shall be increased.}
\newcommand{\rtranslation}{And of peace there shall be no end.}

\newcommand{\magantinitial}{H}
\newcommand{\maganttex}{MagnificatAntiphon}
\newcommand{\maganttranslation}{He hath on His garment and on His thigh written: King of kings and Lord of lords.  To Him be glory and empire for ever and ever.}
\def\magsolemn{T}
\definemag{7}{a}
}
\def\beginchaptercols{\begin{parcolumns}[rulebetween,colwidths={1=0.46\linewidth}]{2}}
%\def\begincollectcols{\begin{parcolumns}[rulebetween,colwidths={1=0.45\linewidth}]{2}}

\printvespers[../OctoberLastSunday-ChristTheKing]{inc-ChristTheKing}
\noindent
\printnote{If today is October 31, the First Vespers of All Saints is commemorated with \emph{Magnificat antiphon}, p.~\pageref{allsaints1-magnificat}; \emph{\Vbar{}~Lætámini}.~in simple commemoration tone, p.~\pageref{allsaints1-vr}; and \emph{Collect}, p.~\pageref{allsaints-collect}.}

\bigskip
\benedicamusdomino[1]{}
}

%Nov 1: All Saints (1st class)
{
\section{November 1: All Saints}
\subtitle{\nth{1} Class, White or Gold}
%\vspace{-2\baselineskip}
\subtitle{I \& II Vespers}

\def\postdeusinadjutorium{\pagebreak}
\def\postpsalmtitleone{\oldneedspace{12\baselineskip}}
%\def\prepsalmoneverses{\vspace{-0.2\baselineskip}}
%\def\presalmtitleone{\vspace{-0.8\baselineskip}}
\def\prerepeatantiphonone{}
%\def\preanttwo{\vspace{-0.5\baselineskip}}
%\def\prepsalmtitletwo{\vspace{-0.3\baselineskip}}
\def\postpsalmtitletwo{\needspace{12\baselineskip}}
\def\prepsalmtitlethree{\vspace{-\baselineskip}}
\def\prerepeatantiphontwo{}
\def\postpsalmtitlefour{\needspace{12\baselineskip}}
\def\prevespers{
  \let\oldthinga=\antonetranslation
  \def\antonetranslation{\vspace{-0.2\baselineskip}\oldthinga\vspace{-0.5\baselineskip}}
  %\let\oldthing=\anttwotranslation
  %\def\anttwotranslation{\vspace{-0.5\baselineskip}\oldthing}
}
\def\definevesperspropers{\definepsalm{5}{115}{8}{G}

\newcommand{\vrtex}{vrExsultabunt}
\newcommand{\vtranslation}{The Saints will rejoice in glory.}
\newcommand{\rtranslation}{They will be joyful upon their beds.}

\newcommand{\maganttex}{an--o_quam_gloriosum--solesmes}
\newcommand{\magantinitial}{O}
\newcommand{\maganttranslation}{Oh! how glorious is the kingdom where all the Saints rejoice with Christ; clothed in white robes, they follow the Lamb whithersoever he goeth!}
\def\magsolemn{T}
\definemag{6}{F}

  %\def\prepsalmtitlefive{\bigskip}
}
\def\definevesperspropersalt{\definepsalm{5}{116}{8}{G}

\newcommand{\vrtex}{vrLaetamini}
\newcommand{\vtranslation}{Be glad in the Lord, and rejoice ye righteous.}
\newcommand{\rtranslation}{And shout for joy, all ye that are upright in heart.}

\newcommand{\maganttex}{an--angeli_archangeli_all_saints--solesmes}
\newcommand{\magantinitial}{A}
\newcommand{\maganttranslation}{O ye Angels, Archangels, Thrones and Dominions, Principalities and Powers, Virtues, Cherubim and Seraphim, Patriarchs and Prophets, holy Teachers of the Law, all Apostles, Martyrs of Christ, holy Confessors, Virgins of the Lord, Hermits, and all Saints, intercede for us.}
\def\magsolemn{T}
\definemag{1}{D}

  \def\premag{\label{allsaints1-magnificat}}
}
\def\vraltlabel{allsaints1-vr}
\def\vesperspropersnote{\needspace{15\baselineskip}At II Vespers:}
\def\vesperspropersaltnote{At I Vespers:}
\def\prechapter{\needspace{6\baselineskip}}
\def\begincollectcols{\label{allsaints-collect}\begin{parcolumns}[rulebetween]{2}}

\printvespers[../November1-AllSaints]{inc-AllSaints}

\medskip
\printnote{\sundaycommemnote{}

\begin{multicols}{2}
\noindent\emph{\nth{21} Sunday after Pentecost}, p.~\pageref{pentecost21}.\\
\emph{\nth{22} Sunday after Pentecost}, p.~\pageref{pentecost22}.\\
\emph{\nth{23} Sunday after Pentecost}, p.~\pageref{pentecost23}.\\
\emph{\nth{4} Sunday after Epiphany}, p.~\pageref{epiphany4}.
\end{multicols}
}
\bigskip{}
\benedicamusdomino{}
}

%Nov 9: Dedication of Archbasilica of Holy Savior (2nd class)
{
\section{November 9: Dedication of Archbasilica of Holy Savior}
\subtitle{\nth{2} Class, White}
\subtitle{I \& II Vespers}
\printnote{All from the Common of the Dedication of a Church.}

\def\prepsalmtwoverses{\vspace{-0.05\baselineskip}}
\def\prerepeatantiphontwo{}
\def\postpsalmtitlethree{\medskip\needspace{10\baselineskip}}
\def\prepsalmthreeverses{\medskip}
%\def\prepsalmthreeverses{\vspace{-0.1\baselineskip}}
%\def\prerepeatantiphonthree{}
%\def\prerepeatantiphonfive{}
%\def\prepsalmtitlefive{\bigskip}
\def\prepsalmfiveverses{\smallskip}
%\def\prevespers{
%  \let\oldthing=\antfivetranslation
%  \def\antfivetranslation{\vspace{-0.6\baselineskip}\oldthing\vspace{-1\baselineskip}}
%}
\def\prevr{\needspace{10\baselineskip}}
\def\definevesperspropers{\newcommand{\vrtex}{vrDomumTuam}
\newcommand{\vtranslation}{Holiness becometh thy house, O Lord.}
\newcommand{\rtranslation}{Forever.}

\newcommand{\maganttex}{MagnificatAntiphon2-OQuamMetuendusEst}
\newcommand{\magantinitial}{O}
\newcommand{\maganttranslation}{How dreadful is this place. Surely this is none other but the house of God, and the gate of heaven.}
\def\magsolemn{T}
\def\magoneline{T}
\definemag{6}{F}

  \let\oldthinga=\maganttranslation
  \def\maganttranslation{\oldthinga\oldneedspace{8\baselineskip}}
}
\def\definevesperspropersalt{\newcommand{\vrtex}{vrHaecEstDomusDomini}
\newcommand{\vtranslation}{This is the house of the Lord, strongly built.}
\newcommand{\rtranslation}{It is well founded upon strong rock.}

\newcommand{\maganttex}{an--sanctificavit_dominus--solesmes}
\newcommand{\magantinitial}{S}
\newcommand{\maganttranslation}{The Lord has hallowed His dwelling; for this is the house of God; there they call on His Name, of which it is written: And My Name shall be there, saith the Lord.}
\def\magsolemn{T}
\definemag{1}{g}

}
\def\vesperspropersnote{At II Vespers:}
\def\vesperspropersaltnote{At I Vespers:}

\printvespers[../CommonOfDedicationOfChurch]{inc-DedicationOfChurch}
\bigskip
\benedicamusdomino[2]{}
}
}