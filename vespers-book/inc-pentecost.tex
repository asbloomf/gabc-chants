\chapter{Proper of the Time -- Pentecost Octave}
{
\newcommand{\benedicamusdomino}[1][1]{
	\noindent\printnote{\Vbar~\emph{Benedicámus Dómino}, page \pageref{benedicamusdomino-#1}.}
	\bigskip
	\hrule
}

{
\section{Pentecost Sunday}
\subtitle{\nth{1} Class}
\subtitle{I \& II Vespers}

\def\definevesperspropers{\definepsalm{5}{113}{7}{c2}

\newcommand{\vrtex}{vrLoquebantur}
\newcommand{\vtranslation}{The apostles spoke in divers tongues.}
\newcommand{\rtranslation}{The wonderful works of God.}

\newcommand{\maganttex}{MagnificatAntiphon2}
\newcommand{\magantinitial}{H}
\newcommand{\maganttranslation}{Today the days of Pentecost are complete, alleluia; today the Holy Ghost appeared in fire to the disciples, gave them gifts and graces, sent them into all the world to preach and to bear witness; whoever believes and is baptised shall be saved, alleluia.}

	\def\prepsalmfive{\greseteolcustos{manual}}
}
\def\definevesperspropersalt{\definepsalm{5}{116}{7}{c2}

\newcommand{\vrtex}{vrRepletiSunt}
\newcommand{\vtranslation}{They were all filled with the Holy Ghost, alleluia.}
\newcommand{\rtranslation}{And they began to speak, alleluia.}

\newcommand{\maganttex}{MagnificatAntiphon1}
\newcommand{\magantinitial}{N}
\newcommand{\maganttranslation}{I will not leave you orphans, alleluia; I go away, and I come unto you, alleluia; and your heart shall rejoice, alleluia.}
}
\def\vesperspropersnote{At II Vespers:}
\def\vesperspropersaltnote{At I Vespers:}
%\def\premag{\def\noeuouae{T}}
\def\premagverses{\greseteolcustos{manual}}
\def\printfullhymn{
	{\printhymn{\oldstylenums{\hymnlinetwo}}{\hymninitial}{\hymntex}{\hymntranslation}}
		% print all versicles that could follow hymn?
	{
		\def\vrlinebreak{T}
		%\label{vr-rorate}
		\printnote{\vesperspropersaltnote}
		\definevesperspropersalt
		\printvr[\greseteolcustos{manual}]{\vrtex}{\vtranslation}{\rtranslation}
	}
	\bigskip
	{
		\def\vrlinebreak{T}
		%\label{vr-rorate}
		\printnote{\vesperspropersnote}
		\definevesperspropers
		\printvr[\greseteolcustos{manual}]{\vrtex}{\vtranslation}{\rtranslation}
	}
}
\printvespers[../Pentecost]{inc-PentecostVespers}
\bigskip
\benedicamusdomino{}
}

{
\section{Trinity Sunday}
\subtitle{\nth{1} Class}

%\def\premag{\def\noeuouae{T}}
\def\premagverses{\greseteolcustos{manual}}
\def\printfullhymn{
	{
		\printhymn{\oldstylenums{\hymnlinetwo}}{\hymninitial}{\hymntex}{\hymntranslation}
		\def\vrlinebreak{T}
		\printvr[\greseteolcustos{manual}]{\vrtex}{\vtranslation}{\rtranslation}
	}
}
\def\begincollectcols{\vspace{-0.5\baselineskip}\begin{parcolumns}[rulebetween,colwidths={1=0.44\linewidth}]{2}}
\printvespers[../TrinitySunday]{inc-TrinitySunday-Vespers}
\bigskip
\benedicamusdomino{}
}

}


{
Easter is March 22 to April 25
    TODO Feast of Nativity of St John the Baptist (6/24) could have commem of 2nd through 6th Sunday after Pentecost
    TODO Sts Peter and Paul (6/29) could have commem of 3rd through 7th Sunday after Pentecost
    probably just give them a page number and say to use the simple tone for the versicle and response

    TODO Feast of Assumption (8/15) could have commem of Saturday before 3rd Sunday of August or 9th to 13th Sunday after Pentecost

    TODO Feast of St Michael the Archangel (9/29) could have commem of 16th to 20th Sunday after Pentecost

    TODO Feast of Christ the King if on October 31, commem for 1st vespers of All Saints

    TODO Feast of All Saints (11/1) commem of Sunday, Saturday before 1st Sunday of November, 21st - 23rd Sunday after Pentecost or 4th after Epiphany
}

\chapter{Proper of the Time -- Time After Pentecost}
{
\def\printcommonvespers{
	\subtitle{\nth{2} Class}
	\printnote{From Vespers of Sundays throughout the year on page \pageref{sundayvespers}.}
}
\newcommand{\benedicamusdomino}[1][sunday]{
	\noindent\printnote{\Vbar~\emph{Benedicámus Dómino}, page \pageref{benedicamusdomino-#1}.}
	\bigskip
	\hrule
}
\newcommand{\printhymnnote}{}
\newcommand{\printvespersafterpentecost}[1]{
	{
	\section{\nth{#1}\ifnum#1=24{th or Last }\fi	Sunday after Pentecost}
	\printcommonvespers{}
	\printvespersmag[../TimeAfterPentecost]{inc-VespersMagnificatPentecost#1}
	\smallskip
	\benedicamusdomino{}
	}
	
}

\printvespersafterpentecost{2}
\printvespersafterpentecost{3}
\printvespersafterpentecost{4}
\printvespersafterpentecost{5}
\printvespersafterpentecost{6}
\printvespersafterpentecost{7}
\printvespersafterpentecost{8}
\printvespersafterpentecost{9}
\printvespersafterpentecost{10}
\printvespersafterpentecost{11}
\printvespersafterpentecost{12}
\printvespersafterpentecost{13}
\printvespersafterpentecost{14}
\printvespersafterpentecost{15}
\printvespersafterpentecost{16}
\printvespersafterpentecost{17}
\printvespersafterpentecost{18}
\printvespersafterpentecost{19}
\printvespersafterpentecost{20}
\printvespersafterpentecost{21}
\printvespersafterpentecost{22}
\printvespersafterpentecost{23}
\printvespersafterpentecost{24}

}