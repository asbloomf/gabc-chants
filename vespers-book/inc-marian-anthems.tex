\chapter{Marian Anthems}
{
\def\nogabcbreaks{T}
\newcommand{\printsimpletone}{
\vspace{0ex plus 1ex minus 1ex}
\oldneedspace{3\baselineskip}
{\centering\textbf{Simple Tone.}\par}
\vspace{0ex plus 1ex minus 1ex}
}
\newcommand{\printsolemntone}{
\vspace{0ex plus 1ex minus 1ex}
\oldneedspace{3\baselineskip}
{\centering\textbf{Solemn Tone.}\par}
\vspace{0ex plus 1ex minus 1ex}
}
\newcommand{\afterant}{
\ifx\note\undefined\else%
\noindent\textit{\note}

\smallskip
\fi
\sloppy

\begin{columns}
\versicle{\vlatin}{\venglish}
\response{\rlatin}{\renglish}
\colchunk{}
\colplacechunks{}
\colchunk{\hspace*{3em}Orémus\label{\prayerlabel}.}\colchunk{\hspace*{3em}Let us pray,}
\colplacechunks{}
\prayer{\prayerlatin}{\prayerenglish}
\end{columns}
\ifx\notetwo\undefined\else%
\medskip

\needspace{2\baselineskip}
\noindent\textit{\notetwo}

\smallskip
\begin{columns}
\versicle{\vlatintwo}{\venglishtwo}
\response{\rlatintwo}{\renglishtwo}
\colchunk{}
\colplacechunks{}
\colchunk{\hspace*{3em}Orémus.}\colchunk{\hspace*{3em}Let us pray,}
\colplacechunks{}
\prayer{\prayerlatintwo}{\prayerenglishtwo}
\end{columns}
\fi
}
\newcommand{\shortafterant}{
\ifx\note\undefined\else%
\noindent\textit{\note}

\smallskip
\fi
\sloppy
%\begin{columns}
%\colchunk{\selectlanguage{latin}\Vbar{}~\vlatin}
%\colchunk{\selectlanguage{latin}\Rbar{}~\rlatin}
%\end{columns}
\selectlanguage{latin}\Vbar{}~\vlatin

\selectlanguage{latin}\Rbar{}~\rlatin

\ifx\notetwo\undefined\else%
\medskip

\needspace{2\baselineskip}
\noindent\textit{\notetwo}

\smallskip
%\begin{columns}
%\colchunk{\selectlanguage{latin}\Vbar{}~\vlatintwo}
%\colchunk{\selectlanguage{latin}\Rbar{}~\rlatintwo}
%\end{columns}
\selectlanguage{latin}\Vbar{}~\vlatintwo

\selectlanguage{latin}\Rbar{}~\rlatintwo
\fi

\ifx\postvr\undefined
  \smallskip
\else
  \postvr
\fi
\noindent{}\textit{Prayer on page \pageref{\prayerlabel}}.
}

\def\gabcfolder{../MarianAntiphons}

\label{almaredemptorismater}
\section{Alma Redemptoris Mater}
\printnote{From Vespers of Saturday before the \nth{1} Sunday of Advent to \nth{2} Vespers of the Purification.}
\ifthenelse{\boolean{includepolyphony}}{

  \noindent
  \emph{Polyphonic version by Palestrina is found on page \pageref{alma_poly}.}
  \smallskip
}{}
\printsimpletone{}
{
\grechangenextscorelinedim{1}{spacelinestext}{0.65 cm}{scalable}%
\printgabc{Ant.}{5.}{A}{AlmaSimple}
}
\medskip
\printsolemntone{}
\printgabc{Ant.}{5.}{A}{AlmaSolemn}

\begin{quote}{
Beloved Mother of the Savior, and wide open Gate of heaven,
Star of the Sea, bring help to the people that weakens and wavers.
O thou, whom marvelling Nature saw give birth to thy Creator;
Virgin before and after, taking up that ``Hail'' from the mouth of Gabriel, show pity, to us sinners.%
%Mother of Christ! hear thou thy people's cry
%Star of the deep, and portal of the sky!
%Mother of Him who thee from nothing made,
%Sinking we strive and call to thee for aid;
%Oh, by that joy which Gabriel brought to thee,
%Thou Virgin first and last, let us thy mercy see.%
}\end{quote}

{
\newcommand{\note}{During Advent:}
\newcommand{\vlatin}{Angelus Dómini nuntiávit Maríæ.}
\newcommand{\venglish}{The Angel of the Lord declared unto Mary.}
\newcommand{\rlatin}{Et concépit de Spíritu Sancto.}
\newcommand{\renglish}{And she conceived by the Holy Ghost.}
\newcommand{\prayerlatin}{Grátiam tuam, quǽsumus Dómine, méntibus nostris infúnde~:~\dag{} ut qui, Angelo nuntiánte, Christi Fílii tui incarnatiónem cognóvimus,~* per passiónem ejus et crucem ad resurrectiónis glóriam perducámur.  Per eúmdem Christum Dóminum nostrum. \Rbar~Amen.}
\newcommand{\prayerenglish}{Pour forth, we beseech Thee, O Lord, Thy grace into our hearts; that we, to whom the Incarnation of Christ Thy Son was made known by the message of an Angel, may, by His Passion and Cross, be brought to the glory of His Resurrection. Through the same Christ our Lord. \Rbar~Amen.}

\newcommand{\notetwo}{From \nth{1} Vespers of Christmas to \nth{2} Vespers of the Purification:}
\newcommand{\vlatintwo}{Post pártum Vírgo invioláta permansísti.}
\newcommand{\venglishtwo}{After childbirth thou didst remain a pure Virgin.}
\newcommand{\rlatintwo}{Déi Génitrix intercéde pro nóbis.}
\newcommand{\renglishtwo}{Intercede for us, O Mother of God.}
\newcommand{\prayerlatintwo}{Deus, qui salútis ætérnæ, beátæ Maríæ virginitáte f\oe cúnda, humáno géneri prǽmia præstitísti~:~\dag{} tríbue, quǽsumus; ut ipsam pro nobis intercédere sentiámus,~* per quam merúimus auctórem vitæ suscípere, Dóminum nostrum Jesum Christum Fílium tuum. \Rbar~Amen.}
\newcommand{\prayerenglishtwo}{O God, who, by the fruitful virginity of blessed Mary, hast given to mankind the rewards of eternal salvation; grant, we beseech Thee, that we may experience her intercession for us, through whom we have deserved to receive the Author of life, our Lord Jesus Christ, Thy Son. \Rbar~Amen.}
\def\prayerlabel{almaprayer}

\afterant{}
}

% Ave Regina Cælorum
\oldneedspace{8\baselineskip}
\label{avereginacaelorum}
\section{Ave Regina Cælorum}
\printnote{From Compline of February \nth{2} (even if the Feast of the Purification be transferred) until Compline of Wednesday in Holy Week.}
\ifthenelse{\boolean{includepolyphony}}{

  \noindent
  \emph{Polyphonic version by Soriano is found on page \pageref{averegina_poly}.}
  \smallskip
}{}
\printsimpletone{}
\printgabc{Ant.}{6.}{A}{AveReginaSimple}
\bigskip
\printsolemntone{}
\printgabc{Ant.}{6.}{A}{AveReginaSolemn}
\begin{quote}{Hail, Queen of Heaven! Hail, Queen of Angels! Hail, blest Root and Gate, from which came light upon the world! Rejoice, O glorious Virgin, that surpassest all in beauty! Hail, O most lovely, and pray for us to Christ.}\end{quote}

{
\newcommand{\vlatin}{Dignáre me laudáre te Vírgo sacráta.}
\newcommand{\venglish}{Voucshafe, O holy Virgin, that I may praise thee.}
\newcommand{\rlatin}{Da míhi virtútem cóntra hóstes túos.}
\newcommand{\renglish}{Give me power against thine enemies.}
\newcommand{\prayerlatin}{Concéde, miséricors Deus, fragilitáti nostræ præ\-sí\-di\-um :~\gredagger{}~ut qui sanctæ Dei Genitrícis memóriam ágimus, * intercessiónis ejus auxílio a nostris iniquitátibus resurgámus. Per eúmdem Christum Dóminum nostrum. \Rbar~Amen.}
\newcommand{\prayerenglish}{Grant, O merciful God, Thy protection to us in our weakness; that we, who celebrate the memory of the holy Mother of God, may, through the aid of her intercession, rise again from our sins. Through the same Christ our Lord. \Rbar~Amen.}
\def\prayerlabel{avereginacaelorumprayer}

\afterant{}
}

\vspace{-0.5\baselineskip}
% Regina Cæli
\needspace{9\baselineskip}
\label{reginacaeli}
\section{Regina cæli}
\printnote{From Compline of Easter Sunday to Compline of Friday after the Feast of Pentecost inclusively.}
\ifthenelse{\boolean{includepolyphony}}{

  \noindent
  \emph{Polyphonic version by Palestrina is found on page \pageref{reginacaeli_poly}.}
  \smallskip
}{}
\printsimpletone{}
\printgabc{Ant.}{6.}{R}{ReginaCaeliSimple}
\oldneedspace{12\baselineskip}
\bigskip
\printsolemntone{}
\printgabc{Ant.}{6.}{R}{ReginaCaeliSolemn}

\def\reginacaelienglish{%
\begin{quote}{O Queen of heaven, rejoice, alleluia.\\
For He whom thou didst merit to bear, alleluia;\\
Has risen as He said, alleluia.\\
Pray for us to God, alleluia.}\end{quote}%
}
\reginacaelienglish{}

\bigskip{}
{
\newcommand{\vlatin}{Gáude et lætáre Virgo María, allelúia.}
\newcommand{\venglish}{Rejoice and be glad, O Virgin Mary, alleluia.}
\newcommand{\rlatin}{Quia surréxit Dóminus vere, allelúia.}
\newcommand{\renglish}{For the Lord is risen indeed, alleluia.}
\newcommand{\prayerlatin}{Deus, qui per resurrectiónem Fílii tui Dómini nostri Jesu Christi mundum lætificáre dignátus es~:~\gredagger{} præsta, quǽsumus; ut per ejus Genitrícem Vírginem Maríam~* perpétuæ capiámus gáudia vitæ. Per eúmdem Christum Dóminum nostrum. \Rbar~Amen.}
\newcommand{\prayerenglish}{O God, who didst vouchsafe to give joy to the world through the resurrection of Thy Son our Lord Jesus Christ; grant, we beseech Thee, that through His Mother, the Virgin Mary, we may obtain the joys of everlasing life. Through the same Christ our Lord. \Rbar~Amen.}
\def\prayerlabel{reginacaeliprayer}

\afterant{}
}


% Salve Regina
\def\salvereginaenglish{
  \begin{quote}{Hail, holy Queen, Mother of mercy; hail, our life, our sweetness, and our hope! To thee do we cry, poor banished children of Eve; to thee do we send up our sighs, mourning and weeping in this vale of tears.  Turn then, most gracious advocate, thine eyes of mercy towards us; and after this our exile, show unto us the blessed fruit of thy womb, Jesus.  O clement, O loving, O sweet Virgin Mary.}\end{quote}
}

\needspace{10\baselineskip}
\label{salveregina}
\section{Salve Regina}
{
\newcommand{\vlatin}{Ora pro nóbis sáncta Déi Génitrix.}
\newcommand{\venglish}{Pray for us, O holy Mother of God.}
\newcommand{\rlatin}{Ut dígni efficiámur promissiónibus Chrísti.}
\newcommand{\renglish}{That we may be made worthy of the promises of Christ.}
\newcommand{\prayerlatin}{Omnípotens sempitérne Deus, qui gloriósæ Vírginis Matris Maríæ corpus et ánimam, ut dignum Fílii tui habitáculum éffici mererétur, Spíritu Sancto cooperánte præparásti~:~\gredagger{} da, ut cujus commemoratióne lætámur,~* ejus pia intercessióne ab instántibus malis et a morte perpétua liberémur. Per eúmdem Christum Dóminum nostrum. \Rbar~Amen.}
\newcommand{\prayerenglish}{Almighty, everlasting God, who with the cooperation of the Holy Ghost didst prepare the body and soul of the glorious Virgin Mary, to make it fit to be the worthy dwelling of Thy Son; grant that by the loving intercession of her in whose commemoration we rejoice, we may be delivered from present ills, and from everlasting death.  Through the same Christ our Lord. \Rbar~Amen.}
\def\prayerlabel{salveprayer}

\printnote{From \nth{1} Vespers of the Feast of the Blessed Trinity to None on Saturday before the \nth{1} Sunday of Advent.}
\ifthenelse{\boolean{includepolyphony}}{

  \medskip\noindent
  \emph{Polyphonic version by Soriano is found on page \pageref{salveregina_poly}.}
  \smallskip
}{}
\printsimpletone{}
\printgabc{Ant.}{5.}{S}{SalveReginaSimple}
  \bigskip{}
\ifthenelse{\boolean{includesalvewithdrone}}{
  \salvereginaenglish
  \medskip
}{}
\printsolemntone{}
\ifthenelse{\boolean{includesalvewithdrone}}{
  \emph{The solemn tone is found on the next page.}

  \bigskip
  \afterant{}
  \bigskip

  \printsolemntone{}
  The puncta cava (open notes) indicate where a male chorus can sing a drone.  These drones are always a fifth apart on either the RE and the LA or the DO and the SOL.  Then for the sections with a brace above, instead of droning, the melody is sung in parallel, either a fifth above or below.  The drone should only be interrupted at the double bars.
  \smallskip
  {
    \grechangedim{spacelinestext}{0.62 cm}{scalable}%
    \printgabc{Ant.}{1.}{S}{SalveReginaSolemn-drone}

    \bigskip
    \shortafterant{}
  }
}{
  \printgabc{Ant.}{1.}{S}{SalveReginaSolemn}

  \needspace{\baselineskip}
  \salvereginaenglish
}

%\vfil
\ifthenelse{\boolean{includesalvewithdrone}}{}{
  \bigskip
  \afterant{}
}

%\vfil
\bigskip

}

}
