{
\def\printcommemnote{\smallskip
\noindent
\printnote{\commemorations{}  Otherwise \Vbar~\emph{Bendicámus Dómino}, page \pageref{benedicamusdomino-easter}.}
}
{
\chapter{Proper of the Time -- Easter}
\section{Easter Sunday}
\subtitle{\nth{1} Class}
\def\printfullhymn{
    \emph{Chapter, Hymn, and Versicle are all omitted, but the following Antiphon is said :}

    \bigskip
    \def\annot{\small{Ant.}}
    \def\annottwo{\small{\chapterhymnversicleantiphonmode.}}
    \alsetinitialspacing{\chapterhymnversicleantiphoninitial}
    \gregorioscore{\gabcfolder/\chapterhymnversicleantiphontex}
    \translation[]{\chapterhymnversicleantiphontranslation}
    \bigskip
}
\def\chapterreplacement{\bigskip}
\def\begincollectcols{\begin{parcolumns}[rulebetween,colwidths={1=0.45\linewidth}]{2}}
\def\postmag{\vspace{-0.05\baselineskip}}
\printvespers[../Easter]{inc-EasterVespers}
\newcommand{\printbenedicamusdomino}[2]{
	\def\annot{\small{#1}}
	\def\annottwo{}
	\alsetinitialspacing{B}
    \greseteolcustos{manual}
	\gregorioscore{#2}
    \bigskip
    \hrule
}
\def\breakbeforeresp{T}
\printbenedicamusdomino{\Vbar}{../BenedicamusDomino/BenedicamusDomino_Easter}
}

\newcommand{\printcommonvespers}[1][2]{
    \subtitle{\nth{#1} Class}
    \printnote{From Vespers of Sundays in Eastertide on page \pageref{sundayvespers-easter}.\\}
}
{
\newcommand{\benedicamusdomino}[1][easter]{
    \noindent\printnote{\Vbar~\emph{Benedicámus Dómino}, page \pageref{benedicamusdomino-#1}.}
    \bigskip
    \hrule
}

\newcommand{\printhymnnote}{
    \noindent\printnote{Hymn. \emph{Ad Régias Agni Dapes}, page \pageref{hymn-adregiasagnidapes}.
    \Vbar~\emph{Mane nobíscum}, page \pageref{vr-manenobiscum}.}
}

{
\section{Low Sunday}
\printcommonvespers[1]
%\def\begincollectcols{\begin{parcolumns}[rulebetween,colwidths={1=0.42\linewidth}]{2}}
\printvespersmag[../TimeAfterEaster]{inc-VespersMagnificatEaster1}

\def\commemorations{If the Feast of the Annunciation has been transferred to the Monday following Low Sunday, First Vespers is commemorated as on page \pageref{annunciation-commem}.  If today is April 30, May 1, or May 2, then First Vespers of St Joseph the Worker is commemorated as follows.}
\printcommemnote{}
}

\medskip
\hrule
{
\label{stjoseph-worker-commem}
\def\begincollectcols{\begin{parcolumns}[rulebetween,colwidths={1=0.43\linewidth}]{2}}
\def\vrlinebreak{T}
\printcommemoration[../May1-StJosephWorker]{commemorationStJosephWorker-Vespers1}

\bigskip
\benedicamusdomino{}
}

%{
%    TODO If feast of St Joseph the worker falls from 2nd through 5th Sunday after Easter, it outranks the Sunday and the Sunday is commemorated
%   Only Low Sunday and the Ascension itself are likely to outrank it
%}

{
\section{\nth{2} Sunday after Easter}
\printcommonvespers{}
\printvespersmag[../TimeAfterEaster]{inc-VespersMagnificatEaster2}
\benedicamusdomino{}
}
{
\section{\nth{3} Sunday after Easter}
\printcommonvespers{}
\def\begincollectcols{\begin{parcolumns}[rulebetween,colwidths={1=0.44\linewidth}]{2}}
\printvespersmag[../TimeAfterEaster]{inc-VespersMagnificatEaster3}
\benedicamusdomino{}
}

{
\section{\nth{4} Sunday after Easter}
\printcommonvespers{}
\printvespersmag[../TimeAfterEaster]{inc-VespersMagnificatEaster4}
\benedicamusdomino{}
}

{
\section{\nth{5} Sunday after Easter}
\printcommonvespers{}
\printvespersmag[../TimeAfterEaster]{inc-VespersMagnificatEaster5}
\benedicamusdomino{}
}

% {
% \section{Ascension of Our Lord}
% \subtitle{\nth{1} Class}
% TODO

% Also need to do St Joseph the Worker

% \def\commemorations{If today is April 30 or May 1, the First Vespers of St Joseph the Worker is commemorated as on page \pageref{stjoseph-worker-commem}.}
% \printcommemnote{}
% }

{
\section{Sunday after the Ascension}
\printcommonvespers{}
\let\printhymnnote=\undefined
\printvespersmag[../TimeAfterEaster]{inc-VespersMagnificatSundayAfterAscension}
\benedicamusdomino{}
}
}
}