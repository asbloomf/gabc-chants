{
\def\printcommemnote{\smallskip
\noindent
\printnote{\commemorations{}  Otherwise \Vbar~\emph{Bendicámus Dómino}, page \pageref{benedicamusdomino-easter}.}
}
{
\chapter{Proper of the Time -- Easter}
\section{Easter Sunday}
\subtitle{1st Class}
\def\printfullhymn{
    \emph{Chapter, Hymn, and Versicle are all omitted, but the following Antiphon is said :}

    \bigskip
    \def\annot{\small{Ant.}}
    \def\annottwo{\small{\chapterhymnversicleantiphonmode.}}
    \setinitialspacing{\chapterhymnversicleantiphoninitial}
    \includescore{\gabcfolder/\chapterhymnversicleantiphontex}
    \translation[]{\chapterhymnversicleantiphontranslation}
    \bigskip
}
\def\chapterreplacement{\bigskip}
\def\begincollectcols{\begin{parcolumns}[rulebetween,colwidths={1=0.45\linewidth}]{2}}
\def\postmag{\vspace{-0.05\baselineskip}}
\printvespers[../Easter]{inc-EasterVespers}
\newcommand{\printbenedicamusdomino}[2]{
	\def\annot{\small{#1}}
	\def\annottwo{}
	\setinitialspacing{B}
    \greblockcustos{}
	\includescore{#2}
    \bigskip
    \hrule
}
\def\breakbeforeresp{T}
\printbenedicamusdomino{\Vbar}{../BenedicamusDomino/BenedicamusDomino_Easter}
}

\newcommand{\printcommonvespers}[1][2nd]{
    \subtitle{#1 Class}
    \printnote{From Vespers of Sundays in Eastertide on page \pageref{sundayvespers-easter}.}
}
{
\newcommand{\benedicamusdomino}[1][easter]{
    \noindent\printnote{\Vbar~\emph{Benedicámus Dómino}, page \pageref{benedicamusdomino-#1}.}
    \bigskip
    \hrule
}

{
\section{Low Sunday}
\printcommonvespers[1st]
\def\printfullhymn{
    \label{hymn-adregiasagnidapes}
    {
        \def\hymnlinetwo{\oldstylenums{8.}}
\def\hymntex{hymn-AdRegiasAgniDapes}
\def\hymninitial{A}
\def\hymntranslation{\item At the Lamb's high feast we sing
praise to our victorious King,
Who hath washed us in the tide
flowing from His pierced side.
\item Praise we Him Whose love divine
gives the guests His Blood for wine,
gives His Body for the feast,
Love the victim, Love the priest.
\item Where the Paschal blood is poured,
Death's dark Angel sheathes his sword;
Israel's hosts triumphant go
through the wave that drowns the foe.
\item Christ, the Lamb Whose Blood was shed,
Paschal victim, Paschal bread;
with sincerity and love
eat we manna from above.
\item Mighty Victim from the sky,
powers of hell beneath Thee lie;
Death is conquered in the fight;
Thou hast brought us life and light.
\item Now Thy banner Thou dost wave;
vanquished Satan and the grave;
see the prince of darkness quelled;
heaven's bright gates are open held.
\item Paschal triumph, Paschal joy,
only sin can this destroy;
from sin's death do Thou set free
souls re-born, dear Lord, in Thee.
\item Hymns of glory, songs of praise,
Father, unto Thee we raise;
risen Lord, all praise to Thee,
ever with the Spirit be.
Amen.}

\def\vrtex{vrManeNobiscum}
\def\vtranslation{Stay with us O Lord, alleluia.}
\def\rtranslation{Because it is towards evening, alleluia.}

        \printhymn{\oldstylenums{\hymnlinetwo}}{\hymninitial}{\hymntex}{\hymntranslation}
        \def\vrlinebreak{T}
        \label{vr-manenobiscum}
        \printvr[\greblockcustos]{\vrtex}{\vtranslation}{\rtranslation}
        \bigskip
    }
}
%\def\begincollectcols{\begin{parcolumns}[rulebetween,colwidths={1=0.42\linewidth}]{2}}
\printvespersmag[../TimeAfterEaster]{inc-VespersMagnificatEaster1}

\def\commemorations{If the Feast of the Annunciation has been transferred to the Monday following Low Sunday, First Vespers is commemorated as on page \pageref{annunciation-commem}.  If today is April 30, May 1, or May 2, then First Vespers of St Joseph the Worker is commemorated as follows.}
\printcommemnote{}
}

{
\label{stjoseph-worker-commem}
\def\vrlinebreak{T}
%\printcommemoration[../May1-StJosephWorker]{commemorationStJosephWorker-Vespers1}

\bigskip
\benedicamusdomino{}
}

{
    TODO Feast of St Joseph the worker could have commem of 2nd through 5th Sunday after Easter
    probably just give them a page number and say to use the simple tone for the versicle and response
}

\newcommand{\printhymnnote}{
    \noindent\printnote{Hymn. \emph{Ad Régias Agni Dapes}, page \pageref{hymn-adregiasagnidapes}.
    \Vbar~\emph{Mane nobíscum}, page \pageref{vr-manenobiscum}.}
}
{
\section{2nd Sunday after Easter}
\printcommonvespers{}
\printvespersmag[../TimeAfterEaster]{inc-VespersMagnificatEaster2}
\benedicamusdomino{}
}
{
\section{3rd Sunday after Easter}
\printcommonvespers{}
\printvespersmag[../TimeAfterEaster]{inc-VespersMagnificatEaster3}
\benedicamusdomino{}
}

{
\section{4th Sunday after Easter}
\printcommonvespers{}
\printvespersmag[../TimeAfterEaster]{inc-VespersMagnificatEaster4}
\benedicamusdomino{}
}

{
\section{5th Sunday after Easter}
\printcommonvespers{}
\printvespersmag[../TimeAfterEaster]{inc-VespersMagnificatEaster5}
\benedicamusdomino{}
}

{
\section{Ascension of Our Lord}
\subtitle{1st Class}
TODO

TODO could be commem of 1st vespers of St Joseph the worker if today is April 30 or May 1

}

{
\section{Sunday after the Ascension}
\printcommonvespers{}
\def\printfullhymn{
    {
        \printhymn{\oldstylenums{\hymnlinetwo}}{\hymninitial}{\hymntex}{\hymntranslation}
        \def\vrlinebreak{T}
        \printvr[\greblockcustos]{\vrtex}{\vtranslation}{\rtranslation}
    }
}
\printvespersmag[../TimeAfterEaster]{inc-VespersMagnificatSundayAfterAscension}
\benedicamusdomino{}
}
}
}