\chapter{Proper of the Time -- Christmas Season}
{
\newcommand{\benedicamusdomino}[1][1]{
  \noindent\printnote{\Vbar~\emph{Benedicámus Dómino \IfInteger{#1}{#1}{\csname benedicamusdominoname#1\endcsname}}, page \pageref{benedicamusdomino-#1}.}
	\bigskip
	\hrule
}

{
\label{christmas}
\section{December 24: Nativity of our Lord -- I Vespers}
\subtitle{\nth{1} Class}

\newcommand{\printhymnnote}{
	\noindent\printnote{Hymn. \emph{Jesu Redémptor ómnium}, page \pageref{hymn-jesuredemptoromnium}.\\}
}
\def\definevesperspropers{\newcommand{\chaptertext}{\dropcap{latin}{Appáruit benígnitas et humánitas Salvatóris nostri} \textbf{Dé}\-i~\dag{} non ex opéribus justítiæ, quæ \emph{fé\-ci}\-\textbf{mus} nos,~* sed secúndum suam misericórdiam salvos nos \textbf{fé}cit.}
\newcommand{\chaptertranslation}{The goodness and kindness of God our Saviour appeared; not by the works of justice which we have done, but according to His mercy He saved us.}

\newcommand{\vrtex}{vrCrastinaDie}
\newcommand{\vtranslation}{Tomorrow shall the iniquity of the land be blotted out.}
\newcommand{\rtranslation}{And the Saviour of the world shall reign over us.}

\newcommand{\maganttex}{an--cum_ortus_fuerit--solesmes}
\newcommand{\magantinitial}{C}
\newcommand{\maganttranslation}{When the sun has risen in the sky, you shall see the King of kings coming forth from the Father, like a bridegroom from his chamber.}
\def\magsolemn{T}
\definemag{8}{G}

\newcommand{\collect}{Concéde, quǽsumus omnípotens Deus~:~\dag{} ut nos Unigéniti tui nova per carnem Natívitas líberet;~* quos sub peccáti jugo vetústa sérvitus tenet. Per eúm\-dem Dóminum.}
\newcommand{\collecttranslation}{Grant, we beseech Thee, almighty God, that the new birth of Thine only-begotten Son in the flesh may set us free, who are held by the old bondage under the yoke of sin.  Through the same our Lord.}
}
\newcommand{\preantfour}{\needspace{5\baselineskip}}
\printvespers[../Christmas-Vespers1]{inc-Christmas-Vespers1-psalms}

\benedicamusdomino{}
}

{
\section{December 25: Nativity of our Lord -- II Vespers}
\subtitle{\nth{1} Class}
\vspace{-1.5\baselineskip}
\subtitle{\&}
\vspace{-1.5\baselineskip}
\section{Sunday within the Octave}
\subtitle{\nth{2} Class}

\def\definevesperspropersalt{\newcommand{\chaptertext}{\dropcap{latin}{Multifáriam multísque modis olim Deus loquens pátribus in} pro\-\textbf{phé}\-tis~:~\dag{} novíssime diébus istis locútus est nobis in Fílio, quem constítuit hærédem ú\-\emph{ni\-ver}\-\textbf{só}\-rum,~* per quem fecit et \textbf{saé}\-cu\-la.}
\newcommand{\chaptertranslation}{God, who at sundry times and in divers manners spoke in times past to the fathers by the prophets, last of all in these days hath spoken to us by His Son, whom He hath appointed heir of all things, by whom also He made the world.}

\newcommand{\vrtex}{vrNotumFecitDominusSolemn}
\newcommand{\vtranslation}{The Lord hath made known, alleluia.}
\newcommand{\rtranslation}{His salvation, alleluia.}

\newcommand{\maganttex}{MagnificatAntiphon}
\newcommand{\magantinitial}{H}
\newcommand{\maganttranslation}{This day Christ is born: this day the Saviour hath appeared: this day the Angels sing on earth, and the Archangels rejoice: this day the just exult, saying: Glory to God in the highest, alleluia.}
\def\magsolemn{T}
\definemag{1}{g2}

\newcommand{\collect}{Concéde, quaésumus omnípotens Deus~:~\dag{} ut nos Unigéniti tui nova per carnem Natívitas líberet;~* quos sub peccáti jugo vetústa sérvitus tenet. Per eúmdem Dóminum.}
\newcommand{\collecttranslation}{Grant, we beseech Thee, almighty God, that the new birth of Thine only-begotten Son in the flesh may set us free, who are held by the old bondage under the yoke of sin.  Through the same Jesus Christ our Lord.}
}
\def\definevesperspropers{\newcommand{\chaptertext}{\dropcap{latin}{Fratres~: Quanto témpore hæres párvulus est, nihil differt a servo, cum sit dóminus} {\textbf{óm}\-ni\-um~:~\dag{} sed sub tutóribus et ac\-\emph{tó\-ri}\-\textbf{bus} est,~* usque ad præfinítum tempus a \textbf{pá}tre.}}
\newcommand{\chaptertranslation}{Brethren, as long as the heir is a child, he differeth nothing from a servant, though he be lord of all: but is under tutors and governors until the time appointed by the father.}

\newcommand{\vrtex}{vrVerbumCaroFactumEst}
\newcommand{\vtranslation}{The Word was made flesh, alleluia.}
\newcommand{\rtranslation}{And dwelt among us, alleluia.}

\newcommand{\maganttex}{MagAnt-PuerJesus}
\newcommand{\magantinitial}{P}
\newcommand{\maganttranslation}{The Child Jesus advanced in age and wisdom before God and men.}
\def\magsolemn{F}
\definemag{6}{F}

\newcommand{\collect}{Omnípotens sempitérne Deus, dírige actus nostros in beneplácito tuo~:~\dag{} ut in nómine dilécti Fílii tui,~* mereámur bonis opéribus abundáre.  Qui tecum vivit et regnat.}
\newcommand{\collecttranslation}{O almighty and everlasting God, direct our actions according to Thy good pleasure; that in the Name of Thy beloved Son we may deserve to abound in good works: Who with Thee liveth and reigneth.}

}
\def\vesperspropersaltnote{At II Vespers of Christmas:}
\def\vesperspropersnote{At II Vespers of the Sunday within the Octave of Christmas:}
\def\premag{\def\noeuouae{T}}
\def\premagverses{\greseteolcustos{manual}}
\def\hymnlabel{hymn-jesuredemptoromnium}
\def\hymnlinetwo{\oldstylenums{1.}}
\def\hymntex{hymn-JesuRedemptorOmnium}
\def\hymninitial{J}
\def\hymntranslation{\item Jesus, Redeemer of the world,
Begotten ere the dawn of light,
Wast of the Father's glory born,
Immense in glory as in might.

\item Thou art the Father's splendid light,
Thou art th'eternal hope of all
Throughout the world to Thee we pray,
O, hear Thy servants as they call.

\item Remember, O creator Lord!
That from the Virgin's sacred womb
Thou didst come forth, and of her flesh
Thou didst our mortal form assume.

\item This day, recurring, year by year,
Bears witness true that all alone
To save the world Thou camest forth,
Proceeding from the Father's throne.

\item This day, the stars, the earth, and sea,
And all creation welcome sing.
This day which brought our liberty.
When came our Lord, our Saviour, King.

\item And we, too, Lord, who have been washed
In Thine own font of Blood divine,
Offer the tribute of our praise
On this blest natal day of Thine.

\item O Jesu, of the Virgin born,
Unceasing glory be to Thee;
And to the Father infinite,
And Holy Ghost eternally.  Amen.
}

\printvespers[../Christmas]{inc-Christmas-Vespers2-psalms}

\benedicamusdomino{}
}

{
\section{January 1: Octave of the Nativity of our Lord}
\subtitle{\nth{1} Class}

\subtitle{I \& II Vespers}
\label{circumcision}

\def\definevesperspropers{\newcommand{\vrtex}{vrNotumFecitDominus}
\newcommand{\vtranslation}{The Lord hath made known, alleluia.}
\newcommand{\rtranslation}{His salvation, alleluia.}

\newcommand{\maganttex}{MagnificatAntiphon}
\newcommand{\magantinitial}{M}
\newcommand{\maganttranslation}{Great the mystery of our inheritance: the womb that knew not man is become the temple of God: taking flesh from her, he is not defiled: all nations shall come, and say: Glory to Thee, O Lord.}
\def\magsolemn{T}
\definemag{2}{A}
}
\def\preantthree{\needspace{10\baselineskip}}
 \newcommand{\printhymnnote}{
 	\noindent\printnote{Hymn. \emph{Jesu Redémptor ómnium}, page \pageref{hymn-jesuredemptoromnium}.\\}
 }
\def\definevesperspropersalt{\newcommand{\vrtex}{vrVerbumCaro}
\newcommand{\vtranslation}{The Word was made flesh, alleluia.}
\newcommand{\rtranslation}{And dwelt amongst us, alleluia.}

\newcommand{\maganttex}{an--propter_nimiam--solesmes}
\newcommand{\magantinitial}{P}
\newcommand{\maganttranslation}{For His exceeding charity wherewith He loved us, God sent His own Son, in the likeness of sinful flesh, alleluia.}
\def\magsolemn{T}
\definemag{8}{G}
}
\def\vesperspropersnote{At II Vespers:}
\def\vesperspropersaltnote{At I Vespers:}
\def\prechapter{\printnote{For the Feast of the Purification \& Presentation, continue with the chapter on page \pageref{purification-chapter}.}}
\printvespers[../ChristmasOctave-Circumcision]{inc-Circumcision-Vespers-common}

\benedicamusdomino{}
}

{
\needspace{5\baselineskip}
\section{The Most Holy Name of Jesus}
\subtitle{\nth{2} Class}

\printnote{This feast is celebrated on the Sunday between the Octave Day of the Nativity and the Epiphany of Our Lord.  If no Sunday occurs within that time, this feast is then celebrated on January 2.}

\printnote{When the Feast of the Most Holy Name of Jesus occurs on January 5, the First Vespers of the Epiphany on page \pageref{epiphany} are sung without a commemoration of the Holy Name of Jesus.}

\smallskip{}
%\def\definevesperspropersalt{\newcommand{\maganttex}{MagnificatAntiphon1}
\newcommand{\magantinitial}{T}
\newcommand{\maganttranslation}{Thou art the Shepherd of the sheep and the Prince of the Apostles, and unto thee are given the keys of the kingdom of heaven.}
\def\magsolemn{T}
\definemag{1}{f}
}
\def\definevesperspropers{\newcommand{\maganttex}{an--vocabis_nomen_ejus--solesmes}
\newcommand{\magantinitial}{V}
\newcommand{\maganttranslation}{Thou shalt call His Name Jesus, for He shall save His people from their sins, alleluia.}
\def\magsolemn{T}
\definemag{1}{g}
}
%\def\vesperspropersaltnote{At I Vespers of the Most Holy Name of Jesus:}
%\def\vesperspropersnote{At II Vespers of the Most Holy Name of Jesus:}
\def\hymnlabel{hymn-jesudulcismemoria}

\printvespers[../HolyName]{inc-HolyName-common}

\benedicamusdomino[2]{}
}

{
\chapter{Proper of the Time -- Epiphany Season}
\label{epiphany}
\section{January 6: The Epiphany of Our Lord}
\subtitle{\nth{1} Class}

\subtitle{I \& II Vespers}

\def\definevesperspropersalt{\definepsalm{5}{116}{7}{c2}

\newcommand{\maganttex}{MagnificatAntiphon1}
\newcommand{\magantinitial}{M}
\newcommand{\maganttranslation}{When the Wise Men saw the star, they said one to another: This is the sign of the great King: let us go and search for Him, and offer Him gifts, gold, frankincense, and myrrh.}
\definemag{8}{G}
}
\def\definevesperspropers{\definepsalm{5}{113}{7}{c2}

\newcommand{\maganttex}{MagnificatAntiphon2}
\newcommand{\magantinitial}{T}
\newcommand{\maganttranslation}{This day we keep a holiday in honour of three wonders: today a star led the wise men to the manger; today water was made wine at the marriage; today Christ was pleased to be baptised in the Jordan by John for our salvation, alleluia.}
\definemag{1}{D}
}
\def\vesperspropersaltnote{At I Vespers:}
\def\vesperspropersnote{At II Vespers:}

\printvespers[../Epiphany]{inc-Epiphany-Vespers}

\benedicamusdomino{}
}

{
\section{Holy Family of Jesus, Mary \& Joseph}
\subtitle{\nth{2} Class}

\subtitle{First Sunday after the Epiphany.}

\printvespers[../HolyFamily]{inc-HolyFamily-Vespers2}

\benedicamusdomino[2]{}
}

}