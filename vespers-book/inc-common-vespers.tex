\chapter{Common of Festal Vespers}

%\subtitle{Festal Tone}
\label{deusinadjutorium}
\sectionmark{Deus in adjutórium}
\addcontentsline{toc}{section}{Deus in adjutórium}
\printnote{The Festal Tone may be sung on any Sunday or Feast.}
\printnote{From Septuagesima to Wednesday in Holy Week, the \emph{Laus tibi} is said instead of \emph{Allelúia}.}

\bigskip
\def\deusinadjutoriumsolemn{F}
\printdeusinadjutorium{}
\vfil

\pagebreak
%\subtitle{Solemn Tone}
\printnote{The Solemn Tone may only be sung on Feasts of the First or Second Class.}
\printnote{From Septuagesima to Wednesday in Holy Week, the \emph{Laus tibi} is said instead of \emph{Allelúia}.}

\bigskip
\def\deusinadjutoriumsolemn{T}
\label{deusinadjutoriumsolemn}
\printdeusinadjutorium{}
\let\deusinadjutoriumsolemn=\undefined

\printnote{Vespers then proceed with the Proper Antiphons, Psalms, Chapter, Hymn, Versicle,
Magnificat, and Collect given for the respective Sunday or Feast.

For Sundays throughout the year, \emph{p.~\pageref{sundayvespers}}.

For Sundays in Paschaltide, \emph{p.~\pageref{sundayvespers-easter}}.}

\bigskip

\newcommand{\printvrwithtranslation}{
    {\normalsize
    \ifx\beginvrcols\undefined\def\beginvrcols{\begin{parcolumns}[rulebetween]{2}}\fi
    \beginvrcols
    \colchunk{
      \par\vspace{-\baselineskip}\noindent\selectlanguage{latin}%
      \Vbar{}~\vlatin{}
    }
    \colchunk{%
      \par\vspace{-\baselineskip}\noindent\selectlanguage{american}%
      \Vbar{}~\vtranslation{}
    }%
    \colplacechunks%
    \ifx\rlatin\undefined\else
    \colchunk{
      \par\vspace{-\baselineskip}\noindent\selectlanguage{latin}%
      \Rbar{}~\rlatin{}%
    }
    \colchunk{%
      \par\vspace{-\baselineskip}\noindent\selectlanguage{american}%
      \Rbar{}~\rtranslation{}%
    }%
    \colplacechunks%
    \fi
    \end{parcolumns}
    }
}

{
  \printnote{All Vespers conclude with the following:}\par
  \def\beginvrcols{\begin{parcolumns}[rulebetween,colwidths={1=0.48\linewidth}]{2}}
  \def\vlatin{Dóminus vobíscum.}
  \def\rlatin{Et cum spíritu tuo.}
  \def\vtranslation{The Lord be with you.}
  \def\rtranslation{And with thy spirit.}
  \printvrwithtranslation{}

  %\Vbar{}~Dóminus vobíscum.\\%\hspace{5em}
  %\Rbar{}~Et cum spíritu tuo.

  \printnote{Or, in the absence of a priest or deacon:}\par
  \def\vlatin{Dómine, exáudi oratiónem meam.}
  \def\rlatin{Et clamor meus ad te véniat.}
  \def\vtranslation{Lord, hear my prayer.}
  \def\rtranslation{And let my cry come unto Thee.}
  \printvrwithtranslation{}

  %\Vbar{}~Dómine, exáudi oratiónem meam.\\
  %\Rbar{}~Et clamor meus ad te véniat.

  \bigskip

  \noindent\printnote{The ``Benedicámus Dómino'' is then sung in one of the tones on the next pages.}

  \bigskip

  \let\rlatin\undefined
  \let\rtranslation\undefined
  \def\vlatin{Fidélium ánimæ per misericóridam Dei requiéscant in pace.\\\Rbar~Amen.}
  %\def\rlatin{Amen.}
  \def\vtranslation{May the souls of the faithful departed through the mercy of God rest in peace.\hfill\Rbar~Amen.}
  %\def\rtranslation{Amen.}
  \printvrwithtranslation{}
}
%\noindent\Vbar{}~Fidélium ánimæ per misericóridam Dei requiéscant in pace.\\
%\Rbar{}~Amen.

\def\noeuouae{T}
\def\dontrepeatantiphon{T}
