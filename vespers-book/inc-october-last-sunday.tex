{
\section{Last Sunday in October: Jesus Christ, King}
\subtitle{\nth{1} Class, White or Gold}
\subtitle{II Vespers}
\medskip

\def\deusinadjutoriumsolemn{T}
\ifthenelse{\boolean{birmingham}}{
	\def\prepsalmtitleone{\vspace{-0.5\baselineskip}}
	\def\prepsalmtitletwo{\vspace{-0.5\baselineskip}}
	\def\prerepeatantiphontwo{}
	\def\preantthree{\vspace{-0.2\baselineskip}}
	\def\prepsalmtitlethree{\vspace{-0.1\baselineskip}}
}{
	\def\postpsalmtitleone{\needspace{8\baselineskip}}
	\def\prepsalmoneverses{}
	\def\prerepeatantiphonone{}
	\def\preanttwo{\vspace{-0.5\baselineskip}}
	\def\preanttranslationtwo{\vspace{-0.5\baselineskip}}
	\def\prepsalmtitletwo{\vspace{-0.5\baselineskip}}
	\def\postpsalmtitletwo{\needspace{12\baselineskip}}
	\def\prerepeatantiphontwo{}
}
\def\prehymn{\vfill}
\def\definevesperspropers{\newcommand{\vrtex}{vr}
\newcommand{\vtranslation}{His dominion shall be increased.}
\newcommand{\rtranslation}{And of peace there shall be no end.}

\newcommand{\magantinitial}{H}
\newcommand{\maganttex}{MagnificatAntiphon}
\newcommand{\maganttranslation}{He hath on His garment and on His thigh written: King of kings and Lord of lords.  To Him be glory and empire for ever and ever.}
\def\magsolemn{T}
\definemag{7}{a}
}
\def\beginchaptercols{\begin{parcolumns}[rulebetween,colwidths={1=0.46\linewidth}]{2}}
%\def\begincollectcols{\begin{parcolumns}[rulebetween,colwidths={1=0.45\linewidth}]{2}}

\printvespers[../OctoberLastSunday-ChristTheKing]{inc-ChristTheKing}
\noindent
\printnote{If today is October 31, \emph{First Vespers of All Saints} is commemorated with \emph{Magnificat antiphon}, p.~\pageref{allsaints1-magnificat}; \emph{\Vbar{}~Lætámini}.~in simple commemoration tone, p.~\pageref{allsaints1-vr}; and \emph{Collect}, p.~\pageref{allsaints-collect}.}

\bigskip
\benedicamusdomino[1]{}
}