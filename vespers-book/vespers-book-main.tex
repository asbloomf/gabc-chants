% !TEX TS-program = lualatex
% !TEX encoding = UTF-8

% This is a simple template for a LuaLaTeX document using gregorio scores.

% easter can be from march 22 to april 25

\documentclass[letterpaper,12pt]{book} % use larger type; default would be 10pt
\usepackage{../definepsalms}
\usepackage{titlesec}
\usepackage{titletoc}
\usepackage{titleps}
\usepackage{letltxmacro}
\usepackage{changepage} % gives us \ifoddpage use [strict]
\usepackage[super]{nth}
\usepackage[savepos]{zref}
\usepackage{xparse}
\usepackage{setspace}
\LetLtxMacro{\oldnth}{\nth}
\renewcommand{\nth}[1]{\addfontfeature{Numbers=Lining}\oldnth{#1}}
\LetLtxMacro{\oldneedspace}{\needspace}
\renewcommand{\needspace}[1]{
	\checkoddpage\ifoddpage\oldneedspace{#1}\else\fi
}
\let\gredagger=\dag
\newcommand*\cleartoleftpage{%
  \clearpage
  \ifodd\value{page}\hbox{}\newpage\fi
}

%\usepackage{hyperref}
\newcommand{\phantomsection}{}
\newcommand{\magnote}{
\IfStrEq{\anttone\nostarend}{8G}{\noindent\emph{A 3 part arrangement of the even verses by Grassi may be found on page \pageref{magnificat-grassi}.\\}}{}
}
\presetkeys{mymag}{note=\magnote}{}
%\LetLtxMacro{\oldprintmag}{\printmag}
%\renewcommand{\printmag}[4][grassilabel=magnificat-grassi]{
%\oldprintmag[#1]{#2}{#3}{#4}
%}

\setcounter{secnumdepth}{-1}

\def\mywidth{6in}
\def\myheight{9in}

\usepackage{../definepsalms}
\usepackage{adjustbox}
\sloppy
\usepackage{../definepsalms}
\usepackage{adjustbox}
\sloppy
\usepackage{../definepsalms}
\usepackage{adjustbox}
\sloppy
\input{../inc_header} %


\let\oldVbar=\Vbar
\def\Vbar{\oldVbar\hspace{-2pt}}
\let\oldRbar=\Rbar
\def\Rbar{\oldRbar\hspace{-2pt}}

\ifbook{
\lfoot[\thepage]{}
\rfoot[]{\thepage}
}
\ifnotbook{
\cfoot{%
  \ifnum\thepage>1
    \thepage
  \fi
}
}
%\renewcommand\headrulewidth{\oldheadrulewidth}
\chead{
  \ifnum\thepage>1
    \addfontfeature{Numbers=Lining}%
    \heading{} \ifx\matinsnocturn\undefined\else(\matinsnocturn)\fi
  \fi
}

\newcommand{\writeheading}[1]{
  \begin{center}{
  \addfontfeature{Numbers=Lining}
  \textsc{#1}
  }\end{center}
  \medskip
}
\newcommand{\printseparation}{
  %\hfil\rule{3in}{0.5pt}\hfil
  \bigskip
  \bigskip
}

\def\nogloriapatri{T}
 %


\let\oldVbar=\Vbar
\def\Vbar{\oldVbar\hspace{-2pt}}
\let\oldRbar=\Rbar
\def\Rbar{\oldRbar\hspace{-2pt}}

\ifbook{
\lfoot[\thepage]{}
\rfoot[]{\thepage}
}
\ifnotbook{
\cfoot{%
  \ifnum\thepage>1
    \thepage
  \fi
}
}
%\renewcommand\headrulewidth{\oldheadrulewidth}
\chead{
  \ifnum\thepage>1
    \addfontfeature{Numbers=Lining}%
    \heading{} \ifx\matinsnocturn\undefined\else(\matinsnocturn)\fi
  \fi
}

\newcommand{\writeheading}[1]{
  \begin{center}{
  \addfontfeature{Numbers=Lining}
  \textsc{#1}
  }\end{center}
  \medskip
}
\newcommand{\printseparation}{
  %\hfil\rule{3in}{0.5pt}\hfil
  \bigskip
  \bigskip
}

\def\nogloriapatri{T}
 %


\let\oldVbar=\Vbar
\def\Vbar{\oldVbar\hspace{-2pt}}
\let\oldRbar=\Rbar
\def\Rbar{\oldRbar\hspace{-2pt}}

\ifbook{
\lfoot[\thepage]{}
\rfoot[]{\thepage}
}
\ifnotbook{
\cfoot{%
  \ifnum\thepage>1
    \thepage
  \fi
}
}
%\renewcommand\headrulewidth{\oldheadrulewidth}
\chead{
  \ifnum\thepage>1
    \addfontfeature{Numbers=Lining}%
    \heading{} \ifx\matinsnocturn\undefined\else(\matinsnocturn)\fi
  \fi
}

\newcommand{\writeheading}[1]{
  \begin{center}{
  \addfontfeature{Numbers=Lining}
  \textsc{#1}
  }\end{center}
  \medskip
}
\newcommand{\printseparation}{
  %\hfil\rule{3in}{0.5pt}\hfil
  \bigskip
  \bigskip
}

\def\nogloriapatri{T}

\setlength\headheight{0.25in+15pt}
\setlength\headsep{1pc}
\setlength\topskip{0pc}
\setlength\footskip{1pc}
\geometry{outer=0.4in,inner=0.85in,top=0pc+\headheight+\headsep,bottom=0.4in,twoside=true}
\newpagestyle{main}{
\sethead[\garamond{\thepage}][\garamond{\chaptertitle}][] % even
{}{\garamond{\sectiontitle}}{\garamond{\thepage}} % odd
\setfoot[][][] % even
{}{}{} % odd
}
\pagestyle{main}

%\grechangeglyph{Porrectus*}{*}{.alt}
%\grechangeglyph{TorculusResupinus*}{*}{.alt}

\titleformat
{\section} % command
[block] % shape
{\phantomsection\large\addfontfeature{Numbers=Lining}} % format
{} % label
{} % sep
{
    % \rule{\textwidth}{1pt}
    % \vspace{1ex}
    \centering
} % before-code
%[
% \vspace{-0.5ex}%
% \rule{\textwidth}{0.3pt}
%] % after-code
 
 
\titleformat{\chapter}[block]
{\thispagestyle{empty}\phantomsection\Large\scshape\addfontfeature{Numbers=Lining}}
{}{0.5em}{\centering}
 
\titlespacing{\chapter}{0pt}{6pt-\headheight}{1pc}
\titlespacing{\section}{0pt}{*2.5}{*1}
\titleclass{\chapter}{top}
\newcommand{\chapterbreak}{\clearpage}
%\titleclass{\section}{top}

\contentsmargin{2pc}
\dottedcontents{chapter}[2.3em]{}{2.3em}{1pc}
\dottedcontents{section}[5.5em]{}{3.2em}{1pc}

\newcommand{\printnote}[1]{
	{\normalsize \emph{#1}}
}
\newcommand{\subtitle}[1]{
\begin{center}{
	{\addfontfeature{Numbers=Lining} \normalsize \emph{#1}}
}\end{center}
}

\newcommand{\deusinadjutorium}{\noindent\printnote{\Vbar~\emph{Deus in adjutórium}, page \pageref{deusinadjutorium}.}}
\newcommand{\printcollect}[2]{
	\ifx\undefined\begincollectcols\def\begincollectcols{\begin{parcolumns}[rulebetween]{2}}\fi
	\ifx\printcollectheading\undefined\def\printcollectheading{T}\fi
	\if\printcollectheading T
	\oldneedspace{3\baselineskip}
	\begin{center}{\large Collect.}\end{center}
	\vspace{-0.4\baselineskip}
	\fi
	\begincollectcols
	\sloppy
	\prayer{#1}{#2}
	\end{parcolumns}
	\let\begincollectcols=\undefined
}
\newcommand{\benedicamusdominoreference}[1]{%
	\Vbar~\emph{Benedicámus Dómino \IfInteger{#1}{#1}{\csname benedicamusdominoname#1\endcsname}}, page \pageref{benedicamusdomino-#1}.
}
\newcommand{\benedicamusdominomaster}[1]{
	\noindent\printnote{\benedicamusdominoreference{#1}}
	\bigskip
	\hrule
}
\newcommand{\psalmcolsoverride}[1][0]{
\def\beginpsalmcols{\begin{parcolumns}[rulebetween,colwidths={1=0.45\linewidth}]{2}}
\ifnum#1=110
\def\beginpsalmcols{\begin{parcolumns}[rulebetween,colwidths={1=0.45\linewidth}]{2}}
\fi
\ifnum#1=111
\def\beginpsalmcols{\begin{parcolumns}[rulebetween,colwidths={1=0.475\linewidth}]{2}}
\fi
%\ifnum#1=112
%\def\beginpsalmcols{\begin{parcolumns}[rulebetween,colwidths={1=0.445\linewidth}]{2}}
%\fi
\ifnum#1=116
\def\beginpsalmcols{\begin{parcolumns}[rulebetween,colwidths={1=0.565\linewidth}]{2}}
\fi
\ifnum#1=121
\def\beginpsalmcols{\begin{parcolumns}[rulebetween,colwidths={1=0.46\linewidth}]{2}}
\fi
\ifnum#1=129
\def\beginpsalmcols{\begin{parcolumns}[rulebetween,colwidths={1=0.475\linewidth}]{2}}
\fi
\ifnum#1=131
\def\beginpsalmcols{\begin{parcolumns}[rulebetween,colwidths={1=0.475\linewidth}]{2}}
\fi
\ifnum#1=137
\def\beginpsalmcols{\begin{parcolumns}[rulebetween]{2}}
\fi
\ifnum#1=147
\def\beginpsalmcols{\begin{parcolumns}[rulebetween,colwidths={1=0.465\linewidth}]{2}}
\fi
}
\newcommand{\printvrcommem}{
		{\normalsize
		\ifx\beginvrcols\undefined\def\beginvrcols{\begin{parcolumns}[rulebetween]{2}}\fi
		\beginvrcols
		\colchunk{%
			\selectlanguage{latin}%
	    \Vbar{}~\commvlatin{}%
    }
		\colchunk{%
			\selectlanguage{american}%
	    \Vbar{}~\commvtranslation{}%
		}%
		\colplacechunks%
    \colchunk{%
			\selectlanguage{latin}%
	    \Rbar{}~\commrlatin{}%
		}%
		\colchunk{%
			\selectlanguage{american}%
	    \Rbar{}~\commrtranslation{}%
		}
		\end{parcolumns}
		}
}
\newcommand{\printvrdirigatur}{
	\smallskip{}
	\oldneedspace{3\baselineskip}
	\noindent\printnote{If the Sunday is being commemorated:\\}\vspace{-0.5\baselineskip}
	{
		\def\beginvrcols{\begin{parcolumns}[rulebetween,colwidths={1=0.51\linewidth}]{2}}
		\def\commvlatin{Dirigátur Dómine orátio \textbf{mé}a.}
		\def\commrlatin{Sicut incénsum in conspéctu \textbf{tú}o.}
		\def\commvtranslation{\noindent{}Let my prayer be directed, O Lord.}
		\def\commrtranslation{\noindent{}As incense in Thy sight.}
		\printvrcommem{}
	}
}
\newcommand{\printvrmanenobiscum}{
	\smallskip{}
	\oldneedspace{3\baselineskip}
	\noindent\printnote{If the Sunday is being commemorated:\\}\vspace{-0.5\baselineskip}
	{
		\def\commvlatin{Mane nobíscum Dómine, alle\textbf{lú}ia.}
		\def\commrlatin{Quóniam advesperáscit, alle\textbf{lú}ia.}
		\def\commvtranslation{Stay with us O Lord, alleluia.}
		\def\commrtranslation{Because it is towards evening, alleluia.}
		\printvrcommem{}
	}
}
\newcommand{\printcommemoration}[2][.]{
	\def\gabcfolder{#1}
	\input{\gabcfolder/#2}
	\subtitle{\comheadingtext}
	\sectionmark{\comheadingtext}
	{
	\def\noeuouae{T}
	\printgabc{At Magn.}{\oldstylenums{\commagantlinetwo}}{\commagantinitial}{\commaganttex}
	}
	\translation[]{\englishcommagantiphon}

	\smallskip
	\ifx\commvlatin\undefined
		\printvr[\greseteolcustos{manual}]{\commvrtex}{\commvtranslation}{\commrtranslation}
	\else
	{
		\printvrcommem{}
	}
	\fi

	\printcollect{\latincomcollect}{\englishcomcollect}
}
\DeclareDocumentCommand{\printbothversions}{ O{\undefined} O{\undefined} m m }{
% #1 & #2 are the label
% #3 is what to check for
% #4 is the body of what to print
	\ifx#3\undefined
		\ifx\definevesperspropersalt\undefined\else
	  {
			\ifx\vesperspropersaltnote\undefined\else
		    \oldneedspace{3\baselineskip}
				\printnote{\vesperspropersaltnote}
			\fi
			\definevesperspropersalt
			\ifx#2\undefined\else\label{#2}\fi
	  	#4
		}
		\medskip
		\fi
		\ifx\definevesperspropers\undefined\else
	  {
			\ifx\vesperspropersnote\undefined\else
	    	\oldneedspace{3\baselineskip}
				\printnote{\vesperspropersnote}
			\fi
			\definevesperspropers
			\ifx#1\undefined\else\label{#1}\fi
	  	#4
		}
		\fi
	\else
  {
		\ifx#1\undefined\else\label{#1}\fi
  	#4
	}
	\fi
}
\newcommand{\printvespersmag}[2][.]{
	\grechangestaffsize{15}
	\def\gabcfolder{#1}
	\input{\gabcfolder/#2}
	\ifx\prevespers\undefined\else\prevespers\fi

	\ifx\chaptertext\undefined\else{
		\oldneedspace{3\baselineskip}
		\printchapter{\chaptertext}{\chaptertranslation}
		\smallskip
	}\fi

	\ifx\printfullhymn\undefined
		\ifx\printhymnnote\undefined
			%print the hymn
			\ifx\hymninput\undefined\else
				\input{\hymninput}
			\fi
			\printbothversions[\hymnlabel][\hymnaltlabel]{\hymntex}{\printhymn{\oldstylenums{\hymnlinetwo}}{\hymninitial}{\hymntex}{\hymntranslation}}
			\printbothversions[\vrlabel][\vraltlabel]{\vrtex}{%
		    \ifx\vrlinebreak\undefined\def\vrlinebreak{T}\fi
		    \printvr[\greseteolcustos{manual}]{\vrtex}{\vtranslation}{\rtranslation}
			}
		\else
			\printhymnnote
		\fi
	\else
		\printfullhymn
	\fi

	{
		\ifx\premagnificat\undefined\else\premagnificat\fi
		\let\anttranslation=\englishmagantiphon
		\let\preverses=\premagverses
	    \let\preanttwo=\premagtwo
	    \let\preant=\premag
	  \oldneedspace{3\baselineskip}
		\printmag{\magtone\magend}{\magantinitial}{\maganttex}
	}
	\ifx\precollect\undefined\vspace{-1\baselineskip}\else\precollect\fi
	\oldneedspace{3\baselineskip}
	\printcollect{\latincollect}{\englishcollect}
}
\newcommand{\printpsalms}{
	\printpsalm{1}{\psalmonenum}{\psalmonetone\psalmoneend}{\antonetex}{\antoneinitial}
	\medskip
	\needspace{3\baselineskip}
	\printpsalm{2}{\psalmtwonum}{\psalmtwotone\psalmtwoend}{\anttwotex}{\anttwoinitial}
	\medskip
	\needspace{3\baselineskip}
	\printpsalm{3}{\psalmthreenum}{\psalmthreetone\psalmthreeend}{\antthreetex}{\antthreeinitial}
	\medskip
	\needspace{3\baselineskip}
	\printpsalm{4}{\psalmfournum}{\psalmfourtone\psalmfourend}{\antfourtex}{\antfourinitial}
	\medskip
	\ifx\psalmfivenum\undefined{
		\ifx\antfivetex\undefined{%then it is a different antiphon for each
			\ifx\definevesperspropersalt\undefined\else{
				\definevesperspropersalt
				\ifx\vesperspropersaltnote\undefined\else
					\oldneedspace{3\baselineskip}
					\noindent\printnote{\vesperspropersaltnote}
				\fi
				\printpsalm{5}{\psalmfivenum}{\psalmfivetone\psalmfiveend}{\antfivetex}{\antfiveinitial}
			}\fi
			\ifx\definevesperspropers\undefined\else{
				\definevesperspropers
				\ifx\vesperspropersnote\undefined\else
					\oldneedspace{3\baselineskip}
					\noindent\printnote{\vesperspropersnote}
				\fi
				\printpsalm{5}{\psalmfivenum}{\psalmfivetone\psalmfiveend}{\antfivetex}{\antfiveinitial}
			}\fi
		}\else{%they share the same antiphon
			\ifx\definevesperspropersalt\undefined\else{
				\definevesperspropersalt
				\ifx\vesperspropersaltnote\undefined\else
					\def\prepsalmtitle{
						\smallskip
						\oldneedspace{3\baselineskip}
						\noindent\printnote{\vesperspropersaltnote}\vspace{-1\baselineskip}

					}
				\fi
				\printpsalm{5}{\psalmfivenum}{\psalmfivetone\psalmfiveend}{\antfivetex}{\antfiveinitial}
			}\fi
			\def\onlyoneant{T}
			\ifx\definevesperspropers\undefined\else{
				\definevesperspropers
				\ifx\vesperspropersnote\undefined\else
					\oldneedspace{5\baselineskip}
					\noindent\printnote{\vesperspropersnote}\vspace{-1\baselineskip}
				\fi
				\def\prepsalmtitle{\def\onlyoneant{F}}
				\printpsalm{5}{\psalmfivenum}{\psalmfivetone\psalmfiveend}{\antfivetex}{\antfiveinitial}
			}\fi
		}\fi
	}\else{
		\printpsalm{5}{\psalmfivenum}{\psalmfivetone\psalmfiveend}{\antfivetex}{\antfiveinitial}
	}\fi
}
\newcommand{\printvespers}[2][.]{
	\grechangestaffsize{15}
	\def\gabcfolder{#1}
	\input{\gabcfolder/#2}
	\ifx\prevespers\undefined\else%
	\prevespers{}
	\fi

	\deusinadjutorium{}
	\medskip
	\ifx\psalmonenum\undefined{
		\ifx\definevesperspropersalt\undefined\else{
			\definevesperspropersalt
			\ifx\prevesperspsalmsalt\undefined\else\prevesperspsalmsalt{}\fi
			\ifx\vesperspropersaltnote\undefined\else
				\oldneedspace{3\baselineskip}
				\noindent\printnote{\vesperspropersaltnote}
			\fi
			\ifx\vesperspsalmsaltlabel\undefined\else\vesperspsalmsaltlabel{}\fi
			\printpsalms{}
		}\fi
		\ifx\definevesperspropers\undefined\else{
			\definevesperspropers
			\ifx\prevesperspsalms\undefined\else\prevesperspsalms{}\fi
			\ifx\vesperspropersnote\undefined\else
				\oldneedspace{3\baselineskip}
				\noindent\printnote{\vesperspropersnote}
			\fi
			\ifx\vesperspsalmslabel\undefined\else\vesperspsalmslabel{}\fi
			\printpsalms{}
		}\fi
	}\else
		\printpsalms{}
	\fi
	%\medskip
	%\vspace{-\baselineskip}
	\needspace{3\baselineskip}
	\ifx\chapterreplacement\undefined{
		\ifx\chaptertext\undefined
			\ifx\prechapter\undefined\else\prechapter\fi
			\def\printchapterheading{F}
	    \begin{center}{\large\textbf Chapter.}\end{center}
	    \vspace{-0.3\baselineskip}
			\ifx\definevesperspropersalt\undefined\else
			{
				\ifx\vesperspropersaltnote\undefined\else
					\oldneedspace{\baselineskip}
					\printnote{\vesperspropersaltnote}
				\fi
				\definevesperspropersalt
				\printchapter{\chaptertext}{\chaptertranslation}
			}
			\medskip
			\fi
			\ifx\definevesperspropers\undefined\else
			{
				\ifx\vesperspropersnote\undefined\else
					\oldneedspace{3\baselineskip}
					\printnote{\vesperspropersnote}
				\fi
				\definevesperspropers
				\printchapter{\chaptertext}{\chaptertranslation}
			}
			\fi
		\else
			%\vspace{-\baselineskip}
			\oldneedspace{3\baselineskip}
			\ifx\prechapter\undefined\else\prechapter\fi
			\printchapter{\chaptertext}{\chaptertranslation}
		\fi
	}
	\else
	{\chapterreplacement}
	\fi

	\medskip
	\ifx\printfullhymn\undefined
		\ifx\prehymn\undefined\else\prehymn\fi
		\ifx\printhymnnote\undefined
			\ifx\hymninput\undefined\else
				\input{\hymninput}
			\fi
			\printbothversions[\hymnlabel][\hymnaltlabel]{\hymntex}{\printhymn{\oldstylenums{\hymnlinetwo}}{\hymninitial}{\hymntex}{\hymntranslation}}
		\else
			\printhymnnote
		\fi
		\printbothversions[\vrlabel][\vraltlabel]{\vrtex}{%
	    \ifx\vrlinebreak\undefined\def\vrlinebreak{T}\fi
	    \printvr[\if\vrlinebreak T\greseteolcustos{manual}\fi]{\vrtex}{\vtranslation}{\rtranslation}
		}
		\ifx\vrpttex\undefined\else%
			\smallskip
			\noindent
			\printnote{In Paschaltide:}
			\ifx\vrlinebreak\undefined\def\vrlinebreak{T}\fi
			\if\vrlinebreak T\greseteolcustos{manual}\fi
			\gregorioscore{\gabcfolder/\vrpttex}
		\fi
	\else
		\printfullhymn
	\fi

	\medskip

	\ifx\magreplacement\undefined
	{
		\ifx\maganttex\undefined
		\ifx\definevesperspropersalt\undefined\else{
			{
			\ifx\definevesperspropers\undefined\else
				{
					\definevesperspropersalt
					\global\let\magtonealt=\magtone
					\global\let\magendalt=\magend
				}
				{
					\definevesperspropers
					\IfStrEq{\magtonealt\magendalt}{\magtone\magend}
					{\gdef\altmagsame{T}}
					{\gdef\altmagsame{F}}

				}
			\fi
			}
			\definevesperspropersalt
  		\hrule
  		\needspace{5\baselineskip}
	    \begin{center}{\large Magnificat.}\end{center}
    	\vspace{-1ex}

    	\ifx\vesperspropersaltnote\undefined\else
    		\oldneedspace{7\baselineskip}
				\printnote{\vesperspropersaltnote\\\vspace{-0.5\baselineskip}}
			\fi
			\let\anttranslation=\maganttranslation
			\let\preverses=\premagverses
		    \let\preanttwo=\premagtwo
		    \let\preant=\premag
		    \def\nomagtitle{T}
		    \if\altmagsame T{
		    	\printgabc{At Magn.}{Ant.~\magtone{}.~\magend}{\magantinitial}{\maganttex}
		    	\translation[]{\anttranslation}
		    	\bigskip
		    }\else{
				\printmag{\magtone\magend}{\magantinitial}{\maganttex}
			}\fi
	    \hrule
	    \medskip
		}\fi
		\ifx\definevesperspropers\undefined\else
			\definevesperspropers
	    	\ifx\vesperspropersnote\undefined\else
	    		\needspace{6\baselineskip}
	    		\oldneedspace{4\baselineskip}
				\printnote{\vesperspropersnote\\\vspace{-0.5\baselineskip}}
			\fi
			\def\nomagtitle{T}
		\fi
		\fi
		\let\anttranslation=\maganttranslation
		\let\preverses=\premagverses
	    \let\preanttwo=\premagtwo
	    \let\preant=\premag
		\printmag{\magtone\magend}{\magantinitial}{\maganttex}
		\global\let\altmagsame\undefined
		\global\let\magtonealt=\undefined
		\global\let\magendalt=\undefined
	}
	\else
	{\magreplacement}
	\fi
	{\ifx\postmag\undefined\else\postmag\fi}

	\vspace{-\baselineskip}
	\ifx\precollect\undefined\else\precollect\fi
	\ifx\collectreplacement\undefined{
		\ifx\collect\undefined
			\def\printcollectheading{F}
			\needspace{3\baselineskip}
			\ifx\precollect\undefined\else\precollect\fi
			\begin{center}{\large Collect.}\end{center}

			\vspace{-0.5\baselineskip plus 0.25\baselineskip}
			\ifx\definevesperspropersalt\undefined\else
			{
				\ifx\vesperspropersaltnote\undefined\else
					\printnote{\vesperspropersaltnote}
				\fi
				\definevesperspropersalt
				\printcollect{\collect}{\collecttranslation}
			}
			\medskip
			\fi
			\ifx\definevesperspropers\undefined\else
			{
				\ifx\vesperspropersnote\undefined\else
					\printnote{\vesperspropersnote}
				\fi
				\definevesperspropers
				\printcollect{\collect}{\collecttranslation}
			}
			\fi
		\else
			\oldneedspace{3\baselineskip}
			\needspace{5\baselineskip}
			\ifx\precollect\undefined\else\precollect\fi
			\printcollect{\collect}{\collecttranslation}
		\fi
	}
	\else
	{\collectreplacement}
	\fi
	\medskip
}

\sloppy
\begin{document}
\normalsize
\grechangestaffsize{15}
{
	\thispagestyle{empty}
	% title page
	{\vspace*{5\baselineskip}

	\centering

	{\Huge
	Vespers with Gregorian Chant

	}

	{\Large\medskip
	\emph{for}

	\medskip}

	{\Huge
	Sundays \& Holy Days

	}
	\vfill
	\pagebreak
	}
	% copyright page
	\thispagestyle{empty}

	Vespers with Gregorian Chant for Sundays \& Holy Days: \emph{newly tyseset, based on \emph{The Liber Usualis}, edited by the Benedictines of Solemnes (Desclee Company, 1961).}

	\vfill

	Many thanks to all the people answering questions on the GABC forums, especially {names, Bro. \& Bro.}, Fr. {name} for setting up the gregobase, which makes it much easier to find all the chant that many people, but especially Andrew Hinkley, have contributed to the public domain.  Also many thanks to my brother Benjamin, whose website makes it incredibly easy to typeset the psalms and considerably easier than otherwise to typeset Gregorian chant.

	\vfill

	This work is free of known copyright.
}
%\frontmatter
%\pagenumbering{roman}

\tableofcontents

%\pagenumbering{arabic}
%\mainmatter

\chapter{Common of Festal Vespers}

%\subtitle{Festal Tone}
\label{deusinadjutorium}
\sectionmark{Deus in adjutórium}
\addcontentsline{toc}{section}{Deus in adjutórium}
\printnote{The Festal Tone may be sung on any Sunday or Feast.}
\printnote{From Septuagesima to Wednesday in Holy Week, the \emph{Laus tibi} is said instead of \emph{Allelúia}.}

\bigskip
\def\deusinadjutoriumsolemn{F}
\printdeusinadjutorium{}
\vfil

\pagebreak
%\subtitle{Solemn Tone}
\printnote{The Solemn Tone may only be sung on Feasts of the First or Second Class.}
\printnote{From Septuagesima to Wednesday in Holy Week, the \emph{Laus tibi} is said instead of \emph{Allelúia}.}

\bigskip
\def\deusinadjutoriumsolemn{T}
\printdeusinadjutorium{}


\printnote{Vespers then proceed with the Proper Antiphons, Psalms, Chapter, Hymn, Versicle,
Magnificat, and Collect given for the respective Sunday or Feast.}

\bigskip

\printnote{All Vespers conclude with the following:}
\\
\Vbar{} Dóminus vobíscum.\\%\hspace{5em}
\Rbar{} Et cum spíritu tuo.

\printnote{Or, in the absence of a priest or deacon:}
\\
\Vbar{} Dómine, exáudi oratiónem meam.\\
\Rbar{} Et clamor meus ad te véniat.

\bigskip

\def\noeuouae{T}
\def\dontrepeatantiphon{T}
%\vfil
{
\newcommand{\printbenedicamusdomino}[2]{
	\greseteolcustos{manual}
	\gresetinitiallines{1}
	\def\annot{\small{#1}}
	\alsetinitialspacing{B}
	\gregorioscore{#2}
  \greseteolcustos{auto}
}

\vfil
\grechangestaffsize{15}
\oldneedspace{4\baselineskip}\label{benedicamusdomino-1}
{{\centering \bfseries 1.~On feasts of the I class.\\}
\smallskip
\def\breakbeforeresp{T}
\printbenedicamusdomino{5.}{../BenedicamusDomino/or--benedicamus_(in_festis_i_classis_ad_laudes)--solesmes}
\vfil
}
\oldneedspace{4\baselineskip}\label{benedicamusdomino-2}
{{\centering \bfseries 2.~On feasts of the II class.\\}
\smallskip
{\def\breakbeforeresp{T}
\printbenedicamusdomino{2.}{../BenedicamusDomino/or--benedicamus_domino_laudes_--solesmes_1961}
}
}
\vfil

\oldneedspace{4\baselineskip}\label{benedicamusdomino-mary}
\gdef\benedicamusdominonamemary{3}
{{\centering \bfseries 3.~On feasts of the Blessed Virgin.\\}
\smallskip
%\def\breakbeforeresp{T}
\printbenedicamusdomino{1.}{../BenedicamusDomino/BenedicamusDomino_blessedVirgin}}
\vfil

\needspace{6\baselineskip}
\label{benedicamusdomino-sunday}
\gdef\benedicamusdominonamesunday{4}
{{\centering \bfseries 4.~On Sundays during the Year\\and Septuagesima, Sexagesima, and Quinquagesima.\\}
\smallskip
%\def\breakbeforeresp{T}
\printbenedicamusdomino{1.}{../BenedicamusDomino/BenedicamusDomino_Sundays}}

\vfil
\label{benedicamusdomino-lent}\label{benedicamusdomino-advent}
\gdef\benedicamusdominonamelent{5}
\gdef\benedicamusdominonameadvent{\benedicamusdominonamelent}
{{\centering \bfseries 5.~On Sundays of Advent and Lent.\\}
\smallskip
%\def\breakbeforeresp{T}
\printbenedicamusdomino{6.}{../BenedicamusDomino/BenedicamusDomino_SundaysOfAdventAndLent}
\ifthenelse{\boolean{birmingham}}{
	\medskip
	\emph{\normalsize or from \emph{Mass XVII}:}

	\smallskip
	\printbenedicamusdomino{6.}{../BenedicamusDomino/ky--benedicamus_xviia--solesmes}
}{}
}
\vfil

\label{benedicamusdomino-easter}
\gdef\benedicamusdominonameeaster{6}
{{\centering \bfseries 6.~On Sundays of Paschal Time.\\}
\smallskip
%\def\breakbeforeresp{T}
\printbenedicamusdomino{7.}{../BenedicamusDomino/BenedicamusDomino_SundaysOfPaschalTime}}

}

\chapter{Marian Anthems}
{
\def\nogabcbreaks{T}
\newcommand{\printsimpletone}{
\vspace{0ex plus 0ex minus 1ex}
\needspace{3\baselineskip}
\begin{center}\textbf{Simple Tone.}\end{center}
\vspace{0ex plus 0ex minus 1ex}
}
\newcommand{\printsolemntone}{
%\vspace{0ex plus 0ex minus 0.5ex}
\oldneedspace{3\baselineskip}
\begin{center}\textbf{Solemn Tone.}\end{center}
\vspace{0ex plus 0ex minus 1ex}
}
\newcommand{\afterant}{
\ifx\note\undefined\else%
\textit{\note}

\smallskip
\fi
\sloppy
\begin{columns}
\versicle{\vlatin}{\venglish}
\response{\rlatin}{\renglish}
\colchunk{}
\colplacechunks{}
\colchunk{\hspace*{3em}Orémus.}\colchunk{\hspace*{3em}Let us pray,}
\colplacechunks{}
\prayer{\prayerlatin}{\prayerenglish}
\end{columns}
\ifx\notetwo\undefined\else%
\medskip

\needspace{2\baselineskip}
\textit{\notetwo}

\smallskip
\begin{columns}
\versicle{\vlatintwo}{\venglishtwo}
\response{\rlatintwo}{\renglishtwo}
\colchunk{}
\colplacechunks{}
\colchunk{\hspace*{3em}Orémus.}\colchunk{\hspace*{3em}Let us pray,}
\colplacechunks{}
\prayer{\prayerlatintwo}{\prayerenglishtwo}
\end{columns}
\fi
}

\def\gabcfolder{../MarianAntiphons}

\section{Alma Redemptoris Mater}
\printnote{From Vespers of Saturday before the 1st Sunday of Advent to 2nd Vespers of the Purification.}
\printsimpletone{}
\printgabc{Ant.}{5.}{A}{AlmaSimple}
\printsolemntone{}
\printgabc{Ant.}{5.}{A}{AlmaSolemn}
\begin{quote}{Mother of Christ! hear thou thy people's cry
Star of the deep, and portal of the sky!
Mother of Him who thee from nothing made,
Sinking we strive and call to thee for aid;
Oh, by that joy which Gabriel brought to thee,
Thou Virgin first and last, let us thy mercy see.}\end{quote}

{
\newcommand{\note}{During Advent:}
\newcommand{\vlatin}{Angelus Dómini nuntiávit Maríæ.}
\newcommand{\venglish}{The Angel of the Lord declared unto Mary.}
\newcommand{\rlatin}{Et concépit de Spíritu Sancto.}
\newcommand{\renglish}{And she conceived by the Holy Ghost.}
\newcommand{\prayerlatin}{Grátiam tuam, quaésumus Dómine, méntibus nostris infúnde~:~\dag{} ut qui, Angelo nuntiánte, Christi Fílii tui incarnatiónem cognóvimus,~* per passiónem ejus et crucem ad resurrectiónis glóriam perducámur.  Per eúmdem Christum Dóminum nostrum. \Rbar~Amen.}
\newcommand{\prayerenglish}{Pour forth, we beseech Thee, O Lord, Thy grace into our hearts; that we, to whom the Incarnation of Christ Thy Son was made known by the message of an Angel, may, by His Passion and Cross, be brought to the glory of His Resurrection. Through the same Christ our Lord. \Rbar~Amen.}

\newcommand{\notetwo}{From 1st Vespers of Christmas to 2nd Vespers of the Purification:}
\newcommand{\vlatintwo}{Post pártum Vírgo invioláta permansísti.}
\newcommand{\venglishtwo}{After childbirth thou didst remain a pure Virgin.}
\newcommand{\rlatintwo}{Déi Génitrix intercéde pro nóbis.}
\newcommand{\renglishtwo}{Intercede for us, O Mother of God.}
\newcommand{\prayerlatintwo}{Deus, qui salútis ætérnæ, beátæ Maríæ virginitáte f\oe cúnda, humáno géneri praémia præstitísti~:~\dag{} tríbue, quaésumus; ut ipsam pro nobis intercédere sentiámus,~* per quam merúimus auctórem vitæ suscípere, Dóminum nostrum Jesum Christum Fílium tuum. \Rbar~Amen.}
\newcommand{\prayerenglishtwo}{O God, who, by the fruitful virginity of blessed Mary, hast given to mankind the rewards of eternal salvation; grant, we beseech Thee, that we may experience her intercession for us, through whom we have deserved to receive the Author of life, our Lord Jesus Christ, Thy Son. \Rbar~Amen.}

\afterant{}
}



% Ave Regina Cælorum
\section{Ave Regina Cælorum}
\printnote{From Compline of Feb 2nd (even if the Feast of the Purification be transferred) until Compline of Wednesday in Holy Week.}
\printsimpletone{}
\printgabc{Ant.}{6.}{A}{AveReginaSimple}
\printsolemntone{}
\printgabc{Ant.}{6.}{A}{AveReginaSolemn}
\begin{quote}{Hail, Queen of Heaven! Hail, Queen of Angels! Hail, blest Root and Gate, from which came light upon the world! Rejoice, O glorious Virgin, that surpassest all in beauty! Hail, O most lovely, and pray for us to Christ.}\end{quote}

\bigskip{}
{
\newcommand{\vlatin}{Dignáre me laudáre te Vírgo sacráta.}
\newcommand{\venglish}{Voucshafe, O holy Virgin, that I may praise thee.}
\newcommand{\rlatin}{Da míhi virtútem cóntra hóstes túos.}
\newcommand{\renglish}{Give me power against thine enemies.}
\newcommand{\prayerlatin}{Concéde, miséricors Deus, fragilitáti nostræ præ\-sí\-di\-um :~\gredagger{}~ut qui sanctæ Dei Genitrícis memóriam ágimus, * intercessiónis ejus auxílio a nostris iniquitátibus resurgámus. Per eúmdem Christum Dóminum nostrum. \Rbar~Amen.}
\newcommand{\prayerenglish}{Grant, O merciful God, Thy protection to us in our weakness; that we, who celebrate the memory of the holy Mother of God, may, through the aid of her intercession, rise again from our sins. Through the same Christ our Lord. \Rbar~Amen.}

\afterant{}
}




% Regina Cæli
\needspace{10\baselineskip}
\section{Regina cæli}
\printnote{From Compline of Easter Sunday to Compline of Friday after the Feast of Pentecost inclusively.}
\printsimpletone{}
\printgabc{Ant.}{6.}{R}{ReginaCaeliSimple}
\printsolemntone{}
\printgabc{Ant.}{6.}{R}{ReginaCaeliSolemn}
\begin{quote}{O Queen of heaven, rejoice, alleluia.
For He whom thou didst merit to bear, alleluia;
Has risen as He said, alleluia.
Pray for us to God, alleluia.}\end{quote}

\bigskip{}
{
\newcommand{\vlatin}{Gáude et lætáre Virgo María, allelúia.}
\newcommand{\venglish}{Rejoice and be glad, O Virgin Mary, alleluia.}
\newcommand{\rlatin}{Quia surréxit Dóminus vere, allelúia.}
\newcommand{\renglish}{For the Lord is risen indeed, alleluia.}
\newcommand{\prayerlatin}{Deus, qui per resurrectiónem Fílii tui Dómini nostri Jesu Christi mundum lætificáre dignátus es~:~\gredagger{} præsta, quaésumus; ut per ejus Genitrícem Vírginem Maríam~* perpétuæ capiámus gáudia vitæ. Per eúmdem Christum Dóminum nostrum. \Rbar~Amen.}
\newcommand{\prayerenglish}{O God, who didst vouchsafe to give joy to the world through the resurrection of Thy Son our Lord Jesus Christ; grant, we beseech Thee, that through His Mother, the Virgin Mary, we may obtain the joys of everlasing life. Through the same Christ our Lord. \Rbar~Amen.}

\afterant{}
}





% Salve Regina
\section{Salve Regina}
\printnote{From 1st Vespers of the Feast of the Blessed Trinity to None on Saturday before the 1st Sunday of Advent.}
\printsimpletone{}
\printgabc{Ant.}{5.}{S}{SalveReginaSimple}
\printsolemntone{}
\printgabc{Ant.}{1.}{S}{SalveReginaSolemn}
\begin{quote}{Hail, holy Queen, Mother of mercy; hail, our life, our sweetness, and our hope! To thee do we cry, poor banished children of Eve; to thee do we send up our sighs, mourning and weeping in this vale of tears.  Turn then, most gracious advocate, thine eyes of mercy towards us; and after this our exile, show unto us the blessed fruit of thy womb, Jesus.  O clement, O loving, O sweet Virgin Mary.}\end{quote}

\bigskip{}
{
\newcommand{\vlatin}{Ora pro nóbis sáncta Déi Génitrix.}
\newcommand{\venglish}{ Pray for us, O holy Mother of God.}
\newcommand{\rlatin}{Ut dígni efficiámur promissiónibus Chrísti.}
\newcommand{\renglish}{That we may be made worthy of the promises of Christ.}
\newcommand{\prayerlatin}{Omnípotens sempitérne Deus, qui gloriósæ Vírginis Matris Maríæ corpus et ánimam, ut dignum Fílii tui habitáculum éffici mererétur, Spíritu Sancto cooperánte præparásti~:~\gredagger{} da, ut cujus commemoratióne lætámur,~* ejus pia intercessióne ab instántibus malis et a morte perpétua liberémur. Per eúmdem Christum Dóminum nostrum. \Rbar~Amen.}
\newcommand{\prayerenglish}{Almighty, everlasting God, who with the cooperation of the Holy Ghost didst prepare the body and soul of the glorious Virgin Mary, to make it fit to be the worthy dwelling of Thy Son; grant that by the loving intercession of her in whose commemoration we rejoice, we may be delivered from present ills, and from everlasting death.  Through the same Christ our Lord. \Rbar~Amen.}

\afterant{}
}

}


\clearpage
{
	\label{sundayvespers}
	\chapter{Sunday at Vespers}
	\section{Sunday at Vespers throughout the year}
\newcommand{\includelilypond}[1]{
  \noindent
  \import{../misc/}{#1}%
}
\def\betweenLilyPondSystem#1{
  \ifnum#1>1
    \vfil\noindent
    %\linebreak
    %\bigskip
  \else
    \linebreak
  \fi
  %\vspace{0.5\baselineskip}
}
\def\printvrdirigatur{
	\label{vr-dirigatur}
  \def\vrlinebreak{T}
  \printvr[\greseteolcustos{manual}]{\vrtex}{\vtranslation}{\rtranslation}
}
\def\printvrdirigaturnote{
	\Vbar{}~Dirigátur Dómine orátio mea.

	\Rbar{}~Sicut incénsum in conspéctu tuo.

	\emph{For chant, see page \pageref{vr-dirigatur}.}
	\bigskip
}

	\def\printfullhymn{
	    {
	    	\oldneedspace{5\baselineskip}
	    	\label{hymn-luciscreator}
	    	\ifthenelse{\boolean{includepolyphony}}{
	        	\printhymn{\oldstylenums{\hymnlinetwo}}{\hymninitial}{\hymntex}{\hymntranslation}
	        }{
				\def\annot{\small{Hymn.}}
				\def\annottwo{\small{\hymnlinetwo}}
				\alsetinitialspacing{\hymninitial}
				\normalsize
				\oldneedspace{5\baselineskip}
				\gregorioscore{\gabcfolder/\hymntex}
			}
			\bigskip
			\printvrdirigaturnote{}
	        \def\hymnlinetwo{\oldstylenums{8.}}
\def\hymntex{hymn-LucisCreatorOptime2}
\def\hymninitial{L}

\def\hymntextwo{hymn-LucisCreatorOptime3}
\def\hymntwolinetwo{\oldstylenums{1.}}

			{
				\def\annot{\small{Hymn.}}
				\def\annottwo{\small{\hymnlinetwo}}
				\alsetinitialspacing{\hymninitial}
				\normalsize
				\oldneedspace{5\baselineskip}
				{\centering\addfontfeature{Numbers=Lining}\textbf{2.~Another Chant (ad libitum).}\par}
				\gregorioscore{\gabcfolder/\hymntex}
			}
			\bigskip
		  \printvrdirigaturnote{}
			{
				\def\annot{\small{Hymn.}}
				\def\annottwo{\small{\hymntwolinetwo}}
				\alsetinitialspacing{\hymninitial}
				\normalsize
				\oldneedspace{5\baselineskip}
				{\centering\addfontfeature{Numbers=Lining}\textbf{3.~Another Chant.}\par}

				\gregorioscore{\gabcfolder/\hymntextwo}
			}
			\bigskip
		  \printvrdirigaturnote{}
	    }
\ifthenelse{\boolean{includepolyphony}}{
		    \vfil\noindent
				\oldneedspace{8\baselineskip}
		    {\centering\addfontfeature{Numbers=Lining}\textbf{4. Chant alternating with polyphony by Victoria.}\par}
				\def\annot{\small{Hymn.}}
				\def\annottwo{\small{8.}}
				\alsetinitialspacing{L}
		    \noindent\gabcsnippet{
		    (c3)LU(e)cis(fh) Cre(h)á(hg)tor(fe) óp(h)ti(i)me,(gvFE.) (,)
	Lú(hi)cem(jh) di(i)é(hvGF')rum(g) pró(g')fe(f)rens,(e.) (;)
	Pri(e)mór(fh)di(h)is(hg) lú(fe)cis(h) nó(i)væ(gvFE.) (,)
	Mún(hi)di(jh) pá(i)rans(hf) o(g)rí(fg)gi(f)nem :(e.) (::) (e+)
		    }
		    \vfil\noindent
	\includelilypond{lucis-creator-optime-systems-v2}%
	\gresetinitiallines{0}
	\gabcsnippet{
	(c3)3. Ne(e) mens(fh) gra(h)vá(hg)ta(fe) crí(h)mi(i)ne,(gvFE.) (,)
	Ví(hi)tæ(jh) sit(i) éx(hvGF')sul(g) mú(g')ne(f)re,(e.) (;)
	Dum(e) nil(fh) per(h)én(hg)ne(fe) có(h)gi(i)tat,(gvFE.) (,)
	Se(hi)sé(jh)que(i) cúl(hf)pis(g) íl(fg)li(f)gat.(e.) (::) (e+)
	}
	\vfil\noindent
	\includelilypond{lucis-creator-optime-systems-v4}%
	\bigskip
	\gabcsnippet{
		(c3)5. Præ(e)sta,(fh) Pá(h)ter(hg) pi(fe)ís(h)si(i)me,(gvFE.) (,)
	Pa(hi)trí(jh)que(i) cóm(hvGF')par(g) U(g')ni(f)ce,(e.) (;)
	Cum(e) Spí(fh)ri(h)tu(hg) Pa(fe)rá(h)cli(i)to,(gvFE.) (,)
	Ré(hi)gnans(jh) per(i) óm(hf)ne(g) s<sp>'ae</sp>(fg)cu(f)lum.(e.) (::) 
	A(efe)men.(d.e.) (::)
	}
}{}
    {
    	\normalsize%\vspace{0pt minus 20pt}
    	\oldneedspace{5\baselineskip}
      \selectlanguage{american}\vspace{-\baselineskip}
      \begin{multicols}{2}
      \begin{psalmverses}
      {\hymntranslation}
      \end{psalmverses}
      \end{multicols}
    }
	\printvrdirigatur{}
	}
	\def\vrlinebreak{T}
	\newcommand{\printsundaysthroughtheyear}{
		\def\dotting{\leaders\hbox to 1em{\hfil.\hfil}\hfill}
		\printnote{Time after Epiphany,\dotting \emph{p.~\pageref{epiphany2} to \pageref{epiphany6}}}

		\printnote{Shrovetide \& Lent,\dotting \emph{p.~\pageref{septuagesima} to \pageref{lent6}}}

		\printnote{Time after Pentecost,\dotting \emph{p.~\pageref{pentecost2} to \pageref{pentecost24}}}

	}
	\newcommand{\magreplacement}{\vspace{-\baselineskip}\noindent\begin{centering}

		\vfil
		\emph{Magnificat \& Collect of the Sunday.}

		\end{centering}
		\medskip{}
		%\vspace{-\baselineskip}
		\printsundaysthroughtheyear{}
		\vfil
		\Vbar~\emph{Benedicámus Dómino \benedicamusdominonamesunday{}}, page \pageref{benedicamusdomino-sunday}; or in Lent, \emph{\benedicamusdominonamelent{}}, page \pageref{benedicamusdomino-lent}.
	}
	\newcommand{\collectreplacement}{\bigskip}
	\def\prepsalmtitlethree{\vspace{-0.25\baselineskip}}
	\def\preantfour{\needspace{10\baselineskip}}
	%\def\prepsalmfive{\greseteolcustos{manual}}
	%\def\prechapter{\vspace{-\baselineskip}}
	\def\beginchaptercols{
		\noindent{}\printnote{In Shrovetide \& Lent, from the Sunday, \emph{p.~\pageref{septuagesima} to \pageref{lent6}.}}

		\bigskip{}
		\noindent{}\printnote{Otherwise:}
		\begin{parcolumns}[rulebetween,colwidths={1=0.47\linewidth}]{2}
	}
	\printvespers[../SundayAtVespers]{inc-SundayAtVespers}
	\bigskip
	\bigskip
	\hrule
	\bigskip
}


{
	\label{advent}
	\chapter{Advent}
	\section{First Sunday of Advent}

  {
	  \def\printhymn{%
		{\centering Hymn.\par}\label{hymn-advent}

	  	{\def\gabcfolder{.}
	  	\printgabc{1.}{}{N}{hy--nunc_sancte_nobis_(in_adventu)--solesmes_1961}}

		\bigskip
	  }
    \def\printshortresp{%
    \label{shortresp-advent}%
    {\def\gabcfolder{.}
    \printgabc{Short}{Resp.}{V}{re--veni_ad_liberandum--solesmes}}

    \translation[]{\Vbar{}~Come to my rescue O God, Lord of hosts.
    \Rbar{}~Come\dots{}
    \Vbar{}~Show us thy face, and we shall be saved.
    \Rbar{}~O God, Lord of hosts.
    \Vbar{}~Glory be to the Father, and to the Son, and to the Holy Ghost.
    \Rbar{}~Come\dots{}
    }

    \bigskip
    \gresetinitiallines{0}
    \gregorioscore{vr-timebunt-gentes}
    \let\myhwidth\relax
    \let\myhhwidth\relax
    \newlength{\myhwidth}
    \settowidth{\myhwidth}{tu}
    \newlength{\myhhwidth}
    \settowidth{\myhhwidth}{n}
    \addtolength{\myhhwidth}{-\myhwidth}
    \def\myhspace{\hspace{1.4ex}}
    \begin{nstabbing}
    \>\Rbar{}~Et \myhspace{} omnes \myhspace{} reges \myhspace{} terræ \myhspace{} glóriam \>\hspace{\myhhwidth}tuam.
    \end{nstabbing}

    \translation[]{\Vbar{}~And the Gentiles shall fear thy name, O Lord.\\
    \Rbar{}~And all the kings of the earth thy glory.}

    \bigskip
    }

	  \printterce[../Advent1]{inc-Advent1}
  }

  \def\printhymn{
  	\noindent\emph{Hymn}, page \pageref{hymn-advent}

  	\medskip
  }
  \def\printshortresp{
    \noindent\emph{Short Resp. \emph{Veni ad liberándum nos}, p.} \pageref{shortresp-advent}, \Vbar{}~Timébunt gentes. 
  }
  \section{Second Sunday of Advent}
  \printterce[../Advent2]{inc-Advent2-Vespers2}

  \section{Third Sunday of Advent}
  \printterce[../Advent3]{inc-Advent3}

  \section{Fourth Sunday of Advent}
  \printterce[../Advent4]{inc-Advent4}

}

%christmas through Holy Family
{
	\label{christmas}
	\chapter{Christmas}
	\section{Christmas Eve}
  \def\printbenedicamusdomino{
    \noindent\emph{Benedicamus Domino}, p. \pageref{benedicamus-domino-sunday}.
  }
  {
    \def\printhymn{%
    	, \emph{Hymn}, page \pageref{hymn-advent}

    	\medskip
    }
    \def\printshortresp{%
      \label{shortresp-christmas-eve}%
      {\def\gabcfolder{.}
      \printgabc{Short}{Resp.}{H}{re--hodie_scietis--solesmes_1961}}

      \translation[]{\Vbar{}~This day ye shall know that the Lord cometh.
      \Rbar{}~This\dots{}
      \Vbar{}~And in the morning, then ye shall see His glory.
      \Rbar{}~That the Lord cometh.
      \Vbar{}~Glory be to the Father, and to the Son, and to the Holy Ghost.
      \Rbar{}~This\dots{}
      }

      \bigskip
      \gresetinitiallines{0}\label{vr-christmas-eve}
      \gregorioscore{vr-constantes-estote}
      \let\myhwidth\relax
      \let\myhhwidth\relax
      \newlength{\myhwidth}
      \settowidth{\myhwidth}{v} %this is what precedes the last vowel in the response
      \newlength{\myhhwidth}
      \settowidth{\myhhwidth}{t} %this is what preceded the last vowel in the verse
      \addtolength{\myhhwidth}{-\myhwidth}
      \def\myhspace{\hspace{1.4ex}}
      \begin{nstabbing}
      \>\Rbar{}~Vidébitis auxílium Dómini super \>\hspace{\myhhwidth}vos.
      \end{nstabbing}

      \translation[]{\Vbar{}~Stand ye still.\\
      \Rbar{}~And ye shall see the salvation of the Lord with you.}

      \bigskip
    }
    \printterce{inc-christmas-eve}{christmas-eve}
  }
  {
    \newcommand{\printrefshymn}[1]{%
      \def\dotting{\hfill%\leaders\hbox to 1em{\hfil.\hfil}\hfill
        }%
      \begin{multicols}{2}%
      \noindent{}Christmas \& Sunday,\dotting \emph{below}\\
      Circumcision,\dotting \emph{p.~\pageref{circumcision-#1}}\\
      Holy Name,\dotting \emph{p.~\pageref{holyname-#1}}
      \end{multicols}%
      \smallskip
    }
    \def\printhymn{

      {\centering Hymn.\par}\label{hymn-christmas}

      {\def\gabcfolder{.}
       \printgabc{8.}{}{N}{hy--nunc_sancte_nobis_(in_nativitate_domini)--solesmes_1961}
      }

      \printrefshymn{ant}
      \bigskip
    }
    \newcommand{\printrefs}[1]{%
      \def\dotting{\hfill%\leaders\hbox to 1em{\hfil.\hfil}\hfill
        }%
      \begin{multicols}{2}%
      \noindent{}Christmas,\dotting \emph{below}\\
      Sunday within the octave,\dotting \emph{p.~\pageref{christmas-sunday-#1}}
      \end{multicols}%
      \smallskip
    }
    \newcommand{\anttwotex}{an--genuit_puerpera_regem--solesmes_1961}
    \newcommand{\anttwoinitial}{G}
    \newcommand{\anttwotranslation}{The Mother brought forth the King, Whose name is called The Eternal; the joy of a Mother was hers, remaining a Virgin unsullied; neither before nor henceforth hath there been or shall be such another, alleluia.}
    \definepsalm{2}{109}{2}{D}

    \def\printshortresp{%
      \label{shortresp-christmas}%
      {\def\gabcfolder{.}
      \printgabc{Short}{Resp.}{V}{rb--verbum_caro_factum_est--solesmes_1961}}

      \translation[]{\Vbar{}~The Word was made flesh. Alleluia, Alleluia.
      \Rbar{}~The Word\dots{}
      \Vbar{}~And dwelt among us.
      \Rbar{}~Alleluia, Alleluia.
      \Vbar{}~Glory be to the Father, and to the Son, and to the Holy Ghost.
      \Rbar{}~The Word\dots{}
      }

      \bigskip
      \gresetinitiallines{0}\label{vr-christmas}
      \gregorioscore{vr-ipse-invocabit}
      \let\myhwidth\relax
      \let\myhhwidth\relax
      \newlength{\myhwidth}
      \settowidth{\myhwidth}{allelui} %this is what precedes the last vowel in the response
      \newlength{\myhhwidth}
      \settowidth{\myhhwidth}{i} %this is what preceded the last vowel in the verse
      \addtolength{\myhhwidth}{-\myhwidth}
      \def\myhspace{\hspace{0.3ex}}
      \begin{nstabbing}
      \>\Rbar{}~Pater \myhspace{} meus \myhspace{} es \myhspace{} tu, \>\hspace{\myhhwidth}allelúia.
      \end{nstabbing}

      \translation[]{\Vbar{}~He shall cry unto Me, Alleluia.\\
      \Rbar{}~Thou art My Father, Alleluia.}

      \bigskip
    }
    \def\prechapter{\printrefs{chapter}}
    \section{Christmas}
    \printterce{../Christmas/inc-Christmas-Vespers2}{christmas}
  }

  \def\printshortresp{
    \noindent\emph{Short Resp. \emph{Verbum caro}}, p. \pageref{shortresp-christmas}, \Vbar{}~Ipse invocabit, p. \pageref{vr-christmas}.
  }
  {
    \def\printhymn{%
      , \emph{Hymn}, page \pageref{hymn-sunday}%
      , Ant. \emph{Genuit puerpera Regem}, \pageref{christmas-ant}

      \medskip
    }

    \section{Sunday within the octave}
    \printterce{../Christmas/inc-SundayWithinOctaveOfChristmas-Vespers2}{christmas-sunday}
  }
  \def\printhymn{%
    , \emph{Hymn}, page \pageref{hymn-christmas}%

    \medskip
  }
  {
    \section{Octave of the Nativity}
    \printterce[../ChristmasOctave-Circumcision]{inc-Circumcision-Vespers-common}{circumcision}
  }
  {
    % holy name
    % proper ant, chapter from vespers, short resp
    \newcommand{\antonetex}{an--omnis_qui_invocaverit--solesmes}
\newcommand{\antoneinitial}{O}
\newcommand{\antonetranslation}{All who shall call on the name of the Lord shall be saved.}
\definepsalm{1}{109}{8}{G}

\newcommand{\anttwotex}{an--sanctum_et_terribile--solesmes}
\newcommand{\anttwoinitial}{S}
\newcommand{\anttwotranslation}{Holy and terrible is His name; the fear of the Lord is the beginning of wisdom.}
\definepsalm{2}{110}{5}{a}

\newcommand{\antthreetex}{an--ego_autem_in_domino--solesmes}
\newcommand{\antthreeinitial}{E}
\newcommand{\antthreetranslation}{Yet I will rejoice in the Lord and exult in the God of my salvation.}
\definepsalm{3}{111}{3}{a2}

\newcommand{\antfourtex}{an--a_solis_ortu--solesmes}
\newcommand{\antfourinitial}{A}
\newcommand{\antfourtranslation}{From the rising of the sun to its setting, the Lord's Name is to be praised.}
\definepsalm{4}{112}{4}{E}

\newcommand{\antfivetex}{an--sacrificabo_hostiam--solesmes}
\newcommand{\antfiveinitial}{S}
\newcommand{\antfivetranslation}{I will offer the sacrifice of praise, and will call upon the Name of the Lord.}
\def\psalmclef{2} %if I didn't use definepsalm, I would define this as psalmcleffive
\definepsalm{5}{115}{8}{c}
\let\psalmclef=\undefined % but then I have to undefine it!
%\renewcommand{\psalmcleffive}{2}

\newcommand{\chaptertext}{\dropcap{latin}{Fratres~: Christus humiliávit semetípsum, factus obédiens usque ad mortem, mortem autem} \textbf{crú}\-cis.~\dag{} Propter quod et Deus exaltávit illum, et donávit illi nomen quod est super \emph{om\-ne} \textbf{nó}\-men~:~* ut in nómine Jesu omne genu flec\textbf{tá}tur.}
\newcommand{\chaptertranslation}{Christ humbled Himself, becoming obedient unto death, even the death of the cross. For which cause God also hath exalted Him, and given Him a name which is above all names: That in the name of Jesus every knee should bow.}

\newcommand{\hymnlinetwo}{1.}
\newcommand{\hymntex}{hy--jesu_dulcis_memoria--solesmes}
\newcommand{\hymninitial}{J}
\newcommand{\hymntranslation}{
\item Jesu, the very thought of Thee
With sweetness fills my breast;
But sweeter far Thy face to see,
And in Thy presence rest.

\item Nor voice can sing, nor heart can frame,
Nor can the memory find,
A sweeter sound than Thy blest Name,
O Saviour of mankind!

\item O Hope of every contrite heart,
O Joy of all the meek,
To those who fall, how kind Thou art!
How good to those who seek!

\item But what to those who find? Ah! this
Nor tongue nor pen can show:
The love of Jesus, what it is
None but His loved ones know.

\item Jesu, our only joy be Thou,
As Thou our prize wilt be;
Jesu, be Thou our glory now,
And through eternity.
Amen.%\grechangestaffsize{15}
}

\newcommand{\vrtex}{vrSitNomenDominiBenedictum}
\newcommand{\vtranslation}{Blessed be the Name of the Lord, alleluia.}
\newcommand{\rtranslation}{From this time forth, and for evermore, alleluia.}

\newcommand{\collect}{Deus, qui unigénitum Fílium tuum constituísti humáni géneris Salvatórem, et Jesum vocári jussísti~:~\dag{} concéde propítius; ut cujus sanctum nomen venerámur in terris,~* ejus quoque aspéctu perfruámur in cælis. Per eúmdem Dóminum.}
\newcommand{\collecttranslation}{O God, Who hast appointed Thine Only-begotten Son to be the Saviour of mankind, and commanded that He should be called Jesus, mercifully grant that we, who here on earth do worship His Holy Name, may be made glad in heaven by His Presence. Through the same our Lord.}

    \renewcommand{\anttwotex}{an--scitote_quia_dominus--solesmes_1961}
    \renewcommand{\anttwoinitial}{S}
    \renewcommand{\anttwotranslation}{}
    \definepsalm{2}{109}{3}{a}

    \def\printshortresp{%
      \label{shortresp-holyname}%
      {\def\gabcfolder{.}
      \printgabc{Short}{Resp.}{S}{rb--sit_nomen--solesmes}}

      \translation[]{\Vbar{}~Blessed be the Name of the Lord, Alleluia, alleluia.
        \Rbar{}~Blessed\dots{}
        \Vbar{}~From this time forth, and for evermore.
        \Rbar{}~Alleluia, alleluia.
        \Vbar{}~Glory be to the Father, and to the Son, and to the Holy Ghost.
        \Rbar{}~Blessed\dots{}
      }

      \bigskip
      \gresetinitiallines{0}\label{vr-holyname}
      \gregorioscore{vr-afferte-domino}
      \let\myhwidth\relax
      \let\myhhwidth\relax
      \newlength{\myhwidth}
      \settowidth{\myhwidth}{allelui} %this is what precedes the last vowel in the response
      \newlength{\myhhwidth}
      \settowidth{\myhhwidth}{i} %this is what preceded the last vowel in the verse
      \addtolength{\myhhwidth}{-\myhwidth}
      \def\myhspace{\hspace{0.5ex}}
      \begin{nstabbing}
      %\>\Rbar{}~Pater \myhspace{} meus \myhspace{} es \myhspace{} tu, \>\hspace{\myhhwidth}allelúia.
      \>\Rbar{}~Afférte \myhspace{}Dómino\myhspace{} glóriam\myhspace{} nómini\myhspace{} ejus, \>\hspace{\myhhwidth}allelúia.
      \end{nstabbing}

      \translation[]{\Vbar{}~Give unto the Lord glory and honour, alleluia.\\
        \Rbar{}~Give unto the Lord the glory due unto His Name, alleluia.}

      \bigskip
    }
    \section{Holy Name}
    \def\gabcfolder{.}
    \printterce{}{holyname}
  }
  {
    % epiphany
    % proper hymn, ant & chapter from vespers, short resp
    \def\printhymn{

      {\centering Hymn.\par}\label{hymn-epiphany}

      {\def\gabcfolder{.}
       \printgabc{8.}{}{N}{hy--nunc_sancte_nobis_(in_epiphania_domini)--solesmes_1961}
      }

      %\printrefshymn{ant}
      \bigskip
    }
    \def\printshortresp{%
      \label{shortresp-epiphany}%
      {\def\gabcfolder{.}
      \printgabc{Short}{Resp.}{R}{re--reges_tharsis--solesmes_1961}}

      \translation[]{\Vbar{}~The kings of Tarshish and of the isles shall bring presents. Alleluia, Alleluia.
        \Rbar{}~The kings\dots{}
        \Vbar{}~The kings of Arabia and Saba shall offer gifts.
        \Rbar{}~Alleluia, Alleluia.
        \Vbar{}~Glory be to the Father, and to the Son, and to the Holy Ghost.
        \Rbar{}~The kings\dots{}
      }

      \bigskip
      \gresetinitiallines{0}\label{vr-epiphany}
      \gregorioscore{vr-omnes-de-saba}
      \let\myhwidth\relax
      \let\myhhwidth\relax
      \newlength{\myhwidth}
      \settowidth{\myhwidth}{allelui} %this is what precedes the last vowel in the response
      \newlength{\myhhwidth}
      \settowidth{\myhhwidth}{i} %this is what preceded the last vowel in the verse
      \addtolength{\myhhwidth}{-\myhwidth}
      \def\myhspace{\hspace{0.1ex}}
      \begin{nstabbing}
      %\>\Rbar{}~Pater \myhspace{} meus \myhspace{} es \myhspace{} tu, \>\hspace{\myhhwidth}allelúia.
      \>\Rbar{}~Aurum et thus deferéntes, \>\hspace{\myhhwidth}allelúia.
      \end{nstabbing}

      \translation[]{\Vbar{}~All they from Saba shall come. Alleluia.\\
        \Rbar{}~They shall bring gold and incense. Alleluia.}

      \bigskip
    }
    \section{Epiphany}
    \printterce[../Epiphany]{inc-Epiphany-vespers}{epiphany}
  }
  {
    % holy family
    % ant & chapter from vespers, proper short resp
    % slightly proper hymn (rest from epiphany)
    \def\printhymn{

      {\centering Hymn.\par}\label{hymn-holyfamily}

      {\def\gabcfolder{.} %TODO: this has the correct verse 3 but first 2 verses are wrong!
       \printgabc{8.}{}{N}{hy--te_lucis_ante_terminum_(holy_family)--solesmes_1961}
      }

      %\printrefshymn{ant}
      \bigskip
    }
    \def\printshortresp{%
      \label{shortresp-holyfamily}%
      {\def\gabcfolder{.}
      \printgabc{Short}{Resp.}{P}{re--propter_nos--solesmes_1961}}

      \translation[]{\Vbar{}~For our sake he became poor, Being rich.
        \Rbar{}~For our sake\dots{}
        \Vbar{}~That through his poverty we might be rich.
        \Rbar{}~Being rich.
        \Vbar{}~Glory be to the Father, and to the Son, and to the Holy Ghost.
        \Rbar{}~For our sake\dots{}
      }

      \bigskip
      \gresetinitiallines{0}\label{vr-holyfamily}
      \gregorioscore{vr-dominus-vias-suas}
      \let\myhwidth\relax
      \let\myhhwidth\relax
      \newlength{\myhwidth}
      \settowidth{\myhwidth}{ej} %this is what precedes the last vowel in the response
      \newlength{\myhhwidth}
      \settowidth{\myhhwidth}{n} %this is what preceded the last vowel in the verse
      \addtolength{\myhhwidth}{-\myhwidth}
      \def\myhspace{\hspace{0.1ex}}
      \begin{nstabbing}
      %\>\Rbar{}~Pater \myhspace{} meus \myhspace{} es \myhspace{} tu, \>\hspace{\myhhwidth}allelúia.
      \>\Rbar{}~Et \myhspace{} ambulábimus \myhspace{} in \myhspace{} sémitis \>\hspace{\myhhwidth}ejus.
      \end{nstabbing}

      \translation[]{\Vbar{}~The Lord will teach us his ways.\\
        \Rbar{}~And we will walk in his paths.}

      \bigskip
    }
    \section{Holy Family}
    \printterce[../HolyFamily]{inc-HolyFamily-Vespers2}{holyfamily}
  }
}

\chapter{Proper of the Time -- After Epiphany}
{
\newcommand{\benedicamusdomino}[1][sunday]{
  \benedicamusdominomaster{#1}
}
\def\printhymnnote{}
\def\printcommonvespers{
	\subtitle{\nth{2} Class, Green}
	\printnote{From \emph{Vespers of Sundays throughout the year}, page \pageref{sundayvespers}.\\}
}

{
\section{\nth{2} Sunday after Epiphany}
\label{epiphany2}
\printcommonvespers{}
\printvespersmag[../TimeAfterEpiphany]{inc-VespersMagnificatEpiphany2}

\bigskip
\benedicamusdomino{}
}

{
\section{\nth{3} Sunday after Epiphany}
\label{epiphany3}
\printcommonvespers{}
\printvespersmag[../TimeAfterEpiphany]{inc-VespersMagnificatEpiphany3}

\bigskip
\benedicamusdomino{}
}

{
\section{\nth{4} Sunday after Epiphany}
\label{epiphany4}
\def\precollect{\printvrdirigatur}
\printcommonvespers{}
\printvespersmag[../TimeAfterEpiphany]{inc-VespersMagnificatEpiphany4}

\bigskip
\benedicamusdomino{}
}

{
\section{\nth{5} Sunday after Epiphany}
\label{epiphany5}
\printcommonvespers{}
\printvespersmag[../TimeAfterEpiphany]{inc-VespersMagnificatEpiphany5}

\bigskip
\benedicamusdomino{}
}

{
\section{\nth{6} Sunday after Epiphany}
\label{epiphany6}
\printcommonvespers{}
\def\prevespers{%
  \let\oldthing=\englishmagantiphon
  \def\englishmagantiphon{\oldthing\pagebreak}
}
\printvespersmag[../TimeAfterEpiphany]{inc-VespersMagnificatEpiphany6}

\bigskip
\benedicamusdomino{}
}
}

%from shrovetide through passiontide
{
	\label{septuagesima}
	\chapter{Septuagesima \& Lent}
  \def\printbenedicamusdomino{
    \noindent\emph{Benedicamus Domino}, p. \pageref{benedicamus-domino-sunday}.
  }
  \section{Septuagesima}
  \printterce{inc-septuagesima}{septuagesima}

  \section{Sexagesima}
  \printterce{inc-sexagesima}{sexagesima}

  \section{Quinquagesima}
  \printterce{inc-quinquagesima}{quinquagesima}

  {
    \newcommand{\printrefs}[1]{%
      \def\dotting{\hfill%\leaders\hbox to 1em{\hfil.\hfil}\hfill
        }%
      \begin{multicols}{2}%
      \noindent{}1st Sunday of Lent,\dotting \emph{below}\\
      2nd Sunday of Lent,\dotting \emph{p.~\pageref{lent2-#1}}\\
      3rd Sunday of Lent,\dotting \emph{p.~\pageref{lent3-#1}}\\
      4th Sunday of Lent,\dotting \emph{p.~\pageref{lent4-#1}}
      \end{multicols}%
      \smallskip
    }

	  \def\printhymn{

		  {\centering Hymn.\par}\label{hymn-lent}

	  	{\def\gabcfolder{.}
	  	\printgabc{1.}{}{N}{hy--nunc_sancte_nobis_(in_quadragesima)--solesmes_1961}}

      \printrefs{ant}
		  \bigskip
	  }

    \def\printshortresp{%
    \label{shortresp-lent}%
    {\def\gabcfolder{.}
    \printgabc{Short}{Resp.}{I}{re--ipse_liberavit--solesmes}}

    \translation[]{\Vbar{}~For he hath delivered me from the snare of the hunters.
    \Rbar{}~For he hath\dots{}
    \Vbar{}~And from the sharp word.
    \Rbar{}~From the snare of the hunters.
    \Vbar{}~Glory be to the Father, and to the Son, and to the Holy Ghost.
    \Rbar{}~For he hath\dots{}
    }

    \bigskip
    \gresetinitiallines{0}\label{vr-lent}
    \gregorioscore{vr-scapulis-suis}
    \let\myhwidth\relax
    \let\myhhwidth\relax
    \newlength{\myhwidth}
    \settowidth{\myhwidth}{speráb} % text in last word before last vowel of response
    \newlength{\myhhwidth}
    \settowidth{\myhhwidth}{b} % text in last syllable before vowel of versicle
    \addtolength{\myhhwidth}{-\myhwidth}
    \def\myhspace{\hspace{1ex}}
    \begin{nstabbing}
    %\>\Rbar{}~Et \myhspace{} omnes \myhspace{} reges \myhspace{} terræ \myhspace{} glóriam \>\hspace{\myhhwidth}tuam.
    \>\Rbar{}~Et \myhspace{} sub \myhspace{} pennis \myhspace{} ejus \>\hspace{\myhhwidth}sperábis.
    \end{nstabbing}

    \translation[]{\Vbar{}~He will overshadow thee with his shoulders.\\
    \Rbar{}~And under his wings thou shalt trust.}

    \printrefs{collect}
    \bigskip
    }

    \section{First Sunday of Lent}
	  \printterce{inc-Lent1}{lent1}
  }

  \def\printhymn{%
  	, \emph{Hymn}, page \pageref{hymn-lent}

  	\medskip
  }
  \def\printshortresp{
    \noindent\emph{Short Resp. \emph{Ipse liberavit me}}, p. \pageref{shortresp-lent}, \Vbar{}~Scapulis suis, p. \pageref{vr-lent}.
  }
  \section{Second Sunday of Lent}
  \printterce{inc-Lent2}{lent2}

  \section{Third Sunday of Lent}
  \printterce{inc-Lent3}{lent3}

  \section{Fourth Sunday of Lent}
  \printterce{inc-Lent4}{lent4}

  {
    \newcommand{\printrefs}[1]{%
      \def\dotting{\hfill%\leaders\hbox to 1em{\hfil.\hfil}\hfill
        }%
      \begin{multicols}{2}%
      \noindent{}1st Sunday of the Passion,\dotting \emph{below}\\
      2nd Sunday of the Passion,\dotting \emph{p.~\pageref{passion2-#1}}
      \end{multicols}%
      \smallskip
    }

    \def\printhymn{

      {\centering Hymn.\par}\label{hymn-passiontide}

      {\def\gabcfolder{.}
      \printgabc{2.}{}{N}{hy--nunc_sancte_nobis_(in_tempore_passionis)--solesmes_1961}}

      \printrefs{ant}
      \bigskip
    }

    \def\printshortresp{%
    \label{shortresp-passiontide}%
    {\def\gabcfolder{.}
    \printgabc{Short}{Resp.}{E}{re--erue_a_frama--solesmes}}

    \translation[]{\Vbar{}~Deliver from the sword, O God, my soul.
    \Rbar{}~Deliver\dots{}
    \Vbar{}~My only one from the hand of the dog.
    \Rbar{}~O God, my soul.
    \Rbar{}~Deliver, * O God, my soul from the sword.For he hath delivered me from the snare of the hunters.
    }

    \bigskip
    \gresetinitiallines{0}\label{vr-passiontide}
    \gregorioscore{vr-de-ore-leonis}
    \let\myhwidth\relax
    \let\myhhwidth\relax
    \newlength{\myhwidth}
    \settowidth{\myhwidth}{me} % text in last word before last vowel of response
    \newlength{\myhhwidth}
    \settowidth{\myhhwidth}{n} % text in last syllable before vowel of versicle
    \addtolength{\myhhwidth}{-\myhwidth}
    \def\myhspace{\hspace{1.4ex}}
    \begin{nstabbing}
    %\>\Rbar{}~Et \myhspace{} omnes \myhspace{} reges \myhspace{} terræ \myhspace{} glóriam \>\hspace{\myhhwidth}tuam.
    \>\Rbar{}~Et a córnibus unicórnium humilitátem \>\hspace{\myhhwidth}meam.
    \end{nstabbing}

    \translation[]{\Vbar{}~From the lion's mouth, O Lord, save me.
    \Rbar{}~And my lowness from the horns of the unicorns.}

    \printrefs{collect}
    \bigskip
    }

    \section{First Sunday of the Passion}
    \printterce{inc-Passion1}{passion1}
  }
 
  \def\printhymn{%
    , \emph{Hymn}, page \pageref{hymn-passiontide}

    \medskip
  }
  \def\printshortresp{
    \noindent\emph{Short Resp. \emph{Erue a framea}}, p. \pageref{shortresp-passiontide}, \Vbar{}~De ore leonis, p. \pageref{vr-passiontide}.
  }
  \section{Second Sunday of the Passion}
  \printterce{inc-Passion2}{passion2}
}

{
\newcommand{\printcommemnote}[1][easter]{\smallskip
\noindent
\printnote{\commemorations{}  Otherwise \benedicamusdominoreference{#1}}
}
{
\chapter{Proper of the Time -- Easter}
\section{Easter Sunday}
\subtitle{\nth{1} Class, White or Gold}
\def\printfullhymn{
    \vspace{-0.5\baselineskip}
    \emph{Chapter, Hymn, and Versicle are all omitted, but the following Antiphon is said :}

    \bigskip
    \def\annot{\small{Ant.}}
    \def\annottwo{\small{\chapterhymnversicleantiphonmode.}}
    \alsetinitialspacing{\chapterhymnversicleantiphoninitial}
    \gregorioscore{\gabcfolder/\chapterhymnversicleantiphontex}
    \translation[]{\chapterhymnversicleantiphontranslation}
    \vspace{-0.5\baselineskip}
    \bigskip
}
\def\chapterreplacement{\bigskip}
\def\begincollectcols{\begin{parcolumns}[rulebetween,colwidths={1=0.45\linewidth}]{2}}
\def\postmag{\vspace{-0.05\baselineskip}}
\printvespers[../Easter]{inc-EasterVespers}
\newcommand{\printbenedicamusdomino}[2]{
	\def\annot{\small{#1}}
	\def\annottwo{}
	\alsetinitialspacing{B}
    \greseteolcustos{manual}
	\gregorioscore{#2}
    \greseteolcustos{auto}
    \bigskip
    \hrule
}
\def\breakbeforeresp{T}
\printbenedicamusdomino{\Vbar}{../BenedicamusDomino/BenedicamusDomino_Easter}
}

\newcommand{\printcommonvespers}[1][2]{
    \subtitle{\nth{#1} Class, \liturgicalcolor{}}
    \deusinadjutorium{}
    \printnote{\emph{Vespers of Sundays in Paschaltide}, p.~\pageref{sundayvespers-easter}.\\}
}
\def\liturgicalcolor{White}
{
\newcommand{\benedicamusdomino}[1][easter]{
  \benedicamusdominomaster{#1}
}

\newcommand{\printhymnnote}{
    \noindent\printnote{Hymn.~\emph{Ad Régias Agni Dapes}, page \pageref{hymn-adregiasagnidapes}.
    \Vbar~\emph{Mane nobíscum}, page \pageref{vr-manenobiscum}.}
}

{
\section{Low Sunday}
\label{easter1}
\printcommonvespers[1]
\def\begincollectcols{\begin{parcolumns}[rulebetween,colwidths={1=0.42\linewidth}]{2}}
\printvespersmag[../TimeAfterEaster]{inc-VespersMagnificatEaster1}

\def\commemorations{If the Feast of the Annunciation has been transferred to the Monday following Low Sunday, First Vespers is commemorated as on page \pageref{annunciation-commem}.  If today is April 30, May 1, or May 2, First Vespers of St Joseph the Worker is commemorated as follows.}
\printcommemnote{}
}

\medskip
\hrule
{
\label{stjoseph-worker-commem}
\def\begincollectcols{\begin{parcolumns}[rulebetween,colwidths={1=0.43\linewidth}]{2}}
\def\vrlinebreak{T}
\printcommemoration[../May1-StJosephWorker]{commemorationStJosephWorker-Vespers1}

\bigskip
\benedicamusdomino{}
}

{
\section{\nth{2} Sunday after Easter}
\label{easter2}
\printcommonvespers{}
\def\precollect{\printvrmanenobiscum}
\printvespersmag[../TimeAfterEaster]{inc-VespersMagnificatEaster2}
\benedicamusdomino{}
}
{
\section{\nth{3} Sunday after Easter}
\label{easter3}
\printcommonvespers{}
\def\begincollectcols{\begin{parcolumns}[rulebetween,colwidths={1=0.44\linewidth}]{2}}
\def\precollect{\printvrmanenobiscum}
\printvespersmag[../TimeAfterEaster]{inc-VespersMagnificatEaster3}
\benedicamusdomino{}
}

{
\section{\nth{4} Sunday after Easter}
\label{easter4}
\printcommonvespers{}
\def\precollect{\printvrmanenobiscum}
\printvespersmag[../TimeAfterEaster]{inc-VespersMagnificatEaster4}
\benedicamusdomino{}
}

{
\section{\nth{5} Sunday after Easter}
\label{easter5}
\printcommonvespers{}
\def\precollect{\printvrmanenobiscum}
\printvespersmag[../TimeAfterEaster]{inc-VespersMagnificatEaster5}
\benedicamusdomino{}
}

{
\section{Ascension of Our Lord}
\label{ascension}
\subtitle{\nth{1} Class, White or Gold}
\vspace{-0.5\baselineskip}
\subtitle{I \& II Vespers}
\vspace{-0.5\baselineskip}

\def\premagverses{\greseteolcustos{manual}}
\def\begincollectcols{%\vspace{-0.5\baselineskip}
\begin{parcolumns}[rulebetween,colwidths={1=0.46\linewidth}]{2}}
\def\definevesperspropers{\newcommand{\maganttex}{an--o_rex_gloriae--solesmes}
\newcommand{\magantinitial}{O}
\newcommand{\maganttranslation}{O King of glory, Lord of hosts, who today didst ascend in triumph above all heavens, leave us not orphans, but send us the promise of the Father, the Spirit of truth, alleluia.}
\def\magsolemn{T}
\definemag{2}{D}

    \def\prepsalmfive{\greseteolcustos{manual}}
}
\def\definevesperspropersalt{\newcommand{\maganttex}{MagnificatAntiphon1}
\newcommand{\magantinitial}{P}
\newcommand{\maganttranslation}{Father, I have manifested Thy Name unto the men whom Thou gavest Me; and now I pray for them, not for the world, because I come to Thee, alleluia.}
\def\magsolemn{T}
\definemag{6}{F}

    %\def\premagverses{\pagebreak}
}
\def\vesperspropersnote{At II Vespers:}
\def\vesperspropersaltnote{At I Vespers:}
\let\printhymnnote=\undefined
\def\hymnlabel{hymn-salutishumanaesator}
\printvespers[../Ascension]{inc-Ascension}

\def\commemorations{If today is April 30 or May 1, First Vespers of St Joseph the Worker is commemorated as on page \pageref{stjoseph-worker-commem}.}
\printcommemnote[1]{}
\medskip
\hrule
}

{
\section{Sunday after the Ascension}
\label{easter6}\label{sundayafterascension}
\printcommonvespers{}
%\let\printhymnnote=\undefined
\renewcommand{\printhymnnote}{
    \noindent\printnote{Hymn.~\emph{Salútis humánæ Sator}, page \pageref{hymn-salutishumanaesator}.}
    \def\vrlinebreak{T}
    \printvr[\greseteolcustos{manual}]{\vrtex}{\vtranslation}{\rtranslation}
}
\def\precollect{\printvrmanenobiscum}
%\def\premagverses{\pagebreak}
\printvespersmag[../TimeAfterEaster]{inc-VespersMagnificatSundayAfterAscension}
\benedicamusdomino{}
}
}
} % need to add St Joseph the Worker and the Ascension

\chapter{Proper of the Time -- Pentecost Octave}
{
\newcommand{\benedicamusdomino}[1][1]{
  \benedicamusdominomaster{#1}
}
{
\section{Pentecost Sunday}
\subtitle{\nth{1} Class}
\subtitle{I \& II Vespers}

\def\deusinadjutoriumsolemn{T}
\def\definevesperspropers{\definepsalm{5}{113}{7}{c2}

\newcommand{\vrtex}{vrLoquebantur}
\newcommand{\vtranslation}{The apostles spoke in divers tongues.}
\newcommand{\rtranslation}{The wonderful works of God.}

\newcommand{\maganttex}{MagnificatAntiphon2}
\newcommand{\magantinitial}{H}
\newcommand{\maganttranslation}{Today the days of Pentecost are complete, alleluia; today the Holy Ghost appeared in fire to the disciples, gave them gifts and graces, sent them into all the world to preach and to bear witness; whoever believes and is baptised shall be saved, alleluia.}

	%\def\prepsalmfivetitle{\medskip}
	\def\prepsalmfive{\greseteolcustos{manual}}
}
\def\definevesperspropersalt{\definepsalm{5}{116}{7}{c2}

\newcommand{\vrtex}{vrRepletiSunt}
\newcommand{\vtranslation}{They were all filled with the Holy Ghost.}
\newcommand{\rtranslation}{And they began to speak.}

\newcommand{\maganttex}{MagnificatAntiphon1}
\newcommand{\magantinitial}{N}
\newcommand{\maganttranslation}{I will not leave you orphans, alleluia; I go away, and I come unto you, alleluia; and your heart shall rejoice, alleluia.}
%
%\let\oldthing=\maganttranslation
%\def\maganttranslation{\oldthing\vspace{-0.5\baselineskip}}
%\def\postmagtitle{\vspace{-\baselineskip}}%
}%
%
%\def\prepsalmtitleone{\vspace{-0.5\baselineskip}}
%\def\prepsalmtitlethree{\vspace{-0.5\baselineskip}}
\ifthenelse{\boolean{birmingham}}{}{
	\def\prepsalmtitleone{\bigskip{}}
	\def\postpsalmtitleone{\bigskip{}}
}
\def\preantthree{\bigskip}
\def\prepsalmtitlefour{\needspace{4\baselineskip}}
\def\vesperspropersnote{At II Vespers:}
\def\vesperspropersaltnote{At I Vespers:}

%\def\premag{\def\noeuouae{T}}
\def\premagverses{\greseteolcustos{manual}}
\def\prehymn{\printnote{All kneel for the first stanza of the following hymn.}}
\def\prehymntranslation{\vspace{-0.3\baselineskip}}
\def\prevespers{%
%	\let\oldthing\antthreetranslation
%	\def\antthreetranslation{\oldthing\vspace{-\baselineskip}}%
	%\let\oldthingb\antfivetranslation
	%\def\antfivetranslation{\oldthingb\vspace{-\baselineskip}}%
}
%\def\prevr{\vspace{-0.7\baselineskip}}
\printvespers[../Pentecost]{inc-PentecostVespers}
\bigskip
\benedicamusdomino{}
}

{
\section{Trinity Sunday}
\subtitle{\nth{1} Class}

\def\deusinadjutoriumsolemn{T}
%\def\premag{\def\noeuouae{T}}
%\def\prepsalmtwoverses{\vspace{-0.05\baselineskip}}
%\def\prepsalmtitleone{\needspace{10\baselineskip}}
\def\prepsalmone{\needspace{10\baselineskip}}

\def\prerepeatantiphontwo{}
\def\preantfive{\bigskip}
\newcommand{\psalmcolsoverrideoverride}[1][0]{
	\psalmcolsoverride[#1]
	\ifnum#1=110
		\def\beginpsalmcols{\begin{parcolumns}[rulebetween,colwidths={1=0.49\linewidth},distance=1.3em]{2}}
	\fi
}
%\def\prepsalmtitlethree{\vspace{-0.5\baselineskip}}
%\def\prerepeatantiphonthree{\vspace{-0.25\baselineskip}}
\def\premagnificat{\needspace{12\baselineskip}}
\def\premagverses{\greseteolcustos{manual}}
\def\prehymn{\oldneedspace{16\baselineskip}}
\def\begincollectcols{\begin{parcolumns}[rulebetween,colwidths={1=0.44\linewidth}]{2}}
\printvespers[../TrinitySunday]{inc-TrinitySunday-Vespers}
\bigskip
\noindent
\printnote{If the feast of St Philip Neri is celebrated tomorrow, then \emph{First Vespers} is commemorated as follows.
\ifthenelse{\boolean{includebenedicamusdominoreferences}}{
	Otherwise \benedicamusdomino{}%
}{}
}
\medskip
%\hrule
%\medskip
}

% commemoration of First Vespers of Immaculate Conception
{
\def\beginvrcols{\begin{parcolumns}[rulebetween,colwidths={1=0.45\linewidth}]{2}}
\def\vrlinebreak{T}
\printcommemoration[../StPhilipNeri]{Commemoration-StPhilipNeri-Vespers1}

%\bigskip
\benedicamusdomino{}
}

}


\clearpage
\chapter{Proper of the Saints}
{
\newcommand{\benedicamusdomino}[1][1]{
  \noindent\printnote{\Vbar~\emph{Benedicámus Dómino \IfInteger{#1}{#1}{\csname benedicamusdominoname#1\endcsname}}, p.~\pageref{benedicamusdomino-#1}.}
  \bigskip
  \hrule
}

%December 8: Immaculate Conception
%Second Vespers of Immaculate Conception
{
\label{immaculateconception}
\section{December 8: Immaculate Conception}
\subtitle{\nth{1} Class}
\subtitle{I \& II Vespers}

\def\premagverses{\greseteolcustos{manual}}
\def\definevesperspropersalt{\def\noeuouae{T}\newcommand{\maganttex}{MagnificatAntiphon1}
\newcommand{\magantinitial}{B}
\newcommand{\maganttranslation}{All generations shall call me blessed, because he that is mighty, hath done great things for me, alleluia.}
\def\magpsalmclef{c3}
\definemag{8}{G}
}
\def\definevesperspropers{\def\noeuouae{T}\newcommand{\maganttex}{MagnificatAntiphon2}
\newcommand{\magantinitial}{H}
\newcommand{\maganttranslation}{This day a rod came forth from the root of Jesse: this day Mary was conceived without any stain of sin: this day the head of the old serpent was crushed by her.  Alleluia.}
\definemag{1}{f}
}
\def\vesperspropersaltnote{At I Vespers:}
\def\vesperspropersnote{At II Vespers:}
\def\prehymn{\printnote{All kneel for the first stanza of the following hymn.}}
\def\hymnlabel{hymn-avemarisstella}
\def\vrlinebreak{F}
\printvespers[../December8-ImmaculateConception]{inc-ImmaculateConceptionVespers}
}

{
\bigskip

\bigskip
\noindent
\printnote{Then follows a Commemoration of the Advent Sunday or Feria according to the day of the week on which the Feast of the Immaculate Conception falls.}
\bigskip
}

{
  \oldneedspace{5\baselineskip}
  \subtitle{First Week of Advent.}
  \vspace{-\baselineskip}
  \subtitle{\small{Thursday.}}
  {
  \def\noeuouae{T}
  \printgabc{At Magn.}{\oldstylenums{Ant.~4.}}{E}{../Advent1/MagAntThursday-Exspectabo}
  }
  \translation[]{I will look for the Lord my Saviour, and await Him, while He is near, alleluia.}
  \medskip

  \oldneedspace{5\baselineskip}
  \vspace{-\baselineskip}
  \subtitle{\small{Friday.}}
  {
  \def\noeuouae{T}
  \printgabc{At Magn.}{\oldstylenums{Ant.~4.}}{E}{../Advent1/MagAntFriday-ExAegypto}
  }
  \translation[]{Out of Egypt I have called my Son; he shall come to save His people.}
  \medskip

  \oldneedspace{5\baselineskip}
  \subtitle{Second Week of Advent.}
  \vspace{-\baselineskip}
  \subtitle{\small{Saturday.}}
  {
  \def\noeuouae{T}
  \printgabc{At Magn.}{\oldstylenums{Ant.~7.}}{V}{../Advent1/MagAntSaturday-VeniDomine}
  }
  \translation[]{Come, Lord, to visit us in peace, that we may rejoice before Thee with a perfect heart.}
  \medskip

  \oldneedspace{5\baselineskip}
  \vspace{-\baselineskip}
  \subtitle{\small{Sunday.}}
  {
  \def\noeuouae{T}
  \printgabc{At Magn.}{\oldstylenums{Ant.~8.~G}}{T}{../Advent2/MagnificatAntiphon-noEuouae}
  }
  \translation[]{Art Thou He that art to come, or look we for another? Relate to John what you have seen: The blind recover their sight, the dead rise again, the poor have the Gospel preached to them, alleluia.}
  \medskip

  \oldneedspace{5\baselineskip}
  \vspace{-\baselineskip}
  \subtitle{\small{Monday.}}
  {
  \def\noeuouae{T}
  \printgabc{At Magn.}{\oldstylenums{Ant.~4.}}{E}{../Advent2/MagAntMonday-EcceRexVeniet}
  }
  \translation[]{Behold, the King shall come, the Lord of the land; and He shall take away the yoke of our captivity.}
  \medskip

  \oldneedspace{5\baselineskip}
  \vspace{-\baselineskip}
  \subtitle{\small{Tuesday.}}
  {
  \def\noeuouae{T}
  \printgabc{At Magn.}{\oldstylenums{Ant.~5.}}{V}{../Advent2/MagAntTuesday-VoxClamantis}
  }
  \translation[]{A voice of one crying in the desert, Prepare ye the way of the Lord, make straight His paths}
  \medskip

  \oldneedspace{5\baselineskip}
  \vspace{-\baselineskip}
  \subtitle{\small{Wednesday.}}
  {
  \def\noeuouae{T}
  \printgabc{At Magn.}{\oldstylenums{Ant.~4.}}{S}{../Advent2/MagAntWednesday-Sion}
  }
  \translation[]{Sion, thou shalt be restored, and shalt see the Just One who shall appear in thee.}
  \medskip

  \oldneedspace{5\baselineskip}
  \vspace{-\baselineskip}
  \subtitle{\small{Thursday.}}
  {
  \def\noeuouae{T}
  \printgabc{At Magn.}{\oldstylenums{Ant.~4.}}{Q}{../Advent2/MagAntThursday-QuiPostMeVenit}
  }
  \translation[]{He that shall come after me is preferred before me; whose shoes I am not worthy to loose.}
  \medskip
  \hrule
  \medskip
  {
      \newcommand{\commvlatin}{Roráte cæli désuper, et nubes pluant \textbf{ju}stum.}
      \newcommand{\commrlatin}{Aperiátur terra, et gérminet Salva\textbf{tó}rem.}
      \newcommand{\commvtranslation}{Ye heavens, drop down dew from above, and let the clouds rain down the Just One.}
      \newcommand{\commrtranslation}{Let the earth open and bud forth the Saviour.}
  \printvrcommem{}
  }

  \oldneedspace{3\baselineskip}
  \begin{center}{\large Collect.}\end{center}
  \vspace{-\baselineskip}
  \def\printcollectheading{F}
  {
  \begin{center}{First Week of Advent.}\end{center}
  \def\gabcfolder{../Advent1}
  \newcommand{\antonetex}{Ant1-InIllaDie}
\newcommand{\antoneinitial}{I}
\newcommand{\antonetranslation}{In that day the mountains shall drop down sweetness, and the hills shall flow with milk and honey, alleluia.}
\definepsalm{1}{109}{8}{G}

\newcommand{\anttwotex}{Ant2-Jucundare}
\newcommand{\anttwoinitial}{J}
\newcommand{\anttwotranslation}{Shout for joy, O daughter of Sion, rejoice greatly, O daughter of Jerusalem, alleluia.}
\definepsalm{2}{110}{8}{G*}

\newcommand{\antthreetex}{Ant3-EcceDominusVeniet}
\newcommand{\antthreeinitial}{E}
\newcommand{\antthreetranslation}{Behold, the Lord shall come, and all His Saints with Him: and there shall be in that day a great light, alleluia.}
\definepsalm{3}{111}{5}{a}

\newcommand{\antfourtex}{Ant4-Omnes}
\newcommand{\antfourinitial}{O}
\newcommand{\antfourtranslation}{All ye that thirst come to the waters: seek the Lord while He can be found, alleluia.}
\definepsalm{4}{112}{7}{c}

\newcommand{\antfivetex}{Ant5-EcceVeniet}
\newcommand{\antfiveinitial}{E}
\newcommand{\antfivetranslation}{Behold there shall come a great Prophet, and He shall renew Jerusalem, alleluia.}
\definepsalm{5}{113}{4}{A*}

\newcommand{\chaptertext}{\dropcap{latin}{Fratres~: Hora est jam nos de somno} \textbf{súr}\-ge\-re~:~\gredagger{} nunc enim própior est \emph{no\-stra} \textbf{sá}\-lus,~* quam cum cre\-\textbf{dí}\-dimus.}
\newcommand{\chaptertranslation}{Brethren: it is now the hour for us to rise from sleep.  For now our salvation is nearer than when we believed.}

\newcommand{\magantinitial}{N}
\newcommand{\maganttex}{MagnificatAntiphon}
\newcommand{\maganttranslation}{Fear not, Mary, for thou hast found grace with the Lord: behold thou shalt conceive and bring forth a son, alleluia.}
\def\magsolemn{F}
\definemag{8}{G}

\newcommand{\collect}{Excita, quǽsumus Dómine, poténtiam tuam, et veni~:~† ut ab imminéntibus peccatórum nostrórum perículis, te mereámur protegénte éripi,~* te liberánte salvári.  Qui vivis et regnas cum Deo Patre in unitáte Spíritus Sancti Deus~:~* per ómnia sǽcula sæculórum.}
\newcommand{\collecttranslation}{Stir up Thy power, we beseech Thee, O Lord, and come: that from the threatening dangers of our sins we may deserve to be rescued by Thy protection, and to be saved by Thy deliverance: Who livest and reignest with God the Father in the unity of the Holy Ghost, world without end.}

  \printcollect{\collect}{\collecttranslation}
  }
  {
  \medskip
  \begin{center}{Second Week of Advent.}\end{center}
  \def\gabcfolder{../Advent2}
  % !TEX TS-program = lualatex
% !TEX encoding = UTF-8

% This is a simple template for a LuaLaTeX document using gregorio scores.

\newcommand{\comheadingtext}{Commemoration of 2nd Sunday of Advent}

\newcommand{\latincomcollect}{Excita Dómine corda nostra ad præparándas Unigéniti tui vias~:~† ut per ejus advéntum,~* purificátis tibi méntibus servíre mereámur. Qui tecum vivit et regnat.}
\newcommand{\englishcomcollect}{Stir up our hearts, O Lord, to prepare the ways of Thine only-begotten Son: that through His coming we may deserve to serve Thee with purified minds: Who with Thee liveth and reigneth.}

\newcommand{\englishcommagantiphon}{Art Thou He that art to come, or look we for another? Relate to John what you have seen: The blind recover their sight, the dead rise again, the poor have the Gospel preached to them, alleluia.}

\newcommand{\commagantlinetwo}{Ant. 8. G}
\newcommand{\commaganttex}{MagnificatAntiphon-noEuouae}
\newcommand{\commagantinitial}{T}
\newcommand{\commagantinitialsize}{35}

\newcommand{\commvrtex}{../Advent/vr-commemoration}
\newcommand{\commvtranslation}{Ye heavens, drop down dew from above, and let the clouds rain down the Just One.}
\newcommand{\commrtranslation}{Let the earth open and bud forth the Saviour.}

\setgrefactor{17}
  \printcollect{\latincomcollect}{\englishcomcollect}
  }

  \bigskip
  \benedicamusdomino{}
}


%Purification & Presentation (2nd class no commem of Sunday because it is a feast of the Lord)
{
\section{February 2: Purification \& Presentation}
\subtitle{\nth{2} Class}
\subtitle{I \& II Vespers}
\printnote{At I Vespers, Psalms and Antiphons of the Circumcision, p.~\pageref{circumcision}, continuing with the chapter on p.~\pageref{purification-chapter}.}

\def\definevesperspropers{\newcommand{\maganttex}{an--hodie_beata_virgo--solesmes}
\newcommand{\magantinitial}{H}
\newcommand{\maganttranslation}{Today did the Blessed Virgin Mary present the Child Jesus in the temple; and Simeon, filled with the Holy Ghost, took Him up in his arms, and blessed God for ever.}
\def\magpsalmclef{c3}
\definemag{8}{G*}

  \def\prepsalmfive{\greseteolcustos{manual}}
}
\def\definevesperspropersalt{\newcommand{\maganttex}{an--senex_puerum--solesmes}
\newcommand{\magantinitial}{S}
\newcommand{\maganttranslation}{The old man carried the Child, but the Child led the old man. The Virgin bore the Child, and after child-bearing was virgin still: Whom she bore, Him she adored.}
\definemag{1}{D}
}
\def\vesperspropersnote{At II Vespers:}
\def\vesperspropersaltnote{At I Vespers:}
%\def\premag{\def\noeuouae{T}}
\def\premagverses{\greseteolcustos{manual}}
\def\prechapter{\label{purification-chapter}}
\def\printhymnnote{
  {
    \oldneedspace{3\baselineskip}
    \printnote{Hymn.~\emph{Ave Maris Stella}, p.~\pageref{hymn-avemarisstella}.\\}
    %
    % \def\vrlinebreak{T}
    % \oldneedspace{3\baselineskip}
    % \printvr[\greseteolcustos{manual}]{\vrtex}{\vtranslation}{\rtranslation}
  }
}
\printvespers[../February2-PurificationOfBlessedVirginMary]{inc-purification}
\bigskip
\benedicamusdomino[2]{}
}

%May 1: St Joseph the Worker (1st class)
{
\section{May 1: St Joseph the Worker}
\subtitle{\nth{1} Class}
\subtitle{I \& II Vespers}

\def\definevesperspropers{\newcommand{\vrtex}{vrOraProNobis}
\newcommand{\vtranslation}{Pray for us, St Joseph, alleluia.}
\newcommand{\rtranslation}{Faithful protector of our labors, alleluia.}

\newcommand{\maganttex}{an--et_ipse_jesus--solesmes}
\newcommand{\magantinitial}{E}
\newcommand{\maganttranslation}{}
\definemag{7}{d}

  \def\prepsalmfive{\greseteolcustos{manual}}
}
\def\definevesperspropersalt{\newcommand{\vrtex}{vrSolemnitasEstHodie}
\newcommand{\vtranslation}{Today is the solemnity of St Joseph, alleluia.}
\newcommand{\rtranslation}{Who ministered with his hands to the Son of God, alleluia.}

\newcommand{\maganttex}{an--christus_dominus--solesmes}
\newcommand{\magantinitial}{C}
\newcommand{\maganttranslation}{}
\definemag{7}{c2}

  \def\vrlinebreak{F}
}
\def\vesperspropersnote{At II Vespers:}
\def\vesperspropersaltnote{At I Vespers:}
%\def\premag{\def\noeuouae{T}}
\def\premagverses{\greseteolcustos{manual}}

\printvespers[../May1-StJosephWorker]{inc-StJosephWorker}
%if feast of St Joseph the worker falls from 2nd through 5th Sunday after Easter, it outranks the Sunday and the Sunday is commemorated
\medskip
\printnote{If today is Sunday, the Vespers of the Sunday is commemorated with \emph{Magnificat antiphon}, \emph{\Vbar{} Mane nobíscum.} in simple commemoration tone, p.~\pageref{vr-manenobiscum} and \emph{Collect}.  Otherwise \Vbar~\emph{Benedicámus Dómino 1}, p.~\pageref{benedicamusdomino-1}.

\begin{multicols}{2}
\noindent\emph{\nth{2} Sunday after Easter}, p.~\pageref{easter2}.\\
\emph{\nth{3} Sunday after Easter}, p.~\pageref{easter3}.\\
\emph{\nth{4} Sunday after Easter}, p.~\pageref{easter4}.\\
\emph{\nth{5} Sunday after Easter}, p.~\pageref{easter5}.
\end{multicols}
}
% \bigskip
% \benedicamusdomino{}
}

%June 24: Nativity of St John the Baptist (1st class)
{
\section{June 24: Nativity of St John the Baptist}
\subtitle{\nth{1} Class}
\subtitle{I \& II Vespers}

\def\definevesperspropers{\newcommand{\antonetex}{Ant1-ElisabethZachariae}
\newcommand{\antoneinitial}{E}
\newcommand{\antonetranslation}{Elizabeth, the wife of Zacharias, gave birth to a great man, John the Baptist, the forerunner of the Lord.}
\definepsalm{1}{109}{3}{a}

\newcommand{\anttwotex}{Ant2-Innuebant}
\newcommand{\anttwoinitial}{I}
\newcommand{\anttwotranslation}{They made signs unto his father, by what name he should be called: and he wrote, saying: His name is John.}
\definepsalm{2}{110}{4}{E}

\newcommand{\antthreetex}{Ant3-JoannesVocabitur}
\newcommand{\antthreeinitial}{J}
\newcommand{\antthreetranslation}{His name shall be called John, and many shall rejoice in his birth.}
\definepsalm{3}{111}{1}{f}

\newcommand{\antfourtex}{Ant4-InterNatos}
\newcommand{\antfourinitial}{I}
\newcommand{\antfourtranslation}{Among those born of women, there hath not risen a greater than John the Baptist.}
\definepsalm{4}{112}{3}{b}

\newcommand{\antfivetex}{Ant5-TuPuer}
\newcommand{\antfiveinitial}{T}
\newcommand{\antfivetranslation}{Thou, child, shalt be called the Prophet of the Most High: thou shalt go before the Lord to prepare His ways.}
\definepsalm{5}{116}{3}{b}

\newcommand{\vrtex}{vrIstePuerMagnus}
\newcommand{\vtranslation}{This child is great before the Lord.}
\newcommand{\rtranslation}{For in truth His hand is with him.}

\newcommand{\maganttex}{MagnificatAntiphon2}
\newcommand{\magantinitial}{P}
\newcommand{\maganttranslation}{The child that is born to us is more than a prophet; for this is he of whom the Saviour said: Among those born of women there hath not risen a greater than John the Baptist.}
\def\magsolemn{T}
\definemag{7}{d}

  \def\prepsalmfive{\greseteolcustos{manual}}
}
\def\definevesperspropersalt{\newcommand{\antonetex}{Ant1-IpsePraeibit}
\newcommand{\antoneinitial}{I}
\newcommand{\antonetranslation}{He shall go before Him in the spirit and power of Elias, to prepare unto the Lord a perfect people.\vspace{0ex plus 0ex minus 3ex}}
\definepsalm{1}{109}{7}{a}

\newcommand{\anttwotex}{Ant2-Joannes}
\newcommand{\anttwoinitial}{J}
\newcommand{\anttwotranslation}{John is his name.  Wine and strong drink he shall not drink, and many shall rejoice in his birth.}
\definepsalm{2}{110}{8}{G}

\newcommand{\antthreetex}{Ant3-ExUteroSenectutis}
\newcommand{\antthreeinitial}{E}
\newcommand{\antthreetranslation}{From the barren womb of age was born John, the forerunner of the Lord.}
\definepsalm{3}{111}{1}{f}

\newcommand{\antfourtex}{Ant4-IstePuer}
\newcommand{\antfourinitial}{I}
\newcommand{\antfourtranslation}{This child is great before the Lord, for the hand of God is with him.}
\definepsalm{4}{112}{4}{A*}

\newcommand{\antfivetex}{Ant5-Nazaraeus}
\newcommand{\antfiveinitial}{N}
\newcommand{\antfivetranslation}{This child shall be called a Nazarite; wine and strong drink he shall not drink, and from his mother's womb he shall eat nothing unclean.}
\definepsalm{5}{116}{5}{a}

\newcommand{\vrtex}{vrFuitHomo}
\newcommand{\vtranslation}{There was a man sent from God.}
\newcommand{\rtranslation}{Whose name was John.}

\newcommand{\maganttex}{MagnificatAntiphon1}
\newcommand{\magantinitial}{I}
\newcommand{\maganttranslation}{When Zacharias had entered the temple of the Lord, there appeared to him the Angel Gabriel, standing at the right hand of the altar of incense.\vspace{-4pt plus 4pt}}
\def\magpsalmclef{c3}
\def\magsolemn{T}
\definemag{8}{G}
}
\def\vesperspropersnote{At II Vespers:}
\def\vesperspropersaltnote{At I Vespers:}
\def\prevesperspsalms{\noindent\printnote{Chapter and following, page \pageref{june24-chapter}.\\}}
\def\vesperspsalmslabel{\label{june24-2vespers}}
\def\prevesperspsalmsalt{\noindent\printnote{II Vespers psalms and antiphons, page \pageref{june24-2vespers}.}\medskip}
\def\prechapter{\label{june24-chapter}}
%\def\premag{\def\noeuouae{T}}
\def\premagverses{\greseteolcustos{manual}}

\printvespers[../June24-BirthOfJohnTheBaptist]{inc-BirthOfJohnTheBaptist}

\medskip
\printnote{If today is Sunday, the Vespers of the Sunday is commemorated with \emph{Magnificat antiphon}, \emph{\Vbar{} Dirigátur.} in simple commemoration tone, p.~\pageref{vr-dirigatur} and \emph{Collect}.  Otherwise \Vbar~\emph{Benedicámus Dómino 1}, p.~\pageref{benedicamusdomino-1}.

\begin{multicols}{2}
\noindent\emph{\nth{2} Sunday after Pentecost}, p.~\pageref{pentecost2}.\\
\emph{\nth{3} Sunday after Pentecost}, p.~\pageref{pentecost3}.\\
\emph{\nth{4} Sunday after Pentecost}, p.~\pageref{pentecost4}.\\
\emph{\nth{5} Sunday after Pentecost}, p.~\pageref{pentecost5}.\\
\emph{\nth{6} Sunday after Pentecost}, p.~\pageref{pentecost6}.
\end{multicols}
}
% \bigskip
% \benedicamusdomino{}
}

%June 29: Sts Peter \& Paul (1st class)
{
\section{June 29: Sts Peter \& Paul}
\subtitle{\nth{1} Class}
\subtitle{I \& II Vespers}

\def\definevesperspropers{\import{../CommonOfApostles/}{inc-CommonOfApostles-2Vespers-psalms}
\edef\antonetex{../CommonOfApostles/\antonetex}
\edef\anttwotex{../CommonOfApostles/\anttwotex}
\edef\antthreetex{../CommonOfApostles/\antthreetex}
\edef\antfourtex{../CommonOfApostles/\antfourtex}
\edef\antfivetex{../CommonOfApostles/\antfivetex}

\newcommand{\vrtex}{vrAnnuntiaverunt}
\newcommand{\vtranslation}{They declared the works of God.}
\newcommand{\rtranslation}{And understood His doings.}

\newcommand{\maganttex}{MagnificatAntiphon2-Hodie}
\newcommand{\magantinitial}{H}
\newcommand{\maganttranslation}{Today, Simon Peter went up upon the gibbet of the cross, alleluia; today, he that holdeth the keys of the kingdom, departed with joy to be with Christ; today, the Apostle Paul, the light of the world, bowing his head, for Christ's sake was crowned with martyrdom, alleluia.}
\def\magsolemn{T}
\definemag{1}{D}

  \def\prepsalmfive{\greseteolcustos{manual}}
}
\def\definevesperspropersalt{\newcommand{\antonetex}{Ant1-PetrusEtJoannes}
\newcommand{\antoneinitial}{P}
\newcommand{\antonetranslation}{Peter and John went up together into the Temple at the hour of prayer, being the ninth hour.}
\definepsalm{1}{109}{8}{G}

\newcommand{\anttwotex}{Ant2-Argentum}
\newcommand{\anttwoinitial}{A}
\newcommand{\anttwotranslation}{Silver and gold have I none, but such as I have, give I thee.}
\definepsalm{2}{110}{7}{b}

\newcommand{\antthreetex}{Ant3-DixitAngelusAdPetrum}
\newcommand{\antthreeinitial}{D}
\newcommand{\antthreetranslation}{The Angel said unto Peter: Cast thy garment about thee, and follow me.}
\definepsalm{3}{111}{8}{c}

\newcommand{\antfourtex}{Ant4-MisitDominus}
\newcommand{\antfourinitial}{M}
\newcommand{\antfourtranslation}{The Lord hath sent His Angel, and hath delivered me out of the hand of Herod.  Alleluia.}
\definepsalm{4}{112}{7}{c2}

\newcommand{\antfivetex}{Ant5-TuEsPetrus}
\newcommand{\antfiveinitial}{T}
\newcommand{\antfivetranslation}{Thou art Peter and upon this Rock I will build My Church.}
\definepsalm{5}{116}{7}{c}

\newcommand{\vrtex}{vrInOmnemTerram}
\newcommand{\vtranslation}{Their sound hath gone forth into all the earth.}
\newcommand{\rtranslation}{And their words unto the ends of the world.}

\newcommand{\maganttex}{MagnificatAntiphon1}
\newcommand{\magantinitial}{T}
\newcommand{\maganttranslation}{Thou art the Shepherd of the sheep and the Prince of the Apostles, and unto thee are given the keys of the kingdom of heaven.}
\def\magsolemn{T}
\definemag{1}{f}
}
\def\vesperspropersnote{At II Vespers:}
\def\vesperspropersaltnote{At I Vespers:}
\def\prevesperspsalms{\noindent\printnote{Chapter and following, page \pageref{june29-chapter}.\\}}
\def\vesperspsalmslabel{\label{june29-2vespers}}
\def\prevesperspsalmsalt{\noindent\printnote{II Vespers psalms and antiphons, page \pageref{june29-2vespers}.}\medskip}
\def\prechapter{\label{june29-chapter}}
%\def\premag{\def\noeuouae{T}}
\def\premagverses{\greseteolcustos{manual}}

\def\begincollectcols{\begin{parcolumns}[rulebetween,colwidths={1=0.45\linewidth}]{2}}
\printvespers[../June29-StsPeterAndPaul]{inc-StsPeterAndPaul}

\medskip
\printnote{If today is Sunday, the Vespers of the Sunday is commemorated with \emph{Magnificat antiphon}, \emph{\Vbar{} Dirigátur.} in simple commemoration tone, p.~\pageref{vr-dirigatur} and \emph{Collect}.  Otherwise \Vbar~\emph{Benedicámus Dómino 1}, p.~\pageref{benedicamusdomino-1}.

\begin{multicols}{2}
\noindent\emph{\nth{3} Sunday after Pentecost}, p.~\pageref{pentecost3}.\\
\emph{\nth{4} Sunday after Pentecost}, p.~\pageref{pentecost4}.\\
\emph{\nth{5} Sunday after Pentecost}, p.~\pageref{pentecost5}.\\
\emph{\nth{6} Sunday after Pentecost}, p.~\pageref{pentecost6}.\\
\emph{\nth{7} Sunday after Pentecost}, p.~\pageref{pentecost7}.
\end{multicols}
}
% \bigskip
% \benedicamusdomino{}
}

%July 1: Most Precious Blood (1st class)
{
\global\let\psalmclefthree=\undefined
\section{July 1: The Precious Blood of Our Lord Jesus Christ}
\subtitle{\nth{1} Class}
\subtitle{I \& II Vespers}

\def\definevesperspropers{\definepsalm{5}{147}{2}{D}

\newcommand{\vrtex}{vrTeErgo}
\newcommand{\vtranslation}{We therefore pray Thee help Thy servants.}
\newcommand{\rtranslation}{Whom Thou hast redeemed with Thy precious Blood.}

\newcommand{\maganttex}{MagnificatAntiphon2}
\newcommand{\magantinitial}{H}
\newcommand{\maganttranslation}{\vspace{-16pt plus 16pt}Ye shall observe this day for a memorial: and ye shall keep it holy unto the Lord, in your generations with an everlasting worship.\vspace{-4pt plus 4pt}}
\newcommand{\magsolemn}{F}
\definemag{1}{D2}

  \def\prepsalmfive{\greseteolcustos{manual}}
}
\def\definevesperspropersalt{\definepsalm{5}{116}{2}{D}

\newcommand{\vrtex}{vrRedemistiNos}
\newcommand{\vtranslation}{Thou hast redeemed us, O Lord, in Thy Blood.}
\newcommand{\rtranslation}{And hast made of us a kingdom unto our God.}

\newcommand{\maganttex}{MagnificatAntiphon1}
\newcommand{\magantinitial}{A}
\newcommand{\maganttranslation}{Ye are come to Mount Sion, to the city of the living God, the heavenly Jerusalem, and to Jesus the Mediator of the new Testament, and to the sprinkling of blood which speaketh better than that of Abel.}
\newcommand{\magsolemn}{F}
\definemag{3}{a}
}
\def\vesperspropersnote{At II Vespers:}
\def\vesperspropersaltnote{At I Vespers:}
%\def\premag{\def\noeuouae{T}}
\def\premagverses{\greseteolcustos{manual}}

\def\begincollectcols{\begin{parcolumns}[rulebetween,colwidths={1=0.45\linewidth}]{2}}
\printvespers[../July1-MostPreciousBloodOfChrist]{inc-MostPreciousBloodOfChrist}
\bigskip
\benedicamusdomino{}
}

%Aug 6: Transfiguration (2nd class)
{
\section{August 6: Transfiguration of Our Lord Jesus Christ}
\subtitle{\nth{2} Class}
\subtitle{I \& II Vespers}

\def\definevesperspropers{\newcommand{\maganttex}{MagnificatAntiphon2}
\newcommand{\magantinitial}{E}
\newcommand{\maganttranslation}{And the disciples hearing, fell on their faces, and were sore afraid; and Jesus came, and touched them, and said to them, Arise, and fear not, alleluia.}
\newcommand{\magsolemn}{F}
\definemag{8}{G}

  \def\prepsalmfive{\greseteolcustos{manual}}
}
\def\definevesperspropersalt{\newcommand{\maganttex}{an--christus_jesus--solesmes}
\newcommand{\magantinitial}{C}
\newcommand{\maganttranslation}{Christ Jesus, radiance of the Father and image of His Being, upholding all things by the word of His power; making atonement for sins, has deigned to appear today in glory on the high mountain.}
\newcommand{\magsolemn}{F}
\definemag{4}{E}
}
\def\vesperspropersnote{At II Vespers:}
\def\vesperspropersaltnote{At I Vespers:}
%\def\premag{\def\noeuouae{T}}
\def\premagverses{\greseteolcustos{manual}}

\def\begincollectcols{\begin{parcolumns}[rulebetween,colwidths={1=0.44\linewidth}]{2}}
\printvespers[../August6-TransfigurationOfOurLord]{inc-Transfiguration}
\bigskip
\benedicamusdomino[2]{}
}

%Aug 15: Assumption (1st class)
{
\section{August 15: Assumption of the B.~V.~M.}
\subtitle{\nth{1} Class}
\subtitle{I \& II Vespers}

\def\definevesperspropers{% hymn is ave maris stella
%\input{inc-hymn-avemarisstella}

\newcommand{\maganttex}{an--hodie_maria_virgo--solesmes}
\newcommand{\magantinitial}{H}
\newcommand{\maganttranslation}{Today the Virgin Mary has gone up to heaven: rejoice, for with Christ she reigns forever.}
\newcommand{\magsolemn}{T}
\definemag{8}{G*}

  \def\prepsalmfive{\greseteolcustos{manual}}
}
\def\definevesperspropersalt{\newcommand{\hymnlinetwo}{2.}
\newcommand{\hymntex}{Hymn-OPrimaVirgoProdita}
\newcommand{\hymninitial}{O}
\newcommand{\hymntranslation}{
\item O Virgin who was first to receive
The Creator’s grace by the spirit,
Who was predestined by the Most High
To bear in her womb the Son.

\item O woman, who was foretold to be
The perpetual enemy of the demon;
Who alone was filled with grace,
Undefiled from conception.

\item Thou who conceives Life itself in thy womb,
Life that was lost by Adam;
Furnishing the divine Victim,
A body for his sacrifice.

\item Death, the recompense for sin,
Had no victory over thee, and now departs;
And then thou hastened bodily to heaven
To be thy loving Son’s companion.

\item Illuminated by so great a Glory,
All nature is raised up;
And in thee is called to reach
The pinnacle of all glory and splendour.

\item In thy triumph O our Queen,
Turn thine eyes to us exiles;
That through thy patronage,
We may come to heaven, our blessed homeland.

\item Praise to the Father! praise to Him,
The Virgin’s holy Son!
Praise to the Spirit Paraclete,
While endless ages run! 
Amen.
}

\newcommand{\maganttex}{MagAntiphon-VirgoPrudentissima}
\newcommand{\magantinitial}{V}
\newcommand{\maganttranslation}{O Virgin most prudent, whither goest thou, like the golden dawn?  Daughter of Sion, thou art all beautiful and sweet; fair as the moon, bright as the sun.}
\newcommand{\magsolemn}{T}
\definemag{1}{f}
}
\def\vesperspropersnote{At II Vespers:}
\def\vesperspropersaltnote{At I Vespers:}
%\def\premag{\def\noeuouae{T}}
\def\premagverses{\greseteolcustos{manual}}
\def\printfullhymn{
  {
    \oldneedspace{3\baselineskip}
    \printnote{At II Vespers: Hymn.~\emph{Ave Maris Stella}, p.~\pageref{hymn-avemarisstella}. \Vbar{} \emph{Exaltata.} p.~\pageref{vr-assumption}.\\}

    \printnote{\vesperspropersaltnote}
    \definevesperspropersalt
    \printhymn{\oldstylenums{\hymnlinetwo}}{\hymninitial}{\hymntex}{\hymntranslation}
  }
  {
    \def\vrlinebreak{T}
    \oldneedspace{3\baselineskip}
    \label{vr-assumption}
    \printvr[\greseteolcustos{manual}]{\vrtex}{\vtranslation}{\rtranslation}
  }
}

\printvespers[../August15-AssumptionOfTheBlessedVirginMary]{inc-Assumption}

%     TODO Feast of Assumption (8/15) could have commem of Saturday before 3rd Sunday of August or 9th to 13th Sunday after Pentecost
\medskip
\printnote{If today is Sunday, the Vespers of the Sunday is commemorated with \emph{Magnificat antiphon}, \emph{\Vbar{} Dirigátur.} in simple commemoration tone, p.~\pageref{vr-dirigatur} and \emph{Collect}.  Otherwise \Vbar~\emph{Benedicámus Dómino 1}, p.~\pageref{benedicamusdomino-1}.

\begin{multicols}{2}
\noindent\emph{\nth{9} Sunday after Pentecost}, p.~\pageref{pentecost9}.\\
\emph{\nth{10} Sunday after Pentecost}, p.~\pageref{pentecost10}.\\
\emph{\nth{11} Sunday after Pentecost}, p.~\pageref{pentecost11}.\\
\emph{\nth{12} Sunday after Pentecost}, p.~\pageref{pentecost12}.\\
\emph{\nth{13} Sunday after Pentecost}, p.~\pageref{pentecost13}.
\end{multicols}
}
\medskip
\hrule
% \bigskip
% \benedicamusdomino{}
}

%Sep 14: Exaltation of Holy Cross (2nd class)
{
\section{September 14: Exaltation of the Holy Cross}
\subtitle{\nth{2} Class}
\subtitle{I \& II Vespers}

\def\definevesperspropers{\newcommand{\maganttex}{MagnificatAntiphon2-OCrux}
\newcommand{\magantinitial}{O}
\newcommand{\maganttranslation}{O blessed art thou, O Cross which wast counted the only tree worthy to bear the Lord and King of heaven. Alleluia.}
\def\magsolemn{F}
\definemag{1}{D2}

  \def\prepsalmfive{\greseteolcustos{manual}}
}
\def\definevesperspropersalt{\newcommand{\maganttex}{MagnificatAntiphon1-OCrux}
\newcommand{\magantinitial}{T}
\newcommand{\maganttranslation}{Hail, O Cross Brighter than all the stars thy name is honourable upon earth; To the eyes of men thou art exceeding lovely; holy art thou among all things that are earthly; thy transom made the one worthy balance whereon the price of the world was weighed; sweetest wood and sweetest iron, sweetest weight is hung on thee; O that every one that is here gathered this day to praise thee may find that thou art indeed salvation for him.}
\def\magsolemn{F}
\definemag{1}{D}
}
\def\vesperspropersnote{At II Vespers:}
\def\vesperspropersaltnote{At I Vespers:}
%\def\premag{\def\noeuouae{T}}
\def\premagverses{\greseteolcustos{manual}}

\def\begincollectcols{\begin{parcolumns}[rulebetween,colwidths={1=0.42\linewidth}]{2}}
\printvespers[../September14-ExaltationOfTheHolyCross]{inc-ExaltationOfTheHolyCross}
\bigskip
\benedicamusdomino[2]{}
}

%Sep 29: Dedication of St Michael (1st class)
{
\section{September 29: Dedication of St Michael the Archangel}
\subtitle{\nth{1} Class}
\subtitle{I \& II Vespers}

\def\definevesperspropers{\definepsalm{5}{137}{7}{c}

\newcommand{\vrtex}{vrInConspectuAngelorum}
\newcommand{\vtranslation}{In the sight of the Angels, I will sing praise to Thee, O my God.}
\newcommand{\rtranslation}{I will worship towards Thy holy temple, and give glory to Thy name.}

\newcommand{\maganttex}{MagnificatAntiphon2}
\definemag{1}{D2}
\newcommand{\magantinitial}{P}
\newcommand{\maganttranslation}{O most glorious prince, Michael the Archangel, be mindful of us, and here and everywhere entreat the Son of God for us, alleluia, alleluia.}

  \def\prepsalmfive{\greseteolcustos{manual}}
}
\def\definevesperspropersalt{\definepsalm{5}{116}{7}{c}

\newcommand{\vrtex}{vrStetitAngelus}
\newcommand{\vtranslation}{The Angel stood by the altar of the temple.}
\newcommand{\rtranslation}{Holding in his hand a censer of gold.}

\newcommand{\maganttex}{MagnificatAntiphon1}
\newcommand{\magantinitial}{D}
\definemag{8}{G}
\newcommand{\maganttranslation}{While John was beholding the sacred Mystery, the Archangel Michael sounded a trumpet.  Forgive us, O Lord our God, Thou who openest the book, and loosest the seals thereof.  Alleluia.\vspace{-1ex}}
}
\def\vesperspropersnote{At II Vespers:}
\def\vesperspropersaltnote{At I Vespers:}
%\def\premag{\def\noeuouae{T}}
\def\premagverses{\greseteolcustos{manual}}

\printvespers[../September29-DedicationOfChurchOfStMichaelArchangel]{inc-DedicationStMichael}

\medskip
\printnote{If today is Sunday, the Vespers of the Sunday is commemorated with \emph{Magnificat antiphon}, \emph{\Vbar{} Dirigátur.} in simple commemoration tone, p.~\pageref{vr-dirigatur} and \emph{Collect}.  Otherwise \Vbar~\emph{Benedicámus Dómino 1}, p.~\pageref{benedicamusdomino-1}.

\begin{multicols}{2}
\noindent\emph{\nth{16} Sunday after Pentecost}, p.~\pageref{pentecost16}.\\
\emph{\nth{17} Sunday after Pentecost}, p.~\pageref{pentecost17}.\\
\emph{\nth{18} Sunday after Pentecost}, p.~\pageref{pentecost18}.\\
\emph{\nth{19} Sunday after Pentecost}, p.~\pageref{pentecost19}.\\
\emph{\nth{20} Sunday after Pentecost}, p.~\pageref{pentecost20}.
\end{multicols}
}
\medskip
\hrule
% \bigskip
% \benedicamusdomino{}
}

%Last Sunday in October: Christ the King (1st class)
{
\section{Last Sunday in October: Jesus Christ, King}
\subtitle{\nth{1} Class}
\subtitle{II Vespers}

\def\definevesperspropers{\newcommand{\vrtex}{vr}
\newcommand{\vtranslation}{His dominion shall be increased.}
\newcommand{\rtranslation}{And of peace there shall be no end.}

\newcommand{\magantinitial}{H}
\newcommand{\maganttex}{MagnificatAntiphon}
\newcommand{\maganttranslation}{He hath on His garment and on His thigh written: King of kings and Lord of lords.  To Him be glory and empire for ever and ever.}
\def\magsolemn{T}
\definemag{7}{a}

  \def\prepsalmfive{\greseteolcustos{manual}}
}
%\def\premag{\def\noeuouae{T}}
\def\premagverses{\greseteolcustos{manual}}
\def\beginchaptercols{\begin{parcolumns}[rulebetween,colwidths={1=0.46\linewidth}]{2}}
%\def\begincollectcols{\begin{parcolumns}[rulebetween,colwidths={1=0.45\linewidth}]{2}}

\printvespers[../OctoberLastSunday-ChristTheKing]{inc-ChristTheKing}
\noindent
\printnote{If today is October 31, the First Vespers of All Saints is commemorated with \emph{Magnificat}, p.~\pageref{allsaints1-magnificat}; \emph{\Vbar{}~Lætámini}.~in simple commemoration tone, p.~\pageref{allsaints1-vr}; and \emph{Collect}, p.~\pageref{allsaints-collect}.  Otherwise \Vbar~\emph{Benedicámus Dómino 1}, p.~\pageref{benedicamusdomino-1}.}
\bigskip
\hrule
%\bigskip
%\benedicamusdomino{}
}

%Nov 1: All Saints (1st class)
{
\section{November 1: All Saints}
\subtitle{\nth{1} Class}
\subtitle{I \& II Vespers}

\def\definevesperspropers{\definepsalm{5}{115}{8}{G}

\newcommand{\vrtex}{vrExsultabunt}
\newcommand{\vtranslation}{The Saints will rejoice in glory.}
\newcommand{\rtranslation}{They will be joyful upon their beds.}

\newcommand{\maganttex}{an--o_quam_gloriosum--solesmes}
\newcommand{\magantinitial}{O}
\newcommand{\maganttranslation}{Oh! how glorious is the kingdom where all the Saints rejoice with Christ; clothed in white robes, they follow the Lamb whithersoever he goeth!}
\def\magsolemn{T}
\definemag{6}{F}

  \def\prepsalmfive{\greseteolcustos{manual}}
}
\def\definevesperspropersalt{\definepsalm{5}{116}{8}{G}

\newcommand{\vrtex}{vrLaetamini}
\newcommand{\vtranslation}{Be glad in the Lord, and rejoice ye righteous.}
\newcommand{\rtranslation}{And shout for joy, all ye that are upright in heart.}

\newcommand{\maganttex}{an--angeli_archangeli_all_saints--solesmes}
\newcommand{\magantinitial}{A}
\newcommand{\maganttranslation}{O ye Angels, Archangels, Thrones and Dominions, Principalities and Powers, Virtues, Cherubim and Seraphim, Patriarchs and Prophets, holy Teachers of the Law, all Apostles, Martyrs of Christ, holy Confessors, Virgins of the Lord, Hermits, and all Saints, intercede for us.}
\def\magsolemn{T}
\definemag{1}{D}
\def\premag{\label{allsaints1-magnificat}}}
\def\vraltlabel{allsaints1-vr}
\def\vesperspropersnote{At II Vespers:}
\def\vesperspropersaltnote{At I Vespers:}
\def\begincollectcols{\label{allsaints-collect}\begin{parcolumns}[rulebetween]{2}}
%\def\premag{\def\noeuouae{T}}
\def\premagverses{\greseteolcustos{manual}}

\printvespers[../November1-AllSaints]{inc-AllSaints}

\medskip
\printnote{If today is Sunday, the Vespers of the Sunday is commemorated with \emph{Magnificat antiphon}, \emph{\Vbar{} Dirigátur.} in simple commemoration tone, p.~\pageref{vr-dirigatur} and \emph{Collect}.  Otherwise \Vbar~\emph{Benedicámus Dómino 1}, p.~\pageref{benedicamusdomino-1}.

\begin{multicols}{2}
\noindent\emph{\nth{21} Sunday after Pentecost}, p.~\pageref{pentecost21}.\\
\emph{\nth{22} Sunday after Pentecost}, p.~\pageref{pentecost22}.\\
\emph{\nth{23} Sunday after Pentecost}, p.~\pageref{pentecost23}.\\
\emph{\nth{4} Sunday after Epiphany}, p.~\pageref{epiphany4}.
\end{multicols}
}
\medskip
\hrule

%     TODO Feast of All Saints (11/1) commem of Sunday, Saturday before 1st Sunday of November, 21st - 23rd Sunday after Pentecost or 4th after Epiphany
% \bigskip
% \benedicamusdomino{}
}

%Nov 9: Dedication of Archbasilica of Holy Savior (2nd class)
{
\section{November 9: Dedication of Archbasilica of Holy Savior}
\subtitle{\nth{2} Class}
\subtitle{I \& II Vespers}
\printnote{All from the Common of the Dedication of a Church.}

\def\definevesperspropers{\newcommand{\vrtex}{vrDomumTuam}
\newcommand{\vtranslation}{Holiness becometh thy house, O Lord.}
\newcommand{\rtranslation}{Forever.}

\newcommand{\maganttex}{MagnificatAntiphon2-OQuamMetuendusEst}
\newcommand{\magantinitial}{O}
\newcommand{\maganttranslation}{How dreadful is this place. Surely this is none other but the house of God, and the gate of heaven.}
\def\magsolemn{T}
\def\magoneline{T}
\definemag{6}{F}

  \def\prepsalmfive{\greseteolcustos{manual}}
}
\def\definevesperspropersalt{\newcommand{\vrtex}{vrHaecEstDomusDomini}
\newcommand{\vtranslation}{This is the house of the Lord, strongly built.}
\newcommand{\rtranslation}{It is well founded upon strong rock.}

\newcommand{\maganttex}{MagnificatAntiphon1-Sanctificavit}
\newcommand{\magantinitial}{S}
\newcommand{\maganttranslation}{.}
\def\magsolemn{T}
\definemag{1}{g}
}
\def\vesperspropersnote{At II Vespers:}
\def\vesperspropersaltnote{At I Vespers:}
%\def\premag{\def\noeuouae{T}}
\def\premagverses{\greseteolcustos{manual}}

\printvespers[../CommonOfDedicationOfChurch]{inc-DedicationOfChurch}
\bigskip
\benedicamusdomino[2]{}
}

}


\chapter{Appendix}
{
\section{Chants at Benediction}
\def\gabcfolder{../BenedictionChants}
\printgabc{8.}{}{O}{OSalutaris}

%\def\translationmargin{68pt}
\translation[]{
\textbf{1.} O saving Victim of the world,
Who openest wide the gates on high,
The foe his bands on us hath hurled,
O, give us strength; for aid we cry.
%\hspace{65ex}
\textbf{2.} To thee, one Lord, yet Three in One,
Let everlasting glory be :
O grant us life that end hath none,
In Fatherland to spend with thee. Amen.}
\let\translationmargin=\undefined
\bigskip

\printgabc{7.}{}{O}{OSalutaris_v2}


\needspace{4\baselineskip}
\printgabc{3.}{}{T}{TantumErgo}

\needspace{4\baselineskip}
{\begin{center}
2. \emph{Another Chant.}
%\vspace{-1ex}
\end{center}
}
\printgabc{1.}{}{T}{TantumErgo_v2}

\needspace{4\baselineskip}
\translation[]{
\textbf{1.} Therefore we, before it bending,
this great sacrament adore;
type and shadows have their ending
in the new rite evermore;
faith, our outward sense amending,
maketh good defects before.
\textbf{2.} Honour, laud, and praise addressing
to the Father and the Son,
might ascribe we, virtue, blessing,
and eternal benison;
Holy Ghost, from both progressing,
equal laud to thee be done. Amen.
}

\needspace{4\baselineskip}
{\begin{center}
3. \emph{Modern Chant.}
\end{center}
}

\needspace{4\baselineskip}
\printgabc{5.}{}{T}{TantumErgo_v3}

\needspace{4\baselineskip}
{\begin{center}
4. \emph{Spanish Chant.}
\end{center}
}

\needspace{4\baselineskip}
\printgabc{5.}{}{T}{TantumErgo-Spanish}


\needspace{4\baselineskip}
\begin{columns}
\versicle{Pánem de cǽlo præstitísti éis. (P. T. allelúia.)}{You have given them bread from heaven.}
\response{Omne delectaméntum in se habéntem. (P. T. allelúia.)}{Containing all sweetness within it.}
\colchunk{}
\colplacechunks{}
\colchunk{\hspace*{3em}Orémus.}\colchunk{\hspace*{3em}Let us pray,}
\colplacechunks{}
\prayer{Deus, qui nobis sub Sacraménto mirábili passiónis tuæ memóriam reliquísti~:~\dag{} tríbue, quǽsumus, ita nos córporis et sánguinis tui sacra mystéria venerári;~* ut redemptiónis tuæ fructum in nobis júgiter sentiámus.  Qui vivis et regnas in sǽcula sæculórum.  Amen.}
{O God, who hast left us a memorial of Thy passion under a wonderful Sacrament: Grant, we bessech Thee, that we may so venerate the sacred mysteries of Thy body and blood that we may continually sense in ourselves the fruit of Thy redemption. Who livest and reignest\dots{}}
\end{columns}

%TODO: we could add the Divine Praises might be nice...

\needspace{4\baselineskip}
{
\printgabc{5.}{}{A}{AdoremusInAeternum}}%

\begin{psalmverses}[1]
\item \emph{Quóniam} confirmáta est super nos misericórdia \textbf{é}jus~:~* et véritas Dómini mánet \textbf{in} æ\textbf{tér}num.
\item[\Rbar{}] \emph{Adorémus}.
\item \emph{Glória} Pátri, et \textbf{Fí}lio,~* et Spi\textbf{rí}tui \textbf{Sán}cto.
\item \emph{Sicut é}\-rat in princípio, et núnc, et \textbf{sém}per,~* et in sǽcula sæcu\textbf{ló}rum. \textbf{A}men.
\item[\Rbar{}] \emph{Adorémus}.
\end{psalmverses}

\translation[]{Let us adore for ever the Most Holy Sacrament.
{\emph Ps.} All ye nations, praise the Lord : all ye peoples, sing his glory.
Because his merciful kindness is great towards us, and the truth of the Lord endureth for ever. Let us adore\ldots Glory.}
\vfil
}
{
\clearpage
\label{magnificat-grassi}
\sectionmark{Magnificat by Grassi}
\addcontentsline{toc}{section}{Magnificat by Grassi}

\def\betweenLilyPondSystem#1{
  \ifnum#1>1
    \vfil\noindent
  \else
    \linebreak
  \fi
  %\vspace{0.5\baselineskip}
}
\newcommand{\includelilypond}[1]{
  \noindent
  \import{../misc/}{#1}
}
\def\magsolemn{F}
\def\maggrassi{T}
\definemag{8}{G}
\def\annot{\magtone. \magend}
\def\greinitialformat#1{%
{\fontsize{50}{50}\selectfont #1}%
}
\setinitialspacing{M}
\def\magoddverses{\input{../psalms/magnificat-\magtone\nostarendmag\if\magsolemn Tsolemn\else simple\fi-grassi-chant-verses}}
\global\def\magtex{../psalms/Magnificat\if\magsolemn Tsolemn\else simple\fi\magtone\nostarendmag}
\greblockcustos
\begin{oddversesmagnificat}{\magtex}
\magoddverses
\end{oddversesmagnificat}

\includelilypond{magnificat-grassi-small-systems}



\def\magsolemn{T}
\let\magant=\undefined
\let\magantlinetwo=\undefined
\let\magtex=\undefined
\let\magverses=\undefined
\definemag{8}{G}
\def\annot{\magtone. \magend}
\setinitialspacing{M}
\global\def\magtex{../psalms/Magnificat\if\magsolemn Tsolemn\else simple\fi\magtone\nostarendmag}
\begin{oddversesmagnificat}{\magtex}
\magoddverses
\end{oddversesmagnificat}
}

\end{document}

