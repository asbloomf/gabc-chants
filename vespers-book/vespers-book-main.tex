% !TEX TS-program = lualatex
% !TEX encoding = UTF-8

% This is a simple template for a LuaLaTeX document using gregorio scores.

% easter can be from march 22 to april 25

\documentclass[letterpaper,12pt]{book} % use larger type; default would be 10pt
\usepackage{../definepsalms}
\usepackage{titlesec}
\usepackage{titletoc}
\usepackage{titleps}
\usepackage{letltxmacro}
\usepackage{changepage} % gives us \ifoddpage use [strict]

\LetLtxMacro{\oldneedspace}{\needspace}
\renewcommand{\needspace}[1]{
	\checkoddpage\ifoddpage\oldneedspace{#1}\else\fi
}
\let\gredagger=\dag

%\usepackage{hyperref}
\newcommand{\phantomsection}{}
\newcommand{\magnote}{
\IfStrEq{\anttone\nostarend}{8G}{\noindent\emph{A 3 part arrangement of the even verses by Grassi may be found on page \pageref{magnificat-grassi}.\\}}{}
}
\presetkeys{mymag}{note=\magnote}{}
%\LetLtxMacro{\oldprintmag}{\printmag}
%\renewcommand{\printmag}[4][grassilabel=magnificat-grassi]{
%\oldprintmag[#1]{#2}{#3}{#4}
%}

\setcounter{secnumdepth}{-1}

\def\mywidth{6in}
\def\myheight{9in}

% !TEX TS-program = lualatex
% !TEX encoding = UTF-8
% usual packages loading:
%\usepackage{luatextra}
%\usepackage{graphicx} % support the \includegraphics command and options
\usepackage{geometry} % See geometry.pdf to learn the layout options. There are lots.
\ifx\undefined\mywidth
    \geometry{letterpaper} % or letterpaper (US) or a5paper or....
\else
    \geometry{papersize={\mywidth,\myheight}}
\fi
\usepackage{expl3}
\let\luatexlocalrightbox\localrightbox
\let\luatexlocalleftbox\localleftbox
\usepackage{gregoriotex} % for gregorio score inclusion
\usepackage{import}

% If you use usual TeX fonts, here is a starting point:
%\usepackage{palatino}
%\input{glyphtounicode} \pdfglyphtounicode{f_f}{FB00} \pdfglyphtounicode{f_f_i}{FB03} \pdfglyphtounicode{f_f_l}{FB04}
%\pdfglyphtounicode{Q_u}{E048} \pdfglyphtounicode{O_e}{0152} \pdfglyphtounicode{o_e}{0153}
%\pdfgentounicode=1
% to change the font to something better, you can install the kpfonts package (if not already installed). To do so
% go open the "TeX Live Manager" in the Menu Start->All Programs->TeX Live 2010
% the additional width of the additional lines (compared to the width of the glyph they're associated with)
\grechangedim{additionallineswidth}{0.14584 cm}{scalable}%
% width of the additional lines, used only for the custos (maybe should depend on the width of the custos...)
% the width is the one for the custos at end of lines, the line for custos in the middle of a score is the same
% multiplied by 2.
\grechangedim{additionalcustoslineswidth}{0.09114 cm}{scalable}%
% null space
\grechangedim{zerowidthspace}{0 cm}{scalable}%
% space between glyphs in the same element
\grechangedim{interglyphspace}{0.06927 cm plus 0.00363 cm minus 0.00363 cm}{scalable}%
% space between an alteration (flat or natural) and the next glyph
\grechangedim{alterationspace}{0.07747 cm plus 0.01276 cm minus 0.00455 cm}{scalable}%
% space between a clef and a flat (for clefs with flat)
\grechangedim{clefflatspace}{0.05469 cm plus 0.00638 cm minus 0.00638 cm}{scalable}%
% space before a choral sign
\grechangedim{beforelowchoralsignspace}{0.04556 cm plus 0.00638 cm minus 0.00638 cm}{scalable}%
% when bolshifts are enabled, minimal space between a clef at the beginning of the line and a leading alteration glyph (should be larger than clefflatspace so that a flatted clef can be distinguished from a flat which is part of the first glyph on a line, but also smaller than spaceafterlineclef, the distance from the clef to the first notes)
\grechangedim{beforealterationspace}{0.1 cm}{scalable}%
% space between elements
\grechangedim{interelementspace}{0.06927 cm plus 0.00182 cm minus 0.00363 cm}{scalable}%
% larger space between elements
\grechangedim{largerspace}{0.10938 cm plus 0.01822 cm minus 0.00911 cm}{scalable}%
% space between elements in ancient notation
\grechangedim{nabcinterelementspace}{0.06927 cm plus 0.00182 cm minus 0.00363 cm}{scalable}%
% larger space between elements in ancient notation
\grechangedim{nabclargerspace}{0.10938 cm plus 0.01822 cm minus 0.00911 cm}{scalable}%
% space between elements which has the size of a note
\grechangedim{glyphspace}{0.21877 cm plus 0.01822 cm minus 0.01822 cm}{scalable}%
% space before custos
\grechangedim{spacebeforecustos}{0.1823 cm plus 0.31903 cm minus 0.0638 cm}{scalable}%
% space before punctum mora and augmentum duplex
\grechangedim{spacebeforesigns}{0.05469 cm plus 0.00455 cm minus 0.00455 cm}{scalable}%
% space after punctum mora and augmentum duplex
\grechangedim{spaceaftersigns}{0.08203 cm plus 0.0082 cm minus 0.0082 cm}{scalable}%
% space after a clef at the beginning of a line
\grechangedim{spaceafterlineclef}{0.27345 cm plus 0.14584 cm minus 0.01367 cm}{scalable}%
% minimal space between notes of different words
%\grechangedim{interwordspacenotes}{0.27 cm plus 0.15 cm minus 0.05 cm}{scalable}%
\grechangedim{interwordspacenotes}{0.27 cm plus 0.08 cm minus 0.05 cm}{scalable}%
% minimal space between notes of the same syllable.
% Warning: always keep minus to 0; also keep plus very low, or some words won't be hyphenated
%\grechangedim{intersyllablespacenotes}{0.24 cm plus 0.04cm minus 0cm}{scalable}%
\grechangedim{intersyllablespacenotes}{0.24 cm plus 0.04cm minus 0cm}{scalable}%
% minimal space between letters of different words. Makes sense to have
% the same plus and minus as interwordspacenotes.
%\grechangedim{interwordspacetext}{0.38 cm plus 0.15 cm minus 0.05 cm}{scalable}%
\grechangedim{interwordspacetext}{0.18 cm plus 0.08 cm minus 0.05 cm}{scalable}%
% Versions of interword spaces for euouae blocks
%\grechangedim{interwordspacenotes@euouae}{0.19 cm plus 0.1 cm minus 0.05 cm}{scalable}%
\grechangedim{interwordspacenotes@euouae}{0.13 cm plus 0.1 cm minus 0.05 cm}{1}%
%\grechangedim{interwordspacetext@euouae}{0.27 cm plus 0.1 cm minus 0.05 cm}{scalable}%
\grechangedim{interwordspacetext@euouae}{0.13 cm plus 0.1 cm minus 0.05 cm}{1}%
% space between notes of a bivirga or trivirga
\grechangedim{bitrivirspace}{0.06927 cm plus 0.00182 cm minus 0.00546 cm}{scalable}%
% space between notes of a bistropha or tristrophae
\grechangedim{bitristrospace}{0.06927 cm plus 0.00182 cm minus 0.00546 cm}{scalable}%
% space between two punctum inclinatum
\grechangedim{punctuminclinatumshift}{-0.03918 cm plus 0.0009 cm minus 0.0009 cm}{scalable}%
% space before puncta inclinata
\grechangedim{beforepunctainclinatashift}{0.05286 cm plus 0.00728 cm minus 0.00455 cm}{scalable}%
% space between a punctum inclinatum and a punctum inclinatum deminutus
\grechangedim{punctuminclinatumanddebilisshift}{-0.02278 cm plus 0.0009 cm minus 0.0009 cm}{scalable}%
% space between two punctum inclinatum deminutus
\grechangedim{punctuminclinatumdebilisshift}{-0.00728 cm plus 0.0009 cm minus 0.0009 cm}{scalable}%
% space between puncta inclinata, larger ambitus (range=3rd)
\grechangedim{punctuminclinatumbigshift}{0.07565 cm plus 0.0009 cm minus 0.0009 cm}{scalable}%
% space between puncta inclinata, larger ambitus (range=4th -or more?-)
\grechangedim{punctuminclinatummaxshift}{0.17865 cm plus 0.0009 cm minus 0.0009 cm}{scalable}%
% space for the bars (inside syllables)
%first for virgula and divisio minima
\grechangedim{spacearoundsmallbar}{0.1823 cm plus 0.22787 cm minus 0.00469 cm}{scalable}%
%then divisio minor
\grechangedim{spacearoundminor}{0.1823 cm plus 0.22787 cm minus 0.00469 cm}{scalable}%
%divisio major
\grechangedim{spacearoundmaior}{0.1823 cm plus 0.22787 cm minus 0.00469 cm}{scalable}%
%divisio finalis
\grechangedim{spacearoundfinalis}{0.1823 cm plus 0.22787 cm minus 0.00469 cm}{scalable}%
%a special space for finalis, for when it is the last glyph
\grechangedim{spacebeforefinalfinalis}{0.29169 cm plus 0.07292 cm minus 0.27345 cm}{scalable}%
% additional space that will appear around bars that are preceded by a custos and followed by a key.
\grechangedim{spacearoundclefbars}{0.03645 cm plus 0.00455 cm minus 0.0009 cm}{scalable}%
% space between the text and the text of the bar
\grechangedim{textbartextspace}{0.24611 cm plus 0.13672 cm minus 0.04921 cm}{scalable}%
% minimal space between a note and a bar
\grechangedim{notebarspace}{0.31903 cm plus 0.27345 cm minus 0.02824 cm}{scalable}%
% maximal space between two syllables for which we consider a dash is not needed
\grechangedim{maximumspacewithoutdash}{0.00 cm}{scalable}%
% an extensible space for the beginning of lines
\grechangedim{afterclefnospace}{0 cm plus 0.27345 cm minus 0 cm}{scalable}%
% space between the initial and the beginning of the score
\grechangedim{afterinitialshift}{0.2457 cm}{scalable}%
% space before the initial
\grechangedim{beforeinitialshift}{0.2457 cm}{scalable}%
% when bolshifts are enabled, minimum space between beginning of line and first syllable text
\grechangedim{minimalspaceatlinebeginning}{0.05 cm}{scalable}%
% space to force the initial width to.  Ignored when 0.
\grechangedim{manualinitialwidth}{0 cm}{scalable}%
% distance to move the initial up by
\grechangedim{initialraise}{0 cm}{scalable}%
% Space between lines in the annotation
\grechangedim{annotationseparation}{0.05cm}{scalable}%
% Amount to raise (positive) or lower (negative) the annotations from the default position (base line of top annotation aligned with top line of staff)
\grechangedim{annotationraise}{0cm}{scalable}%
% space at the beginning of the lines if there is no clef
\grechangedim{noclefspace}{0.1 cm}{scalable}%
% space around a clef change
\grechangedim{clefchangespace}{0.01768 cm plus 0.00175 cm minus 0.01768 cm}{scalable}%
%When \gre@clivisalignment is 2, this distance is the maximum length of the consonants after vowels for which the clivis will be aligned on its center.
\grechangedim{clivisalignmentmin}{0.3 cm}{scalable}%



%%%%%%%%%%%%%%%%%%
% vertical spaces
%%%%%%%%%%%%%%%%%%

% first, we have two spaces for the chironomic signs
\grechangedim{abovesignsspace}{0.8 cm}{scalable}%
\grechangedim{belowsignsspace}{0 cm}{scalable}%
% the amount to shift down:
% (a) low choral signs that are not lower than the note, regardless of whether
%     it's on a line or in a space
% (b) high choral signs and low choral signs that are lower than the note which
%     are in a space
\grechangedim{choralsigndownshift}{0.00911 cm}{scalable}%
% the amount to shift up:
% (a) high choral signs and low choral signs that are lower than the note which
%     are on a line
\grechangedim{choralsignupshift}{0.04556 cm}{scalable}%
% the space for the translation
\grechangedim{translationheight}{0.5 cm}{scalable}%
%the space above the lines
\grechangedim{spaceabovelines}{0.45576 cm plus 0.36461 cm minus 0.09114 cm}{scalable}%
%the space between the lines and the bottom of the text
\grechangedim{spacelinestext}{0.60617 cm}{scalable}%
%the space beneath the text
\grechangedim{spacebeneathtext}{0 cm}{scalable}%
% height of the text above the note line
\grechangedim{abovelinestextraise}{-0.1 cm}{scalable}%
% height that is added at the top of the lines if there is text above the lines (it must be bigger than the text for it to be taken into consideration)
\grechangedim{abovelinestextheight}{0.3 cm}{scalable}%
% an additional shift you can give to the brace above the bars if you don't like it
\grechangedim{braceshift}{0 cm}{scalable}%
% a shift you can give to the accentus above the curly brace
\grechangedim{curlybraceaccentusshift}{-0.05 cm}{scalable}%


%\def\greinitialformat#1{{\fontsize{37}{37}\selectfont #1}}
%small > footnotesize > scriptsize > tiny


% my stuff
\usepackage[garamond]{../mypackage}
% end my stuff

\setgrefactor{17}

%\marginsize{25pt}{25pt}{25pt}{30pt}
\usepackage{calc}
%\setlength\headsep{20pt}
%\setlength\footskip{15pt}
\setlength\headheight{15pt}
\setlength\headsep{22pt}
\ifx\undefined\tenebrae
    \geometry{outer=25pt,inner=25pt,top=22pt+\headsep+\headheight,bottom=25pt+\footskip}
\else
    \ifx\undefined\mywidth
        %had been .3 outer, .4 inner
        %let's try .75 for inner and .5 for outer
        %now let's go to .35 outer, .9 inner
        %this time let's try .4 and .85
        \ifbook{\geometry{outer=0.4in,inner=0.85in,top=25pt+\headsep+\headheight,bottom=25pt+\footskip,twoside=true}}
        \ifnotbook{\geometry{outer=0.625in,inner=0.625in,top=25pt+\headsep+\headheight,bottom=25pt+\footskip,twoside=true}}
    \else
        \setlength\headsep{0.25in}
        \setlength\footskip{0.3in}
        \geometry{outer=0.5in,inner=0.5in,top=0.25in+\headsep+\headheight,bottom=0.25in+\footskip,twoside=true}
    \fi
\fi

\pagestyle{fancy} % no header or footers
\let\oldheadrulewidth\headrulewidth
\renewcommand\headrulewidth{\ifnum\thepage=1
0pt
\else
\oldheadrulewidth
\fi}

\ifx\undefined\ifbook
    \newcommand{\ifbook}[1]{}
    \newcommand{\ifnotbook}[1]{#1}
\fi
\ifx\undefined\ifsmallbook
    \newcommand{\ifsmallbook}[1]{}
    \newcommand{\ifnotsmallbook}[1]{#1}
\fi
%\cfoot{\thepage}

\setlength\headheight{0.25in+15pt}
\setlength\headsep{1pc}
\setlength\topskip{0pc}
\setlength\footskip{1pc}
\geometry{outer=0.4in,inner=0.85in,top=0pc+\headheight+\headsep,bottom=0.4in,twoside=true}
\newpagestyle{main}{
\sethead[\garamond{\thepage}][\garamond{\chaptertitle}][] % even
{}{\garamond{\sectiontitle}}{\garamond{\thepage}} % odd
\setfoot[][][] % even
{}{}{} % odd
}
\pagestyle{main}


\titleformat
{\section} % command
[block] % shape
{\phantomsection\large\addfontfeature{Numbers=Lining}} % format
{} % label
{} % sep
{
    % \rule{\textwidth}{1pt}
    % \vspace{1ex}
    \centering
} % before-code
%[
% \vspace{-0.5ex}%
% \rule{\textwidth}{0.3pt}
%] % after-code
 
 
\titleformat{\chapter}[block]
{\thispagestyle{empty}\phantomsection\Large\scshape\addfontfeature{Numbers=Lining}}
{}{0.5em}{\centering}
 
\titlespacing{\chapter}{0pt}{-\headheight}{1pc}
\titlespacing{\section}{0pt}{*2.5}{*1}
\titleclass{\chapter}{top}
\newcommand{\chapterbreak}{\clearpage}
%\titleclass{\section}{top}

\contentsmargin{1pc}
\dottedcontents{chapter}[0pc]{}{2pc}{1pc}
\dottedcontents{section}[0pc]{}{0pc}{1pc}

\newcommand{\printnote}[1]{
	{\normalsize \emph{#1}}
}
\newcommand{\subtitle}[1]{
\begin{center}{
	{\addfontfeature{Numbers=Lining} \normalsize \emph{#1}}
}\end{center}
}

\newcommand{\deusinadjutorium}{\noindent\printnote{\Vbar~\emph{Deus in adjutórium}, page \pageref{deusinadjutorium}.}}
\newcommand{\printcollect}[2]{
	\ifx\undefined\begincollectcols\def\begincollectcols{\begin{parcolumns}[rulebetween]{2}}\fi
	\ifx\printcollectheading\undefined\def\printcollectheading{T}\fi
	\if\printcollectheading T
	\needspace{3\baselineskip}
	\begin{center}{\large Collect.}\end{center}
	\vspace{-0.4\baselineskip}
	\fi
	\begincollectcols
	\sloppy
	\prayer{#1}{#2}
	\end{parcolumns}
	\let\begincollectcols=\undefined
}
\newcommand{\psalmcolsoverride}[1][0]{
\def\beginpsalmcols{\begin{parcolumns}[rulebetween,colwidths={1=0.45\linewidth}]{2}}
\ifnum#1=110
\def\beginpsalmcols{\begin{parcolumns}[rulebetween,colwidths={1=0.45\linewidth}]{2}}
\fi
\ifnum#1=111
\def\beginpsalmcols{\begin{parcolumns}[rulebetween,colwidths={1=0.475\linewidth}]{2}}
\fi
\ifnum#1=129
\def\beginpsalmcols{\begin{parcolumns}[rulebetween,colwidths={1=0.475\linewidth}]{2}}
\fi
\ifnum#1=131
\def\beginpsalmcols{\begin{parcolumns}[rulebetween,colwidths={1=0.475\linewidth}]{2}}
\fi
}
\newcommand{\printcommemoration}[2][.]{
	\def\gabcfolder{#1}
	\input{\gabcfolder/#2}
	\subtitle{\comheadingtext}
	\sectionmark{\comheadingtext}
	{
	\def\noeuouae{T}
	\printgabc{At Magn.}{\oldstylenums{\commagantlinetwo}}{\commagantinitial}{\commaganttex}
	}
	\translation[]{\englishcommagantiphon}

	\smallskip
	\printvr[\greseteolcustos{manual}]{\commvrtex}{\commvtranslation}{\commrtranslation}

	\printcollect{\latincomcollect}{\englishcomcollect}
}
\newcommand{\printvespersmag}[2][.]{
	\grechangestaffsize{15}
	\def\gabcfolder{#1}
	\input{\gabcfolder/#2}

	\ifx\chaptertext\undefined\else{
		\oldneedspace{3\baselineskip}
		\printchapter{\chaptertext}{\chaptertranslation}
		\smallskip
	}\fi

	\ifx\printfullhymn\undefined
		\printhymnnote
	\else
		\printfullhymn
	\fi

	{
		\let\anttranslation=\englishmagantiphon
		\let\preverses=\premagverses
	    \let\preanttwo=\premagtwo
	    \let\preant=\premag
		\printmag{\magtone\magend}{\magantinitial}{\maganttex}
	}
	\vspace{-1\baselineskip}
	\oldneedspace{3\baselineskip}
	\printcollect{\latincollect}{\englishcollect}
}
\newcommand{\printvespers}[2][.]{
	\grechangestaffsize{15}
	\def\gabcfolder{#1}
	\input{\gabcfolder/#2}

	\deusinadjutorium{}
	\medskip
	\printpsalm{1}{\psalmonenum}{\psalmonetone\psalmoneend}{\antonetex}{\antoneinitial}
	\medskip
	\printpsalm{2}{\psalmtwonum}{\psalmtwotone\psalmtwoend}{\anttwotex}{\anttwoinitial}
	\medskip
	\needspace{3\baselineskip}
	\printpsalm{3}{\psalmthreenum}{\psalmthreetone\psalmthreeend}{\antthreetex}{\antthreeinitial}
	\medskip
	\needspace{3\baselineskip}
	\printpsalm{4}{\psalmfournum}{\psalmfourtone\psalmfourend}{\antfourtex}{\antfourinitial}
	\medskip
	\ifx\psalmfivenum\undefined{
		\ifx\antfivetex\undefined{%then it is a different antiphon for each
			\ifx\definevesperspropersalt\undefined\else{
				\definevesperspropersalt
				\ifx\vesperspropersaltnote\undefined\else
					\noindent\printnote{\vesperspropersaltnote}
				\fi
				\printpsalm{5}{\psalmfivenum}{\psalmfivetone\psalmfiveend}{\antfivetex}{\antfiveinitial}
			}\fi
			\ifx\definevesperspropers\undefined\else{
				\definevesperspropers
				\ifx\vesperspropersnote\undefined\else
					\noindent\printnote{\vesperspropersnote}
				\fi
				\printpsalm{5}{\psalmfivenum}{\psalmfivetone\psalmfiveend}{\antfivetex}{\antfiveinitial}
			}\fi
		}\else{%they share the same antiphon
			\ifx\definevesperspropersalt\undefined\else{
				\definevesperspropersalt
				\ifx\vesperspropersaltnote\undefined\else
					\def\prepsalmtitle{
						\smallskip
						\noindent\printnote{\vesperspropersaltnote}\vspace{-0.5\baselineskip}

					}
				\fi
				\printpsalm{5}{\psalmfivenum}{\psalmfivetone\psalmfiveend}{\antfivetex}{\antfiveinitial}
			}\fi
			\def\onlyoneant{T}
			\ifx\definevesperspropers\undefined\else{
				\definevesperspropers
				\ifx\vesperspropersnote\undefined\else
					\noindent\printnote{\vesperspropersnote}\vspace{-0.5\baselineskip}
				\fi
				\def\prepsalmtitle{\def\onlyoneant{F}}
				\printpsalm{5}{\psalmfivenum}{\psalmfivetone\psalmfiveend}{\antfivetex}{\antfiveinitial}
			}\fi
		}\fi
	}\else{
		\printpsalm{5}{\psalmfivenum}{\psalmfivetone\psalmfiveend}{\antfivetex}{\antfiveinitial}
	}\fi
	%\medskip
	\vspace{-\baselineskip}
	\needspace{3\baselineskip}
	\ifx\chapterreplacement\undefined{
		\ifx\chaptertext\undefined
			\def\printchapterheading{F}
		    \begin{center}{\large\textbf Chapter.}\end{center}

		    \vspace{-0.25\baselineskip plus 0.125\baselineskip}
			\ifx\definevesperspropersalt\undefined\else
			\medskip
			{
				\ifx\vesperspropersaltnote\undefined\else
					\printnote{\vesperspropersaltnote}
				\fi
				\definevesperspropersalt
				\printchapter{\chaptertext}{\chaptertranslation}
			}
			\fi
			\ifx\definevesperspropers\undefined\else
			{
				\ifx\vesperspropersnote\undefined\else
					\printnote{\vesperspropersnote}
				\fi
				\definevesperspropers
				\printchapter{\chaptertext}{\chaptertranslation}
			}
			\fi
		\else
			\printchapter{\chaptertext}{\chaptertranslation}
		\fi
	}
	\else
	{\chapterreplacement}
	\fi

	\medskip
	\ifx\printfullhymn\undefined
		\ifx\printhymnnote\undefined\else
			\printhymnnote
			\medskip
		\fi
	\else
		\printfullhymn
		\medskip
	\fi
	\ifx\magreplacement\undefined
	{
		\ifx\definevesperspropersalt\undefined\else{
			{
			\ifx\definevesperspropers\undefined\else
				{
					\definevesperspropersalt
					\global\let\magtonealt=\magtone
					\global\let\magendalt=\magend
				}
				{
					\definevesperspropers
					\IfStrEq{\magtonealt\magendalt}{\magtone\magend}
					{\gdef\altmagsame{T}}
					{\gdef\altmagsame{F}}

				}
			\fi
			}
			\definevesperspropersalt
		    \begin{center}{\large Magnificat.}\end{center}
	    	\vspace{-1ex}

	    	\ifx\vesperspropersaltnote\undefined\else
	    		\needspace{14\baselineskip}
				\printnote{\vesperspropersaltnote\\\vspace{-0.5\baselineskip}}
			\fi
			\let\anttranslation=\maganttranslation
			\let\preverses=\premagverses
		    \let\preanttwo=\premagtwo
		    \let\preant=\premag
		    \def\nomagtitle{T}
		    \if\altmagsame T{
		    	\printgabc{At Magn.}{Ant. \magtone{}. \magend}{\magantinitial}{\maganttex}
		    	\translation[]{\anttranslation}
		    	\bigskip
		    }\else{
				\printmag{\magtone\magend}{\magantinitial}{\maganttex}
			}\fi
		}\fi
		\ifx\definevesperspropers\undefined\else
			\definevesperspropers
	    	\ifx\vesperspropersnote\undefined\else
	    		\needspace{14\baselineskip}
				\printnote{\vesperspropersnote\\\vspace{-0.5\baselineskip}}
			\fi
			\def\nomagtitle{T}
		\fi
		\let\anttranslation=\maganttranslation
		\let\preverses=\premagverses
	    \let\preanttwo=\premagtwo
	    \let\preant=\premag
		\printmag{\magtone\magend}{\magantinitial}{\maganttex}
		\global\let\altmagsame\undefined
		\global\let\magtonealt=\undefined
		\global\let\magendalt=\undefined
	}
	\else
	{\magreplacement}
	\fi
	{\ifx\postmag\undefined\else\postmag\fi}

	\vspace{-\baselineskip}
	\ifx\precollect\undefined\else\precollect\fi
	\ifx\collectreplacement\undefined{
		\ifx\collect\undefined
			\def\printcollectheading{F}
			\needspace{3\baselineskip}
			\begin{center}{\large Collect.}\end{center}

			\vspace{-0.5\baselineskip plus 0.25\baselineskip}
			\ifx\definevesperspropers\undefined\else
			{
				\ifx\vesperspropersnote\undefined\else
					\printnote{\vesperspropersnote}
				\fi
				\definevesperspropers
				\printcollect{\collect}{\collecttranslation}
			}
			\fi
			\ifx\definevesperspropersalt\undefined\else
			\medskip
			{
				\ifx\vesperspropersaltnote\undefined\else
					\printnote{\vesperspropersaltnote}
				\fi
				\definevesperspropersalt
				\printcollect{\collect}{\collecttranslation}
			}
			\fi
		\else
			\printcollect{\collect}{\collecttranslation}
		\fi
	}
	\else
	{\collectreplacement}
	\fi
	\medskip
}

\sloppy
\begin{document}
\normalsize
\grechangestaffsize{15}
%\pagenumbering{roman}
%\frontmatter

\tableofcontents

%\pagenumbering{arabic}
%\mainmatter

\chapter{Common of Festal Vespers}

%\subtitle{Festal Tone}
\label{deusinadjutorium}
\sectionmark{Deus in adjutórium}
\addcontentsline{toc}{section}{Deus in adjutórium}
\printnote{The Festal Tone may be sung on any Sunday or Feast.}

\smallskip
\def\deusinadjutoriumsolemn{F}
\printdeusinadjutorium{}

\pagebreak
%\subtitle{Solemn Tone}
\printnote{The Solemn Tone may only be sung on Feasts of the First or Second Class.}

\medskip
\def\deusinadjutoriumsolemn{T}
\printdeusinadjutorium{}


\printnote{Vespers then proceed with the Proper Antiphons, Psalms, Chapter, Hymn, Versicle,
Magnificat, and Collect given for the respective Sunday or Feast.}

\medskip

\printnote{All Vespers conclude with the following:}
\\
\Vbar{} Dóminus vobíscum.\\%\hspace{5em}
\Rbar{} Et cum spíritu tuo.

\printnote{Or, in the absence of a priest or deacon:}
\\
\Vbar{} Dómine, exáudi oratiónem meam.\\
\Rbar{} Et clamor meus ad te véniat.

\medskip

\def\noeuouae{T}
{
\newcommand{\printbenedicamusdomino}[2]{
	\greseteolcustos{manual}
	\def\annot{\small{#1}}
	\alsetinitialspacing{B}
	\gregorioscore{#2}
  \greseteolcustos{auto}
}

\pagebreak
\phantomsection\printnote{The ``Benedicámus Dómino'' is then sung in one of the different tones as follows:}
\bigskip
\label{benedicamusdomino}
\sectionmark{Benedicámus Dómino}
\addcontentsline{toc}{section}{Benedicámus Dómino}

%\vfil
\grechangestaffsize{15}
\label{benedicamusdomino-1}
{{\centering \bfseries 1.~On feasts of the I class.\\}
\smallskip
{\centering \normalsize At \nth{1} Vespers.\\}
\smallskip
\def\breakbeforeresp{T}
\printbenedicamusdomino{2.}{../BenedicamusDomino/BenedicamusDomino_1class1stvespers}
\bigskip\needspace{8\baselineskip}
%\vfil
{\centering \normalsize At \nth{2} Vespers.\\}
\smallskip
{\def\breakbeforeresp{T}
\printbenedicamusdomino{6.}{../BenedicamusDomino/BenedicamusDomino_1class2ndvespers}
}
\medskip
\emph{\normalsize or more commonly:}

\smallskip
\def\breakbeforeresp{T}
\printbenedicamusdomino{5.}{../BenedicamusDomino/BenedicamusDomino_1class2ndVespersAlt}}
%\vfil
{\bigskip
\def\dotting{\leaders\hbox to 1em{\hfil.\hfil}\hfill}
\emph{Marian anthems can be found on the following pages:}\\
Alma Redemptoris Mater \emph{(Advent through February 2)},\dotting p.~\pageref{almaredemptorismater}\\
Ave Regina Cælorum \emph{(February 2 through Lent)},\dotting p.~\pageref{avereginacaelorum}\\
Regina Cæli \emph{(Easter through the Friday after Pentecost)},\dotting p.~\pageref{reginacaeli}\\
Salve Regina \emph{(1st Vespers of Trinity Sunday until just before Advent)},\dotting p.~\pageref{salveregina}

}
\pagebreak

\label{benedicamusdomino-2}
{{\centering \bfseries 2.~On feasts of the II class.\\}
\smallskip
{\centering \normalsize At \nth{1} Vespers.\\}
{\def\breakbeforeresp{T}
\printbenedicamusdomino{2.}{../BenedicamusDomino/BenedicamusDomino_2class1stvespers}
}
\bigskip
{\centering \normalsize At \nth{2} Vespers.\\}
{%\def\breakbeforeresp{T}
\printbenedicamusdomino{8.}{../BenedicamusDomino/BenedicamusDomino_2class2ndvespers}}
}
\bigskip

%{{\centering \bfseries On feasts of the III class.\\}
%\printbenedicamusdomino{2.}{../BenedicamusDomino/BenedicamusDomino_3class}}
%\bigskip\vfill

\needspace{3\baselineskip}
\label{benedicamusdomino-mary}
\gdef\benedicamusdominonamemary{3}
{{\centering \bfseries 3.~On feasts of the Blessed Virgin.\\}
\smallskip
%\def\breakbeforeresp{T}
\printbenedicamusdomino{1.}{../BenedicamusDomino/BenedicamusDomino_blessedVirgin}}
\bigskip\bigskip

\needspace{6\baselineskip}
\label{benedicamusdomino-sunday}
\gdef\benedicamusdominonamesunday{4}
{{\centering \bfseries 4.~On Sundays during the Year\\and Septuagesima, Sexagesima, and Quinquagesima.\\}
\smallskip
%\def\breakbeforeresp{T}
\printbenedicamusdomino{1.}{../BenedicamusDomino/BenedicamusDomino_Sundays}}
\bigskip\bigskip

\label{benedicamusdomino-lent}\label{benedicamusdomino-advent}
\gdef\benedicamusdominonamelent{5}
\gdef\benedicamusdominonameadvent{\benedicamusdominonamelent}
{{\centering \bfseries 5.~On Sundays of Advent and Lent.\\}
\smallskip
%\def\breakbeforeresp{T}
\printbenedicamusdomino{6.}{../BenedicamusDomino/BenedicamusDomino_SundaysOfAdventAndLent}}
\bigskip\bigskip

\label{benedicamusdomino-easter}
\gdef\benedicamusdominonameeaster{6}
{{\centering \bfseries 6.~On Sundays of Paschal Time.\\}
\smallskip
%\def\breakbeforeresp{T}
\printbenedicamusdomino{7.}{../BenedicamusDomino/BenedicamusDomino_SundaysOfPaschalTime}}

}

\def\dontrepeatantiphon{T}
\clearpage
%{
	\label{sundayvespers}
	\section{Sunday at Vespers}
	\def\printfullhymn{
	    {
	    	\label{hymn-luciscreator}
	        \printhymn{\oldstylenums{\hymnlinetwo}}{\hymninitial}{\hymntex}{\hymntranslation}
	        \def\hymnlinetwo{\oldstylenums{8.}}
\def\hymntex{hymn-LucisCreatorOptime2}
\def\hymninitial{L}

\def\hymntextwo{hymn-LucisCreatorOptime3}
\def\hymntwolinetwo{\oldstylenums{1.}}

			{
				\def\annot{\small{Hymn.}}
				\def\annottwo{\small{\hymnlinetwo}}
				\setinitialspacing{\hymninitial}
				\normalsize
				\begin{center}\addfontfeature{Numbers=Lining}\textbf{2. Another Chant (ad libitum).}\end{center}

				\includescore{\gabcfolder/\hymntex}
			}
			{
				\def\annot{\small{Hymn.}}
				\def\annottwo{\small{\hymntwolinetwo}}
				\setinitialspacing{\hymninitial}
				\normalsize
				\begin{center}\addfontfeature{Numbers=Lining}\textbf{3. Another Chant.}\end{center}

				\includescore{\gabcfolder/\hymntextwo}
			}

			\bigskip\bigskip

	        \def\vrlinebreak{T}
	        \label{vr-dirigatur}
	        \printvr[\greblockcustos]{\vrtex}{\vtranslation}{\rtranslation}
	    }
	}
	\newcommand{\magreplacement}{\noindent\printnote{Magnificat. Collect.\\\\Benedicámus Dómino \emph{ for Sundays during the year, page \pageref{benedicamusdomino-sunday}}.}}
	\def\preantfour{\needspace{10\baselineskip}}
	\printvespers[../SundayAtVespers]{inc-SundayAtVespers}
}

{
	\let\dontrepeatantiphon=\undefined
	\label{sundayvespers-easter}
	\section{Sunday at Vespers in Paschaltide}
	\def\printfullhymn{
	    {
	    	\label{hymn-adregiasagnidapes}
	        \def\hymnlinetwo{\oldstylenums{8.}}
\def\hymntex{hymn-AdRegiasAgniDapes}
\def\hymninitial{A}
\def\hymntranslation{\item At the Lamb's high feast we sing
praise to our victorious King,
Who hath washed us in the tide
flowing from His pierced side.
\item Praise we Him Whose love divine
gives the guests His Blood for wine,
gives His Body for the feast,
Love the victim, Love the priest.
\item Where the Paschal blood is poured,
Death's dark Angel sheathes his sword;
Israel's hosts triumphant go
through the wave that drowns the foe.
\item Christ, the Lamb Whose Blood was shed,
Paschal victim, Paschal bread;
with sincerity and love
eat we manna from above.
\item Mighty Victim from the sky,
powers of hell beneath Thee lie;
Death is conquered in the fight;
Thou hast brought us life and light.
\item Now Thy banner Thou dost wave;
vanquished Satan and the grave;
see the prince of darkness quelled;
heaven's bright gates are open held.
\item Paschal triumph, Paschal joy,
only sin can this destroy;
from sin's death do Thou set free
souls re-born, dear Lord, in Thee.
\item Hymns of glory, songs of praise,
Father, unto Thee we raise;
risen Lord, all praise to Thee,
ever with the Spirit be.
Amen.}

\def\vrtex{vrManeNobiscum}
\def\vtranslation{Stay with us O Lord, alleluia.}
\def\rtranslation{Because it is towards evening, alleluia.}

	        \printhymn{\oldstylenums{\hymnlinetwo}}{\hymninitial}{\hymntex}{\hymntranslation}
			\bigskip

	        \def\vrlinebreak{T}
	        \label{vr-dirigatur}
	        \printvr[\greblockcustos]{\vrtex}{\vtranslation}{\rtranslation}
	    }
	}	
	\newcommand{\magreplacement}{\noindent\printnote{Magnificat. Collect.\\\\Benedicámus Dómino \emph{ for Sundays in Paschaltide, page \pageref{benedicamusdomino-easter}}.}}
	\printvespers[../TimeAfterEaster]{inc-SundayAtVespers-PaschalTime}
}

%\chapter{Marian Anthems}
{
\def\nogabcbreaks{T}
\newcommand{\printsimpletone}{
\vspace{0ex plus 0ex minus 1ex}
\needspace{3\baselineskip}
\begin{center}\textbf{Simple Tone.}\end{center}
\vspace{0ex plus 0ex minus 1ex}
}
\newcommand{\printsolemntone}{
%\vspace{0ex plus 0ex minus 0.5ex}
\oldneedspace{3\baselineskip}
\begin{center}\textbf{Solemn Tone.}\end{center}
\vspace{0ex plus 0ex minus 1ex}
}
\newcommand{\afterant}{
\ifx\note\undefined\else%
\textit{\note}

\smallskip
\fi
\sloppy
\begin{columns}
\versicle{\vlatin}{\venglish}
\response{\rlatin}{\renglish}
\colchunk{}
\colplacechunks{}
\colchunk{\hspace*{3em}Orémus.}\colchunk{\hspace*{3em}Let us pray,}
\colplacechunks{}
\prayer{\prayerlatin}{\prayerenglish}
\end{columns}
\ifx\notetwo\undefined\else%
\medskip

\needspace{2\baselineskip}
\textit{\notetwo}

\smallskip
\begin{columns}
\versicle{\vlatintwo}{\venglishtwo}
\response{\rlatintwo}{\renglishtwo}
\colchunk{}
\colplacechunks{}
\colchunk{\hspace*{3em}Orémus.}\colchunk{\hspace*{3em}Let us pray,}
\colplacechunks{}
\prayer{\prayerlatintwo}{\prayerenglishtwo}
\end{columns}
\fi
}

\def\gabcfolder{../MarianAntiphons}

\section{Alma Redemptoris Mater}
\printnote{From Vespers of Saturday before the 1st Sunday of Advent to 2nd Vespers of the Purification.}
\printsimpletone{}
\printgabc{Ant.}{5.}{A}{AlmaSimple}
\printsolemntone{}
\printgabc{Ant.}{5.}{A}{AlmaSolemn}
\begin{quote}{Mother of Christ! hear thou thy people's cry
Star of the deep, and portal of the sky!
Mother of Him who thee from nothing made,
Sinking we strive and call to thee for aid;
Oh, by that joy which Gabriel brought to thee,
Thou Virgin first and last, let us thy mercy see.}\end{quote}

{
\newcommand{\note}{During Advent:}
\newcommand{\vlatin}{Angelus Dómini nuntiávit Maríæ.}
\newcommand{\venglish}{The Angel of the Lord declared unto Mary.}
\newcommand{\rlatin}{Et concépit de Spíritu Sancto.}
\newcommand{\renglish}{And she conceived by the Holy Ghost.}
\newcommand{\prayerlatin}{Grátiam tuam, quaésumus Dómine, méntibus nostris infúnde~:~\dag{} ut qui, Angelo nuntiánte, Christi Fílii tui incarnatiónem cognóvimus,~* per passiónem ejus et crucem ad resurrectiónis glóriam perducámur.  Per eúmdem Christum Dóminum nostrum. \Rbar~Amen.}
\newcommand{\prayerenglish}{Pour forth, we beseech Thee, O Lord, Thy grace into our hearts; that we, to whom the Incarnation of Christ Thy Son was made known by the message of an Angel, may, by His Passion and Cross, be brought to the glory of His Resurrection. Through the same Christ our Lord. \Rbar~Amen.}

\newcommand{\notetwo}{From 1st Vespers of Christmas to 2nd Vespers of the Purification:}
\newcommand{\vlatintwo}{Post pártum Vírgo invioláta permansísti.}
\newcommand{\venglishtwo}{After childbirth thou didst remain a pure Virgin.}
\newcommand{\rlatintwo}{Déi Génitrix intercéde pro nóbis.}
\newcommand{\renglishtwo}{Intercede for us, O Mother of God.}
\newcommand{\prayerlatintwo}{Deus, qui salútis ætérnæ, beátæ Maríæ virginitáte f\oe cúnda, humáno géneri praémia præstitísti~:~\dag{} tríbue, quaésumus; ut ipsam pro nobis intercédere sentiámus,~* per quam merúimus auctórem vitæ suscípere, Dóminum nostrum Jesum Christum Fílium tuum. \Rbar~Amen.}
\newcommand{\prayerenglishtwo}{O God, who, by the fruitful virginity of blessed Mary, hast given to mankind the rewards of eternal salvation; grant, we beseech Thee, that we may experience her intercession for us, through whom we have deserved to receive the Author of life, our Lord Jesus Christ, Thy Son. \Rbar~Amen.}

\afterant{}
}



% Ave Regina Cælorum
\section{Ave Regina Cælorum}
\printnote{From Compline of Feb 2nd (even if the Feast of the Purification be transferred) until Compline of Wednesday in Holy Week.}
\printsimpletone{}
\printgabc{Ant.}{6.}{A}{AveReginaSimple}
\printsolemntone{}
\printgabc{Ant.}{6.}{A}{AveReginaSolemn}
\begin{quote}{Hail, Queen of Heaven! Hail, Queen of Angels! Hail, blest Root and Gate, from which came light upon the world! Rejoice, O glorious Virgin, that surpassest all in beauty! Hail, O most lovely, and pray for us to Christ.}\end{quote}

\bigskip{}
{
\newcommand{\vlatin}{Dignáre me laudáre te Vírgo sacráta.}
\newcommand{\venglish}{Voucshafe, O holy Virgin, that I may praise thee.}
\newcommand{\rlatin}{Da míhi virtútem cóntra hóstes túos.}
\newcommand{\renglish}{Give me power against thine enemies.}
\newcommand{\prayerlatin}{Concéde, miséricors Deus, fragilitáti nostræ præ\-sí\-di\-um :~\gredagger{}~ut qui sanctæ Dei Genitrícis memóriam ágimus, * intercessiónis ejus auxílio a nostris iniquitátibus resurgámus. Per eúmdem Christum Dóminum nostrum. \Rbar~Amen.}
\newcommand{\prayerenglish}{Grant, O merciful God, Thy protection to us in our weakness; that we, who celebrate the memory of the holy Mother of God, may, through the aid of her intercession, rise again from our sins. Through the same Christ our Lord. \Rbar~Amen.}

\afterant{}
}




% Regina Cæli
\needspace{10\baselineskip}
\section{Regina cæli}
\printnote{From Compline of Easter Sunday to Compline of Friday after the Feast of Pentecost inclusively.}
\printsimpletone{}
\printgabc{Ant.}{6.}{R}{ReginaCaeliSimple}
\printsolemntone{}
\printgabc{Ant.}{6.}{R}{ReginaCaeliSolemn}
\begin{quote}{O Queen of heaven, rejoice, alleluia.
For He whom thou didst merit to bear, alleluia;
Has risen as He said, alleluia.
Pray for us to God, alleluia.}\end{quote}

\bigskip{}
{
\newcommand{\vlatin}{Gáude et lætáre Virgo María, allelúia.}
\newcommand{\venglish}{Rejoice and be glad, O Virgin Mary, alleluia.}
\newcommand{\rlatin}{Quia surréxit Dóminus vere, allelúia.}
\newcommand{\renglish}{For the Lord is risen indeed, alleluia.}
\newcommand{\prayerlatin}{Deus, qui per resurrectiónem Fílii tui Dómini nostri Jesu Christi mundum lætificáre dignátus es~:~\gredagger{} præsta, quaésumus; ut per ejus Genitrícem Vírginem Maríam~* perpétuæ capiámus gáudia vitæ. Per eúmdem Christum Dóminum nostrum. \Rbar~Amen.}
\newcommand{\prayerenglish}{O God, who didst vouchsafe to give joy to the world through the resurrection of Thy Son our Lord Jesus Christ; grant, we beseech Thee, that through His Mother, the Virgin Mary, we may obtain the joys of everlasing life. Through the same Christ our Lord. \Rbar~Amen.}

\afterant{}
}





% Salve Regina
\section{Salve Regina}
\printnote{From 1st Vespers of the Feast of the Blessed Trinity to None on Saturday before the 1st Sunday of Advent.}
\printsimpletone{}
\printgabc{Ant.}{5.}{S}{SalveReginaSimple}
\printsolemntone{}
\printgabc{Ant.}{1.}{S}{SalveReginaSolemn}
\begin{quote}{Hail, holy Queen, Mother of mercy; hail, our life, our sweetness, and our hope! To thee do we cry, poor banished children of Eve; to thee do we send up our sighs, mourning and weeping in this vale of tears.  Turn then, most gracious advocate, thine eyes of mercy towards us; and after this our exile, show unto us the blessed fruit of thy womb, Jesus.  O clement, O loving, O sweet Virgin Mary.}\end{quote}

\bigskip{}
{
\newcommand{\vlatin}{Ora pro nóbis sáncta Déi Génitrix.}
\newcommand{\venglish}{ Pray for us, O holy Mother of God.}
\newcommand{\rlatin}{Ut dígni efficiámur promissiónibus Chrísti.}
\newcommand{\renglish}{That we may be made worthy of the promises of Christ.}
\newcommand{\prayerlatin}{Omnípotens sempitérne Deus, qui gloriósæ Vírginis Matris Maríæ corpus et ánimam, ut dignum Fílii tui habitáculum éffici mererétur, Spíritu Sancto cooperánte præparásti~:~\gredagger{} da, ut cujus commemoratióne lætámur,~* ejus pia intercessióne ab instántibus malis et a morte perpétua liberémur. Per eúmdem Christum Dóminum nostrum. \Rbar~Amen.}
\newcommand{\prayerenglish}{Almighty, everlasting God, who with the cooperation of the Holy Ghost didst prepare the body and soul of the glorious Virgin Mary, to make it fit to be the worthy dwelling of Thy Son; grant that by the loving intercession of her in whose commemoration we rejoice, we may be delivered from present ills, and from everlasting death.  Through the same Christ our Lord. \Rbar~Amen.}

\afterant{}
}

}


\chapter{Proper of the Time -- Advent Season}
{
\newcommand{\benedicamusdomino}[1][advent]{
	\benedicamusdominomaster{#1}
}
\newcommand{\printhymnnote}{
	\noindent\printnote{Hymn.~\emph{Creátor alme síderum}, page \pageref{hymn-creatoralmesiderum}.
	\Vbar~\emph{Roráte}, page \pageref{vr-rorate}.}
}
\newcommand{\printoant}[2]{
%	#1 number
%	#2 translation
{
\oldneedspace{5\baselineskip}
\subtitle{December #1.}
\printgabc{\small At Magn.}{\small \oldstylenums{Ant.~2.~D}}{O}{December#1-MagAntiphon}
\translation[]{#2}
\medskip
\emph{\emph{Magnificat}.~page \pageref{oantiphon-magnificat}.}
}
}

\normalsize
%\def\nogloriapatri{T}
%\def\breakbeforeEuouae{T}
%1st sunday of advent
{
\section{First Sunday of Advent}
\subtitle{\nth{1} Class, Violet}

\def\premag{\def\noeuouae{T}}
\def\premagverses{\greseteolcustos{manual}}
\def\printfullhymn{
	\label{hymn-creatoralmesiderum}
	{
	\def\gabcfolder{../Advent}
	\def\prehymntranslation{\vspace{-\baselineskip}}
	\printhymn{\oldstylenums{4.}}{C}{hymn-CreatorAlmeSiderum}
	{\item Creator of the stars of night
	Thy people's everlasting light,
	Jesu, Redeemer, save us all,
	And hear Thy servants when they call.

	\item Thou, lest the demon's ancient curse
	Should doom to death a universe,
	In love wast made, Thyself alone,
	The means to save a world undone.

	\item Towards the Cross Thou wentest forth,
	That Thou might'st heal the crimes of earth;
	Proceeding from a virgin shrine,
	The spotless Victim all divine.

	\item At whose dread Name, majestic now,
	All knees must bend, all hearts must bow;
	And things celestial Thee shall own,
	And things terrestrial, Lord alone.

	\item O Thou, whose coming is with dread,
	To judge and doom the quick and dead.
	Thy heavenly grace on us bestow,
	To shield us from our ghostly foe.

	\item To God the Father, God the Son,
	And God the Spirit, Three in One,
	Laud, honour, might, and glory be
	From age to age eternally.
	Amen.}

	{
		\def\vrlinebreak{T}
		\label{vr-rorate}
		\printvr[\greseteolcustos{manual}]{vr}
		{Ye heavens, drop down dew from above, and let the clouds rain down the Just One.}
		{Let the earth open and bud forth the Saviour.}
	}
	}
}
\def\prepsalmfive{\greseteolcustos{manual}}
\def\prevespers{%
	\let\oldthing\antfourtranslation
	\def\antfourtranslation{\oldthing\vspace{-\baselineskip}}%
	\let\oldthingb\antfivetranslation
	\def\antfivetranslation{\oldthingb\vspace{-\baselineskip}}%
}
\printvespers[../Advent1]{inc-Advent1}
\bigskip
\benedicamusdomino{}
}



% Advent 2
{
\section{Second Sunday of Advent}
\subtitle{\nth{1} Class, Violet}
\printnote{If today is December 8, the Second Vespers of the Immaculate Conception is sung with a Commemoration of this Sunday, page \pageref{immaculateconception}.}
\medskip
\def\preanttwo{\needspace{5\baselineskip}}
\def\prepsalmthree{\needspace{5\baselineskip}}
\def\prepsalmfive{\greseteolcustos{manual}}
\def\premag{\def\noeuouae{T}}
\def\premagverses{\greseteolcustos{manual}}
%\def\prechapter{\vspace{-\baselineskip}}
\printvespers[../Advent2]{inc-Advent2-Vespers2}

\noindent
\printnote{If today is December 7, the First Vespers of the Immaculate Conception is commemorated as follows.  Otherwise \benedicamusdominoreference{advent}}
\bigskip
\hrule
}



% commemoration of First Vespers of Immaculate Conception
{
\def\beginvrcols{\begin{parcolumns}[rulebetween,colwidths={1=0.45\linewidth}]{2}}
\printcommemoration[../December8-ImmaculateConception]{commemorationImmaculateConception-Vespers1}

\bigskip
\benedicamusdomino{}
}


% 3rd Sunday of Advent
{
\section{Third Sunday of Advent}
\subtitle{\nth{1} Class, Violet or Rose}

\def\prepsalmfive{\greseteolcustos{manual}}
\def\preantfour{\needspace{15\baselineskip}}
\def\premag{\def\noeuouae{T}
\printnote{When the third Sunday of Advent falls on December 17, the Antiphon \emph{O Sapiéntia} on page \pageref{osapientia} is sung in place of \emph{Beáta es}.}
\medskip

\label{magant-beataes}
}
\def\premagverses{\greseteolcustos{manual}}
\def\postmag{
	\def\nomagtitle{T}
	\def\magsolemn{T}
	%\definemag{2}{D}
	\def\gabcfolder{../Advent}
	\def\noeuouae{T}
		%print o sapientia
	%\printoant{17}{O Wisdom, which camest out of the mouth of the Most High, reaching from end to end and ordering all things mightily and sweetly: come and teach us the way of prudence.}
	\renewcommand{\anttranslation}{O Wisdom, which camest out of the mouth of the Most High, reaching from end to end and ordering all things mightily and sweetly: come and teach us the way of prudence.}
	\oldneedspace{5\baselineskip}
	\subtitle{December 17.}
	\label{osapientia}
	\def\preverses{\greseteolcustos{manual}}
	\printmag{2D}{O}{December17-MagAntiphon}
}
\def\prechapter{\vspace{-\baselineskip}}
\def\beginchaptercols{\begin{parcolumns}[rulebetween,colwidths={1=0.48\linewidth}]{2}}
\printvespers[../Advent3]{inc-Advent3}
\bigskip
\benedicamusdomino{}
}



%Advent 4
{
\section{Fourth Sunday of Advent}
\subtitle{\nth{1} Class, Violet}
\printnote{If today is December 24, First Vespers of Christmas is used on page \pageref{christmas}.}
\medskip

\def\prepsalmtwoverses{\oldneedspace{2\baselineskip}}
\def\prepsalmfive{\greseteolcustos{manual}}
\def\magreplacement{
	%\pagebreak
	\bigskip
	{
	\def\magsolemn{T}
	\definemag{2}{D}
	\def\gabcfolder{../Advent}
	\def\noeuouae{T}
	\printoant{18}{O Adonai, and Leader of the house of Israël, who didst appear to Moses in the flame of the burning bush and didst give unto him the law on Sinai: come and with an outstretched arm redeem us.}
	\printoant{19}{O Root of Jesse, which standest for an ensign of the people, before whom kings shall keep silence, whom the Gentiles shall beseech: come and deliver us, and tarry not.}
	\printoant{20}{O Key of David, and Sceptre of the house of Israël, that openest and no man shutteth, and shuttest and no man openeth: come and bring the prisoner forth from the prison-house, and him that sitteth in darkness and in the shadow of death.}
	\printoant{21}{O Day-spring, Brightness of light eternal, and Sun of Justice, come and enlighten them that sit in darkness and in the shadow of death.}
	\printoant{22}{O King of the Gentiles and the desire thereof, Thou cornerstone that makest both one, come and deliver mankind, whom Thou didst form out of clay.}
	\printoant{23}{O Emmanuel, our King and Lawgiver, the desire of the nations and the Saviour thereof, come to save us, O Lord our God.}

	\bigskip
	\pagebreak
	\vfil
	{\greseteolcustos{manual}
	\label{oantiphon-magnificat}
	\begin{magnificat}{\magtex}
	\magverses
	\end{magnificat}
	\noindent\emph{Repeat antiphon.}}
	\vfil
	}
}
\printvespers[../Advent4]{inc-Advent4}
\bigskip
\benedicamusdomino{}
}
}

% christmas through Holy Family
%\chapter{Proper of the Time -- Christmas Season}
{
\newcommand{\benedicamusdomino}[1][1]{
  \benedicamusdominomaster{#1}
}

{
\label{christmas}
\section{December 24: Nativity of our Lord -- I Vespers}
\subtitle{\nth{1} Class, White or Gold}

\def\deusinadjutoriumsolemn{T}
\newcommand{\printhymnnote}{
	\noindent\printnote{Hymn.~\emph{Jesu Redémptor ómnium}, page \pageref{hymn-jesuredemptoromnium}.\\}
}
\def\definevesperspropers{\newcommand{\chaptertext}{\dropcap{latin}{Appáruit benígnitas et humánitas Salvatóris nostri} \textbf{Dé}\-i~\dag{} non ex opéribus justítiæ, quæ \emph{fé\-ci}\-\textbf{mus} nos,~* sed secúndum suam misericórdiam salvos nos \textbf{fé}cit.}
\newcommand{\chaptertranslation}{The goodness and kindness of God our Saviour appeared; not by the works of justice which we have done, but according to His mercy He saved us.}

\newcommand{\vrtex}{vrCrastinaDie}
\newcommand{\vtranslation}{Tomorrow shall the iniquity of the land be blotted out.}
\newcommand{\rtranslation}{And the Saviour of the world shall reign over us.}

\newcommand{\maganttex}{an--cum_ortus_fuerit--solesmes}
\newcommand{\magantinitial}{C}
\newcommand{\maganttranslation}{When the sun has risen in the sky, you shall see the King of kings coming forth from the Father, like a bridegroom from his chamber.}
\def\magsolemn{T}
\definemag{8}{G}

\newcommand{\collect}{Concéde, quǽsumus omnípotens Deus~:~\dag{} ut nos Unigéniti tui nova per carnem Natívitas líberet;~* quos sub peccáti jugo vetústa sérvitus tenet. Per eúm\-dem Dóminum.}
\newcommand{\collecttranslation}{Grant, we beseech Thee, almighty God, that the new birth of Thine only-begotten Son in the flesh may set us free, who are held by the old bondage under the yoke of sin.  Through the same our Lord.}
%
}
\def\prepsalmtitleone{\vspace{-0.5\baselineskip}}
\def\prepsalmtitletwo{\vspace{-0.5\baselineskip}}
\def\prepsalmtitlethree{\vspace{-0.5\baselineskip}}
\def\preantfour{\needspace{10\baselineskip}}
%\def\prepsalmtitlefour{\vspace{-\baselineskip}}
\def\preantfive{\oldneedspace{10\baselineskip}}
%\def\prechapter{\vspace{-\baselineskip}}
%\def\precollect{\vspace{-0.2\baselineskip}}
\def\begincollectcols{\begin{parcolumns}[rulebetween,colwidths={1=0.44\linewidth}]{2}}
\def\prevr{\label{vrcrastina}}
\printvespers[../Christmas-Vespers1]{inc-Christmas-Vespers1-psalms}

\benedicamusdomino{}
}

{
\section[December 25: Nativity of our Lord -- II Vespers]{December 25: Nativity of our Lord -- II Vespers\sectionmark{December 25: Nativity of our Lord \& Sunday within the Octave}}
\sectionmark{December 25: Nativity of our Lord \& Sunday within the Octave}
\subtitle{\nth{1} Class, White or Gold}
\medskip
\subtitle{\&}
\vspace{-0.5\baselineskip}
\section[Sunday within the Octave]{Sunday within the Octave\sectionmark{December 25: Nativity of our Lord \& Sunday within the Octave}}
\sectionmark{December 25: Nativity of our Lord \& Sunday within the Octave}
\subtitle{\nth{2} Class, White}
\bigskip

\def\deusinadjutoriumsolemn{T}
\def\definevesperspropersalt{\newcommand{\chaptertext}{\dropcap{latin}{Multifáriam multísque modis olim Deus loquens pátribus in} pro\-\textbf{phé}\-tis~:~\dag{} novíssime diébus istis locútus est nobis in Fílio, quem constítuit hærédem ú\-\emph{ni\-ver}\-\textbf{só}\-rum,~* per quem fecit et \textbf{saé}\-cu\-la.}
\newcommand{\chaptertranslation}{God, who at sundry times and in divers manners spoke in times past to the fathers by the prophets, last of all in these days hath spoken to us by His Son, whom He hath appointed heir of all things, by whom also He made the world.}

\newcommand{\vrtex}{vrNotumFecitDominusSolemn}
\newcommand{\vtranslation}{The Lord hath made known, alleluia.}
\newcommand{\rtranslation}{His salvation, alleluia.}

\newcommand{\maganttex}{MagnificatAntiphon}
\newcommand{\magantinitial}{H}
\newcommand{\maganttranslation}{This day Christ is born: this day the Saviour hath appeared: this day the Angels sing on earth, and the Archangels rejoice: this day the just exult, saying: Glory to God in the highest, alleluia.}
\def\magsolemn{T}
\definemag{1}{g2}

\newcommand{\collect}{Concéde, quaésumus omnípotens Deus~:~\dag{} ut nos Unigéniti tui nova per carnem Natívitas líberet;~* quos sub peccáti jugo vetústa sérvitus tenet. Per eúmdem Dóminum.}
\newcommand{\collecttranslation}{Grant, we beseech Thee, almighty God, that the new birth of Thine only-begotten Son in the flesh may set us free, who are held by the old bondage under the yoke of sin.  Through the same Jesus Christ our Lord.}

  \let\oldthing\maganttranslation
  \def\maganttranslation{\oldthing\vspace{-0.5\baselineskip}}
}
\def\definevesperspropers{\newcommand{\chaptertext}{\dropcap{latin}{Fratres~: Quanto témpore hæres párvulus est, nihil differt a servo, cum sit dóminus} {\textbf{óm}\-ni\-um~:~\dag{} sed sub tutóribus et ac\-\emph{tó\-ri}\-\textbf{bus} est,~* usque ad præfinítum tempus a \textbf{pá}tre.}}
\newcommand{\chaptertranslation}{Brethren, as long as the heir is a child, he differeth nothing from a servant, though he be lord of all: but is under tutors and governors until the time appointed by the father.}

\newcommand{\vrtex}{vrVerbumCaroFactumEst}
\newcommand{\vtranslation}{The Word was made flesh, alleluia.}
\newcommand{\rtranslation}{And dwelt among us, alleluia.}

\newcommand{\maganttex}{MagAnt-PuerJesus}
\newcommand{\magantinitial}{P}
\newcommand{\maganttranslation}{The Child Jesus advanced in age and wisdom before God and men.}
\def\magsolemn{F}
\definemag{6}{F}

\newcommand{\collect}{Omnípotens sempitérne Deus, dírige actus nostros in beneplácito tuo~:~\dag{} ut in nómine dilécti Fílii tui,~* mereámur bonis opéribus abundáre.  Qui tecum vivit et regnat.}
\newcommand{\collecttranslation}{O almighty and everlasting God, direct our actions according to Thy good pleasure; that in the Name of Thy beloved Son we may deserve to abound in good works: Who with Thee liveth and reigneth.}


  \let\oldthing\maganttranslation
  \def\maganttranslation{\oldthing\vspace{-0.5\baselineskip}}
}
\def\vesperspropersaltnote{At II Vespers of Christmas:}
\def\vesperspropersnote{\needspace{4\baselineskip}At II Vespers of the Sunday within the Octave of Christmas:}
\def\prepsalmthree{\needspace{8\baselineskip}}
\def\prepsalmthreeverses{\vspace{-0.1\baselineskip}}
\def\prerepeatantiphonthree{\vspace{-0.85\baselineskip}}
\def\prechapter{\vspace{-1\baselineskip}}
\def\premag{\def\noeuouae{T}}
\def\premagtitle{\vspace{-0.3\baselineskip}}
\def\premagverses{\greseteolcustos{manual}}
\def\prevr{On \emph{Dec.~24, \emph{\Vbar{}~Crástina die}, p.~\pageref{vrcrastina}}.

\emph{On Dec.~31, \emph{\Vbar{}~Verbum} \& Jan.~1, \emph{\Vbar{}~Notum fecit}, p.~\pageref{vrverbumcarofactumest}}.
\medskip
}
\def\hymnlabel{hymn-jesuredemptoromnium}
\def\hymnlinetwo{\oldstylenums{1.}}
\def\hymntex{hymn-JesuRedemptorOmnium}
\def\hymninitial{J}
\def\hymntranslation{\item Jesus, Redeemer of the world,
Begotten ere the dawn of light,
Wast of the Father's glory born,
Immense in glory as in might.

\item Thou art the Father's splendid light,
Thou art th'eternal hope of all
Throughout the world to Thee we pray,
O, hear Thy servants as they call.

\item Remember, O creator Lord!
That from the Virgin's sacred womb
Thou didst come forth, and of her flesh
Thou didst our mortal form assume.

\item This day, recurring, year by year,
Bears witness true that all alone
To save the world Thou camest forth,
Proceeding from the Father's throne.

\item This day, the stars, the earth, and sea,
And all creation welcome sing.
This day which brought our liberty.
When came our Lord, our Saviour, King.

\item And we, too, Lord, who have been washed
In Thine own font of Blood divine,
Offer the tribute of our praise
On this blest natal day of Thine.

\item O Jesu, of the Virgin born,
Unceasing glory be to Thee;
And to the Father infinite,
And Holy Ghost eternally.  Amen.
}

\def\prehymn{%
  \grechangedim{spaceabovelines}{0.40 cm plus 0.36461 cm minus 0.09114 cm}{scalable}%
}
\def\begincollectcols{\begin{parcolumns}[rulebetween,colwidths={1=0.42\linewidth}]{2}}
\printvespers[../Christmas]{inc-Christmas-Vespers2-psalms}

\grechangedim{spaceabovelines}{0.45576 cm plus 0.36461 cm minus 0.09114 cm}{scalable}%
\benedicamusdomino{}
}

{
\section{January 1: Octave of the Nativity of our Lord}
\vspace{-0.5\baselineskip}
\subtitle{\nth{1} Class, White or Gold}

\subtitle{I \& II Vespers}
\label{circumcision}
\medskip

\def\deusinadjutoriumsolemn{T}
\def\definevesperspropers{\newcommand{\vrtex}{vrNotumFecitDominus}
\newcommand{\vtranslation}{The Lord hath made known, alleluia.}
\newcommand{\rtranslation}{His salvation, alleluia.}

\newcommand{\maganttex}{MagnificatAntiphon}
\newcommand{\magantinitial}{M}
\newcommand{\maganttranslation}{Great the mystery of our inheritance: the womb that knew not man is become the temple of God: taking flesh from her, he is not defiled: all nations shall come, and say: Glory to Thee, O Lord.}
\def\magsolemn{T}
\definemag{2}{A}
}
\def\preantthree{\needspace{10\baselineskip}}
 \newcommand{\printhymnnote}{
 	\noindent\printnote{Hymn.~\emph{Jesu Redémptor ómnium}, page \pageref{hymn-jesuredemptoromnium}.\\}
 }
\def\definevesperspropersalt{\newcommand{\vrtex}{vrVerbumCaro}
\newcommand{\vtranslation}{The Word was made flesh, alleluia.}
\newcommand{\rtranslation}{And dwelt amongst us, alleluia.}

\newcommand{\maganttex}{an--propter_nimiam--solesmes}
\newcommand{\magantinitial}{P}
\newcommand{\maganttranslation}{For His exceeding charity wherewith He loved us, God sent His own Son, in the likeness of sinful flesh, alleluia.}
\def\magsolemn{T}
\definemag{8}{G}
}
\def\vesperspropersnote{At II Vespers:}
\def\vesperspropersaltnote{At I Vespers:}
\def\prepsalmtitletwo{\needspace{6\baselineskip}}
\def\preantfive{\bigskip}
%\def\precollect{\vspace{-2\baselineskip}}
\def\prechapter{\noindent\printnote{For the Feast of the Purification \& Presentation, continue to \emph{Chapter}, p.~\pageref{purification-chapter}.}\par}
\def\premagtitle{\oldneedspace{6\baselineskip}}
\def\prevr{\label{vrverbumcarofactumest}}
\printvespers[../ChristmasOctave-Circumcision]{inc-Circumcision-Vespers-common}

\benedicamusdomino{}
}

{
\needspace{5\baselineskip}
\section{The Most Holy Name of Jesus}
\vspace{-0.5\baselineskip}
\subtitle{\nth{2} Class, White}
\medskip
%\subtitle{II Vespers}

\printnote{This feast is celebrated on the Sunday between the Octave Day of the Nativity and the Epiphany of Our Lord.  If no Sunday occurs within that time, this feast is then celebrated on January 2.}

\printnote{When the Feast of the Most Holy Name of Jesus occurs on January 5, \emph{First Vespers of the Epiphany} on page \pageref{epiphany} is sung without a commemoration of the Holy Name of Jesus.}

\medskip{}
%\def\definevesperspropersalt{\newcommand{\maganttex}{MagnificatAntiphon1}
\newcommand{\magantinitial}{T}
\newcommand{\maganttranslation}{Thou art the Shepherd of the sheep and the Prince of the Apostles, and unto thee are given the keys of the kingdom of heaven.}
\def\magsolemn{T}
\definemag{1}{f}
}
\def\definevesperspropers{\newcommand{\maganttex}{an--vocabis_nomen_ejus--solesmes}
\newcommand{\magantinitial}{V}
\newcommand{\maganttranslation}{Thou shalt call His Name Jesus, for He shall save His people from their sins, alleluia.}
\def\magsolemn{T}
\definemag{1}{g}
}
%\def\vesperspropersaltnote{At I Vespers of the Most Holy Name of Jesus:}
%\def\vesperspropersnote{At II Vespers of the Most Holy Name of Jesus:}
\def\preantone{\bigskip}
\def\prepsalmtitleone{\needspace{5\baselineskip}}
\def\preanttranslationthree{\vspace{-0.2\baselineskip}}
\def\prepsalmtitlethree{\vspace{-0.1\baselineskip}}
\def\prepsalmthreeverses{}
\def\prerepeatantiphonthree{}
\def\prepsalmtitlefour{\vspace{-0.5\baselineskip}}
\def\hymnlabel{hymn-jesudulcismemoria}
\def\begincollectcols{\begin{parcolumns}[rulebetween,colwidths={1=0.425\linewidth}]{2}}
%\def\prepsalmtitlefive{\vspace{\baselineskip}}
\def\prechapter{\vspace{-0.5\baselineskip}}
\def\prehymn{\needspace{5\baselineskip}}
\printvespers[../HolyName]{inc-HolyName-common}

\benedicamusdomino[2]{}
}

{
\chapter{Proper of the Time -- Epiphany Season}
\vspace{-0.5\baselineskip}
\label{epiphany}
\section{January 6: The Epiphany of Our Lord}
\subtitle{\nth{1} Class, White or Gold}

\subtitle{I \& II Vespers}

\def\deusinadjutoriumsolemn{T}
\def\definevesperspropersalt{\definepsalm{5}{116}{7}{c2}

\newcommand{\maganttex}{MagnificatAntiphon1}
\newcommand{\magantinitial}{M}
\newcommand{\maganttranslation}{When the Wise Men saw the star, they said one to another: This is the sign of the great King: let us go and search for Him, and offer Him gifts, gold, frankincense, and myrrh.}
\definemag{8}{G}

  %\def\prepsalmtitlefive{\medskip}
}
\def\definevesperspropers{\definepsalm{5}{113}{7}{c2}

\newcommand{\maganttex}{MagnificatAntiphon2}
\newcommand{\magantinitial}{T}
\newcommand{\maganttranslation}{This day we keep a holiday in honour of three wonders: today a star led the wise men to the manger; today water was made wine at the marriage; today Christ was pleased to be baptised in the Jordan by John for our salvation, alleluia.}
\definemag{1}{D}
%
  %\def\prepsalmtitlefive{\medskip}
  \let\oldthing\maganttranslation
  \def\maganttranslation{\oldthing\vspace{-0.25\baselineskip}}
}
\def\vesperspropersaltnote{At I Vespers:}
\def\vesperspropersnote{At II Vespers:}
\def\prepsalmthree{\needspace{5\baselineskip}}
\def\prepsalmthreeverses{\vspace{-0.05\baselineskip}}
\def\prerepeatantiphonthree{}
\def\prepsalmtitlefour{\vspace{-0.5\baselineskip}}
%\def\prechapter{\vspace{-1\baselineskip}}
\def\beginchaptercols{\begin{parcolumns}[rulebetween,colwidths={1=0.42\linewidth}]{2}}
\def\prevr{\needspace{8\baselineskip}}

\printvespers[../Epiphany]{inc-Epiphany-Vespers}

\benedicamusdomino{}
}

{
\section{Holy Family of Jesus, Mary \& Joseph}
\subtitle{First Sunday after the Epiphany.}
\subtitle{\nth{2} Class, White}

\subtitle{I \& II Vespers}
%\def\begincollectcols{\begin{parcolumns}[rulebetween,colwidths={1=0.43825\linewidth}]{2}}

\def\preanttwo{\oldneedspace{12\baselineskip}}
%\def\prepsalmtitletwo{\vspace{-0.1\baselineskip}}
%\def\prerepeatantiphontwo{}
%\def\prerepeatantiphonthree{}
%\def\prepsalmtitlethree{\vspace{-1.5\baselineskip}}
%\def\prepsalmtitlefive{\vspace{-1.5\baselineskip}}
\def\prepsalmfiveverses{\vspace{-0.05\baselineskip}}
%\def\prerepeatantiphonfive{\vspace{-0.3\baselineskip}}
\def\prevespers{%
  \let\oldthingb\antthreetranslation
  \def\antthreetranslation{\vspace{-0.1\baselineskip}\oldthingb\vspace{-1\baselineskip}}%
  \let\oldthingc\antfivetranslation
  \def\antfivetranslation{\vspace{-0.4\baselineskip}\oldthingc\vspace{-0.8\baselineskip}}%
  %\def\prehymntranslation{\vspace{-0.3\baselineskip}}
  \def\prehymn{\vfil}
}
\printvespers[../HolyFamily]{inc-HolyFamily-Vespers2}

\benedicamusdomino[2]{}
}

}

%\chapter{Proper of the Time -- After Epiphany}
{
\newcommand{\benedicamusdomino}[1][sunday]{
  \benedicamusdominomaster{#1}
}
\def\printhymnnote{}
\def\printcommonvespers{
  \vspace{-0.5\baselineskip}
	\subtitle{\nth{2} Class, Green}
  \smallskip
  \deusinadjutorium{}
  \hfill
	\printnote{\emph{Vespers of Sundays throughout the year}, p.~\pageref{sundayvespers}.\par}

}

{
\section{\nth{2} Sunday after Epiphany}
\label{epiphany2}
\printcommonvespers{}
\printvespersmag[../TimeAfterEpiphany]{inc-VespersMagnificatEpiphany2}

\bigskip
\benedicamusdomino{}
}

{
\section{\nth{3} Sunday after Epiphany}
\label{epiphany3}
\printcommonvespers{}
\printvespersmag[../TimeAfterEpiphany]{inc-VespersMagnificatEpiphany3}

\bigskip
\benedicamusdomino{}
}

{
\section{\nth{4} Sunday after Epiphany}
\label{epiphany4}
\def\postmagtitle{\label{epiphany4-mag}}
\def\precollect{\printvrdirigatur}
\printcommonvespers{}
\printvespersmag[../TimeAfterEpiphany]{inc-VespersMagnificatEpiphany4}

\bigskip
\benedicamusdomino{}
}

{
\needspace{18\baselineskip}
\section{\nth{5} Sunday after Epiphany}
\label{epiphany5}
\printcommonvespers{}
\printvespersmag[../TimeAfterEpiphany]{inc-VespersMagnificatEpiphany5}

\bigskip
\benedicamusdomino{}
}

{
\section{\nth{6} Sunday after Epiphany}
\label{epiphany6}
\printcommonvespers{}
\def\prevespers{%
%  \let\oldthing=\englishmagantiphon
%  \def\englishmagantiphon{\oldthing\pagebreak}
}
\def\premagverses{\needspace{16\baselineskip}}
\printvespersmag[../TimeAfterEpiphany]{inc-VespersMagnificatEpiphany6}

\bigskip
\benedicamusdomino{}
}
}

%from shrovetide through passiontide
%{
	\label{septuagesima}
	\chapter{Proper of the Time -- Septuagesima \& Lent}
  \def\printbenedicamusdomino{
    \noindent\emph{Benedicamus Domino}, p. \pageref{benedicamus-domino-sunday}.
  }
  \section{Septuagesima}
  \printterce{inc-septuagesima}{septuagesima}

  \section{Sexagesima}
  \printterce{inc-sexagesima}{sexagesima}

  \section{Quinquagesima}
  \printterce{inc-quinquagesima}{quinquagesima}

  {
    \newcommand{\printrefs}[1]{%
      \def\dotting{\hfill%\leaders\hbox to 1em{\hfil.\hfil}\hfill
        }%
      \begin{multicols}{2}%
      \noindent{}1st Sunday of Lent,\dotting \emph{below}\\
      2nd Sunday of Lent,\dotting \emph{p.~\pageref{lent2-#1}}\\
      3rd Sunday of Lent,\dotting \emph{p.~\pageref{lent3-#1}}\\
      4th Sunday of Lent,\dotting \emph{p.~\pageref{lent4-#1}}
      \end{multicols}%
      \smallskip
    }

	  \def\printhymn{

		  {\centering Hymn.\par}\label{hymn-lent}

	  	{\def\gabcfolder{.}
	  	\printgabc{1.}{}{N}{hy--nunc_sancte_nobis_(in_quadragesima)--solesmes_1961}}

      \printrefs{ant}
		  \bigskip
	  }

    \def\printshortresp{%
    \label{shortresp-lent}%
    {\def\gabcfolder{.}
    \printgabc{Short}{Resp.}{I}{re--ipse_liberavit--solesmes}}

    \translation[]{\Vbar{}~For he hath delivered me from the snare of the hunters.
    \Rbar{}~For he hath\dots{}
    \Vbar{}~And from the sharp word.
    \Rbar{}~From the snare of the hunters.
    \Vbar{}~Glory be to the Father, and to the Son, and to the Holy Ghost.
    \Rbar{}~For he hath\dots{}
    }

    \bigskip
    \gresetinitiallines{0}\label{vr-lent}
    \gregorioscore{vr-scapulis-suis}
    \let\myhwidth\relax
    \let\myhhwidth\relax
    \newlength{\myhwidth}
    \settowidth{\myhwidth}{speráb} % text in last word before last vowel of response
    \newlength{\myhhwidth}
    \settowidth{\myhhwidth}{b} % text in last syllable before vowel of versicle
    \addtolength{\myhhwidth}{-\myhwidth}
    \def\myhspace{\hspace{1ex}}
    \begin{nstabbing}
    %\>\Rbar{}~Et \myhspace{} omnes \myhspace{} reges \myhspace{} terræ \myhspace{} glóriam \>\hspace{\myhhwidth}tuam.
    \>\Rbar{}~Et \myhspace{} sub \myhspace{} pennis \myhspace{} ejus \>\hspace{\myhhwidth}sperábis.
    \end{nstabbing}

    \translation[]{\Vbar{}~He will overshadow thee with his shoulders.\\
    \Rbar{}~And under his wings thou shalt trust.}

    \printrefs{collect}
    \bigskip
    }

    \section{First Sunday of Lent}
	  \printterce{inc-Lent1}{lent1}
  }

  \def\printhymn{%
  	, \emph{Hymn}, page \pageref{hymn-lent}

  	\medskip
  }
  \def\printshortresp{
    \noindent\emph{Short Resp. \emph{Ipse liberavit me}}, p. \pageref{shortresp-lent}, \Vbar{}~Scapulis suis, p. \pageref{vr-lent}.
  }
  \section{Second Sunday of Lent}
  \printterce{inc-Lent2}{lent2}

  \section{Third Sunday of Lent}
  \printterce{inc-Lent3}{lent3}

  \section{Fourth Sunday of Lent}
  \printterce{inc-Lent4}{lent4}

  {
    \newcommand{\printrefs}[1]{%
      \def\dotting{\hfill%\leaders\hbox to 1em{\hfil.\hfil}\hfill
        }%
      \begin{multicols}{2}%
      \noindent{}Passion Sunday,\dotting \emph{below}\\
      Palm Sunday,\dotting \emph{p.~\pageref{passion2-#1}}
      \end{multicols}%
      \smallskip
    }

    \def\printhymn{

      {\centering Hymn.\par}\label{hymn-passiontide}

      {\def\gabcfolder{.}
      \printgabc{2.}{}{N}{hy--nunc_sancte_nobis_(in_tempore_passionis)--solesmes_1961}}

      \printrefs{ant}
      \bigskip
    }

    \def\printshortresp{%
    \label{shortresp-passiontide}%
    {\def\gabcfolder{.}
    \printgabc{Short}{Resp.}{E}{re--erue_a_frama--solesmes}}

    \translation[]{\Vbar{}~Deliver from the sword, O God, my soul.
    \Rbar{}~Deliver\dots{}
    \Vbar{}~My only one from the hand of the dog.
    \Rbar{}~O God, my soul.
    \Rbar{}~Deliver, * O God, my soul from the sword.For he hath delivered me from the snare of the hunters.
    }

    \bigskip
    \gresetinitiallines{0}\label{vr-passiontide}
    \gregorioscore{vr-de-ore-leonis}
    \let\myhwidth\relax
    \let\myhhwidth\relax
    \newlength{\myhwidth}
    \settowidth{\myhwidth}{me} % text in last word before last vowel of response
    \newlength{\myhhwidth}
    \settowidth{\myhhwidth}{n} % text in last syllable before vowel of versicle
    \addtolength{\myhhwidth}{-\myhwidth}
    \def\myhspace{\hspace{1.4ex}}
    \begin{nstabbing}
    %\>\Rbar{}~Et \myhspace{} omnes \myhspace{} reges \myhspace{} terræ \myhspace{} glóriam \>\hspace{\myhhwidth}tuam.
    \>\Rbar{}~Et a córnibus unicórnium humilitátem \>\hspace{\myhhwidth}meam.
    \end{nstabbing}

    \translation[]{\Vbar{}~From the lion's mouth, O Lord, save me.
    \Rbar{}~And my lowness from the horns of the unicorns.}

    \printrefs{collect}
    \bigskip
    }

    \section{Passion Sunday}
    \printterce{inc-Passion1}{passion1}
  }
 
  \def\printhymn{%
    , \emph{Hymn}, page \pageref{hymn-passiontide}

    \medskip
  }
  \def\printshortresp{
    \noindent\emph{Short Resp. \emph{Erue a framea}}, p. \pageref{shortresp-passiontide}, \Vbar{}~De ore leonis, p. \pageref{vr-passiontide}.
  }
  \section{Palm Sunday}
  \printterce{inc-Passion2}{passion2}
}

%{
\def\printcommemnote{\smallskip
\noindent
\printnote{\commemorations{}  Otherwise \Vbar~\emph{Bendicámus Dómino}, page \pageref{benedicamusdomino-easter}.}
}
{
\chapter{Proper of the Time -- Easter}
\section{Easter Sunday}
\subtitle{1st Class}
\def\printfullhymn{
    \emph{Chapter, Hymn, and Versicle are all omitted, but the following Antiphon is said :}

    \bigskip
    \def\annot{\small{Ant.}}
    \def\annottwo{\small{\chapterhymnversicleantiphonmode.}}
    %\setinitialspacing{\chapterhymnversicleantiphoninitial}
    \gregorioscore{\gabcfolder/\chapterhymnversicleantiphontex}
    \translation[]{\chapterhymnversicleantiphontranslation}
    \bigskip
}
\def\chapterreplacement{\bigskip}
\def\begincollectcols{\begin{parcolumns}[rulebetween,colwidths={1=0.45\linewidth}]{2}}
\def\postmag{\vspace{-0.05\baselineskip}}
\printvespers[../Easter]{inc-EasterVespers}
\newcommand{\printbenedicamusdomino}[2]{
	\def\annot{\small{#1}}
	\def\annottwo{}
	%\setinitialspacing{B}
    \greseteolcustos{manual}
	\gregorioscore{#2}
    \bigskip
    \hrule
}
\def\breakbeforeresp{T}
\printbenedicamusdomino{\Vbar}{../BenedicamusDomino/BenedicamusDomino_Easter}
}

\newcommand{\printcommonvespers}[1][2nd]{
    \subtitle{#1 Class}
    \printnote{From Vespers of Sundays in Eastertide on page \pageref{sundayvespers-easter}.}
}
{
\newcommand{\benedicamusdomino}[1][easter]{
    \noindent\printnote{\Vbar~\emph{Benedicámus Dómino}, page \pageref{benedicamusdomino-#1}.}
    \bigskip
    \hrule
}

{
\section{Low Sunday}
\printcommonvespers[1st]
\def\printfullhymn{
    \label{hymn-adregiasagnidapes}
    {
        \def\hymnlinetwo{\oldstylenums{8.}}
\def\hymntex{hymn-AdRegiasAgniDapes}
\def\hymninitial{A}
\def\hymntranslation{\item At the Lamb's high feast we sing
praise to our victorious King,
Who hath washed us in the tide
flowing from His pierced side.
\item Praise we Him Whose love divine
gives the guests His Blood for wine,
gives His Body for the feast,
Love the victim, Love the priest.
\item Where the Paschal blood is poured,
Death's dark Angel sheathes his sword;
Israel's hosts triumphant go
through the wave that drowns the foe.
\item Christ, the Lamb Whose Blood was shed,
Paschal victim, Paschal bread;
with sincerity and love
eat we manna from above.
\item Mighty Victim from the sky,
powers of hell beneath Thee lie;
Death is conquered in the fight;
Thou hast brought us life and light.
\item Now Thy banner Thou dost wave;
vanquished Satan and the grave;
see the prince of darkness quelled;
heaven's bright gates are open held.
\item Paschal triumph, Paschal joy,
only sin can this destroy;
from sin's death do Thou set free
souls re-born, dear Lord, in Thee.
\item Hymns of glory, songs of praise,
Father, unto Thee we raise;
risen Lord, all praise to Thee,
ever with the Spirit be.
Amen.}

\def\vrtex{vrManeNobiscum}
\def\vtranslation{Stay with us O Lord, alleluia.}
\def\rtranslation{Because it is towards evening, alleluia.}

        \printhymn{\oldstylenums{\hymnlinetwo}}{\hymninitial}{\hymntex}{\hymntranslation}
        \def\vrlinebreak{T}
        \label{vr-manenobiscum}
        \printvr[\greseteolcustos{manual}]{\vrtex}{\vtranslation}{\rtranslation}
        \bigskip
    }
}
%\def\begincollectcols{\begin{parcolumns}[rulebetween,colwidths={1=0.42\linewidth}]{2}}
\printvespersmag[../TimeAfterEaster]{inc-VespersMagnificatEaster1}

\def\commemorations{If the Feast of the Annunciation has been transferred to the Monday following Low Sunday, First Vespers is commemorated as on page \pageref{annunciation-commem}.  If today is April 30, May 1, or May 2, then First Vespers of St Joseph the Worker is commemorated as follows.}
\printcommemnote{}
}

{
\label{stjoseph-worker-commem}
\def\vrlinebreak{T}
%\printcommemoration[../May1-StJosephWorker]{commemorationStJosephWorker-Vespers1}

\bigskip
\benedicamusdomino{}
}

{
    TODO Feast of St Joseph the worker could have commem of 2nd through 5th Sunday after Easter
    probably just give them a page number and say to use the simple tone for the versicle and response
}

\newcommand{\printhymnnote}{
    \noindent\printnote{Hymn. \emph{Ad Régias Agni Dapes}, page \pageref{hymn-adregiasagnidapes}.
    \Vbar~\emph{Mane nobíscum}, page \pageref{vr-manenobiscum}.}
}
{
\section{2nd Sunday after Easter}
\printcommonvespers{}
\printvespersmag[../TimeAfterEaster]{inc-VespersMagnificatEaster2}
\benedicamusdomino{}
}
{
\section{3rd Sunday after Easter}
\printcommonvespers{}
\printvespersmag[../TimeAfterEaster]{inc-VespersMagnificatEaster3}
\benedicamusdomino{}
}

{
\section{4th Sunday after Easter}
\printcommonvespers{}
\printvespersmag[../TimeAfterEaster]{inc-VespersMagnificatEaster4}
\benedicamusdomino{}
}

{
\section{5th Sunday after Easter}
\printcommonvespers{}
\printvespersmag[../TimeAfterEaster]{inc-VespersMagnificatEaster5}
\benedicamusdomino{}
}

{
\section{Ascension of Our Lord}
\subtitle{1st Class}
TODO

TODO could be commem of 1st vespers of St Joseph the worker if today is April 30 or May 1

}

{
\section{Sunday after the Ascension}
\printcommonvespers{}
\def\printfullhymn{
    {
        \printhymn{\oldstylenums{\hymnlinetwo}}{\hymninitial}{\hymntex}{\hymntranslation}
        \def\vrlinebreak{T}
        \printvr[\greseteolcustos{manual}]{\vrtex}{\vtranslation}{\rtranslation}
    }
}
\printvespersmag[../TimeAfterEaster]{inc-VespersMagnificatSundayAfterAscension}
\benedicamusdomino{}
}
}
}

%\chapter{Proper of the Time -- Pentecost Octave}
{
\newcommand{\benedicamusdomino}[1][1]{
	\noindent\printnote{\Vbar~\emph{Benedicámus Dómino}, page \pageref{benedicamusdomino-#1}.}
	\bigskip
	\hrule
}

{
\section{Pentecost Sunday}
\subtitle{1st Class}
\subtitle{I \& II Vespers}

\def\definevesperspropers{\definepsalm{5}{113}{7}{c2}

\newcommand{\vrtex}{vrLoquebantur}
\newcommand{\vtranslation}{The apostles spoke in divers tongues.}
\newcommand{\rtranslation}{The wonderful works of God.}

\newcommand{\maganttex}{MagnificatAntiphon2}
\newcommand{\magantinitial}{H}
\newcommand{\maganttranslation}{Today the days of Pentecost are complete, alleluia; today the Holy Ghost appeared in fire to the disciples, gave them gifts and graces, sent them into all the world to preach and to bear witness; whoever believes and is baptised shall be saved, alleluia.}

	\def\prepsalmfive{\greseteolcustos{manual}}
}
\def\definevesperspropersalt{\definepsalm{5}{116}{7}{c2}

\newcommand{\vrtex}{vrRepletiSunt}
\newcommand{\vtranslation}{They were all filled with the Holy Ghost, alleluia.}
\newcommand{\rtranslation}{And they began to speak, alleluia.}

\newcommand{\maganttex}{MagnificatAntiphon1}
\newcommand{\magantinitial}{N}
\newcommand{\maganttranslation}{I will not leave you orphans, alleluia; I go away, and I come unto you, alleluia; and your heart shall rejoice, alleluia.}
}
\def\vesperspropersnote{At II Vespers:}
\def\vesperspropersaltnote{At I Vespers:}
%\def\premag{\def\noeuouae{T}}
\def\premagverses{\greseteolcustos{manual}}
\def\printfullhymn{
	{\printhymn{\oldstylenums{\hymnlinetwo}}{\hymninitial}{\hymntex}{\hymntranslation}}
		% print all versicles that could follow hymn?
	{
		\def\vrlinebreak{T}
		%\label{vr-rorate}
		\printnote{\vesperspropersaltnote}
		\definevesperspropersalt
		\printvr[\greseteolcustos{manual}]{\vrtex}{\vtranslation}{\rtranslation}
	}
	\bigskip
	{
		\def\vrlinebreak{T}
		%\label{vr-rorate}
		\printnote{\vesperspropersnote}
		\definevesperspropers
		\printvr[\greseteolcustos{manual}]{\vrtex}{\vtranslation}{\rtranslation}
	}
}
\printvespers[../Pentecost]{inc-PentecostVespers}
\bigskip
\benedicamusdomino{}
}

{
\section{Trinity Sunday}
\subtitle{1st Class}

%\def\premag{\def\noeuouae{T}}
\def\premagverses{\greseteolcustos{manual}}
\def\printfullhymn{
	{
		\printhymn{\oldstylenums{\hymnlinetwo}}{\hymninitial}{\hymntex}{\hymntranslation}
		\def\vrlinebreak{T}
		\printvr[\greseteolcustos{manual}]{\vrtex}{\vtranslation}{\rtranslation}
	}
}
\def\begincollectcols{\vspace{-0.5\baselineskip}\begin{parcolumns}[rulebetween,colwidths={1=0.44\linewidth}]{2}}
\printvespers[../TrinitySunday]{inc-TrinitySunday-Vespers}
\bigskip
\benedicamusdomino{}
}

}


{
Easter is March 22 to April 25
    TODO Feast of Nativity of St John the Baptist (6/24) could have commem of 2nd through 6th Sunday after Pentecost
    TODO Sts Peter and Paul (6/29) could have commem of 3rd through 7th Sunday after Pentecost
    probably just give them a page number and say to use the simple tone for the versicle and response

    TODO Feast of Assumption (8/15) could have commem of Saturday before 3rd Sunday of August or 9th to 13th Sunday after Pentecost

    TODO Feast of St Michael the Archangel (9/29) could have commem of 16th to 20th Sunday after Pentecost

    TODO Feast of Christ the King if on October 31, commem for 1st vespers of All Saints

    TODO Feast of All Saints (11/1) commem of Sunday, Saturday before 1st Sunday of November, 21st - 23rd Sunday after Pentecost or 4th after Epiphany
}

\chapter{Proper of the Time -- Time After Pentecost}
{
\def\printcommonvespers{
	\subtitle{2nd Class}
	\printnote{From Vespers of Sundays throughout the year on page \pageref{sundayvespers}.}
}
\newcommand{\benedicamusdomino}[1][sunday]{
	\noindent\printnote{\Vbar~\emph{Benedicámus Dómino}, page \pageref{benedicamusdomino-#1}.}
	\bigskip
	\hrule
}
\newcommand{\printhymnnote}{}
\newcommand{\printvespersafterpentecost}[1]{
	{
	\def\ordinalending{th}
	\ifnum#1=2\def\ordinalending{nd}\fi
	\ifnum#1=3\def\ordinalending{rd}\fi
	\ifnum#1=21\def\ordinalending{st}\fi
	\ifnum#1=22\def\ordinalending{nd}\fi
	\ifnum#1=23\def\ordinalending{rd}\fi
	\ifnum#1=24\def\ordinalending{th or Last}\fi

	\section{#1\ordinalending{}	Sunday after Pentecost}
	\printcommonvespers{}
	\printvespersmag[../TimeAfterPentecost]{inc-VespersMagnificatPentecost#1}
	\smallskip
	\benedicamusdomino{}
	}
	
}

\printvespersafterpentecost{2}
\printvespersafterpentecost{3}
\printvespersafterpentecost{4}
\printvespersafterpentecost{5}
\printvespersafterpentecost{6}
\printvespersafterpentecost{7}
\printvespersafterpentecost{8}
\printvespersafterpentecost{9}
\printvespersafterpentecost{10}
\printvespersafterpentecost{11}
\printvespersafterpentecost{12}
\printvespersafterpentecost{13}
\printvespersafterpentecost{14}
\printvespersafterpentecost{15}
\printvespersafterpentecost{16}
\printvespersafterpentecost{17}
\printvespersafterpentecost{18}
\printvespersafterpentecost{19}
\printvespersafterpentecost{20}
\printvespersafterpentecost{21}
\printvespersafterpentecost{22}
\printvespersafterpentecost{23}
\printvespersafterpentecost{24}

}

\chapter{Appendix}
%{
\section{Chants at Benediction}
\def\gabcfolder{../BenedictionChants}
\printgabc{8.}{}{O}{OSalutaris}

%\def\translationmargin{68pt}
\translation[]{
\textbf{1.} O saving Victim of the world,
Who openest wide the gates on high,
The foe his bands on us hath hurled,
O, give us strength; for aid we cry.
%\hspace{65ex}
\textbf{2.} To thee, one Lord, yet Three in One,
Let everlasting glory be :
O grant us life that end hath none,
In Fatherland to spend with thee. Amen.}
\let\translationmargin=\undefined
\bigskip

\printgabc{7.}{}{O}{OSalutaris_v2}


\needspace{4\baselineskip}
\printgabc{3.}{}{T}{TantumErgo}

\needspace{4\baselineskip}
{\begin{center}
2.~\emph{Another Chant.}
%\vspace{-1ex}
\end{center}
}
\printgabc{1.}{}{T}{TantumErgo_v2}

\needspace{4\baselineskip}
\translation[]{
\textbf{1.} Therefore we, before it bending,
this great sacrament adore;
type and shadows have their ending
in the new rite evermore;
faith, our outward sense amending,
maketh good defects before.
\textbf{2.} Honour, laud, and praise addressing
to the Father and the Son,
might ascribe we, virtue, blessing,
and eternal benison;
Holy Ghost, from both progressing,
equal laud to thee be done. Amen.
}

\needspace{4\baselineskip}
{\begin{center}
3.~\emph{Modern Chant.}
\end{center}
}

\needspace{4\baselineskip}
\printgabc{5.}{}{T}{TantumErgo_v3}

\needspace{4\baselineskip}
{\begin{center}
4.~\emph{Spanish Chant.}
\end{center}
}

\needspace{4\baselineskip}
\printgabc{5.}{}{T}{TantumErgo-Spanish}


\needspace{4\baselineskip}
\begin{columns}
\versicle{Pánem de cǽlo præstitísti éis. (P.~T.~allelúia.)}{You have given them bread from heaven.}
\response{Omne delectaméntum in se habéntem. (P.~T.~allelúia.)}{Containing all sweetness within it.}
\colchunk{}
\colplacechunks{}
\colchunk{\hspace*{3em}Orémus.}\colchunk{\hspace*{3em}Let us pray,}
\colplacechunks{}
\prayer{Deus, qui nobis sub Sacraménto mirábili passiónis tuæ memóriam reliquísti~:~\dag{} tríbue, quǽsumus, ita nos córporis et sánguinis tui sacra mystéria venerári;~* ut redemptiónis tuæ fructum in nobis júgiter sentiámus.  Qui vivis et regnas in sǽcula sæculórum.  Amen.}
{O God, who hast left us a memorial of Thy passion under a wonderful Sacrament: Grant, we bessech Thee, that we may so venerate the sacred mysteries of Thy body and blood that we may continually sense in ourselves the fruit of Thy redemption. Who livest and reignest\dots{}}
\end{columns}

%TODO: we could add the Divine Praises might be nice...

\needspace{4\baselineskip}
{
\printgabc{5.}{}{A}{AdoremusInAeternum}}%

\begin{psalmverses}[1]
\item \emph{Quóniam} confirmáta est super nos misericórdia \textbf{é}jus~:~* et véritas Dómini mánet \textbf{in} æ\textbf{tér}num.
\item[\Rbar{}] \emph{Adorémus}.
\item \emph{Glória} Pátri, et \textbf{Fí}lio,~* et Spi\textbf{rí}tui \textbf{Sán}cto.
\item \emph{Sicut é}\-rat in princípio, et núnc, et \textbf{sém}per,~* et in sǽcula sæcu\textbf{ló}rum. \textbf{A}men.
\item[\Rbar{}] \emph{Adorémus}.
\end{psalmverses}

\translation[]{Let us adore for ever the Most Holy Sacrament.
{\emph Ps.} All ye nations, praise the Lord : all ye peoples, sing his glory.
Because his merciful kindness is great towards us, and the truth of the Lord endureth for ever. Let us adore\ldots Glory.}
\vfil
}
%{
\pagebreak
\thispagestyle{empty}
\newcommand{\printsimpletone}{
\needspace{3\baselineskip}
\begin{center}\textbf{Simple Tone.}\end{center}
\vspace{0ex plus 0ex minus 2ex}
}
\newcommand{\printsolemntone}{
%\vspace{0ex plus 0ex minus 0.5ex}
\needspace{3\baselineskip}
\begin{center}\textbf{Solemn Tone.}\end{center}
\vspace{0ex plus 0ex minus 1ex}
}

\label{magnificat-grassi}
\vspace*{-\headheight}
\section{Magnificat 8. G.}

\def\betweenLilyPondSystem#1{
  \ifnum#1>1
    \vfil\noindent
  \else
    \linebreak
  \fi
  %\vspace{0.5\baselineskip}
}
\newcommand{\includelilypond}[1]{
  \noindent
  \import{../misc/}{#1}
}
\def\magsolemn{F}
\def\maggrassi{T}
\definemag{8}{G}
\def\annot{\magtone. \magend}
\def\greinitialformat#1{%
{\fontsize{50}{50}\selectfont #1}%
}
\alsetinitialspacing{M}
\def\magoddverses{\input{../psalms/magnificat-\magtone\nostarendmag\if\magsolemn Tsolemn\else simple\fi-grassi-chant-verses}}
\global\def\magtex{../psalms/Magnificat\if\magsolemn Tsolemn\else simple\fi\magtone\nostarendmag}
\greseteolcustos{manual}
\printsimpletone{}
\begin{oddversesmagnificat}{\magtex}
\magoddverses
\end{oddversesmagnificat}

\includelilypond{magnificat-grassi-small-systems}



\def\magsolemn{T}
\let\magant=\undefined
\let\magantlinetwo=\undefined
\let\magtex=\undefined
\let\magverses=\undefined
\definemag{8}{G}
\def\annot{\magtone. \magend}
\alsetinitialspacing{M}
\global\def\magtex{../psalms/Magnificat\if\magsolemn Tsolemn\else simple\fi\magtone\nostarendmag}
\printsolemntone{}
\begin{oddversesmagnificat}{\magtex}
\magoddverses
\end{oddversesmagnificat}
}

\end{document}

