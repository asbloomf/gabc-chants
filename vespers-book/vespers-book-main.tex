% !TEX TS-program = lualatex
% !TEX encoding = UTF-8

% This is a simple template for a LuaLaTeX document using gregorio scores.

\documentclass[letterpaper,12pt]{book} % use larger type; default would be 10pt
\usepackage{../definepsalms}
\usepackage{titlesec}
\usepackage{titletoc}
\usepackage{titleps}
\usepackage{letltxmacro}
\usepackage[strict]{changepage} % gives us \ifoddpage

%\usepackage{hyperref}
\newcommand{\phantomsection}{}
\newcommand{\magnote}{
\IfStrEq{\anttone\nostarend}{8G}{\noindent\emph{A 3 part arrangement of the even verses by Grassi may be found on page \pageref{magnificat-grassi}.}}{}
}
\presetkeys{mymag}{note=\magnote}{}
%\LetLtxMacro{\oldprintmag}{\printmag}
%\renewcommand{\printmag}[4][grassilabel=magnificat-grassi]{
%\oldprintmag[#1]{#2}{#3}{#4}
%}

\setcounter{secnumdepth}{-1}

\def\mywidth{6in}
\def\myheight{9in}

\usepackage{../definepsalms}
\usepackage{adjustbox}
\sloppy
\usepackage{../definepsalms}
\usepackage{adjustbox}
\sloppy
\usepackage{../definepsalms}
\usepackage{adjustbox}
\sloppy
\input{../inc_header} %


\let\oldVbar=\Vbar
\def\Vbar{\oldVbar\hspace{-2pt}}
\let\oldRbar=\Rbar
\def\Rbar{\oldRbar\hspace{-2pt}}

\ifbook{
\lfoot[\thepage]{}
\rfoot[]{\thepage}
}
\ifnotbook{
\cfoot{%
  \ifnum\thepage>1
    \thepage
  \fi
}
}
%\renewcommand\headrulewidth{\oldheadrulewidth}
\chead{
  \ifnum\thepage>1
    \addfontfeature{Numbers=Lining}%
    \heading{} \ifx\matinsnocturn\undefined\else(\matinsnocturn)\fi
  \fi
}

\newcommand{\writeheading}[1]{
  \begin{center}{
  \addfontfeature{Numbers=Lining}
  \textsc{#1}
  }\end{center}
  \medskip
}
\newcommand{\printseparation}{
  %\hfil\rule{3in}{0.5pt}\hfil
  \bigskip
  \bigskip
}

\def\nogloriapatri{T}
 %


\let\oldVbar=\Vbar
\def\Vbar{\oldVbar\hspace{-2pt}}
\let\oldRbar=\Rbar
\def\Rbar{\oldRbar\hspace{-2pt}}

\ifbook{
\lfoot[\thepage]{}
\rfoot[]{\thepage}
}
\ifnotbook{
\cfoot{%
  \ifnum\thepage>1
    \thepage
  \fi
}
}
%\renewcommand\headrulewidth{\oldheadrulewidth}
\chead{
  \ifnum\thepage>1
    \addfontfeature{Numbers=Lining}%
    \heading{} \ifx\matinsnocturn\undefined\else(\matinsnocturn)\fi
  \fi
}

\newcommand{\writeheading}[1]{
  \begin{center}{
  \addfontfeature{Numbers=Lining}
  \textsc{#1}
  }\end{center}
  \medskip
}
\newcommand{\printseparation}{
  %\hfil\rule{3in}{0.5pt}\hfil
  \bigskip
  \bigskip
}

\def\nogloriapatri{T}
 %


\let\oldVbar=\Vbar
\def\Vbar{\oldVbar\hspace{-2pt}}
\let\oldRbar=\Rbar
\def\Rbar{\oldRbar\hspace{-2pt}}

\ifbook{
\lfoot[\thepage]{}
\rfoot[]{\thepage}
}
\ifnotbook{
\cfoot{%
  \ifnum\thepage>1
    \thepage
  \fi
}
}
%\renewcommand\headrulewidth{\oldheadrulewidth}
\chead{
  \ifnum\thepage>1
    \addfontfeature{Numbers=Lining}%
    \heading{} \ifx\matinsnocturn\undefined\else(\matinsnocturn)\fi
  \fi
}

\newcommand{\writeheading}[1]{
  \begin{center}{
  \addfontfeature{Numbers=Lining}
  \textsc{#1}
  }\end{center}
  \medskip
}
\newcommand{\printseparation}{
  %\hfil\rule{3in}{0.5pt}\hfil
  \bigskip
  \bigskip
}

\def\nogloriapatri{T}

\setlength\headheight{0.25in+15pt}
\setlength\headsep{1pc}
\setlength\topskip{0pc}
\setlength\footskip{1pc}
\geometry{outer=0.4in,inner=0.85in,top=0pc+\headheight+\headsep,bottom=0.4in,twoside=true}
\newpagestyle{main}{
\sethead[\garamond{\thepage}][\garamond{\chaptertitle}][] % even
{}{\garamond{\sectiontitle}}{\garamond{\thepage}} % odd
\setfoot[][][] % even
{}{}{} % odd
}
\pagestyle{main}


\titleformat
{\section} % command
[block] % shape
{\phantomsection\large\addfontfeature{Numbers=Lining}} % format
{} % label
{} % sep
{
    % \rule{\textwidth}{1pt}
    % \vspace{1ex}
    \centering
} % before-code
%[
% \vspace{-0.5ex}%
% \rule{\textwidth}{0.3pt}
%] % after-code
 
 
\titleformat{\chapter}[block]
{\thispagestyle{empty}\phantomsection\Large\scshape\addfontfeature{Numbers=Lining}}
{}{0.5em}{\centering}
 
\titlespacing{\chapter}{0pt}{-\headheight}{1pc}
\titlespacing{\section}{0pt}{*2.5}{*1}
\titleclass{\chapter}{top}
\newcommand{\chapterbreak}{\clearpage}
%\titleclass{\section}{top}

\contentsmargin{1pc}
\dottedcontents{chapter}[0pc]{}{2pc}{1pc}
\dottedcontents{section}[0pc]{}{0pc}{1pc}

\newcommand{\printnote}[1]{
	{\normalsize \emph{#1}}
}
\newcommand{\subtitle}[1]{
\begin{center}{
	{\addfontfeature{Numbers=Lining} \normalsize \emph{#1}}
}\end{center}
}

\sloppy
\begin{document}
\normalsize
\setgrefactor{15}
%\pagenumbering{roman}
%\frontmatter

\tableofcontents

%\pagenumbering{arabic}
%\mainmatter

\chapter{Common of Festal Vespers}

%\subtitle{Festal Tone}
\label{deusinadjutorium}
\sectionmark{Deus in adjutórium}
\addcontentsline{toc}{section}{Deus in adjutórium}
\printnote{The Festal Tone may be sung on any Sunday or Feast.}

\smallskip
\def\deusinadjutoriumsolemn{F}
\printdeusinadjutorium{}

%\pagebreak
%\subtitle{Solemn Tone}
\printnote{The Solemn Tone may only be sung on Feasts of the First or Second Class.}

\medskip
\def\deusinadjutoriumsolemn{T}
\printdeusinadjutorium{}


\printnote{Vespers then proceed with the Proper Antiphons, Psalms, Chapter, Hymn, Versicle,
Magnificat, and Collect given for the respective Sunday or Feast.}

\medskip

\printnote{All Vespers conclude with the following:}
\\
\Vbar{} Dóminus vobíscum.\\%\hspace{5em}
\Rbar{} Et cum spíritu tuo.

\printnote{Or, in the absence of a priest or deacon:}
\\
\Vbar{} Dómine, exáudi oratiónem meam.\\
\Rbar{} Et clamor meus ad te véniat.

\medskip

{
\newcommand{\printbenedicamusdomino}[2]{
	\greseteolcustos{manual}
	\gresetinitiallines{1}
	\def\annot{\small{#1}}
	\alsetinitialspacing{B}
	\gregorioscore{#2}
  \greseteolcustos{auto}
}

\vfil
\grechangestaffsize{15}
\oldneedspace{4\baselineskip}\label{benedicamusdomino-1}
{{\centering \bfseries 1.~On feasts of the I class.\\}
\smallskip
\def\breakbeforeresp{T}
\printbenedicamusdomino{5.}{../BenedicamusDomino/or--benedicamus_(in_festis_i_classis_ad_laudes)--solesmes}
\vfil
}
\oldneedspace{4\baselineskip}\label{benedicamusdomino-2}
{{\centering \bfseries 2.~On feasts of the II class.\\}
\smallskip
{\def\breakbeforeresp{T}
\printbenedicamusdomino{2.}{../BenedicamusDomino/or--benedicamus_domino_laudes_--solesmes_1961}
}
}
\vfil

\oldneedspace{4\baselineskip}\label{benedicamusdomino-mary}
\gdef\benedicamusdominonamemary{3}
{{\centering \bfseries 3.~On feasts of the Blessed Virgin.\\}
\smallskip
%\def\breakbeforeresp{T}
\printbenedicamusdomino{1.}{../BenedicamusDomino/BenedicamusDomino_blessedVirgin}}
\vfil

\needspace{6\baselineskip}
\label{benedicamusdomino-sunday}
\gdef\benedicamusdominonamesunday{4}
{{\centering \bfseries 4.~On Sundays during the Year\\and Septuagesima, Sexagesima, and Quinquagesima.\\}
\smallskip
%\def\breakbeforeresp{T}
\printbenedicamusdomino{1.}{../BenedicamusDomino/BenedicamusDomino_Sundays}}

\vfil
\label{benedicamusdomino-lent}\label{benedicamusdomino-advent}
\gdef\benedicamusdominonamelent{5}
\gdef\benedicamusdominonameadvent{\benedicamusdominonamelent}
{{\centering \bfseries 5.~On Sundays of Advent and Lent.\\}
\smallskip
%\def\breakbeforeresp{T}
\printbenedicamusdomino{6.}{../BenedicamusDomino/BenedicamusDomino_SundaysOfAdventAndLent}
\ifthenelse{\boolean{birmingham}}{
	\medskip
	\emph{\normalsize or from \emph{Mass XVII}:}

	\smallskip
	\printbenedicamusdomino{6.}{../BenedicamusDomino/ky--benedicamus_xviia--solesmes}
}{}
}
\vfil

\label{benedicamusdomino-easter}
\gdef\benedicamusdominonameeaster{6}
{{\centering \bfseries 6.~On Sundays of Paschal Time.\\}
\smallskip
%\def\breakbeforeresp{T}
\printbenedicamusdomino{7.}{../BenedicamusDomino/BenedicamusDomino_SundaysOfPaschalTime}}

}

\chapter{Marian Anthems}
{
\def\nogabcbreaks{T}
\newcommand{\printsimpletone}{
\vspace{0ex plus 0ex minus 1ex}
\needspace{3\baselineskip}
\begin{center}\textbf{Simple Tone.}\end{center}
\vspace{0ex plus 0ex minus 1ex}
}
\newcommand{\printsolemntone}{
%\vspace{0ex plus 0ex minus 0.5ex}
\oldneedspace{3\baselineskip}
\begin{center}\textbf{Solemn Tone.}\end{center}
\vspace{0ex plus 0ex minus 1ex}
}
\newcommand{\afterant}{
\ifx\note\undefined\else%
\textit{\note}

\smallskip
\fi
\sloppy
\begin{columns}
\versicle{\vlatin}{\venglish}
\response{\rlatin}{\renglish}
\colchunk{}
\colplacechunks{}
\colchunk{\hspace*{3em}Orémus.}\colchunk{\hspace*{3em}Let us pray,}
\colplacechunks{}
\prayer{\prayerlatin}{\prayerenglish}
\end{columns}
\ifx\notetwo\undefined\else%
\medskip

\needspace{2\baselineskip}
\textit{\notetwo}

\smallskip
\begin{columns}
\versicle{\vlatintwo}{\venglishtwo}
\response{\rlatintwo}{\renglishtwo}
\colchunk{}
\colplacechunks{}
\colchunk{\hspace*{3em}Orémus.}\colchunk{\hspace*{3em}Let us pray,}
\colplacechunks{}
\prayer{\prayerlatintwo}{\prayerenglishtwo}
\end{columns}
\fi
}

\def\gabcfolder{../MarianAntiphons}

\section{Alma Redemptoris Mater}
\printnote{From Vespers of Saturday before the 1st Sunday of Advent to 2nd Vespers of the Purification.}
\printsimpletone{}
\printgabc{Ant.}{5.}{A}{AlmaSimple}
\printsolemntone{}
\printgabc{Ant.}{5.}{A}{AlmaSolemn}
\begin{quote}{Mother of Christ! hear thou thy people's cry
Star of the deep, and portal of the sky!
Mother of Him who thee from nothing made,
Sinking we strive and call to thee for aid;
Oh, by that joy which Gabriel brought to thee,
Thou Virgin first and last, let us thy mercy see.}\end{quote}

{
\newcommand{\note}{During Advent:}
\newcommand{\vlatin}{Angelus Dómini nuntiávit Maríæ.}
\newcommand{\venglish}{The Angel of the Lord declared unto Mary.}
\newcommand{\rlatin}{Et concépit de Spíritu Sancto.}
\newcommand{\renglish}{And she conceived by the Holy Ghost.}
\newcommand{\prayerlatin}{Grátiam tuam, quaésumus Dómine, méntibus nostris infúnde~:~\dag{} ut qui, Angelo nuntiánte, Christi Fílii tui incarnatiónem cognóvimus,~* per passiónem ejus et crucem ad resurrectiónis glóriam perducámur.  Per eúmdem Christum Dóminum nostrum. \Rbar~Amen.}
\newcommand{\prayerenglish}{Pour forth, we beseech Thee, O Lord, Thy grace into our hearts; that we, to whom the Incarnation of Christ Thy Son was made known by the message of an Angel, may, by His Passion and Cross, be brought to the glory of His Resurrection. Through the same Christ our Lord. \Rbar~Amen.}

\newcommand{\notetwo}{From 1st Vespers of Christmas to 2nd Vespers of the Purification:}
\newcommand{\vlatintwo}{Post pártum Vírgo invioláta permansísti.}
\newcommand{\venglishtwo}{After childbirth thou didst remain a pure Virgin.}
\newcommand{\rlatintwo}{Déi Génitrix intercéde pro nóbis.}
\newcommand{\renglishtwo}{Intercede for us, O Mother of God.}
\newcommand{\prayerlatintwo}{Deus, qui salútis ætérnæ, beátæ Maríæ virginitáte f\oe cúnda, humáno géneri praémia præstitísti~:~\dag{} tríbue, quaésumus; ut ipsam pro nobis intercédere sentiámus,~* per quam merúimus auctórem vitæ suscípere, Dóminum nostrum Jesum Christum Fílium tuum. \Rbar~Amen.}
\newcommand{\prayerenglishtwo}{O God, who, by the fruitful virginity of blessed Mary, hast given to mankind the rewards of eternal salvation; grant, we beseech Thee, that we may experience her intercession for us, through whom we have deserved to receive the Author of life, our Lord Jesus Christ, Thy Son. \Rbar~Amen.}

\afterant{}
}



% Ave Regina Cælorum
\section{Ave Regina Cælorum}
\printnote{From Compline of Feb 2nd (even if the Feast of the Purification be transferred) until Compline of Wednesday in Holy Week.}
\printsimpletone{}
\printgabc{Ant.}{6.}{A}{AveReginaSimple}
\printsolemntone{}
\printgabc{Ant.}{6.}{A}{AveReginaSolemn}
\begin{quote}{Hail, Queen of Heaven! Hail, Queen of Angels! Hail, blest Root and Gate, from which came light upon the world! Rejoice, O glorious Virgin, that surpassest all in beauty! Hail, O most lovely, and pray for us to Christ.}\end{quote}

\bigskip{}
{
\newcommand{\vlatin}{Dignáre me laudáre te Vírgo sacráta.}
\newcommand{\venglish}{Voucshafe, O holy Virgin, that I may praise thee.}
\newcommand{\rlatin}{Da míhi virtútem cóntra hóstes túos.}
\newcommand{\renglish}{Give me power against thine enemies.}
\newcommand{\prayerlatin}{Concéde, miséricors Deus, fragilitáti nostræ præ\-sí\-di\-um :~\gredagger{}~ut qui sanctæ Dei Genitrícis memóriam ágimus, * intercessiónis ejus auxílio a nostris iniquitátibus resurgámus. Per eúmdem Christum Dóminum nostrum. \Rbar~Amen.}
\newcommand{\prayerenglish}{Grant, O merciful God, Thy protection to us in our weakness; that we, who celebrate the memory of the holy Mother of God, may, through the aid of her intercession, rise again from our sins. Through the same Christ our Lord. \Rbar~Amen.}

\afterant{}
}




% Regina Cæli
\needspace{10\baselineskip}
\section{Regina cæli}
\printnote{From Compline of Easter Sunday to Compline of Friday after the Feast of Pentecost inclusively.}
\printsimpletone{}
\printgabc{Ant.}{6.}{R}{ReginaCaeliSimple}
\printsolemntone{}
\printgabc{Ant.}{6.}{R}{ReginaCaeliSolemn}
\begin{quote}{O Queen of heaven, rejoice, alleluia.
For He whom thou didst merit to bear, alleluia;
Has risen as He said, alleluia.
Pray for us to God, alleluia.}\end{quote}

\bigskip{}
{
\newcommand{\vlatin}{Gáude et lætáre Virgo María, allelúia.}
\newcommand{\venglish}{Rejoice and be glad, O Virgin Mary, alleluia.}
\newcommand{\rlatin}{Quia surréxit Dóminus vere, allelúia.}
\newcommand{\renglish}{For the Lord is risen indeed, alleluia.}
\newcommand{\prayerlatin}{Deus, qui per resurrectiónem Fílii tui Dómini nostri Jesu Christi mundum lætificáre dignátus es~:~\gredagger{} præsta, quaésumus; ut per ejus Genitrícem Vírginem Maríam~* perpétuæ capiámus gáudia vitæ. Per eúmdem Christum Dóminum nostrum. \Rbar~Amen.}
\newcommand{\prayerenglish}{O God, who didst vouchsafe to give joy to the world through the resurrection of Thy Son our Lord Jesus Christ; grant, we beseech Thee, that through His Mother, the Virgin Mary, we may obtain the joys of everlasing life. Through the same Christ our Lord. \Rbar~Amen.}

\afterant{}
}





% Salve Regina
\section{Salve Regina}
\printnote{From 1st Vespers of the Feast of the Blessed Trinity to None on Saturday before the 1st Sunday of Advent.}
\printsimpletone{}
\printgabc{Ant.}{5.}{S}{SalveReginaSimple}
\printsolemntone{}
\printgabc{Ant.}{1.}{S}{SalveReginaSolemn}
\begin{quote}{Hail, holy Queen, Mother of mercy; hail, our life, our sweetness, and our hope! To thee do we cry, poor banished children of Eve; to thee do we send up our sighs, mourning and weeping in this vale of tears.  Turn then, most gracious advocate, thine eyes of mercy towards us; and after this our exile, show unto us the blessed fruit of thy womb, Jesus.  O clement, O loving, O sweet Virgin Mary.}\end{quote}

\bigskip{}
{
\newcommand{\vlatin}{Ora pro nóbis sáncta Déi Génitrix.}
\newcommand{\venglish}{ Pray for us, O holy Mother of God.}
\newcommand{\rlatin}{Ut dígni efficiámur promissiónibus Chrísti.}
\newcommand{\renglish}{That we may be made worthy of the promises of Christ.}
\newcommand{\prayerlatin}{Omnípotens sempitérne Deus, qui gloriósæ Vírginis Matris Maríæ corpus et ánimam, ut dignum Fílii tui habitáculum éffici mererétur, Spíritu Sancto cooperánte præparásti~:~\gredagger{} da, ut cujus commemoratióne lætámur,~* ejus pia intercessióne ab instántibus malis et a morte perpétua liberémur. Per eúmdem Christum Dóminum nostrum. \Rbar~Amen.}
\newcommand{\prayerenglish}{Almighty, everlasting God, who with the cooperation of the Holy Ghost didst prepare the body and soul of the glorious Virgin Mary, to make it fit to be the worthy dwelling of Thy Son; grant that by the loving intercession of her in whose commemoration we rejoice, we may be delivered from present ills, and from everlasting death.  Through the same Christ our Lord. \Rbar~Amen.}

\afterant{}
}

}


%this is just an example of using pageref
%See Deus in adjutorium on page \pageref{deusinadjutorium} and Benedicamus domino on page \pageref{benedicamusdomino}.

\def\dontrepeatantiphon{T}
{
	\label{advent}
	\chapter{Advent}
	\section{First Sunday of Advent}

  {
	  \def\printhymn{%
		{\centering Hymn.\par}\label{hymn-advent}

	  	{\def\gabcfolder{.}
	  	\printgabc{1.}{}{N}{hy--nunc_sancte_nobis_(in_adventu)--solesmes_1961}}

		\bigskip
	  }
    \def\printshortresp{%
    \label{shortresp-advent}%
    {\def\gabcfolder{.}
    \printgabc{Short}{Resp.}{V}{re--veni_ad_liberandum--solesmes}}

    \translation[]{\Vbar{}~Come to my rescue O God, Lord of hosts.
    \Rbar{}~Come\dots{}
    \Vbar{}~Show us thy face, and we shall be saved.
    \Rbar{}~O God, Lord of hosts.
    \Vbar{}~Glory be to the Father, and to the Son, and to the Holy Ghost.
    \Rbar{}~Come\dots{}
    }

    \bigskip
    \gresetinitiallines{0}
    \gregorioscore{vr-timebunt-gentes}
    \let\myhwidth\relax
    \let\myhhwidth\relax
    \newlength{\myhwidth}
    \settowidth{\myhwidth}{tu}
    \newlength{\myhhwidth}
    \settowidth{\myhhwidth}{n}
    \addtolength{\myhhwidth}{-\myhwidth}
    \def\myhspace{\hspace{1.4ex}}
    \begin{nstabbing}
    \>\Rbar{}~Et \myhspace{} omnes \myhspace{} reges \myhspace{} terræ \myhspace{} glóriam \>\hspace{\myhhwidth}tuam.
    \end{nstabbing}

    \translation[]{\Vbar{}~And the Gentiles shall fear thy name, O Lord.\\
    \Rbar{}~And all the kings of the earth thy glory.}

    \bigskip
    }

	  \printterce[../Advent1]{inc-Advent1}
  }

  \def\printhymn{
  	\noindent\emph{Hymn}, page \pageref{hymn-advent}

  	\medskip
  }
  \def\printshortresp{
    \noindent\emph{Short Resp. \emph{Veni ad liberándum nos}, p.} \pageref{shortresp-advent}, \Vbar{}~Timébunt gentes. 
  }
  \section{Second Sunday of Advent}
  \printterce[../Advent2]{inc-Advent2-Vespers2}

  \section{Third Sunday of Advent}
  \printterce[../Advent3]{inc-Advent3}

  \section{Fourth Sunday of Advent}
  \printterce[../Advent4]{inc-Advent4}

}

%TODO Appendix
\chapter{Appendix}
{
\section{Chants at Benediction}
\def\gabcfolder{../BenedictionChants}
\printgabc{8.}{}{O}{OSalutaris}

%\def\translationmargin{68pt}
\translation[]{
\textbf{1.} O saving Victim of the world,
Who openest wide the gates on high,
The foe his bands on us hath hurled,
O, give us strength; for aid we cry.
%\hspace{65ex}
\textbf{2.} To thee, one Lord, yet Three in One,
Let everlasting glory be :
O grant us life that end hath none,
In Fatherland to spend with thee. Amen.}
\let\translationmargin=\undefined
\bigskip

\printgabc{7.}{}{O}{OSalutaris_v2}


\needspace{4\baselineskip}
\printgabc{3.}{}{T}{TantumErgo}

\needspace{4\baselineskip}
{\begin{center}
2. \emph{Another Chant.}
%\vspace{-1ex}
\end{center}
}
\printgabc{1.}{}{T}{TantumErgo_v2}

\needspace{4\baselineskip}
\translation[]{
\textbf{1.} Therefore we, before it bending,
this great sacrament adore;
type and shadows have their ending
in the new rite evermore;
faith, our outward sense amending,
maketh good defects before.
\textbf{2.} Honour, laud, and praise addressing
to the Father and the Son,
might ascribe we, virtue, blessing,
and eternal benison;
Holy Ghost, from both progressing,
equal laud to thee be done. Amen.
}

\needspace{4\baselineskip}
{\begin{center}
3. \emph{Modern Chant.}
\end{center}
}

\needspace{4\baselineskip}
\printgabc{5.}{}{T}{TantumErgo_v3}

\needspace{4\baselineskip}
{\begin{center}
4. \emph{Spanish Chant.}
\end{center}
}

\needspace{4\baselineskip}
\printgabc{5.}{}{T}{TantumErgo-Spanish}


\needspace{4\baselineskip}
\begin{columns}
\versicle{Pánem de cǽlo præstitísti éis. (P. T. allelúia.)}{You have given them bread from heaven.}
\response{Omne delectaméntum in se habéntem. (P. T. allelúia.)}{Containing all sweetness within it.}
\colchunk{}
\colplacechunks{}
\colchunk{\hspace*{3em}Orémus.}\colchunk{\hspace*{3em}Let us pray,}
\colplacechunks{}
\prayer{Deus, qui nobis sub Sacraménto mirábili passiónis tuæ memóriam reliquísti~:~\dag{} tríbue, quǽsumus, ita nos córporis et sánguinis tui sacra mystéria venerári;~* ut redemptiónis tuæ fructum in nobis júgiter sentiámus.  Qui vivis et regnas in sǽcula sæculórum.  Amen.}
{O God, who hast left us a memorial of Thy passion under a wonderful Sacrament: Grant, we bessech Thee, that we may so venerate the sacred mysteries of Thy body and blood that we may continually sense in ourselves the fruit of Thy redemption. Who livest and reignest\dots{}}
\end{columns}

%TODO: we could add the Divine Praises might be nice...

\needspace{4\baselineskip}
{
\printgabc{5.}{}{A}{AdoremusInAeternum}}%

\begin{psalmverses}[1]
\item \emph{Quóniam} confirmáta est super nos misericórdia \textbf{é}jus~:~* et véritas Dómini mánet \textbf{in} æ\textbf{tér}num.
\item[\Rbar{}] \emph{Adorémus}.
\item \emph{Glória} Pátri, et \textbf{Fí}lio,~* et Spi\textbf{rí}tui \textbf{Sán}cto.
\item \emph{Sicut é}\-rat in princípio, et núnc, et \textbf{sém}per,~* et in sǽcula sæcu\textbf{ló}rum. \textbf{A}men.
\item[\Rbar{}] \emph{Adorémus}.
\end{psalmverses}

\translation[]{Let us adore for ever the Most Holy Sacrament.
{\emph Ps.} All ye nations, praise the Lord : all ye peoples, sing his glory.
Because his merciful kindness is great towards us, and the truth of the Lord endureth for ever. Let us adore\ldots Glory.}
\vfil
}
{
\clearpage
\label{magnificat-grassi}
\sectionmark{Magnificat by Grassi}
\addcontentsline{toc}{section}{Magnificat by Grassi}

\def\betweenLilyPondSystem#1{
  \ifnum#1>1
    \vfil\noindent
  \else
    \linebreak
  \fi
  %\vspace{0.5\baselineskip}
}
\newcommand{\includelilypond}[1]{
  \noindent
  \import{../misc/}{#1}
}
\def\magsolemn{F}
\def\maggrassi{T}
\definemag{8}{G}
\def\annot{\magtone. \magend}
\def\greinitialformat#1{%
{\fontsize{50}{50}\selectfont #1}%
}
\setinitialspacing{M}
\def\magoddverses{\input{../psalms/magnificat-\magtone\nostarendmag\if\magsolemn Tsolemn\else simple\fi-grassi-chant-verses}}
\global\def\magtex{../psalms/Magnificat\if\magsolemn Tsolemn\else simple\fi\magtone\nostarendmag}
\greblockcustos
\begin{oddversesmagnificat}{\magtex}
\magoddverses
\end{oddversesmagnificat}

\includelilypond{magnificat-grassi-small-systems}



\def\magsolemn{T}
\let\magant=\undefined
\let\magantlinetwo=\undefined
\let\magtex=\undefined
\let\magverses=\undefined
\definemag{8}{G}
\def\annot{\magtone. \magend}
\setinitialspacing{M}
\global\def\magtex{../psalms/Magnificat\if\magsolemn Tsolemn\else simple\fi\magtone\nostarendmag}
\begin{oddversesmagnificat}{\magtex}
\magoddverses
\end{oddversesmagnificat}
}

\end{document}

