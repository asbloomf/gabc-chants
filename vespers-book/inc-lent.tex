\chapter{Proper of the Time -- Shrovetide \& Lent}
{
\newcommand{\printcommonvespers}[1][2]{
	\vspace{-0.3\baselineskip}
	\subtitle{\nth{#1} Class, \liturgicalcolor{}}
	\smallskip
	\deusinadjutorium{}
	\hfill
	\printnote{\emph{Vespers of Sundays throughout the year}, p.~\pageref{sundayvespers}.\par}
	\smallskip
}
\def\liturgicalcolor{Violet}
\def\printcommemnote{\smallskip
\noindent\printnote{\commemorations{}  Otherwise \benedicamusdominoreference{lent}}
}
{
\newcommand{\benedicamusdomino}[1][sunday]{
  \benedicamusdominomaster{#1}
}

\newcommand{\printhymnnote}{
	\noindent\printnote{Hymn.~\emph{Lucis Creátor}, page \pageref{hymn-luciscreator}.
	\Vbar~\emph{Dirigátur}, page \pageref{vr-dirigatur}.}
}

{
\section{Septuagesima}
\label{septuagesima}
%\printnote{N.B.~From this Sunday until Palm Sunday inclusive, \emph{Allelúia} is replaced by \emph{Laus tibi\ldots{}} in the \emph{Deus in adjutórium}.}
\printcommonvespers{}

\printnote{N.B.~From this Sunday until Palm Sunday inclusive, at the \emph{Deus, in adjutórium}, the \emph{Allelúia} is replaced by \emph{Laus tibi, Dómine, Rex æternæ glóriæ}.}
\def\beginchaptercols{\begin{parcolumns}[rulebetween,colwidths={1=0.47\linewidth}]{2}}
\def\precollect{\printvrdirigatur}
%\let\oldthing=\printhymnnote
%\def\printhymnnote{\oldthing\bigskip}

%\def\premagverses{\oldneedspace{10\baselineskip}}
\printvespersmag[../Lent]{inc-VespersMagnificatSeptuagesima}

\bigskip
\benedicamusdomino{}
}

{
\section{Sexagesima}
\label{sexagesima}
\printcommonvespers{}
\def\beginchaptercols{\begin{parcolumns}[rulebetween,colwidths={1=0.445\linewidth}]{2}}
\def\begincollectcols{\begin{parcolumns}[rulebetween,colwidths={1=0.46\linewidth}]{2}}
\def\precollect{\printvrdirigatur}

%\def\premagverses{\oldneedspace{18\baselineskip}}
\printvespersmag[../Lent]{inc-VespersMagnificatSexagesima}

\bigskip
\benedicamusdomino{}
}

{
\section{Quinquagesima}
\label{quinquagesima}
\printcommonvespers{}
\printvespersmag[../Lent]{inc-VespersMagnificatQuinquagesima}

\bigskip
\benedicamusdomino{}
}
}

%lent
{
\renewcommand{\printcommonvespers}[1][2]{
	\vspace{-0.3\baselineskip}
	\subtitle{\nth{#1} Class}
	\smallskip
	\deusinadjutorium{}
	\hfill
	\printnote{\emph{Vespers of Sundays throughout the year}, p. \pageref{sundayvespers}.\par}
}

\newcommand{\benedicamusdomino}[1][lent]{
  \benedicamusdominomaster{#1}
}

{
\section{\nth{1} Sunday of Lent}
\label{lent1}
\printcommonvespers[1]
\def\hymnlabel{hymn-audibenigneconditor}
\def\vrlabel{vr-angelissuis}
\def\hymninput{\gabcfolder/inc-hymn-AudiBenigneConditor}
%\def\begincollectcols{\begin{parcolumns}[rulebetween,colwidths={1=0.44\linewidth}]{2}}
%\def\premagnificat{\pagebreak}
\def\premagnificat{\needspace{8\baselineskip}}
\printvespersmag[../Lent]{inc-VespersMagnificatLent1}

\bigskip
\benedicamusdomino{}
}

\newcommand{\printhymnnote}{
	\noindent\printnote{Hymn.~\emph{Audi, benígne Cónditor}, page \pageref{hymn-audibenigneconditor}.
	\Vbar~\emph{Angelis suis}, page \pageref{vr-angelissuis}.}
}

{
\section{\nth{2} Sunday of Lent}
\label{lent2}
\printcommonvespers[1]
\def\precollect{\vspace{-1.5\baselineskip}}
\printvespersmag[../Lent]{inc-VespersMagnificatLent2}

\def\commemorations{If today is March 18 or 19, \emph{First Vespers of St Joseph} is commemorated as follows.}
\printcommemnote{}
}

\medskip
\hrule
\medskip
{
\label{stjoseph-commem}
\def\begincollectcols{\begin{parcolumns}[rulebetween,colwidths={1=0.42\linewidth}]{2}}
\def\vrlinebreak{T}
\printcommemoration[../March19-StJoseph]{commemorationStJoseph-Vespers1}

\bigskip
\benedicamusdomino{}
}

{
\section{\nth{3} Sunday of Lent}
\label{lent3}
\printcommonvespers[1]
\printvespersmag[../Lent]{inc-VespersMagnificatLent3}

\def\commemorations{If today is March 18 or 19, \emph{First Vespers of St Joseph} is commemorated as on page \pageref{stjoseph-commem}.  If today is March 24 or 25, \emph{First Vespers of the Annunciation} is commemorated as follows.}
\printcommemnote{}
}

\medskip
\hrule
\medskip
{
\oldneedspace{12\baselineskip}
\label{annunciation-commem}
\def\begincollectcols{\begin{parcolumns}[rulebetween,colwidths={1=0.43\linewidth}]{2}}

\printcommemoration[../March25-Annunciation]{commemorationAnnunciation-Vespers1}

\bigskip

\benedicamusdominolentoreaster{}
}

\def\commemorations{If today is March 18 or 19, \emph{First Vespers of St Joseph} is commemorated as on page \pageref{stjoseph-commem}.  If today is March 24 or 25, \emph{First Vespers of the Annunciation} is commemorated as on page \pageref{annunciation-commem}.}
{
\section{\nth{4} Sunday of Lent}
\label{lent4}
\def\liturgicalcolor{Violet or Rose}
\printcommonvespers[1]
\needspace{8\baselineskip}
\printvespersmag[../Lent]{inc-VespersMagnificatLent4}

\printcommemnote{}
}

\medskip
\hrule
{
\let\printhymnnote=\undefined
\section{Passion Sunday}
\label{lent5}\label{passionsunday}
\printcommonvespers[1]
\def\hymnlabel{hymn-vexillaregis}
\def\prehymn{\needspace{8\baselineskip}\printnote{All kneel for the sixth verse of the following hymn.}}
\def\vrlabel{vr-eripeme}
\def\hymninput{\gabcfolder/inc-hymn-VexillaRegis}
\def\premagnificat{\needspace{14\baselineskip}}
\def\precollect{\vspace{-1.5\baselineskip}}
\printvespersmag[../Lent]{inc-VespersMagnificatPassion1}

\printcommemnote{}
}

\medskip
\hrule
\medskip
{
\section{Palm Sunday}
\label{lent6}\label{palmsunday}
\printcommonvespers[1]
\renewcommand{\printhymnnote}{
	\noindent\printnote{Hymn.~\emph{Vexílla Regis}, page \pageref{hymn-vexillaregis}.
	\Vbar~\emph{Eripe me, Dómine}, page \pageref{vr-eripeme}.}
}
\printvespersmag[../Lent]{inc-VespersMagnificatPassion2}

\bigskip
\benedicamusdomino{}
}

}
}