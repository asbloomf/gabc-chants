\chapter{Proper of the Time -- Shrovetide \& Lent}
{
\newcommand{\printcommonvespers}[1][2]{
	\subtitle{\nth{#1} Class}
	\printnote{From Vespers of Sundays throughout the year on page \pageref{sundayvespers}.}
}
\def\printcommemnote{\smallskip
\noindent
\printnote{\commemorations{}  Otherwise \Vbar~\emph{Bendicámus Dómino}, page \pageref{benedicamusdomino-lent}.}
}
{
\newcommand{\benedicamusdomino}[1][sunday]{
	\noindent\printnote{\Vbar~\emph{Benedicámus Dómino}, page \pageref{benedicamusdomino-#1}.}
	\bigskip
	\hrule
}
\newcommand{\printhymnnote}{
	\noindent\printnote{Hymn. \emph{Lucis Creátor}, page \pageref{hymn-luciscreator}.
	\Vbar~\emph{Dirigátur}, page \pageref{vr-dirigatur}.}
}

{
\section{Septuagesima}
\printcommonvespers{}
\printvespersmag[../Lent]{inc-VespersMagnificatSeptuagesima}

\bigskip
\benedicamusdomino{}
}

{
\section{Sexagesima}
\printcommonvespers{}
\def\beginchaptercols{\begin{parcolumns}[rulebetween,colwidths={1=0.445\linewidth}]{2}}
\def\begincollectcols{\begin{parcolumns}[rulebetween,colwidths={1=0.46\linewidth}]{2}}
\printvespersmag[../Lent]{inc-VespersMagnificatSexagesima}

\bigskip
\benedicamusdomino{}
}

{
\section{Quinquagesima}
\printcommonvespers{}
\printvespersmag[../Lent]{inc-VespersMagnificatQuinquagesima}

\bigskip
\benedicamusdomino{}
}
}

%lent
{
\newcommand{\benedicamusdomino}[1][lent]{
	\noindent\printnote{\Vbar~\emph{Benedicámus Dómino}, page \pageref{benedicamusdomino-#1}.}
	\bigskip
	\hrule
}
\newcommand{\printhymnnote}{
	\noindent\printnote{Hymn. \emph{Audi, benígne Cónditor}, page \pageref{hymn-audibenigneconditor}.
	\Vbar~\emph{Angelis suis}, page \pageref{vr-angelissuis}.}
}

{
\section{\nth{1} Sunday of Lent}
\printcommonvespers[1]
\def\printfullhymn{
	\label{hymn-audibenigneconditor}
	{
	\def\hymnlinetwo{\oldstylenums{2.}}
\def\hymntex{hymn-AudiBenigneConditor}
\def\hymninitial{A}
\def\hymntranslation{\item Thou loving Maker of mankind, before thy throne we pray and weep; oh, strengthen us with grace divine, duly this sacred Lent to keep.
\item Searcher of hearts! thou dost discern our ills, and all our weakness know: again to thee with tears we turn; again to us thy mercy show.
\item Much have we sinn'd; but we confess our guilt, and all our faults deplore: oh, for the praise of thy great Name, our fainting souls to health restore!
\item And grant us, while by fasts we strive this mortal body to control, to fast from all the food of sin, and so to purify the soul.
\item Hear us, O Trinity thrice blest! Sole Unity! to Thee we cry: vouchsafe us from these fasts below to reap immortal fruit on high.}

\def\vrtex{hymn-AudiBenigneConditorvr}
\def\vtranslation{God hath given His Angels charge over thee.}
\def\rtranslation{To keep thee in all thy ways.}

	\printhymn{\oldstylenums{\hymnlinetwo}}{\hymninitial}{\hymntex}{\hymntranslation}

	{
		\def\vrlinebreak{T}
		\label{vr-angelissuis}
		\printvr[\greseteolcustos{manual}]{\vrtex}{\vtranslation}{\rtranslation}
	}
	}
}
\printvespersmag[../Lent]{inc-VespersMagnificatLent1}

\bigskip
\benedicamusdomino{}
}

{
\section{\nth{2} Sunday of Lent}
\printcommonvespers[1]
\printvespersmag[../Lent]{inc-VespersMagnificatLent2}

\def\commemorations{If today is March 18 or 19, the First Vespers of St Joseph is commemorated as follows.}
\printcommemnote{}
}

{
\label{stjoseph-commem}
\def\begincollectcols{\begin{parcolumns}[rulebetween,colwidths={1=0.42\linewidth}]{2}}
\def\vrlinebreak{T}
\printcommemoration[../March19-StJoseph]{commemorationStJoseph-Vespers1}

\bigskip
\benedicamusdomino{}
}

{
\section{\nth{3} Sunday of Lent}
\printcommonvespers[1]
\printvespersmag[../Lent]{inc-VespersMagnificatLent3}

\def\commemorations{If today is March 18 or 19, the First Vespers of St Joseph is commemorated as on page \pageref{stjoseph-commem}.  If today is March 25 or 26, the First Vespers of the Annunciation is commemorated as follows.}
\printcommemnote{}
}

{
\label{annunciation-commem}
%\def\begincollectcols{\begin{parcolumns}[rulebetween,colwidths={1=0.43\linewidth}]{2}}
\printcommemoration[../March25-Annunciation]{commemorationAnnunciation-Vespers1}

\bigskip
\benedicamusdomino{}
}

\def\commemorations{If today is March 18 or 19, the First Vespers of St Joseph is commemorated as on page \pageref{stjoseph-commem}.  If today is March 25 or 26, the First Vespers of the Annunciation is commemorated as on page \pageref{annunciation-commem}.}
{
\section{\nth{4} Sunday of Lent}
\printcommonvespers[1]
\printvespersmag[../Lent]{inc-VespersMagnificatLent4}

\printcommemnote{}
}

{
\section{Passion Sunday}
\printcommonvespers[1]
\def\printfullhymn{
	\label{hymn-vexillaregis}
	{
	\def\hymnlinetwo{\oldstylenums{1.}}
\def\hymntex{hymn-VexillaRegis}
\def\hymninitial{V}
\def\hymntranslation{\item Forth comes the Standard of the King:
All hail, thou Mystery ador'd!
Hail, Cross! on which the Life Himself
Died, and by death our life restor'd!
\item On which our Saviour's holy side
Rent open with a cruel spear
Of blood and water poured a stream,
To wash us from defilement clear.
\item O sacred wood! in Thee fulfill'd
Was holy David's truthful lay!
Which told the world, that from a tree
The Lord should all the nations sway.
\item Most royally empurpled o'er,
How beauteously thy stem doth shine!
How glorious was its lot to touch
Those limbs so holy and divine!
\item Thrice blest, upon whose arms outstretched
The Saviour of the world reclined;
Balance sublime! upon whose beam
Was weighed the ransom of mankind.
\item Hail, Cross! thou only hope of man,
Hail, on this holy Passionday!
To saints increase the grace they have;
From sinners purge their guilt away.
\item Salvation's spring, blest Trinity,
Be praise to Thee through earth and skies:
Thou through the Cross the victory
Dost give; oh, also give the prize!
Amen.}

\def\vrtex{hymn-VexillaRegisvr}
\def\vtranslation{Deliver me, O Lord, from the wicked man.}
\def\rtranslation{And save me from the evil-doer.}

	\printhymn{\oldstylenums{\hymnlinetwo}}{\hymninitial}{\hymntex}{\hymntranslation}

	{
		\def\vrlinebreak{T}
		\label{vr-eripeme}
		\printvr[\greseteolcustos{manual}]{\vrtex}{\vtranslation}{\rtranslation}
	}
	}
}
\printvespersmag[../Lent]{inc-VespersMagnificatPassion1}

\printcommemnote{}
}

{
\section{Palm Sunday}
\printcommonvespers[1]
\renewcommand{\printhymnnote}{
	\noindent\printnote{Hymn. \emph{Vexílla Regis}, page \pageref{hymn-vexillaregis}.
	\Vbar~\emph{Eripe me, Dómine}, page \pageref{vr-eripeme}.}
}
\printvespersmag[../Lent]{inc-VespersMagnificatPassion2}

\bigskip
\benedicamusdomino{}
}

}
}